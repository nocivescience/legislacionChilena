Párrafo 1°
    Del plan de emergencia y de los planos del condominio

     
    Artículo 40.- Todo condominio deberá tener un plan de emergencia ante siniestros o emergencias, tales como incendios, terremotos, tsunamis u otros eventos que puedan dañar a las personas, a las unidades y/o a los bienes de dominio común del condominio. El plan de emergencia deberá incluir las acciones a tomar antes, durante y después del siniestro o emergencia, con especial énfasis en la alerta temprana y los procedimientos de evacuación ante incendios.
    El primer plan de emergencia, que deberá contener el plan de evacuación, tendrá que ser suscrito por la persona natural o jurídica propietaria del condominio y deberá acompañarse como antecedente al solicitar la recepción definitiva del proyecto acogido al régimen de copropiedad inmobiliaria, salvo que la solicitud para acogerse al referido régimen se presente respecto de una edificación que ya cuenta con recepción definitiva, en cuyo caso el plan de emergencia deberá acompañarse al solicitar el certificado referido en el artículo 48 de esta ley.
    El plan de emergencia deberá ser actualizado por el comité de administración, cuando se modifiquen las condiciones generales de seguridad, de seguridad contra incendios y el buen funcionamiento de las instalaciones de emergencia definidas en el permiso de edificación.
    Respecto al plan de evacuación, deberá ser actualizado al menos una vez al año, considerando el número de residentes y especialmente a las personas ocupantes con discapacidad, con movilidad reducida, infantes y población no hispano parlante, señalando las acciones determinadas para su evacuación segura y expedita, debiendo incluir acciones de capacitación que procedan y los respectivos simulacros de evacuación según los diferentes tipos de eventos o emergencias.
    Siempre deberá mantenerse en la recepción o conserjería del condominio un archivo de los documentos que conforman el plan de emergencia y el plan de evacuación actualizados, incluido un plano del condominio con indicación de las vías de evacuación y las instalaciones de emergencia, tales como los grifos o bocas de incendio, sistemas de respaldo de energía o grupo electrógeno, alumbrado de emergencia, sistema de detección de humos y alarmas, red seca, red húmeda, sistemas de extinción manual o automática; incluyendo además las instalaciones de agua potable, alcantarillado, electricidad y calefacción, con los artefactos a gas contemplados y sus requerimientos de ventilación si correspondiese, y cualquier otra información de instalaciones o recintos que sea necesario conocer frente a los distintos tipos de eventos o emergencias considerados en el plan.
    La elaboración del primer plan de emergencia, así como sus actualizaciones, serán realizadas y suscritas por un ingeniero en prevención de riesgos, debiendo dar cumplimiento a la norma técnica que para dicho efecto se oficialice. La actualización de este plan deberá ser suscrita además por el presidente del comité de administración y por el administrador del condominio.
    El plan de emergencia, incluido el plan de evacuación, así como sus actualizaciones, deberán ser entregados en formato material y digital a la respectiva unidad de Carabineros de Chile y del Cuerpo de Bomberos que corresponda a la comuna donde se emplaza el condominio. Dichas entidades podrán hacer las observaciones que estimen pertinentes a la persona natural o jurídica propietaria que presenta el primer plan de emergencia, o al comité de administración tratándose de las actualizaciones del plan.

     
    Párrafo 2°
    De las revisiones y certificaciones en las unidades

     
    Artículo 41.- Los copropietarios, arrendatarios u ocupantes de las unidades que compongan el condominio están obligados a facilitar la expedición de revisiones o certificaciones en el interior de las mismas, cuando hayan sido dispuestas conforme a la normativa vigente. Si no otorgaren las facilidades para efectuarlas, habiendo sido notificados por escrito por el administrador en la dirección que cada uno registre en la administración, serán sancionados conforme a lo dispuesto en el artículo 27.
     
    Artículo 42.- Si se viere comprometida la seguridad o conservación de un condominio sea respecto de sus bienes comunes o de sus unidades, por efecto de filtraciones, inundaciones, emanaciones de gas u otros desperfectos o imprevistos, para cuya reparación fuere necesario ingresar a una unidad, no encontrándose el propietario, arrendatario u ocupante que facilite o permita el acceso, el administrador del condominio podrá ingresar forzadamente a ella, debiendo hacerlo acompañado de un copropietario, quien deberá levantar acta detallada de la diligencia, conforme al reglamento de esta ley, y remitirla al comité de administración para su incorporación en el libro de actas del mismo, debiendo en todo caso dejar copia del acta en el interior de la unidad. Los gastos que se originen serán de cargo del o los responsables del daño producido.
     
    Párrafo 3°
    De los seguros

     
    Artículo 43.- Los condominios que contemplen el destino habitacional en alguna de sus unidades deberán contratar y mantener vigente un seguro colectivo contra incendio, que cubra los daños que sufran todos los bienes e instalaciones comunes y que otorgue opciones a los copropietarios para cubrir los daños que sufran sus unidades, especialmente cuando éstas formen parte de una edificación continua, pareada o colectiva. Lo anterior es sin perjuicio de otras coberturas complementarias que la asamblea de copropietarios decida incluir en la respectiva póliza para la protección de los bienes comunes y/o sus unidades, tales como sismo o salida de mar.
    La contratación, renovación y término del seguro colectivo contra incendio del condominio deberá efectuarse conforme a las exigencias, normas procedimentales y excepciones que establezca el reglamento de esta ley y la normativa que dicte la Comisión para el Mercado Financiero, en el ejercicio de sus funciones.
    Al respecto, la referida Comisión deberá establecer mecanismos que protejan los derechos de los copropietarios y que eviten el doble pago de seguros por parte de éstos, resguardando:
     
    a) Que los copropietarios puedan presentar la póliza contratada por el condominio ante las entidades crediticias, con el objeto de ejercer el derecho contemplado en el inciso tercero del artículo 40 del decreto con fuerza de ley N° 251, de 1931, del Ministerio de Hacienda.
    b) Que los copropietarios puedan renunciar a las opciones que el seguro del condominio contempla para cubrir los daños que sufra su respectiva unidad, especialmente cuando ésta se encuentra asegurada contra incendio mediante otra póliza vigente, como la que dicho copropietario contrate con ocasión de una operación hipotecaria. Con todo, dicha renuncia no implicará, en caso alguno, que ese copropietario se exima de la obligación de pagar la parte del seguro del condominio correspondiente a la cobertura de los daños que sufran los bienes e instalaciones comunes, la que será plenamente exigible.
    c) Que el pago de indemnizaciones por los daños parciales que sufra una unidad, ya sea que se trate de la liquidación del seguro del condominio o de la liquidación de otra póliza contratada con ocasión de una operación hipotecaria, se destine, en primer lugar, a la reparación del bien asegurado y no al pago del saldo insoluto en favor del acreedor hipotecario de dicha unidad.
     
    La contratación de un seguro colectivo contra incendio también será exigible respecto de aquellos condominios que no contemplen unidades con destino habitacional, salvo que el reglamento de copropiedad del condominio establezca lo contrario.
    En el caso de los condominios de viviendas sociales, la cobertura contra el riesgo de incendio se regirá por las normas especiales y las excepciones que establezca el reglamento de la ley, con el objeto de cautelar que tales condominios cuenten con un adecuado resguardo ante dicho siniestro, pero sin imponerles una carga excesiva a los copropietarios. Lo anterior es sin perjuicio de la posibilidad que tales condominios y las unidades que los conforman puedan postular y acceder, de manera preferente, a los recursos públicos referidos en el artículo 68, con el objeto de solventar el pago de reparaciones o reconstrucciones derivadas de la ocurrencia de un incendio u otra catástrofe.

    TÍTULO VIII
    FÓRMULAS DE RESOLUCIÓN DE CONFLICTOS


    Párrafo 1°
    De la resolución judicial
   
    Artículo 44.- Serán de competencia de los juzgados de policía local correspondientes y se sujetarán a las disposiciones de la ley Nº 18.287 y, en subsidio, a las normas del Libro Primero del Código de Procedimiento Civil, las contiendas que surjan en el ámbito del régimen especial de copropiedad inmobiliaria establecido en esta ley y que se promuevan entre los copropietarios o entre éstos y la asamblea de copropietarios, el comité de administración o el administrador, o entre estos mismos órganos de administración de la copropiedad inmobiliaria, relativas a la administración o funcionamiento del condominio, para lo cual estos tribunales estarán investidos de todas las facultades que sean necesarias a fin de resolver esas controversias. En el ejercicio de estas facultades, el juez podrá:
     
    a) Declarar la nulidad total o parcial del reglamento de copropiedad en conformidad al Párrafo 3° del TÍTULO III de esta ley.
    b) Declarar la nulidad de los acuerdos adoptados por la asamblea con infracción de las normas de esta ley y de su reglamento o de los reglamentos de copropiedad. Para estos efectos, el tribunal deberá sujetarse a lo dispuesto en el inciso quinto del artículo 10 de esta ley.
    c) Citar a asamblea de copropietarios, si el administrador o el presidente del comité de administración no lo hicieren, aplicándose al efecto las normas contenidas en el artículo 654 del Código de Procedimiento Civil, en lo que fuere pertinente. A esta asamblea deberá asistir un notario como ministro de fe, quien levantará acta de lo actuado. La citación a asamblea se notificará mediante carta certificada y/o correo electrónico, sujetándose a lo previsto en el inciso primero del artículo 16 de la presente ley. Para estos efectos, el administrador, a requerimiento del juez, deberá poner a disposición del tribunal la nómina de copropietarios a que se refiere el citado inciso primero, dentro de los cinco días siguientes desde que le fuere solicitada y, si así no lo hiciere, se le aplicará la multa prevista en el inciso tercero del artículo 27.
    d) Exigir al administrador que someta a la aprobación de la asamblea de copropietarios rendiciones de cuentas, fijándole plazo para ello y, en caso de infracción, aplicarle la multa a que alude la letra anterior.
    e) Citar a asamblea de copropietarios a fin de que se proceda a elegir el comité de administración en los casos en que no lo hubiere. La citación a asamblea se notificará mediante carta certificada y/o correo electrónico, conforme a una nómina que deberá ser puesta a disposición del tribunal por los copropietarios que representen, a lo menos, el 5% de los derechos en el condominio. No obstante, tratándose de condominios de viviendas sociales, el juez podrá disponer que un funcionario del tribunal o de la municipalidad respectiva notifique la citación a asamblea mediante la entrega de esta última a cualquier persona adulta que se encontrare en el domicilio del copropietario o a través de su fijación en la puerta de este lugar, conforme a una nómina de copropietarios que deberá ser proporcionada por quien solicitó la citación. Para este efecto, el juez podrá solicitar al conservador de bienes raíces competente que complemente dicha nómina respecto de aquellas unidades cuyos dueños no estuvieren identificados, de acuerdo con las inscripciones de dominio vigentes. Asimismo, podrá disponer que un funcionario del tribunal o de la municipalidad respectiva se desempeñe como ministro de fe.
    f) En general, adoptar todas las medidas necesarias para la solución de los conflictos que afecten a los copropietarios derivados de su condición de tales, pudiendo ejercer siempre labores de amigable componedor, para lo cual podrá proponer bases de arreglo e instar a éstos, en tanto no haya sido posible resolverlos previamente en las asambleas respectivas.
     
    Artículo 45.- Las resoluciones que se dicten en las gestiones a que alude el artículo anterior serán apelables,  aplicándose a  dicho recurso las  normas contempladas en el Título III de la ley Nº 18.287.
     
    Párrafo 2°
    Del arbitraje

     
    Artículo 46.- Sin perjuicio de lo dispuesto en el artículo 44, las contiendas a que se refiere dicho precepto podrán someterse a la resolución de un juez árbitro, en cualquiera de las calidades a que se refiere el artículo 223 del Código Orgánico de Tribunales. En contra de la sentencia arbitral se podrán interponer los recursos de apelación y de casación en la forma, de acuerdo a lo previsto en el artículo 239 de ese mismo Código.
    La designación del árbitro deberá efectuarse de consuno por las partes, quienes también deberán establecer si será de derecho, arbitrador o mixto. A falta de acuerdo, el árbitro será arbitrador y su designación corresponderá al juez de letras competente.
     
    Párrafo 3°
    De la resolución extrajudicial

     
    Artículo 47.- La respectiva municipalidad podrá atender extrajudicialmente los conflictos que se promuevan entre los copropietarios o entre éstos y el comité de administración o el administrador, que previamente no hayan podido solucionarse en las asambleas correspondientes, y para ello estará facultada para citar a reuniones a las partes en conflicto y proponer vías de solución, haciendo constar lo obrado y los acuerdos adoptados en actas que se levantarán al efecto. La copia del acta pertinente, autorizada por el secretario municipal respectivo, constituirá plena prueba de los acuerdos adoptados y deberá agregarse al libro de actas del comité de administración. En todo caso, la municipalidad deberá abstenerse de actuar si alguna de las partes hubiere recurrido o recurriera al juez de policía local o a un árbitro, conforme a lo dispuesto en los artículos 44 y 46 de esta ley.

    TÍTULO IX
    DE LA CONSTITUCIÓN DE LA COPROPIEDAD


    Artículo 48.- Para acogerse al régimen de copropiedad inmobiliaria, todo condominio deberá cumplir con las normas exigidas por esta ley y su reglamento, por la Ley General de Urbanismo y Construcciones, por la Ordenanza General de Urbanismo y Construcciones, por los instrumentos de planificación territorial y por las normas que regulen el área de emplazamiento del condominio, sin perjuicio de las excepciones y normas especiales establecidas en esta ley, en el decreto N° 1.101, del Ministerio de Obras Públicas, que fijó el texto definitivo del decreto con fuerza de ley N° 2, de 1959, sobre Plan Habitacional, y en el Reglamento Especial de Viviendas Económicas.
    Corresponderá a los directores de obras municipales verificar que un condominio cumple con lo dispuesto en el inciso anterior y extender el certificado que lo declare acogido al régimen de copropiedad inmobiliaria, haciendo constar en el mismo la fecha y la notaría en que se redujo a escritura pública el primer reglamento de copropiedad y la foja y el número de su inscripción en el registro de hipotecas y gravámenes del conservador de bienes raíces. Este certificado deberá señalar las unidades que sean enajenables dentro de cada condominio. Tratándose de condominios de viviendas sociales, deberá especificarse dicha condición en el referido certificado, sin perjuicio de lo establecido en el artículo 67 respecto de los condominios existentes a la fecha de publicación de esta ley.
    Otorgado el certificado que acoge un condominio al régimen de copropiedad inmobiliaria, la dirección de obras municipales deberá remitir copia del mismo a la Secretaría Ejecutiva de Condominios del Ministerio de Vivienda y Urbanismo y generar una carpeta física o expediente digital en el que se archivará copia de todos los actos administrativos que emita y sus respectivos antecedentes relacionados con dicho condominio, entre los que se encuentran: i) el mencionado certificado y los que, eventualmente, se emitan para modificar o dejar sin efecto la declaración que acogió el condominio al referido régimen; ii) el permiso de edificación, sus planos y el certificado de recepción definitiva; iii) el primer reglamento de copropiedad y su inscripción en el registro de hipotecas y gravámenes del conservador de bienes raíces, así como las modificaciones a dicho reglamento que digan relación con las materias referidas en el numeral siguiente; y iv) las resoluciones aprobatorias de obras u otras materias que deban contar con autorización de la referida dirección, tales como las construcciones en bienes comunes, incluidos aquellos asignados en uso y goce exclusivos; las obras de alteración, ampliación, restauración, remodelación, rehabilitación, reconstrucción o demolición de unidades del condominio o de construcciones en bienes comunes; los cambios de destino de unidades o bienes comunes; entre otras.
    Podrán acogerse al régimen de copropiedad inmobiliaria los predios con edificaciones existentes o con proyectos de edificación aprobados, así como los predios con sitios urbanizados o con proyectos de urbanización para condominio tipo B aprobados. Con todo, en el caso de los predios con proyectos de edificación o de urbanización aprobados, para acogerse al régimen de copropiedad inmobiliaria se deberá dar cumplimiento a lo establecido en el artículo 136 de la Ley General de Urbanismo y Construcciones respecto de las obras de urbanización en el espacio público existente o afecto a utilidad pública y, además, la enajenación de las unidades solo podrá efectuarse una vez recepcionadas por la dirección de obras municipales las obras de edificación y/o de urbanización de la unidad o sitio que se enajena. Esto, sin perjuicio de que el certificado que declare el proyecto acogido al régimen de copropiedad inmobiliaria permita la reserva o suscripción de contratos de promesa de compraventa respecto de las unidades enajenables, siendo aplicable lo dispuesto en el artículo 138 bis de la Ley General de Urbanismo y Construcciones.
    Si al solicitar el permiso para la ejecución de las obras que contempla el condominio, el interesado informa que el proyecto posteriormente se acogerá al régimen de copropiedad inmobiliaria, el director de obras municipales no solo deberá verificar el cumplimiento de las normas urbanísticas aplicables, sino también el de las exigencias urbanas y de construcción contempladas en esta ley.
    En el caso del inciso anterior, para extender posteriormente el certificado que acoge el proyecto al régimen de copropiedad inmobiliaria, el director de obras únicamente deberá verificar el cumplimiento de las exigencias relacionadas con la reducción a escritura pública e inscripción del primer reglamento de copropiedad, sin perjuicio de la revisión que habrá de efectuar respecto de la existencia del plan de emergencia y del cumplimiento del respectivo permiso, para otorgar la recepción definitiva de las obras.
   
    Artículo 49.- Los planos de un condominio deberán singularizar claramente cada una de las unidades en que se divide un condominio, los sectores en el caso a que se refiere el artículo 38 y los bienes de dominio común. Estos planos deberán contar con la aprobación del director de obras municipales y se archivarán en una sección especial del registro de propiedad del conservador de bienes raíces respectivo, en estricto orden numérico, conjuntamente con el certificado a que se refiere el inciso segundo del artículo 48.
   
    Artículo 50.- Las escrituras públicas que sean título para la transferencia de dominio o constitución de otros derechos reales sobre alguna unidad de un condominio deberán hacer referencia al plano a que alude el artículo anterior. En el caso de la primera de estas transferencias, deberá insertarse el certificado mencionado en el inciso segundo del artículo 48.
     
    Artículo 51.- La inscripción del título de propiedad y de otros derechos reales sobre una unidad contendrá las siguientes menciones:
     
    1) La fecha de la inscripción.
    2) La naturaleza y fecha del título, así como la notaría en que se extendió.
    3) Los nombres, apellidos y domicilios de las partes.
    4) La ubicación y los deslindes del condominio a que pertenezca la unidad.
    5) El número y la ubicación que corresponda a la unidad en el plano de que trata el artículo 49.
    6) La firma del conservador.
    7) En general, todas las demás formalidades que han de cumplir los títulos que deben inscribirse, de acuerdo al Reglamento del Registro Conservatorio de Bienes Raíces.
     
    Artículo 52.- La resolución del director de obras municipales que declare acogido un condominio al régimen de copropiedad inmobiliaria será irrevocable por decisión unilateral de esa autoridad.
    Sin perjuicio de lo dispuesto en el inciso anterior, la asamblea podrá solicitar al director de obras municipales que proceda a modificar o dejar sin efecto dicha declaración, debiendo, en todo caso, cumplirse con las normas vigentes sobre urbanismo y construcciones para la gestión ulterior respectiva y recabarse la autorización de los acreedores hipotecarios o de los titulares de otros derechos reales, si los hubiere. Si se deja sin efecto dicha declaración, la comunidad que se forme entre los copropietarios se regirá por las normas del derecho común.
   
    Artículo 53.- El director de obras municipales tendrá un plazo de treinta días corridos para pronunciarse sobre las solicitudes a que se refieren los artículos 48 y 52, contados desde la fecha de la presentación de la misma. Será aplicable a este requerimiento lo dispuesto en los incisos segundo, tercero y cuarto del artículo 118 de la Ley General de Urbanismo y Construcciones.

    TÍTULO X
    EXIGENCIAS URBANAS Y DE CONSTRUCCIÓN


    Artículo 54.- Podrán acogerse al régimen de copropiedad inmobiliaria las edificaciones que se emplacen en terrenos cuya superficie sea inferior a la superficie de subdivisión predial mínima establecida en el instrumento de planificación territorial, siempre que se trate de predios existentes y que no sean el resultado de un nuevo proceso de división del suelo.
    En un condominio tipo B, la superficie de los sitios resultantes podrá ser inferior a la superficie de subdivisión predial mínima exigida por el respectivo instrumento de planificación territorial, siempre que la superficie total de todos ellos, sumada a la superficie de terreno en dominio común, sea igual o mayor a la que resulte de multiplicar el número de todas las unidades de dominio exclusivo por la superficie de subdivisión predial mínima exigida por el instrumento de planificación territorial. Para los efectos de este cómputo, se excluirán las áreas que deban cederse conforme al artículo 59 de esta ley.
     
    Artículo 55.- Los nuevos condominios deberán respetar la trama vial que, conforme a lo dispuesto en la letra d) del artículo 28 quáter de la Ley General de Urbanismo y Construcciones, hubiere establecido el correspondiente instrumento de planificación territorial.
    Con todo, si el referido instrumento no se hubiere adaptado aún a lo dispuesto en dicha norma y el predio en que se emplazaría un nuevo condominio tiene una superficie total superior a la que establezca la Ordenanza General de Urbanismo y Construcciones, dependiendo del tipo de proyecto y/o su emplazamiento, serán aplicables las siguientes reglas supletorias:
     
    a) El proyecto de nuevo condominio deberá incorporar una trama vial que contemple, en primer lugar, la extensión de vías públicas existentes en el entorno y, si ello no fuere factible, la proyección de nuevas circulaciones destinadas al uso público, dividiendo el condominio en sectores cuya superficie sea igual o inferior a la que establezca la referida Ordenanza General para ese tipo de proyecto y emplazamiento.
    b) Excepcionalmente, la dirección de obras municipales, previo informe técnico del asesor urbanista si la municipalidad contare con dicho cargo, podrá autorizar que los condominios no se dividan, o bien, que uno o más de los sectores que se generen excedan la superficie que establezca la Ordenanza General para ese tipo de proyecto y emplazamiento, atendido que el nuevo condominio no representaría un impedimento para la conectividad del sector. De la misma manera, se podrán autorizar excepciones a lo establecido en el inciso segundo del artículo 59 de esta ley.
    c) Las solicitudes para acogerse a estas autorizaciones excepcionales deberán dar cumplimiento a los requisitos que se establezcan en la Ordenanza General de Urbanismo y Construcciones e incluir un informe fundado, suscrito por el arquitecto del proyecto.
     
    Las vías que se proyecten para cumplir la exigencia contenida en este artículo se incorporarán al dominio nacional de uso público al momento de su recepción definitiva.
     
    Artículo 56.- En cada uno de los sitios urbanizados de un condominio que pertenezcan en dominio exclusivo a uno o más copropietarios solo podrán construirse edificaciones que cumplan con las normas urbanísticas establecidas en el respectivo plan regulador comunal o, en el silencio de éste, con las que resulten de aplicar otras normas de la Ley General de Urbanismo y Construcciones y de su Ordenanza General.
     
    Artículo 57.- En aquellos condominios en los que no se hubiere utilizado todo el potencial edificatorio derivado de las normas del plan regulador comunal aplicable, especialmente en los condominios tipo B de sitios urbanizados, el porcentaje que le corresponderá a cada unidad respecto de dicho potencial edificatorio remanente estará determinado por la proporción de derechos que tenga sobre los bienes comunes.
   
    Artículo 58.- Los terrenos de dominio común y los sitios urbanizados de dominio exclusivo de cada copropietario no podrán subdividirse ni lotearse mientras exista el condominio, salvo que concurran las circunstancias previstas en el artículo 26.
     
    Artículo 59.- Todo condominio debe cumplir con las disposiciones contenidas en los artículos 66, 67, 69, 70, 134, 135 y 136 de la Ley General de Urbanismo y Construcciones, con excepción del inciso cuarto del artículo 136. Las calles, avenidas, plazas y espacios públicos que se incorporen al dominio nacional de uso público conforme al artículo 135, antes citado, serán solo aquellos que estuvieren considerados en el respectivo plan regulador y los necesarios para dar cumplimiento a lo establecido en el artículo 55 de esta ley. Las otras exigencias de urbanización establecidas en el artículo 70 de la Ley General de Urbanismo y Construcciones, relativas a áreas verdes, equipamiento y circulaciones, formarán parte de los bienes de dominio común.
    El terreno en que estuviere emplazado un condominio deberá tener acceso directo a un bien nacional de uso público. Respecto a las unidades y/o edificaciones colectivas que contemple el condominio, dicho acceso podrá ser directo o a través de circulaciones de dominio común cuya longitud no exceda los 400 metros de recorrido peatonal, medidos desde algún acceso al condominio.
    El diseño del conjunto y de las circulaciones interiores deberá asegurar el tránsito y operación expedita de vehículos de emergencia. El administrador será personalmente responsable de velar que esta condición se mantenga permanentemente. Se prohíbe la construcción o colocación de cualquier tipo de elementos que limiten las condiciones de seguridad del conjunto.
    Los deslindes del condominio que enfrenten una vialidad y que contemplen uno o más tramos de cierros opacos, deberán dar cumplimiento a las siguientes exigencias especiales:
     
    a) Cada tramo de cierro opaco no podrá exceder de un tercio de la longitud total del respectivo deslinde, con un máximo de cincuenta metros lineales por tramo. Se considerará como inicio o término de un tramo opaco, la intersección con una vialidad, un acceso al condominio, el acceso a alguna de sus unidades o la intersección con el deslinde del predio vecino.
    b) En los tramos de cierros opacos deberá resguardarse que la vereda destinada a la circulación peatonal cuente con adecuadas condiciones de iluminación nocturna.
   
    Artículo 60.- En todo condominio deberá contemplarse la cantidad de estacionamientos para automóviles y bicicletas, requerida conforme a las normas vigentes y al plan regulador respectivo. No obstante, los condominios de viviendas de interés público deberán contemplar, al menos, un estacionamiento para automóvil por cada unidad destinada a vivienda, resguardando que también exista espacio para estacionar bicicletas, ya sea en los mismos estacionamientos para automóviles o en un área común destinada al efecto, conforme a las exigencias que establezca el reglamento de la ley.
    El referido reglamento también podrá establecer supuestos en los que excepcionalmente se permita al director de obras municipales la aprobación fundada de condominios de viviendas de interés público con una dotación de estacionamientos inferior a la señalada en el inciso precedente, en atención al tamaño acotado del condominio, al emplazamiento del mismo o a otros factores técnicos o urbanísticos que justifiquen una rebaja de dicha exigencia.
    Los estacionamientos definidos que correspondan a la cuota mínima obligatoria, o aquellos que determine el Director de Obras Municipales, en virtud de la atribución contemplada en el inciso anterior, en cuyo caso se considerará el número que éste determine, deberán singularizarse en el plano a que se refiere el artículo 49 y solo podrán enajenarse o adjudicarse en uso y goce exclusivo en favor de personas que adquieran o hayan adquirido una o más unidades en el condominio. Los estacionamientos que excedan la cuota mínima obligatoria serán de libre enajenación.
    En los condominios de viviendas de interés público nuevos, la escritura de compraventa de las viviendas deberá incluir la transferencia o la asignación en uso y goce exclusivo del estacionamiento que le corresponde a dicha unidad, sin perjuicio de la posibilidad que los propietarios de tales viviendas posteriormente transfieran su estacionamiento a otro copropietario del condominio o renuncien al uso y goce exclusivo constituido en su favor; en este último caso, la asamblea de copropietarios podrá asignar a otro copropietario el derecho de uso y goce exclusivo sobre dicho estacionamiento.
    Tratándose de estacionamientos para personas con discapacidad, solo podrán asignarse en uso y goce a copropietarios, ocupantes o arrendatarios de las unidades del condominio que así lo requieran, cuando éstos correspondan a personas con discapacidad, especialmente aquellas con movilidad reducida que cuenten con la respectiva acreditación de esa condición señalada en la ley N° 20.422.
    En tanto los estacionamientos que correspondan a la cuota mínima obligatoria para personas con discapacidad no sean requeridos por las personas señaladas, podrán ser asignados temporalmente en uso y goce a otros copropietarios, concesión que finalizará por el solo ministerio de la ley, cuando sean asignados según se indica en el inciso anterior.
    Los estacionamientos de visitas tendrán el carácter de bienes comunes del condominio, sin perjuicio de su asignación a sectores determinados, conforme establezca el respectivo reglamento de copropiedad, no pudiendo ser enajenados ni asignados en uso y goce exclusivo.
    En los casos en que la Ordenanza General de Urbanismo y Construcciones permite ubicar estacionamientos en otros predios, el plano del condominio a que se refiere el artículo 49 deberá señalar tal circunstancia.



    TÍTULO XI
    DE LA MODIFICACIÓN, AMPLIACIÓN, SUBDIVISIÓN, FUSIÓN Y DEMOLICIÓN DE LA COPROPIEDAD


    Párrafo 1°
    De las solicitudes ante la dirección de obras municipales
   
    Artículo 61.- Tratándose de solicitudes ante la dirección de obras municipales, respecto de cualquiera de las autorizaciones o permisos contemplados en la Ley General de Urbanismo y Construcciones o en la presente ley, deberá identificarse en aquéllas la facultad de representar al condominio, establecida en el reglamento de copropiedad, acta de asamblea extraordinaria o mandato especial.
    La tramitación de solicitudes ante la dirección de obras municipales se efectuará conforme a lo establecido en la Ordenanza General de Urbanismo y Construcciones.
   
    Párrafo 2°
    Del cambio de destino

     
    Artículo 62.- Para cambiar el destino de una unidad se requerirá que el nuevo uso esté permitido por el instrumento de planificación territorial y que el copropietario obtenga, además del permiso de la dirección de obras municipales, el acuerdo previo de la asamblea.
     
    Párrafo 3°
    De la demolición
   
    Artículo 63.- Si la municipalidad decretase la demolición de un condominio, de conformidad a la legislación vigente en la materia, la asamblea de copropietarios, reunida en asamblea extraordinaria, acordará su proceder futuro.
     
    Párrafo 4°
    De la subdivisión
   
    Artículo 64.- Las direcciones de obras municipales podrán aprobar la subdivisión de condominios existentes, debiendo darse cumplimiento en cada uno de los condominios resultantes a las normas urbanísticas que les fueren aplicables.
    La solicitud que presenten los copropietarios podrá contener una propuesta de subdivisión del condominio, que conste de un plano suscrito por un profesional competente y que esté aprobada por los copropietarios que representen, a lo menos, el 66% de los derechos en el condominio.
    El 10% de los copropietarios de condominios de viviendas sociales, alternativamente, podrá solicitar a la dirección de obras municipales que elabore una propuesta de subdivisión. Esta propuesta, con su correspondiente plano, deberá ser aprobada por el 66% de los derechos del condominio.
    La dirección de obras municipales, por propia iniciativa, podrá elaborar propuestas de subdivisión de condominios de viviendas sociales, para facilitar una mejor administración, propuesta que también deberá ser aprobada por el 66% de los derechos del condominio.
    Para acreditar las mayorías establecidas en este artículo bastará el acta de la asamblea suscrita por los copropietarios que reúnan el citado quórum legal o, en su defecto, el instrumento en que conste la aprobación de la propuesta de subdivisión firmada por los respectivos copropietarios, protocolizada ante notario.
    La dirección de obras municipales, después de aprobadas las modificaciones por los copropietarios, dictará, si procediere, una resolución que disponga la subdivisión del condominio, la cual deberá inscribirse en el conservador de bienes raíces conjuntamente con el plano respectivo. Los cambios producidos como consecuencia de la subdivisión de los bienes del condominio regirán desde la fecha de la referida inscripción.
    Las normas de la Ley General de Urbanismo y Construcciones, de la Ordenanza General de Urbanismo y Construcciones y de los respectivos instrumentos de planificación territorial no serán aplicables a las edificaciones y a la división del suelo que se originen con motivo de la subdivisión de los condominios de viviendas sociales que se efectúe en virtud de lo dispuesto en los incisos anteriores.
    Los condominios de viviendas sociales estarán exentos del pago de los derechos municipales que pudieren devengarse respecto de las actuaciones a que se refiere este artículo.

    TÍTULO XII
    DE LOS CONDOMINIOS DE VIVIENDAS DE INTERÉS PÚBLICO


    Párrafo 1°
    Disposiciones especiales

     
    Artículo 65.- Los condominios de viviendas de interés público se regirán por las disposiciones especiales contenidas en este Título y, en lo no previsto por éstas y siempre que no se contrapongan con lo establecido en ellas, se sujetarán a las normas de carácter general contenidas en los restantes Títulos de esta ley.
   
    Artículo 66.- Para los efectos de este Título, se considerarán condominios de viviendas de interés público, los siguientes:
     
    1) Aquellos conjuntos habitacionales en régimen de copropiedad inmobiliaria, constituidos por viviendas económicas que, total o parcialmente, hayan contado para su construcción con financiamiento otorgado por el Ministerio de Vivienda y Urbanismo, o alternativamente, que sean objeto de atención para dicho ministerio mediante iniciativas de acceso a la vivienda, tales como arriendo, integración social o viviendas tuteladas.
    2) Los condominios de viviendas sociales, correspondientes a aquellos constituidos mayoritariamente por viviendas económicas cuyo valor de tasación no exceda en más de un 30% el señalado en el decreto ley Nº 2.552, de 1979, o cuyo financiamiento de construcción proviniere del Ministerio de Vivienda y Urbanismo, a través de los decretos supremos N° 155, de 2001; N° 174, de 2006, y N° 49, de 2012, todos del Ministerio de Vivienda y Urbanismo, o de los que los reemplazaren.
     
    También se considerarán como condominios de viviendas sociales, para todos los efectos, los conjuntos de viviendas preexistentes a la vigencia de esta ley, calificadas como viviendas sociales de acuerdo con los decretos leyes Nº 1.088, de 1975, y Nº 2.552, de 1979, y los construidos por los servicios de vivienda y urbanización y sus antecesores legales, directamente o a través de los planes o programas señalados anteriormente, cuando dentro de sus deslindes existan bienes de dominio común.
    Los condominios de viviendas de interés público se ejecutarán conforme a las condiciones establecidas por el Reglamento Especial de Viviendas Económicas, a que se refiere el decreto con fuerza de ley N° 2, del Ministerio de Hacienda, de 1959, sobre plan habitacional, cuyo texto definitivo fue fijado por el decreto N° 1.101, del Ministerio de Obras Públicas, de 1960.

     
    Artículo 67.- La condición de condominio de viviendas de interés público se acreditará de las siguientes formas:
     
    1) Con la declaración de condominio de interés público, que será sancionada mediante resolución del Ministro de Vivienda y Urbanismo, cuando se verifiquen las circunstancias señaladas en el numeral 1) del artículo precedente. Dicha declaración podrá efectuarse por iniciativa del Ministerio de Vivienda y Urbanismo o a solicitud de la comunidad de copropietarios, la municipalidad o el gobierno regional respectivo.
    2) Con el certificado de condominio de vivienda social, que será extendido por el director de obras municipales respectivo, cuando se constate alguna de las siguientes condiciones:
     
    a. Que el condominio está compuesto mayoritariamente por viviendas económicas de carácter definitivo, cuyo valor de tasación no excede en más de un 30% al señalado en el decreto ley Nº 2.552, de 1979, para lo cual se considerará conjuntamente:
     
    i) El valor del terreno, que será el de su avalúo fiscal vigente en la fecha de la solicitud del permiso.
    ii) El valor de construcción de la vivienda, según el proyecto presentado, que se evaluará conforme a la tabla de costos unitarios a que se refiere el artículo 127 de la Ley General de Urbanismo y Construcciones.
     
    b. Condominios que hayan contado con financiamiento proveniente del Ministerio de Vivienda y Urbanismo, a través de planes o programas dirigidos a promover el acceso de las familias que se encuentran en situación de vulnerabilidad a una solución habitacional, en cuyo caso el director de obras municipales deberá tener una copia del documento oficial que sanciona el otorgamiento del financiamiento ministerial.
    c. Condominios preexistentes a la vigencia de esta ley que hayan sido calificados como vivienda social, pero no cuenten con la certificación de la dirección de obras municipales respectiva; en este caso, el director de obras podrá certificar dicha condición basado en cualquier documento oficial donde se acrediten las circunstancias descritas en el inciso segundo del artículo precedente.
     
    Artículo 68.- Los gobiernos regionales, las municipalidades y los servicios de vivienda y urbanización podrán destinar recursos a condominios de viviendas sociales emplazados en sus respectivos territorios.
    Los recursos destinados solo podrán ser asignados con los siguientes objetos:
     
    a) En la reparación, mejoramiento o dotación de los bienes de dominio común, con el fin de mejorar la calidad de vida y seguridad de los habitantes del condominio.
    b) En gastos que demande la formalización del reglamento de copropiedad a que alude el artículo 69.
    c) En pago de primas de seguros de incendio y adicionales para cubrir riesgos catastróficos de la naturaleza, tales como terremotos, inundaciones, incendios a causa de terremotos u otros del mismo tipo.
    d) En instalaciones de las redes de servicios básicos, dentro de los deslindes del condominio, que no sean bienes comunes.
    e) En programas de mejoramiento o ampliación de las unidades del condominio o de los bienes comunes.
    f) En programas de mantenimiento y pago de servicios básicos de los bienes comunes.
    g) En apoyo de los programas de autofinanciamiento de los condominios a que se refiere el número 9) del inciso sexto del artículo 14.
    h) En programas de capacitación para los miembros del comité de administración y administradores, relativos a materias propias del ejercicio de tales cargos.
    i) En acciones de fortalecimiento de la participación y convivencia comunitaria, mediante mecanismos de difusión y actividades de capacitación dirigidas a promover el adecuado uso, administración y mantención de los bienes comunes.
    j) En la demolición parcial o total, por causas que lo ameriten, cuando sean declarados en ruina según lo establecido en la Ordenanza General de Urbanismo y Construcciones.
    k) En programas de instalación, certificación y mantención de equipos de circulación vertical.
     
    Sin perjuicio que los programas y recursos a que hace referencia este artículo están destinados preferentemente a condominios de viviendas sociales, podrán postular también a ellos los condominios de viviendas de interés público referidos en el número 1) del artículo 66, cuando se acredite que sus propietarios, arrendatarios u ocupantes a cualquier título se encuentran en situación de vulnerabilidad conforme al instrumento de caracterización socioeconómica aplicable.
    Con el objeto de promover acciones integrales y armónicas, los condominios o sus sectores podrán optar a dichos programas y recursos, aun cuando existan copropietarios que individualmente no cumplan los requisitos del respectivo programa.
    Asimismo, los condominios de viviendas sociales podrán postular a los programas financiados con recursos fiscales en las mismas condiciones que las juntas de vecinos, organizaciones comunitarias, organizaciones deportivas y otras entidades de similar naturaleza.
    Los gobiernos regionales, las municipalidades y los servicios de vivienda y urbanización respectivos podrán designar, por una sola vez, en los condominios de viviendas sociales que carezcan de administrador, una persona que actuará provisionalmente como tal, con las mismas facultades y obligaciones que aquél.
    La persona designada deberá ser mayor de edad, capaz de contratar y de disponer libremente de sus bienes y se desempeñará temporalmente mientras se designa el administrador definitivo. La designación de este último deberá realizarse en un plazo no superior a un año desde el nombramiento del administrador provisional. Sin perjuicio de lo anterior, para ejercer el cargo de administrador provisional no será necesario estar inscrito en el Registro Nacional de Administradores de Condominios.
    La asamblea de copropietarios, por acuerdo adoptado en sesión ordinaria, podrá solicitar del gobierno regional, de la municipalidad o del servicio de vivienda y urbanización que hubiere designado al administrador provisional, la sustitución de éste, por causa justificada.
     
    Artículo 69.- En el caso de condominios de viviendas sociales que no cuenten con un reglamento de copropiedad inscrito en el conservador de bienes raíces respectivo, sus copropietarios formalizarán un primer reglamento empleando los quórum señalados en el número 2) del cuadro contenido en el artículo 15.
     
    Artículo 70.- A partir del 1 de enero de 2024 los nuevos condominios de viviendas sociales no podrán contar con más de 160 unidades habitacionales.
    Si en un terreno se contempla la construcción de varios condominios de viviendas sociales, éstos podrán ser evaluados y calificados, de manera conjunta, por el respectivo servicio regional de vivienda y urbanización.
    Asimismo, podrá solicitarse ante la dirección de obras municipales la tramitación conjunta y simultánea de las aprobaciones y permisos necesarios para la construcción de dichos condominios, tales como el permiso de loteo o subdivisión del terreno en que éstos se emplazarían y los correspondientes permisos de edificación para cada uno de los condominios.
    Cada uno de estos condominios independientes deberá tener acceso directo a un bien nacional de uso público y cumplir con las exigencias de urbanización relativas a áreas verdes, equipamiento y circulaciones establecidas en esta ley, en la Ley General de Urbanismo y Construcciones y en el decreto supremo que reglamente el respectivo programa habitacional. Además, deberán contar con su propio reglamento de copropiedad y órganos de administración. Una vez ejecutadas las obras y verificado el cumplimiento de la normativa aplicable, la dirección de obras municipales otorgará las correspondientes recepciones definitivas y emitirá, para cada condominio, el respectivo certificado que lo acoja al régimen de copropiedad inmobiliaria.

     
    Artículo 71.- Para los efectos de esta ley, y sin perjuicio de lo dispuesto en el artículo 98, las municipalidades deberán incorporar a todos los condominios de viviendas sociales de la respectiva comuna en un apartado especial del registro municipal a que se refiere el artículo 6° del decreto supremo N° 58, del Ministerio del Interior, de 1997, que fija texto refundido, coordinado y sistematizado de la ley Nº 19.418, sobre Juntas de Vecinos y demás Organizaciones Comunitarias, especificando los datos de la carpeta física o expediente digital del condominio, referida en el artículo 48.
    El reglamento de copropiedad de los condominios de viviendas sociales, las escrituras públicas que contengan modificaciones de estos reglamentos y las actas que contengan la nómina de los miembros del comité de administración y la designación del administrador, en su caso, deberán quedar bajo custodia del presidente del comité de administración, sin perjuicio de la entrega de copia de tales antecedentes a la municipalidad.
     
    Artículo 72.- Las empresas que proporcionen servicios de energía eléctrica, agua potable, alcantarillado, gas u otros servicios, a un condominio de viviendas sociales, deberán dotar a cada una de las unidades de medidores individuales y cobrar, conjuntamente con las cuentas particulares de cada vivienda, la proporción que le corresponda a dicha unidad en los gastos comunes por concepto del respectivo consumo o reparación de tales instalaciones. Esta contribución se determinará en el correspondiente reglamento de copropiedad o por acuerdo de la asamblea de copropietarios, conforme a lo dispuesto en el artículo 15.
    Sin perjuicio de lo establecido en el artículo 31, para el cobro de gastos comunes los condominios de viviendas sociales podrán celebrar convenios con la municipalidad o con cualquiera de las empresas a que se refiere el inciso anterior. Facúltase a las municipalidades y a las citadas empresas de servicios para efectuar dicha labor.
    Los cobros de gastos comunes que efectúen las citadas empresas de servicios, en su caso, deberán efectuarse en documento separado del cobro de los servicios. Los convenios respectivos deberán archivarse en el registro municipal a que se refiere el artículo 71.
     
    Artículo 73.- Las actuaciones que deban efectuar los condominios de viviendas sociales en cumplimiento de esta ley estarán exentas del pago de los derechos arancelarios que correspondan a los notarios, conservadores de bienes raíces y archiveros. Para tales efectos, la calidad de condominio de viviendas sociales se acreditará mediante certificado emitido por la dirección de obras municipales correspondiente. Asimismo, la exigencia de que un notario intervenga en dichas actuaciones se entenderá cumplida si participa en ellas, como ministro de fe, un funcionario municipal designado al efecto o el oficial de registro civil competente.
    Los condominios de viviendas sociales estarán exentos del pago de los derechos municipales que pudieren devengarse respecto de las actuaciones del ministro de fe, en su caso.
    Las actuaciones requeridas a notarios, conservadores de bienes raíces y archiveros, por parte de condominios de viviendas sociales, deberán efectuarse en un plazo máximo de treinta días a contar de la respectiva solicitud.
     
    Artículo 74.- Serán aplicables a los condominios de viviendas de interés público los artículos 23 y 24 de esta ley, referidos a las subadministraciones.
     
    Artículo 75.- Tratándose de condominios de viviendas sociales, la formación del fondo común de reserva se regirá por las normas especiales que establezca el reglamento de la ley, con el objeto de resguardar que tales condominios cuenten, en forma permanente, con un monto de recursos disponibles para asumir gastos comunes urgentes, extraordinarios e imprevistos, pero sin imponerles un gravamen excesivo. En tal sentido, el reglamento de la ley podrá eximir a estos condominios, bajo determinados supuestos, del porcentaje mínimo de recargo señalado en el artículo 39 y/o de la periodicidad mensual ahí establecida.
    Para determinar dichas normas especiales, el reglamento de la ley podrá considerar factores como la inexistencia de personal contratado y/o la antigüedad del condominio, así como la cantidad y tipo de bienes comunes que, a futuro, pudieren generar la necesidad de solventar gastos comunes extraordinarios, urgentes o imprevistos.
    Lo señalado en este artículo es sin perjuicio de la posibilidad de que tales condominios puedan postular a los recursos públicos referidos en el artículo 68, cuando requieran solventar el pago de este tipo de gastos.
     
    Artículo 76.- En los condominios a que se refiere este Título, la municipalidad correspondiente estará obligada a actuar como instancia de mediación extrajudicial, conforme a lo establecido en el artículo 47, pudiendo ejercer siempre labores de amigable componedor, para lo cual podrá proponer bases de arreglo e instar a éstos. Asimismo, deberá proporcionar su asesoría para la organización de los copropietarios. Para estos efectos, la municipalidad podrá celebrar convenios con instituciones públicas o privadas.
   
    Artículo 77.- Las municipalidades deberán desarrollar programas educativos sobre los derechos y deberes de los habitantes de condominios de viviendas sociales, promover, asesorar, prestar apoyo a su organización y progreso y, sin perjuicio de lo dispuesto en el artículo 68, podrán adoptar todas las medidas necesarias para permitir la adecuación de las comunidades de copropietarios de viviendas sociales a las normas de la presente ley, estando facultadas al efecto para prestar asesoría legal, técnica y contable y para destinar recursos con el objeto de afrontar los gastos que demanden estas gestiones, tales como elaboración de planos u otros de similar naturaleza.
   
    Artículo 78.- Las municipalidades, a través de sus unidades o mediante convenios celebrados con otras instituciones, públicas o privadas, realizarán los trámites que sean necesarios para apoyar a los condominios de viviendas sociales en el buen funcionamiento de los mismos, incluida la asesoría necesaria para el cobro judicial de los gastos comunes adeudados y para que conjuntos de viviendas construidos antes de la entrada en vigencia de esta ley puedan acogerse a sus disposiciones.
     
    Párrafo 2°
    Densificación predial

     
    Artículo 79.- En los predios donde no existan viviendas y en los que cuentan u originalmente contaron con una vivienda económica, social o construida con subsidio del Estado, así como en aquellos provenientes de Operaciones Sitio, podrá permitirse en un mismo predio, por una sola vez, la construcción de hasta cuatro viviendas nuevas en caso de que no existieren en él edificaciones, o hasta tres viviendas adicionales si existiere en dicho predio una vivienda, las que deberán ser destinadas a su adquisición o arriendo por parte de beneficiarios de los programas habitacionales del Estado y constituir un condominio acogido a la presente ley, bajo la denominación de "condominio de densificación predial".
    Con todo, tratándose de programas de densificación impulsados por el Ministerio de Vivienda y Urbanismo, la cantidad total de viviendas podrá alcanzar hasta doce unidades, incluyéndose las existentes y las nuevas que se construyan, en la medida que la densidad neta del predio no supere las doscientas veinte viviendas por hectárea.
    En los llamados que se efectúen para la construcción de estos condominios, el Ministro de Vivienda y Urbanismo podrá eximir de requisitos técnicos y urbanísticos que establecen los programas habitacionales, para la aprobación de los proyectos respectivos.
    Lo anterior, también será aplicable en zonas decretadas como "zonas afectadas por catástrofe".
     
    Artículo 80.- Los condominios de densificación predial no requerirán comité de administración ni administrador, y aquellos de hasta cuatro viviendas, además, no necesitarán contar con fondo de reserva, estacionamientos, seguros ni planes de emergencia. Las normas urbanísticas aplicables serán solo las establecidas en el Reglamento Especial de Viviendas Económicas.
    A falta de reglamento de copropiedad, los condominios de densificación predial se regirán por el que se establezca en el reglamento de esta ley como reglamento tipo, sin necesidad de que éste se encuentre inscrito en el conservador de bienes raíces respectivo.
   
    Artículo 81.- Todo lo concerniente a la administración de este tipo de condominios corresponderá a los copropietarios, que deberán actuar concertadamente en todas aquellas materias que puedan afectar a más de una unidad. Tratándose de obras relacionadas con las condiciones de habitabilidad o de seguridad, el director de obras municipales podrá autorizar su ejecución a solicitud de uno solo de los copropietarios afectados.

    TÍTULO XIII
    DEL REGISTRO NACIONAL DE ADMINISTRADORES DE CONDOMINIOS


    Artículo 82.- Créase el Registro Nacional de Administradores de Condominios, en adelante Registro Nacional, de carácter público, obligatorio y gratuito, que estará a cargo del Ministerio de Vivienda y Urbanismo, en el cual deberán inscribirse todas las personas naturales o jurídicas que ejerzan la actividad de administradores de condominios, siempre que cumplan con las disposiciones de esta ley y su reglamento.
     
    Artículo 83.- La inscripción en el Registro Nacional será requisito previo para ejercer la actividad de administrador o subadministrador de condominios, sea a título gratuito u oneroso.
    Las personas naturales o jurídicas que se inscriban en el Registro Nacional serán responsables de que la prestación de servicios cumpla con todas las leyes, reglamentos, resoluciones y normas que les sean aplicables.
    En el Registro Nacional se consignarán todos los antecedentes que el Ministerio de Vivienda y Urbanismo establezca en el reglamento para supervigilar el cumplimiento normativo por parte de quienes ejerzan la referida actividad, correspondiéndole a las secretarías regionales ministeriales conocer y resolver las reclamaciones que se interpongan en contra de los administradores o subadministradores de condominios.
     
    Artículo 84.- No podrán inscribirse en el Registro Nacional los administradores y subadministradores que hubieren sido condenados por alguno de los delitos contemplados en los Títulos VIII y IX del Libro Segundo del Código Penal.
    Para el caso de los administradores a título oneroso, deberán cumplir además con los siguientes requisitos de inscripción:
     
    1. Acreditar licencia de enseñanza media.
    2. Haber aprobado un curso de capacitación en materias de administración de condominios, que haya sido impartido por una institución de educación superior del Estado o reconocida por éste, u organismo técnico de capacitación acreditado por el Servicio Nacional de Capacitación y Empleo, o bien, contar con certificación de competencia laboral otorgada por un centro acreditado por la Comisión del Sistema Nacional de Certificación de Competencias Laborales, Chile Valora, conforme a lo dispuesto en la ley N° 20.267. El reglamento referido en el artículo 86 precisará, a partir de las funciones que esta ley asigna a los administradores, las competencias mínimas necesarias para ejercer dicho cargo, así como la forma en que debe acreditarse el cumplimiento de este requisito, ya sea por los nuevos administradores o por quienes desempeñan tal función desde antes de la publicación de esta ley. Dicha reglamentación deberá incluir, al menos, el conocimiento de las normas y procedimientos básicos relacionados con: i) el plan de emergencia, la contratación de seguros y demás preceptos sobre seguridad del condominio y mantención de sus instalaciones; ii) la normativa laboral y previsional aplicable al personal del condominio; iii) la rendición de cuentas y el cobro de gastos comunes y demás obligaciones económicas; y iv) las fórmulas de resolución de conflictos.
    En relación a los requisitos establecidos en este artículo, si el administrador fuere una persona jurídica, al menos uno de los socios o el representante legal deberá cumplir con tales requisitos. Sin perjuicio de lo expuesto, la persona natural que ejerza el rol de administrador o subadministrador deberá estar inscrita en el Registro Nacional.
     
    Artículo 85.- La inscripción en el Registro Nacional se realizará por el interesado en la plataforma digital que el Ministerio de Vivienda y Urbanismo disponga al efecto, el que deberá mantener el señalado registro actualizado, identificando los administradores y los condominios en que prestan servicios, las sanciones impuestas, así como las incorporaciones y retiros del Registro Nacional.
     
    Artículo 86.- Un reglamento, expedido mediante decreto supremo del Ministerio de Vivienda y Urbanismo, establecerá las normas necesarias para el procedimiento de inscripción, actualización y funcionamiento del Registro Nacional y las demás condiciones en que han de operar los administradores y subadministradores inscritos, diferenciando entre aquellos que realizan esta labor a título oneroso o gratuito.

    TÍTULO XIV
    DE LAS INFRACCIONES, RECLAMACIONES, SANCIONES Y PROCEDIMIENTO ANTE INCUMPLIMIENTO DE ADMINISTRADORES

     
    Párrafo 1°
    De las infracciones

     
    Artículo 87.- Las infracciones a las normas que regulan la administración de condominios, especialmente las contempladas en el artículo 20 de la presente ley, referido a las funciones de los administradores y subadministradores, serán conocidas por las respectivas secretarías regionales ministeriales de vivienda y urbanismo.
    Sin perjuicio de la responsabilidad civil o penal que pudiere corresponderles a los administradores, las infracciones señaladas en el inciso primero se calificarán en gravísimas, graves, menos graves o leves, conforme al siguiente detalle:
     
    1) Son infracciones gravísimas:
     
    a) Actuar como administrador encontrándose afectado por alguna causal de inhabilidad o habiendo perdido alguno de los requisitos habilitantes para la inscripción en el Registro Nacional.
    b) Proporcionar información falsa relativa al cumplimiento de los requisitos de inscripción.
    c) Aportar datos o antecedentes falsos respecto de la administración del condominio, induciendo a error o impidiendo la correcta evaluación de la gestión por parte del comité de administración o de los copropietarios.
    d) Ser condenado por sentencia ejecutoriada debido a responsabilidades civiles o penales derivadas de la administración de condominios.
    e) Reincidir en la comisión de alguna infracción grave dentro de un período de tres años.
    f) No dar cumplimiento a las obligaciones contempladas en los numerales 1) y 2) del artículo 20 de la presente ley y que dicho incumplimiento hubiese causado daño a la seguridad de las personas, lesiones o muerte.
    g) Suspender o requerir la suspensión del servicio eléctrico, de telecomunicaciones o de calefacción de un propietario, durante la vigencia del decreto de declaración de estado de excepción constitucional de catástrofe.
     
    2) Son infracciones graves:
     
    a) No dar cumplimiento a las obligaciones contempladas en los numerales 1) y 2) del artículo 20 de la presente ley, sin los efectos referidos en la letra f) del numeral precedente.
    b) No dar cumplimiento a la obligación contemplada en el numeral 10) del artículo 20 de la presente ley.
    c) Reincidir en la comisión de alguna infracción menos grave dentro de un período de dos años.
     
    3) Son infracciones menos graves:
     
    a) No dar cumplimiento a las obligaciones contempladas en los numerales 3), 5), 7), 8) y 11) del artículo 20 de la presente ley.
    b) Reincidir en la comisión de alguna infracción leve dentro de un período de dos años.
     
    4) Son infracciones leves:
     
    a) No dar cumplimiento a las obligaciones contempladas en los numerales 4), 6), 9), 12), 13) y 14) del artículo 20 de la presente ley.
    b) Todas las demás transgresiones de la presente ley que no estén indicadas en la enumeración de los numerales anteriores.
     
    Párrafo 2°
    De las sanciones

     
    Artículo 88.- La sanción que corresponda aplicar a cada infracción se determinará, según su gravedad, dentro de los siguientes rangos:
     
    a) Las infracciones gravísimas serán sancionadas con la eliminación del Registro Nacional y/o multa a beneficio fiscal de cinco a diez unidades tributarias mensuales.
    b) Las infracciones graves serán sancionadas con la suspensión por uno a tres años del Registro Nacional y/o multa a beneficio fiscal de cinco a diez unidades tributarias mensuales.
    c) Las infracciones menos graves serán sancionadas con una amonestación por escrito y/o multa a beneficio fiscal de una a cuatro unidades tributarias mensuales.
    d) Las infracciones leves serán sancionadas con una amonestación por escrito.
     
    Artículo 89.- Para la determinación de la sanción a aplicar, el secretario regional ministerial deberá considerar los efectos producidos por la infracción, tales como poner en riesgo la vida o la seguridad de los ocupantes del condominio, afectar los derechos de los copropietarios, incumplir obligaciones que deriven en la necesidad de efectuar gastos extraordinarios, el perjuicio económico provocado a la comunidad producto de la infracción, entre otros.
     
    Párrafo 3°
    De la reclamación y su procedimiento

     
    Artículo 90.- El comité de administración o el porcentaje mínimo de copropietarios o arrendatarios que defina el reglamento de esta ley conforme al número total de unidades del condominio, podrán interponer una reclamación ante la secretaría regional ministerial de vivienda y urbanismo de la región donde se encuentre el condominio, cuando el administrador o subadministrador incumpla alguna de las obligaciones que le impone la presente ley y su reglamento. En el escrito que se presente deberán especificarse las acciones u omisiones en que se funda la reclamación y acompañar copia de los antecedentes que la respaldan.
     
    Artículo 91.- Recibida la reclamación, el secretario regional ministerial de vivienda y urbanismo respectivo podrá, en atención al contenido de la misma, desestimarla por improcedente, solicitar mayores antecedentes u ordenar el inicio de un procedimiento sancionatorio.
    El procedimiento sancionatorio se iniciará mediante una resolución de la secretaría regional ministerial de vivienda y urbanismo, en la que deberán constar los cargos formulados en contra del presunto infractor, la que se le notificará por correo electrónico o carta certificada enviada al domicilio registrado en la plataforma del Registro Nacional de Administradores de Condominios, adjuntando los antecedentes en que se funda la reclamación.
    La formulación de cargos deberá señalar una descripción de los hechos que se estiman constitutivos de infracción, la norma eventualmente infringida y la disposición que establece la sanción asignada a la infracción.
    El presunto infractor tendrá un plazo de diez días hábiles para presentar sus descargos, contado desde la notificación.
    Con todo, si el secretario regional ministerial de vivienda y urbanismo toma conocimiento de que mediante sentencia firme y ejecutoriada se ha determinado la responsabilidad civil o penal de un administrador, por no dar cumplimiento a las obligaciones contempladas en esta ley en un condominio ubicado en su respectiva región, dicha autoridad podrá iniciar de oficio un procedimiento sancionatorio, de conformidad con lo dispuesto en los incisos anteriores.
     
    Artículo 92.- Recibidos los descargos o transcurrido el plazo establecido para ello, la secretaría regional ministerial de vivienda y urbanismo examinará el mérito de los antecedentes y, en caso de ser necesario, ordenará la realización de diligencias destinadas a determinar si hubo incumplimiento por parte del administrador o subadministrador de sus obligaciones y los efectos de dicho incumplimiento, con el objeto de determinar la sanción aplicable.
     
    Artículo 93.- La resolución que resuelva la reclamación deberá dictarse dentro del plazo de treinta días hábiles siguientes a aquel en que se haya evacuado la última diligencia ordenada.
     
    Artículo 94.- Frente a la resolución del secretario regional ministerial de vivienda y urbanismo que aplique una sanción, procederá el recurso de reposición que se deberá interponer dentro del plazo de cinco días hábiles ante la entidad que dictó el acto que se impugna; en subsidio, podrá interponerse el recurso jerárquico para ante el Subsecretario de Vivienda y Urbanismo.
    Rechazado total o parcialmente el recurso de reposición, se elevará el expediente al Subsecretario de Vivienda y Urbanismo, si junto con éste se hubiere interpuesto subsidiariamente recurso jerárquico. Cuando no se deduzca reposición, el recurso jerárquico se interpondrá ante el Subsecretario de Vivienda y Urbanismo, dentro de los cinco días siguientes a su notificación.
     
    Artículo 95.- Interpuesta una reclamación ante la secretaría regional ministerial de vivienda y urbanismo respectiva, no podrá el mismo reclamante deducir igual pretensión en contra del administrador o subadministrador ante el juzgado de policía local o ante la respectiva municipalidad.
     
    Artículo 96.- Las reclamaciones en contra del administrador o subadministrador prescribirán en el plazo de dos años contado desde la acción u omisión reclamada.
 
    TÍTULO FINAL
    DISPOSICIONES GENERALES
   
    Artículo 97.- Corresponderá al Ministerio de Vivienda y Urbanismo impartir las instrucciones para la aplicación de las normas de esta ley y su reglamento, mediante circulares que se mantendrán a disposición de cualquier interesado en su sitio electrónico institucional.
    Dicha función la ejercerá a través de la Secretaría Ejecutiva de Condominios, la que dependerá directamente del Ministro de la cartera y que también será la encargada de:
     
    a) Proponer e implementar la política habitacional y los programas presupuestarios relacionados con la mantención y mejoramiento de condominios de viviendas sociales o condominios de viviendas de interés público que evidencien grave deterioro.
    b) Mantener actualizados los registros referidos en los artículos 82 y 98 de esta ley.
    c) Ejercer labores de supervisión, coordinación y asesoría a otras divisiones o unidades ministeriales en la aplicación del régimen de copropiedad inmobiliaria, siempre que no se trate de materias radicadas en otros órganos de la Administración del Estado.
     
    Asimismo, las secretarías regionales ministeriales de vivienda y urbanismo deberán supervigilar las normas legales, reglamentarias, administrativas y técnicas sobre copropiedad inmobiliaria, pudiendo resolver las reclamaciones interpuestas en contra de las resoluciones dictadas por las direcciones de obras municipales, relacionadas con el certificado que declare un condominio acogido al régimen de copropiedad inmobiliaria, la modificación de tal certificado, el cambio de destino de unidades o la ejecución de obras en un condominio. Tales reclamaciones se regirán por el mismo procedimiento establecido en la Ley General de Urbanismo y Construcciones para las reclamaciones interpuestas en contra de las resoluciones dictadas por las direcciones de obras municipales.
     
    Artículo 98.- Los condominios que incluyan unidades con destino habitacional deberán incorporarse en un registro, a cargo de la Secretaría Ejecutiva de Condominios, en el que se consignarán, al menos:
     
    a) La identificación y ubicación del condominio, precisando el número total de viviendas y especificando, cuando corresponda, si se trata de un condominio de viviendas sociales o de viviendas de interés público, en los términos referidos en los artículos 65 y siguientes.
    b) La especificación de la carpeta física o expediente digital del condominio, referida en el artículo 48.
    c) La especificación de la escritura pública en que consta el reglamento de copropiedad y de su inscripción en el registro de hipotecas y gravámenes del conservador de bienes raíces respectivo.
     
    La incorporación en el Registro de Condominios Habitacionales será efectuada por la Secretaría Ejecutiva de Condominios, a partir de la información que le remita la dirección de obras municipales, en cumplimiento de lo dispuesto en el inciso tercero del artículo 48.
     
    Artículo 99.- La presente ley se aplicará a las comunidades de copropietarios acogidas a la Ley de Propiedad Horizontal con anterioridad a la entrada en vigencia de la ley N° 19.537, salvo que, conforme a lo establecido en el artículo 49 de esta última ley, sus copropietarios hayan acordado continuar aplicando las normas de sus reglamentos de copropiedad en relación al cambio de destino de las unidades del condominio y a la proporción o porcentaje que a cada copropietario corresponde sobre los bienes comunes y en el pago de los gastos comunes. Asimismo, se mantendrán vigentes los derechos de uso y goce exclusivo sobre bienes comunes que hayan sido legalmente constituidos.
    En los casos en que esta ley exija que una determinada facultad o derecho esté establecido en el reglamento de copropiedad, se presumirá tal autorización respecto de los reglamentos de copropiedad formulados con anterioridad a la vigencia de aquélla, salvo acuerdo en contrario de una asamblea extraordinaria de copropietarios.
    Las comunidades a que se refiere este artículo podrán establecer siempre subadministraciones en los términos previstos en los artículos 23 y 24, previo acuerdo adoptado conforme a lo prescrito en el artículo 15. Para estos efectos, la porción correspondiente a cada subadministración deberá constar en un plano complementario de aquel aprobado por la dirección de obras municipales al acogerse el edificio o conjunto de viviendas a la Ley de Propiedad Horizontal.
   
    Artículo 100.- Derógase la ley N° 19.537, sobre copropiedad inmobiliaria, sin perjuicio de lo dispuesto en el artículo 5° transitorio de la presente ley.
    Las comunidades de copropietarios que se hubieren acogido a la ley N° 19.537 se regirán por la presente ley desde su publicación, debiendo ajustarse los reglamentos de copropiedad a sus disposiciones en el plazo de un año. Los acuerdos adoptados por las asambleas de copropietarios con anterioridad a la entrada en vigencia de esta ley no quedarán sin efecto.



NOTA
      El artículo 1º de la ley 21.508, publicada el 10.11.2022, interpreta el presente artículo en el sentido de declarar que tratándose de las materias reguladas por esta ley cuya aplicación requiera, expresa o tácitamente, la dictación de reglamentos u otras normas complementarias, conservarán su eficacia las disposiciones de la ley 19.537 hasta la publicación de dichos textos.
   
    Artículo 101.- Las referencias que se efectúan en la legislación vigente a las disposiciones legales que se derogan por el artículo precedente se entenderán realizadas a las correspondientes de la presente ley, y aquellas efectuadas a las "juntas de vigilancia" a los "comités de administración".
     
    Artículo 102.- Las disposiciones de la presente ley serán aplicables a todos los condominios que se hubieren acogido al régimen de copropiedad inmobiliaria, conforme a lo establecido en el artículo 48, aun cuando sus unidades no fueren transferidas a terceros.
    Si todas las unidades permanecen bajo el dominio de la persona natural o jurídica propietaria del condominio o si el número de copropietarios es inferior a tres, las funciones encomendadas por esta ley al comité de administración y a su presidente deberán ser asumidas por el propietario del condominio o por el copropietario que tenga la mayor proporción de derechos en éste. En ambos casos, no será necesario que las materias referidas en el artículo 15 sean acordadas por la asamblea de copropietarios, pero las decisiones que dicho propietario adopte al respecto deberán constar en un libro de actas y, si la naturaleza de la decisión adoptada lo requiere, el acta deberá reducirse a escritura pública.
    Lo señalado en el inciso precedente no obsta a la designación de un administrador del condominio, con las mismas funciones y responsabilidades establecidas en esta ley.
    Asimismo, cuando el porcentaje de derechos enajenados en el condominio sea inferior al 33%, se deberá convocar anualmente a una asamblea de residentes, con el objeto de informar sobre el funcionamiento y administración del condominio, reportar las actualizaciones al plan de emergencia, programar los simulacros de evacuación y/o acciones de capacitación o prevención de riesgos y tratar cualquier otro asunto relacionado con los intereses de los residentes.
 
 
    DISPOSICIONES TRANSITORIAS


     
    Artículo 1°.- Deberán dotarse de un reglamento de copropiedad aquellos condominios que hubiesen sido creados antes de la entrada en vigencia de esta ley, o que, habiendo surgido con posterioridad, se originen en una comunidad que no signifique copropiedad en los términos de la ley. Si éste no hubiese sido dictado al cabo de un año de la publicación de la presente ley, se entenderá aplicable al condominio el reglamento tipo que deberá sancionar el reglamento de esta ley.
     
    Artículo 2°.- Desde la publicación de la ley y hasta la entrada en vigencia del reglamento del Registro Nacional de Administradores de Condominios, podrán continuar desempeñándose como administradores aquellas personas que se encontraban ejerciendo tal función y que, además, acrediten una antigüedad mínima de tres meses en el cargo.
    Una vez que entre en vigencia el Registro Nacional de Administradores de Condominios, podrán inscribirse en él y desempeñarse como administradores todas aquellas personas que acrediten el cumplimiento de los requisitos establecidos en el artículo 84.
    Los administradores señalados en el inciso primero de este artículo tendrán un plazo de dieciocho meses para acreditar la aprobación de un curso de capacitación u obtener la certificación de competencia laboral referidos en el numeral 2 del inciso segundo del artículo 84, sin perjuicio del cumplimiento de los demás requisitos establecidos en dicho artículo. Si transcurrido ese plazo no se han inscrito en el Registro Nacional, se entenderán inhabilitados para continuar desempeñando el cargo de administrador.
     
    Artículo 3°.- Los condominios de viviendas sociales que no se encuentren organizados podrán postular a los programas financiados con recursos fiscales a que se refiere el artículo 68 de la presente ley. Para lo anterior, bastará la firma de los copropietarios que representen, al menos, la mitad de los derechos en el condominio. Contará para esto también la firma del copropietario donde autoriza al arrendatario u ocupante, por medio de un poder visado por alguno de los ministros de fe mencionados en el artículo 73, para que lo represente en esta instancia y con la finalidad antes señalada.
     
    Artículo 4°.- Los planes de emergencia, incluidos en éstos los planes de evacuación, señalados en el artículo 40 de esta ley y en el artículo 144 de la Ley General de Urbanismo y Construcciones, deberán ser elaborados y actualizados conforme a la Norma Técnica que para dicho efecto oficialice el Ministerio de Vivienda y Urbanismo.
     
    Artículo 5°.- La obligación contemplada en el artículo 43, relacionada con la contratación de un seguro colectivo contra incendio, será exigible una vez transcurridos seis meses desde la publicación del reglamento de esta ley. En el tiempo intermedio, seguirá vigente lo dispuesto en el artículo 36 de la ley N° 19.537.
    Los condominios que se hubieren acogido al régimen de copropiedad inmobiliaria con anterioridad a la fecha de entrada en vigencia del artículo 43, tendrán el plazo de dos años, contado desde la referida fecha, para efectuar una revisión de las pólizas que tuvieren vigentes y adaptarse a lo establecido en dicho artículo, conforme a lo que disponga el referido reglamento y la normativa que dicte la Comisión para el Mercado Financiero.
     
    Artículo 6°.- El reglamento de la ley y el del Registro Nacional de Administradores de Condominios deberán dictarse dentro del plazo de doce meses, contado desde la publicación de la presente ley y deberán ser sometidos a consulta pública, por un plazo no inferior a treinta días.
     
    Artículo 7°.- El mayor gasto fiscal que represente la aplicación de esta ley, durante el primer año presupuestario de vigencia, se financiará con cargo al presupuesto del Ministerio de Vivienda y Urbanismo y, en lo que faltare, con cargo a los recursos de la partida presupuestaria Tesoro Público de la Ley de Presupuestos del Sector Público.
     
    Artículo 8°.- La exigencia de estacionamientos para nuevos condominios de viviendas de interés público, establecida en el inciso primero del artículo 60, será aplicable para los proyectos que soliciten permiso de edificación desde el 1 de enero de 2025, exceptuándose aquellos que contaren con anteproyecto vigente, aprobado con anterioridad.


    Artículo 9°.- Derogado.


     
    Artículo 10.- Los condominios que incluyan unidades con destino habitacional existentes a la fecha de publicación de esta ley, deberán incorporarse en el registro señalado en el artículo 98 en el plazo de dos años contados desde la referida publicación.".

    Artículo segundo.- Modifícase el decreto con fuerza de ley N° 458, de 1975, del Ministerio de Vivienda y Urbanismo, que aprueba la Ley General de Urbanismo y Construcciones, de la siguiente forma:
     
    1) Reemplázase, en el inciso séptimo del artículo 18, la frase "y del revisor del proyecto de cálculo estructural, cuando corresponda," por ", del revisor del proyecto de cálculo estructural y del profesional a cargo de la elaboración del plan de emergencia, cuando corresponda,".
    2) Agrégase, en la letra g) del artículo 105, a continuación de la expresión "salubridad,", lo siguiente: "seguridad,".
    3) Reemplázase, en el inciso cuarto del artículo 142, la frase "plan de evacuación" por "plan de emergencia".
    4) Sustitúyense, en el inciso tercero del artículo 144, las expresiones "plan de evacuación" y "señalética" por "plan de emergencia" y "señalización", respectivamente.".


      Habiéndose cumplido con lo establecido en el Nº 1 del artículo 93 de la Constitución Política de la República y por cuanto he tenido a bien aprobarlo y sancionarlo; por tanto, promúlguese y llévese a efecto como Ley de la República.   
    Santiago, 1 de abril de 2022.- GABRIEL BORIC FONT, Presidente de la República.- Carlos Montes Cisternas, Ministro de Vivienda y Urbanismo.- Izkia Siches Pastén, Ministra del Interior y Seguridad Pública.- Marcela Ríos Tobar, Ministra de Justicia y Derechos Humanos.
    Lo que transcribo para su conocimiento.- Tatiana Valeska Rojas Leiva, Subsecretaria de Vivienda y Urbanismo.
     
    Tribunal Constitucional
     
    Proyecto de ley que establece una nueva ley de copropiedad inmobiliaria, correspondiente al Boletín N° 11.540-14
     
    La Secretaria del Tribunal Constitucional, quien suscribe, certifica que el Honorable Senado de la República envió el proyecto de ley enunciado en el rubro, aprobado por el Congreso Nacional, a fin de que este Tribunal ejerciera el control de constitucionalidad respecto de los artículos 6, inciso final; 10, inciso cuarto; 44; 46; 47; 64, incisos cuarto y sexto; 68; 76; y 77, contenidos en el Artículo Primero del Proyecto de Ley; y por sentencia de 18 de marzo de 2022, en los autos Rol 12874-22-CPR.
     
    Se declara:
     
    1°. Que los artículos 6, inciso final; 10, inciso cuarto; 44; 46 en la expresión "Sin perjuicio de lo dispuesto en el artículo 44, las contiendas a que se refiere dicho precepto podrán someterse a la resolución de un Juez Árbitro, en cualquiera de las calidades a que se refiere el artículo 223 del Código Orgánico de Tribunales"; 47; 64, incisos cuarto y sexto; 68; 76; y 77, contenidos en el artículo primero del proyecto de ley que establece una nueva Ley de Copropiedad Inmobiliaria, correspondiente al Boletín N° 11.540-14, son conformes con la Constitución Política.
    2°. Que no se emite pronunciamiento, en examen preventivo de constitucionalidad, de las restantes disposiciones del proyecto de ley, por no versar sobre materias reguladas en Ley Orgánica Constitucional.
     
    Santiago, 18 de marzo de 2022.- María Angélica Barriga Meza, Secretaria.