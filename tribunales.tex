CODIGO ORGANICO DE TRIBUNALES

    LEY N° 7421

    Santiago, 15 de Junio de 1943.

    HOY SE DECRETO LO QUE SIGUE:

    En uso de la facultad que confiere al Presidente de la República el artículo 32 de la Ley N° 7200, de 18 de Julio de 1942, y teniendo presente el oficio del Decano de la Facultad de Ciencias Jurídicas y Sociales de la Universidad de Chile, de fecha 14 del mes en curso,

    DECRETO:

    1° Téngase por texto definitivo del Código Orgánico de Tribunales el adjunto al oficio referido; y

    2° Dos ejemplares de dicho texto, autorizados por el Presidente de la República y signados con el sello del Ministerio de Justicia, se depositarán en las Secretarías de ambas Cámaras y otro, en el Archivo de dicho Ministerio.

    Dicho texto se tendrá por el auténtico del Código Orgánico de Tribunales, y a él deberán conformarse las demás ediciones y publicaciones que del expresado Código se hicieren.

    Y por cuanto he tenido a bien aprobarlo y sancionarlo, promúlguese y llévese a efecto como Ley de la República.

    J. A. RIOS M. - Oscar Gajardo V.

    Este decreto fué publicado en el Diario Oficial del 9 de Julio de 1943.

    Santiago, 14 de Junio de 1943.

SEÑOR MINISTRO:

    Por Decreto de 19 de Agosto de 1942, el Supremo Gobierno comisionó a la Universidad de Chile para que, por intermedio de la Facultad de Ciencias Jurídicas y Sociales y sin derecho a remuneración, procediera a refundir en un solo texto la Ley de Organización y Atribuciones de los Tribunales, de 15 de Octubre de 1875, y todas las leyes que la han modificado o complementado, en los términos a que se refiere el Artículo 32 de la Ley N° 7200, de 21 de Julio de 1942. Una vez aprobado ese texto por el Ministerio de Justicia, la Universidad debería editarlo en forma esmerada por su cuenta y sin cargo alguno para el Fisco, con la obligación de entregar a ese Ministerio, libres de todo costo, 30 ejemplares de la edición.

    En cumplimiento de este Decreto, y debidamente facultado, al efecto, por el H. Consejo Universitario, el suscrito designó, con fecha 4 de Septiembre de 1942, una Comisión formada por los profesores señores Fernando Alessandri R., Humberto Trucco, Darío Benavente, Manuel Urrutia Salas, Alberto Echavarría, Jaime Galté, Luis Varas Gómez y del Abogado don Víctor García Garzena para que realizaran el referido trabajo. Actuaría como Secretario de esta Comisión, el Ayudante del Seminario de Derecho Privado de esta Facultad, don Patricio Aylwin Azócar.

    La Comisión nombrada, después de celebrar numerosas sesiones y de reunirse, en ocasiones, hasta cuatro veces por semana, acaba de dar término a su cometido.

    La Comisión tomó como base de estudio un anteproyecto presentado por el Profesor don Fernando Alessandri, salvo en las partes relativas a los acuerdos de las Cortes de Apelaciones y a los árbitros, que fueron preparadas por don Víctor García Garzena y por don Patricio Aylwin, respectivamente.

    En el nuevo texto se ha conservado, en general, la estructura de la actual ley de tribunales, y para dar un orden lógico a sus preceptos y a las numerosas disposiciones que la han modificado y complementado, se han agrupado por materias. En esta forma el proyecto gana mucho en claridad y se facilita su consulta y aplicación.

    El Profesor don Fernando Alessandri me ha pedido hacer constar de que fué un gran auxiliar para su trabajo la obra de que son autores los señores Luis Varas Gómez y Víctor García Garzena, intitulada "La Ley de Organización y Atribuciones de los Tribunales, de 15 de Octubre de 1875, las disposiciones que la modifican y complementan".

    Es de justicia, asimismo, dejar testimonio de la labor del Secretario de la Comisión don Patricio Aylwin Azócar, que actuó con abnegación digna del mayor encomio.

    El suscrito espera que el nuevo texto del Código Orgánico de Tribunales que tengo el honor de remitir a US. ha de merecer la aprobación de ese Ministerio y aprovecha la oportunidad para agradecer al Supremo Gobierno la demostración de confianza que ha dispensado a esta Facultad al confiarle tan delicado trabajo.

    Saluda atentamente al señor Ministro.

    ARTURO ALESSANDRI R.
    DECANO DE LA FACULTAD DE CIENCIAS JURIDICAS
    Y SOCIALES DE LA UNIVERSIDAD DE CHILE

    Código Orgánico de Tribunales


    TITULO I

    Del Poder Judicial y de la Administración de Justicia en general


    Artículo 1° La facultad de conocer de las causas civiles y criminales, de juzgarlas y de hacer ejecutar lo juzgado pertenece exclusivamente a los tribunales que establece la ley.


    Art. 2° También corresponde a los tribunales intervenir en todos aquellos actos no contenciosos en que una ley expresa requiera su intervención.


    Art. 3° Los tribunales tienen, además, las facultades conservadoras, disciplinarias y económicas que a cada uno de ellos se asignan en los respectivos títulos de este Código.


    Art. 4° Es prohibido al Poder Judicial mezclarse en las atribuciones de otros poderes públicos y en general ejercer otras funciones que las determinadas en los artículos precedentes.

    Art. 5° A los tribunales mencionados en este artículo corresponderá el conocimiento de todos los asuntos judiciales que se promuevan dentro del territorio de la República, cualquiera que sea su naturaleza o la calidad de las personas que en ellos intervengan, sin perjuicio de las excepciones que establezcan la Constitución y las leyes.
    Integran el Poder Judicial, como tribunales ordinarios de justicia, la Corte Suprema, las Cortes de Apelaciones, los Presidentes y Ministros de Corte, los tribunales de juicio oral en lo penal, los juzgados de letras y los juzgados de garantía.
    Forman parte del Poder Judicial, como tribunales especiales, los juzgados de familia, los Juzgados de Letras del Trabajo, los Juzgados de Cobranza Laboral y Previsional y los Tribunales Militares en tiempo de paz, los cuales se regirán en su organización y atribuciones por las disposiciones orgánicas constitucionales contenidas en la ley Nº 19.968, en el Código del Trabajo, y en el Código de Justicia Militar y sus leyes complementarias, respectivamente, rigiendo para ellos las disposiciones de este Código sólo cuando los cuerpos legales citados se remitan en forma expresa a él.
    Los demás tribunales especiales se regirán por las leyes que los establecen y reglamentan, sin perjuicio de quedar sujetos a las disposiciones generales de este Código.
    Los jueces árbitros se regirán por el Título IX de este Código.



    Art. 6° Quedan sometidos a la jurisdicción chilena los crímenes y simples delitos perpetrados fuera del territorio de la República que a continuación se indican:

    1°) Los cometidos por un agente diplomático o consular de la República, en el ejercicio de sus funciones;
    2°) La malversación de caudales públicos, fraudes y exacciones ilegales, la infidelidad en la custodia de documentos, la violación de secretos, el cohecho, cometidos por funcionarios públicos chilenos o por extranjeros al servicio de la República y el cohecho a funcionarios públicos extranjeros, cuando sea cometido por un chileno o por una persona que tenga residencia habitual en Chile;
    3°) Los que van contra la soberanía o contra la seguridad exterior del Estado, perpetrados ya sea por chilenos naturales, ya por naturalizados, y los contemplados en el Párrafo 14 del Título VI del Libro II del Código Penal, cuando ellos pusieren en peligro la salud de habitantes de la República;
    4°) Los cometidos, por chilenos o extranjeros, a bordo de un buque chileno en alta mar, o a bordo de un buque chileno de guerra surto en aguas de otra potencia;
    5°) La falsificación del sello del Estado, de moneda nacional, de documentos de crédito del Estado, de las Municipalidades o de establecimientos públicos, cometida por chilenos, o por extranjeros que fueren habidos en el territorio de la República;
    6°) Los cometidos por chilenos contra chilenos si el culpable regresa a Chile sin haber sido juzgado por la autoridad del país en que delinquió;
    7°) La piratería;
    8°) Los comprendidos en los tratados celebrados con otras potencias;
    9°) Los sancionados por la ley 6.026 y las que la han modificado, cometidos por chilenos o por extranjeros al servicio de la República;
    10°) Los sancionados en los artículos 367, 367 quáter inciso segundo y 367 septies del Código Penal, cuando pusieren en peligro o lesionaren la indemnidad o la libertad sexual de algún chileno o fueren cometidos por un chileno o por una persona que tuviere residencia habitual en Chile; y el contemplado en el artículo 367 quáter, inciso primero, del mismo cuerpo legal, cuando el material pornográfico objeto de la conducta hubiere sido elaborado utilizando chilenos menores de dieciocho años;
    11°) Los sancionados en el artículo 62 del decreto con fuerza de ley Nº 1, del Ministerio de Economía, Fomento y Reconstrucción, de 2004, que fija el texto refundido, coordinado y sistematizado del decreto ley Nº 211, de 1973, cuando afectaren los mercados chilenos;
    12°. Los delitos cometidos por chilenos, que se encuentran comprendidos en los artículos 34 y 35 de la Ley que Implementa la Convención sobre la Prohibición del Desarrollo, la Producción, el Almacenamiento y el Empleo de Armas Químicas y sobre su Destrucción y la Convención sobre la Prohibición del Desarrollo, la Producción y el Almacenamiento de Armas Biológicas (Bacteriológicas) y Toxínicas y sobre su Destrucción.




NOTA
      El artículo 3° de la Ley 20960, publicada el 18.10.2016, modifica la presente norma en el sentido de reemplazar en el N° 9, la expresión ", y" por un punto y coma (;), y en el número 10, el punto final por ", y" e incorpora un nuevo N° 11 del siguiente tenor: "11. Los delitos y faltas penales sancionados en la ley Nº 18.556 y en la ley Nº 18.700, cometidos por chilenos o extranjeros.". Sin embargo, este precepto ya había sido modificado por la Ley 20945, publicada el 30.08.2016, tal como aparece en el presente texto actualizado, incorporando también un N° 11.

    Art. 7° Los tribunales sólo podrán ejercer su potestad en los negocios y dentro del territorio que la ley les hubiere respectivamente asignado.
    Lo cual no impide que en los negocios de que conocen puedan dictar providencias que hayan de llevarse a efecto en otro territorio.

    Art. 8° Ningún tribunal puede avocarse el conocimiento de causas o negocios pendientes ante otro tribunal, a menos que la ley le confiera expresamente esta facultad.

    Art. 9° Los actos de los tribunales son públicos, salvo las excepciones expresamente establecidas por la ley.

    Art. 10. Los tribunales no podrán ejercer su ministerio sino a petición de parte, salvo los casos en que la ley los faculte para proceder de oficio.
    Reclamada su intervención en forma legal y en negocios de su competencia, no podrán excusarse de ejercer su autoridad ni aún por falta de ley que resuelva la contienda sometida a su decisión.

    Art. 11. Para hacer ejecutar sus sentencias y para practicar o hacer practicar las actuaciones que decreten, podrán los tribunales requerir de las demás autoridades el auxilio de la fuerza pública que de ellas dependiere, o los otros medios de acción conducentes de que dispusieren.
    La autoridad legalmente requerida debe prestar el auxilio, sin que le corresponda calificar el fundamento con que se le pide ni la justicia o legalidad de la sentencia o decreto que se trata de ejecutar.


    Art. 12. El Poder Judicial es independiente de toda otra autoridad en el ejercicio de sus funciones.

    Art. 13. Las decisiones o decretos que los jueces expidan en los negocios de que conozcan no les impondrán responsabilidad sino en los casos expresamente determinados por la ley.
    Título II

    De los juzgados de garantía y de los tribunales de juicio oral en lo penal



    Párrafo 1º

    De los juzgados de garantía.

    Art. 14. Los juzgados de garantía estarán conformados por uno o más jueces con competencia en un mismo territorio jurisdiccional, que actúan y resuelven unipersonalmente los asuntos sometidos a su conocimiento.

    Corresponderá a los jueces de garantía:

    a) Asegurar los derechos del imputado y demás intervinientes en el proceso penal, de acuerdo a la ley procesal penal;
    b) Dirigir personalmente las audiencias que procedan, de conformidad a la ley procesal penal;
    c) Dictar sentencia, cuando corresponda, en el procedimiento abreviado que contemple la ley procesal penal;
    d) Conocer y fallar las faltas penales de conformidad con el procedimiento contenido en la ley procesal penal;
    e) Conocer y fallar, conforme a los procedimientos regulados en el Título I del Libro IV del Código Procesal Penal, las faltas e infracciones contempladas en la Ley de Alcoholes, cualquiera sea la pena que ella les asigne;
    f) Hacer ejecutar las condenas criminales y las medidas de seguridad, y resolver las solicitudes y reclamos relativos a dicha ejecución, de conformidad a la ley procesal penal;
    g) Conocer y resolver todas las cuestiones y asuntos que la ley de responsabilidad penal juvenil les encomienden, y
    h) Conocer y resolver todas las cuestiones y asuntos que este Código, la ley procesal penal y la ley que establece disposiciones especiales sobre el Sistema de Justicia Militar les encomienden.


    Art. 15. La distribución de las causas entre los jueces de los juzgados de garantía se realizará de acuerdo a un procedimiento objetivo y general, que deberá ser anualmente aprobado por el comité de jueces del juzgado a propuesta del juez presidente, o sólo por este último, según corresponda.
    Art. 16. Existirá un juzgado de garantía con asiento en cada una de las siguientes comunas del territorio de la República, con el número de jueces y con la competencia que en cada caso se indican:

    Primera Región de Tarapacá:
    Iquique, con siete jueces, con competencia sobre la misma comuna.

    Segunda Región de Antofagasta:
    Tocopilla, con un juez, con competencia sobre la misma comuna.
    Calama, con cinco jueces, con competencia sobre las comunas de Calama, Ollagüe y San Pedro de Atacama.
    Antofagasta, con nueve jueces, con competencia sobre las comunas de Sierra Gorda y Antofagasta.

    Tercera Región de Atacama:
    Diego de Almagro, con un juez, con competencia en la misma comuna.
    Copiapó, con cinco jueces, con competencia sobre las comunas de Copiapó y Tierra Amarilla.
    Vallenar, con dos jueces, con competencia sobre las comunas de Vallenar y Alto del Carmen.

    Cuarta Región de Coquimbo:
    La Serena, con cuatro jueces, con competencia sobre las comunas de La Serena y La Higuera.
    Vicuña, con un juez, con competencia sobre las comunas de Vicuña y Paihuano.
    Coquimbo, con tres jueces, con competencia sobre la misma comuna.
    Ovalle, con dos jueces, con competencia sobre las comunas de Ovalle, Río Hurtado, Punitaqui y Monte Patria.
    Illapel, con un juez, con competencia sobre las comunas de Illapel y Salamanca.

    Quinta Región de Valparaíso:
    La Ligua, con dos jueces, con competencia sobre las comunas de La Ligua, Cabildo, Papudo y Zapallar.
    Calera, con dos jueces, con competencia sobre las comunas de Nogales, Calera, La Cruz e Hijuelas.
    San Felipe, con tres jueces, con competencia sobre las comunas de San Felipe, Catemu, Santa María, Panquehue y Llay-LLay.
    Los Andes, con tres jueces, con competencia sobre las comunas de San Esteban, Rinconada, Calle Larga y Los Andes.
    Quillota, con dos jueces, con competencia sobre la misma comuna.
    Limache, con dos jueces, con competencia sobre las comunas de Limache y Olmué.
    Viña del Mar, con ocho jueces, con competencia sobre las comunas de Viña del Mar y Concón.
    Valparaíso, con nueve jueces, con competencia sobre las comunas de Valparaíso y Juan Fernández.
    Quilpué, con tres jueces, con competencia sobre la misma comuna.
    Villa Alemana, con dos jueces, con competencia sobre la misma comuna.
    Casablanca, con un juez, con competencia sobre la misma comuna.
    San Antonio, con cinco jueces, con competencia sobre las comunas de Algarrobo, El Quisco, El Tabo, Cartagena, San Antonio y Santo Domingo.

    Sexta Región del Libertador General Bernardo O'Higgins:
    Graneros, con dos jueces, con competencia sobre las comunas de Mostazal, Graneros y Codegua.
    Rancagua, con nueve jueces, con competencia sobre las comunas de Rancagua, Machalí, Doñihue, Coínco y Olivar.
    San Vicente, con dos jueces, con competencia sobre las comunas de Coltauco, Pichidegua y San Vicente.
    Rengo, con tres jueces, con competencia sobre las comunas de Requínoa, Quinta de Tilcoco, Malloa y Rengo.
    San Fernando, con tres jueces, con competencia sobre las comunas de San Fernando, Placilla y Chimbarongo.
    Santa Cruz, con dos jueces, con competencia sobre las comunas de Santa Cruz, Nancagua, Lolol y Chépica.

    Séptima Región del Maule:
    Curicó, con cuatro jueces, con competencia sobre las comunas de Teno, Rauco, Curicó, Romeral y Sagrada Familia.
    Molina, con dos jueces, con competencia sobre la misma comuna.
    Constitución, con dos jueces, con competencia sobre las comunas de Constitución y Empedrado.
    Talca, con seis jueces, con competencia sobre las comunas de Río Claro, Pencahue, Talca, Pelarco, San Clemente, Maule y San Rafael.
    San Javier, con dos jueces, con competencia sobre las comunas de San Javier y Villa Alegre.
    Cauquenes, con un juez, con competencia sobre la misma comuna.
    Linares, con tres jueces, con competencia sobre las comunas de Colbún, Yerbas Buenas, Linares y Longaví.
    Parral, con un juez, con competencia sobre las comunas de Parral y Retiro.

    Octava Región del Bío Bío:
    Tomé, con un juez, con competencia sobre la misma comuna.
    Talcahuano, con cuatro jueces, con competencia sobre las comunas de Talcahuano y Hualpén.
    Concepción, con ocho jueces, con competencia sobre las comunas de Penco y Concepción.
    San Pedro de la Paz, con tres jueces, con competencia sobre la misma comuna.
    Chiguayante, con dos jueces, con competencia sobre las comunas de Chiguayante y Hualqui.
    Coronel, con dos jueces, con competencia sobre la misma comuna.
    Los Angeles, con cuatro jueces, con competencia sobre las comunas de Los Angeles, Quilleco y Antuco.
    Arauco, con un juez, con competencia sobre la misma comuna.
    Cañete, con un juez, con competencia sobre las comunas de Cañete, Contulmo y Tirúa.

    Novena Región de La Araucanía:
    Angol, con dos jueces, con competencia sobre las comunas de Angol y Renaico.
    Victoria, con un juez, con competencia sobre la misma comuna.
    Nueva Imperial, con un juez, con competencia sobre las comunas de Nueva Imperial, Cholchol y Teodoro Schmidt.
    Temuco, con ocho jueces, con competencia sobre las comunas de Temuco, Vilcún, Melipeuco, Cunco y Padre Las Casas.
    Lautaro, con un juez, con competencia sobre las comunas de Galvarino, Perquenco y Lautaro.
    Pitrufquén, con dos jueces, con competencia sobre las comunas de Freire, Pitrufquén y Gorbea.
    Loncoche, con un juez, con competencia sobre la misma comuna.
    Villarrica, con dos jueces, con competencia sobre la misma comuna.

    Décima Región de Los Lagos:
    Osorno, con cuatro jueces, con competencia sobre las comunas de San Juan de la Costa, San Pablo, Osorno y Puyehue.
    Río Negro, con un juez, con competencia sobre las comunas de Río Negro, Puerto Octay y Purranque.
    Puerto Varas, con dos jueces, con competencia sobre las comunas de Fresia, Frutillar, Puerto Varas y Llanquihue.
    Puerto Montt, con seis jueces, con competencia sobre las comunas de Puerto Montt y Cochamó.
    Ancud, con un juez, con competencia sobre las comunas de Ancud y Quemchi.
    Castro, con dos jueces, con competencia sobre las comunas de Dalcahue, Castro, Chonchi, Puqueldón y Queilén.

    Undécima Región de Aisén del General Carlos Ibáñez del Campo:
    Coihaique, con dos jueces, con competencia sobre las comunas de Coihaique y Río Ibáñez.

    Duodécima Región de Magallanes y la Antártica Chilena:
    Punta Arenas, con cuatro jueces, con competencia sobre las comunas de Laguna Blanca, San Gregorio, Río Verde y Punta Arenas.

    Decimocuarta Región de los Ríos:
    Mariquina, con un juez, con competencia sobre las comunas de Mariquina y Lanco.
    Valdivia, con cuatro jueces, con competencia sobre las comunas de Valdivia y Corral.
    Los Lagos, con un juez, con competencia sobre las comunas de Máfil, Los Lagos y Futrono.

    Decimoquinta Región de Arica y Parinacota:
    Arica, con seis jueces, con competencia sobre las comunas de General Lagos, Putre, Arica y Camarones.

    Región Metropolitana de Santiago:
    Colina, con cuatro jueces, con competencia sobre las comunas de Til Til, Colina y Lampa.
    Puente Alto, con nueve jueces, con competencia sobre las comunas de Puente Alto, San José de Maipo y Pirque.
    San Bernardo, con diez jueces, con competencia sobre las comunas de San Bernardo, Calera de Tango, Buin y Paine.
    Melipilla, con tres jueces, con competencia sobre las comunas de Melipilla, San Pedro y Alhué.
    Talagante, con seis jueces, con competencia sobre las comunas de Talagante, El Monte, Isla de Maipo, Peñaflor y Padre Hurtado.
    Curacaví, con dos jueces, con competencia sobre las comunas de Curacaví y María Pinto.

    Habrá además, con asiento en la comuna de Santiago, los siguientes juzgados de garantía:

    Primer Juzgado de Garantía de Santiago, con cinco jueces, con competencia sobre la comuna de Pudahuel.
    Segundo Juzgado de Garantía de Santiago, con diez jueces, con competencia sobre las comunas de Quilicura, Huechuraba, Renca y Conchalí.
    Tercer Juzgado de Garantía de Santiago, con seis jueces, con competencia sobre las comunas de Independencia y Recoleta.
    Cuarto Juzgado de Garantía de Santiago, con doce jueces, con competencia sobre las comunas de Lo Barnechea, Vitacura, Las Condes y La Reina.
    Quinto Juzgado de Garantía de Santiago, con cinco jueces, con competencia sobre las comunas de Cerro Navia y Lo Prado.
    Sexto Juzgado de Garantía de Santiago, con siete jueces, con competencia sobre las comunas de Estación Central y Quinta Normal.
    Séptimo Juzgado de Garantía de Santiago, con catorce jueces, con competencia sobre la comuna de Santiago.
    Octavo Juzgado de Garantía de Santiago, con nueve jueces, con competencia sobre las comunas de Providencia y Ñuñoa.
    Noveno Juzgado de Garantía de Santiago, con nueve jueces, con competencia sobre las comunas de Maipú y Cerrillos.
    Décimo Juzgado de Garantía de Santiago, con cinco jueces, con competencia sobre las comunas de Lo Espejo y Pedro Aguirre Cerda.
    Undécimo Juzgado de Garantía de Santiago, con ocho jueces, con competencia sobre las comunas de San Miguel, La Cisterna y El Bosque.
    Duodécimo Juzgado de Garantía de Santiago, con seis jueces, con competencia sobre las comunas de San Joaquín y La Granja.
    Decimotercer Juzgado de Garantía de Santiago, con siete jueces, con competencia sobre las comunas de Macul y Peñalolén.
    Decimocuarto Juzgado de Garantía de Santiago, con nueve jueces, con competencia sobre la comuna de La Florida.
    Decimoquinto Juzgado de Garantía de Santiago, con siete jueces, con competencia sobre las comunas de San Ramón y La Pintana.

    Región de Ñuble:
    San Carlos, con dos jueces, con competencia sobre las comunas de San Carlos, Ñiquén y San Fabián.
    Chillán, con cuatro jueces, con competencia sobre las comunas de San Nicolás, Chillán, Coihueco, Pinto y Chillán Viejo.
    Yungay, con un juez, con competencia sobre las comunas de El Carmen, Pemuco, Yungay y Tucapel.


 La Ley 21527, Art. 56 N° 1, D.O. 12.01.2023 agregó un Artículo 16 BIS en esta ubicación que depende del siguiente evento para entrar en vigencia: Las modificaciones introducidas a la presente norma por la ley 21527, publicada el 12.01.2023, comenzarán a regir en forma gradual en plazos de 12, 24 y 36 meses desde su fecha de publicación, para las regiones que indica, conforme lo dispone su artículo primero transitorio.
Ver texto diferido
Ver modificatoria
 La Ley 21527, Art. 56 N° 2, D.O. 12.01.2023 agregó un Artículo 16 TER en esta ubicación que depende del siguiente evento para entrar en vigencia: Las modificaciones introducidas a la presente norma por la ley 21527, publicada el 12.01.2023, comenzarán a regir en forma gradual en plazos de 12, 24 y 36 meses desde su fecha de publicación, para las regiones que indica, conforme lo dispone su artículo primero transitorio.
Ver texto diferido
Ver modificatoria
 La Ley 21527, Art. 56 N° 3, D.O. 12.01.2023 agregó un Artículo 16 QUÁTER en esta ubicación que depende del siguiente evento para entrar en vigencia: Las modificaciones introducidas a la presente norma por la ley 21527, publicada el 12.01.2023, comenzarán a regir en forma gradual en plazos de 12, 24 y 36 meses desde su fecha de publicación, para las regiones que indica, conforme lo dispone su artículo primero transitorio.
Ver texto diferido
Ver modificatoria
    Párrafo 2º
    De los tribunales de juicio oral en lo penal




    Art. 17. Los tribunales de juicio oral en lo penal funcionarán en una o más salas integradas por tres de sus miembros.
    Cada sala será dirigida por un juez presidente de sala, quien tendrá las atribuciones a que alude el artículo 92 y las demás de orden que la ley procesal penal indique. Sin perjuicio de lo anterior, podrán integrar también cada sala otros jueces en calidad de alternos, con el solo propósito de subrogar, si fuere necesario, a los miembros que se vieren impedidos de continuar participando en el desarrollo del juicio oral, en los términos que contemplan los artículos 76, inciso final, y 281, inciso quinto, del Código Procesal Penal.
    La integración de las salas de estos tribunales, incluyendo a los jueces alternos de cada una, se determinará mediante sorteo anual que se efectuará durante el mes de enero de cada año.
    La distribución de las causas entre las diversas salas se hará de acuerdo a un procedimiento objetivo y general que deberá ser anualmente aprobado por el comité de jueces del tribunal, a propuesta del juez presidente.


    Art. 18. Corresponderá a los tribunales de juicio oral en lo penal:

    a) Conocer y juzgar las causas por crimen o simple delito, salvo aquellas relativas a simples delitos cuyo conocimiento y fallo corresponda a un juez de garantía;
    b) Resolver, en su caso, sobre la libertad o prisión preventiva de los acusados puestos a su disposición;
    c) Resolver todos los incidentes que se promuevan durante el juicio oral;
    d) Conocer y resolver todas las cuestiones y asuntos que la ley de responsabilidad penal juvenil les encomienden, y
    e) Conocer y resolver los demás asuntos que la ley procesal penal y la ley que establece disposiciones especiales sobre el Sistema de Justicia Militar les encomiende.


    Art. 19. Las decisiones de los tribunales de juicio oral en lo penal se regirán, en lo que no resulte contrario a las normas de este párrafo, por las reglas sobre acuerdos en las Cortes de Apelaciones contenidas en los artículos 72, 81, 83, 84 y 89 de este Código.
    Sólo podrán concurrir a las decisiones del tribunal los jueces que hubieren asistido a la totalidad de la audiencia del juicio oral.
    La decisión deberá ser adoptada por la mayoría de los miembros de la sala.
    Cuando existiere dispersión de votos en relación con una decisión, la sentencia o la determinación de la pena si aquélla fuere condenatoria, el juez que sostuviere la opinión más desfavorable al condenado deberá optar por alguna de las otras.
    Si se produjere desacuerdo acerca de cuál es la opinión que favorece más al imputado, prevalecerá la que cuente con el voto del juez presidente de la sala.
    Sin perjuicio de lo dispuesto en el presente artículo y en el artículo 281 del Código Procesal Penal, podrán ser resueltas por un único juez del tribunal de juicio oral en lo penal la fijación de día y hora para la realización de audiencias. Asimismo, podrán ser resueltas por un único juez del tribunal de juicio oral en lo penal las resoluciones de mero trámite, tales como téngase presente y traslados; pedir cuenta de oficios e informes; y tramitación de exhortos.

      Art. 20. Derogado.


    Art. 21. Existirá un tribunal de juicio oral en lo penal con asiento en cada una de las siguientes comunas del territorio de la República, con el número de jueces y con la competencia que en cada caso se indican:
    Primera Región de Tarapacá:
    Iquique, con trece jueces, con competencia sobre las comunas de Huara, Camiña, Colchane, Iquique, Pozo Almonte, Alto Hospicio y Pica.

    Segunda Región de Antofagasta:
    Calama, con siete jueces, con competencia sobre las comunas de Calama, Ollagüe y San Pedro de Atacama.
    Antofagasta, con trece jueces, con competencia sobre las comunas de Tocopilla, María Elena, Mejillones, Sierra Gorda, Antofagasta y Taltal.

    Tercera Región de Atacama:
    Copiapó, con nueve jueces, con competencia sobre las comunas de Chañaral, Diego de Almagro, Caldera, Copiapó, Tierra Amarilla, Huasco, Vallenar, Freirina y Alto del Carmen.

    Cuarta Región de Coquimbo:
    La Serena, con diez jueces, con competencia sobre las comunas de La Higuera, Vicuña, La Serena, Coquimbo, Andacollo y Paihuano.
    Ovalle, con siete jueces, con competencia sobre las comunas de Ovalle, Río Hurtado, Punitaqui, Monte Patria, Combarbalá, Canela, Illapel, Los Vilos y Salamanca.

    Quinta Región de Valparaíso:
    San Felipe, con cuatro jueces, con competencia sobre las comunas de la provincia de San Felipe.
    Los Andes, con cuatro jueces, con competencia sobre las comunas de la provincia de Los Andes.
    Quillota, con seis jueces, con competencia sobre las comunas de La Ligua, Petorca, Cabildo, Papudo, Zapallar, Nogales, Calera, La Cruz, Quillota, Hijuelas, Limache y Olmué.
    Viña del Mar, con dieciseís jueces, con competencia sobre las comunas de Puchuncaví, Quintero, Viña del Mar, Villa Alemana, Quilpué y Concón.
    Valparaíso, con diecinueve jueces, con competencia sobre las comunas de Juan Fernández, Valparaíso, Casablanca e Isla de Pascua.
    San Antonio, con siete jueces, con competencia sobre las comunas de Algarrobo, El Quisco, El Tabo, Cartagena, San Antonio y Santo Domingo.

    Sexta Región del Libertador General Bernardo O'Higgins:
    Rancagua, con dieciséis jueces, con competencia sobre las comunas de Mostazal, Graneros, Codegua, Rancagua, Machalí, Las Cabras, Coltauco, Doñihue, Olivar, Coinco, Requínoa, Peumo, Quinta de Tilcoco, Pichidegua, San Vicente, Malloa y Rengo.
    San Fernando, con cuatro jueces, con competencia sobre las comunas de San Fernando, Placilla y Chimbarongo.
    Santa Cruz, con siete jueces, con competencia sobre las comunas de Santa Cruz, Navidad, Litueche, La Estrella, Pichilemu, Marchigüe, Paredones, Peralillo, Palmilla, Pumanque, Nancagua, Lolol y Chépica.

    Séptima Región del Maule:
    Curicó, con siete jueces, con competencia sobre las comunas de Teno, Vichuquén, Hualañé, Rauco, Curicó, Romeral, Licantén, Sagrada Familia y Molina.
    Talca, con siete jueces, con competencia sobre las comunas de Curepto, Río Claro, Constitución, Pencahue, Talca, Pelarco, San Clemente, Maule, Empedrado y San Rafael.
    Linares, con seis jueces, con competencia sobre las comunas de San Javier, Villa Alegre, Colbún, Yerbas Buenas, Linares y Longaví .
    Cauquenes, con cuatro jueces, con competencia sobre las comunas de Chanco, Cauquenes, Pelluhue, Retiro y Parral.

    Octava Región del Bío Bío:
    Concepción, con veintidós jueces, con competencia sobre las comunas de Tomé, Penco, Florida, Concepción, Coronel, Hualqui, Lota, Santa Juana, Talcahuano, San Pedro de la Paz, Hualpén y Chiguayante.
    Los Angeles, con seis jueces, con competencia sobre las comunas de San Rosendo, Yumbel, Cabrero, Laja, Los Angeles, Antuco, Quilleco, Nacimiento, Negrete, Mulchén, Santa Bárbara, Alto Biobío y Quilaco.
    Cañete, con seis jueces, con competencia sobre las comunas de Arauco, Curanilahue, Lebu, Los Alamos, Cañete, Contulmo y Tirúa.

    Novena Región de La Araucanía:
    Angol, con cuatro jueces, con competencia sobre las comunas de Angol, Renaico, Collipulli, Purén, Los Sauces, Ercilla, Lumaco, Traiguén y Victoria.
    Temuco, con diez jueces, con competencia sobre las comunas de Lonquimay, Curacautín, Galvarino, Perquenco, Carahue, Nueva Imperial, Temuco, Lautaro, Vilcún, Melipeuco, Saavedra, Teodoro Schmidt, Freire, Cunco, Toltén, Pitrufquén, Gorbea, Cholchol y Padre Las Casas.
    Villarrica, con cuatro jueces, con competencia sobre las comunas de Loncoche, Villarrica, Pucón y Curarrehue.

    Décima Región de Los Lagos:
    Osorno, con seis jueces, con competencia sobre las comunas de San Juan de la Costa, San Pablo, Osorno, Puyehue, Río Negro, Puerto Octay y Purranque.
    Puerto Montt, con seis jueces, con competencia sobre las comunas de Fresia, Frutillar, Puerto Varas, Llanquihue, Los Muermos, Puerto Montt, Cochamó, Maullín, Calbuco, Hualaihué, Chaitén, Futaleufú y Palena.
    Castro, con cuatro jueces, con competencia sobre las comunas de Ancud, Quemchi, Dalcahue, Castro, Curaco de Vélez, Quinchao, Chonchi, Puqueldón, Queilén y Quellón.

    Undécima Región de Aisén del General Carlos Ibáñez del Campo:
    Coihaique, con cuatro jueces, con competencia sobre las comunas de Guaitecas, Cisnes, Aisén, Lago Verde, Coihaique, Río Ibáñez, Chile Chico, Cochrane, Tortel y OHiggins.

    Duodécima Región de Magallanes y la Antártica Chilena:
    Punta Arenas, con seis jueces, con competencia sobre las comunas de Natales, Torres del Paine, Laguna Blanca, San Gregorio, Río Verde, Punta Arenas, Primavera, Porvenir, Timaukel, Cabo de Hornos y Antártica.

    Decimocuarta Región de los Ríos:
    Valdivia, con siete jueces, con competencia sobre las comunas de Mariquina, Lanco, Panguipulli, Máfil, Valdivia, Los Lagos, Corral, Paillaco, Futrono, La Unión, Lago Ranco y Río Bueno.

    Decimoquinta Región de Arica y Parinacota:
    Arica, con diez jueces, con competencia sobre las comunas de General Lagos, Putre, Arica y Camarones

    Región Metropolitana de Santiago:
    Colina, con seis jueces, con competencia sobre las comunas de Til Til, Colina y Lampa.
    Puente Alto, con nueve jueces, con competencia sobre las comunas de Puente Alto, San José de Maipo y Pirque.
    San Bernardo, con nueve jueces, con competencia sobre las comunas de San Bernardo, Calera de Tango, Buin y Paine.
    Melipilla, con seis jueces, con competencia sobre las comunas de Melipilla, San Pedro, Alhué, Curacaví y María Pinto.
    Talagante, con seis jueces, con competencia sobre las comunas de Talagante, El Monte, Isla de Maipo, Peñaflor y Padre Hurtado.

    Habrá además, con asiento en la comuna de Santiago, los siguientes tribunales de juicio oral en lo penal:

    Primer Tribunal de Juicio Oral en lo Penal de Santiago, con doce jueces, con competencia sobre las comunas de Lo Prado, Cerro Navia y Pudahuel.
    Segundo Tribunal de Juicio Oral en lo Penal de Santiago, con veintiún jueces, con competencia sobre las comunas de Quilicura, Huechuraba, Renca, Conchalí, Independencia y Recoleta.
    Tercer Tribunal de Juicio Oral en lo Penal de Santiago, con diecinueve jueces, con competencia sobre las comunas de Lo Barnechea, Vitacura, Las Condes, Providencia, Ñuñoa y La Reina.
    Cuarto Tribunal de Juicio Oral en lo Penal de Santiago, con veinte jueces, con competencia sobre las comunas de Quinta Normal, Estación Central y Santiago.
    Quinto Tribunal de Juicio Oral en lo Penal de Santiago, con nueve jueces, con competencia sobre las comunas de Maipú y Cerrillos.
    Sexto Tribunal de Juicio Oral en lo Penal de Santiago, con veintisiete jueces, con competencia sobre las comunas de Lo Espejo, Pedro Aguirre Cerda, San Miguel, San Joaquín, La Cisterna, San Ramón, La Granja, El Bosque y La Pintana.
    Séptimo Tribunal de Juicio Oral en lo Penal de Santiago, con trece jueces, con competencia sobre las comunas de Macul, Peñalolén y La Florida.

    Región de Ñuble:
    Chillán, con siete jueces, con competencia sobre las comunas de Cobquecura, Quirihue, Ninhue, San Carlos, Ñiquén, San Fabián, San Nicolás, Treguaco, Portezuelo, Chillán, Coihueco, Coelemu, Ránquil, Pinto, Quillón, Bulnes, San Ignacio, El Carmen, Pemuco, Yungay, Tucapel y Chillán Viejo.

    Art. 21 A. Cuando sea necesario para facilitar la aplicación oportuna de la justicia penal, de conformidad a criterios de distancia, acceso físico y dificultades de traslado de quienes intervienen en el proceso, los tribunales de juicio oral en lo penal se constituirán y funcionarán en localidades situadas fuera de su lugar de asiento.
    Corresponderá a la respectiva Corte de Apelaciones determinar anualmente la periodicidad y forma con que los tribunales de juicio oral en lo penal darán cumplimiento a lo dispuesto en este artículo. Sin perjuicio de ello, la Corte podrá disponer en cualquier momento la constitución y funcionamiento de un tribunal de juicio oral en lo penal en una localidad fuera de su asiento, cuando la mejor atención de uno o más casos así lo aconseje.
    La Corte de Apelaciones adoptará esta medida previo informe de la Corporación Administrativa del Poder Judicial y de los jueces presidentes de los comités de jueces de los tribunales de juicio oral en lo penal correspondientes.

    Párrafo 3º

    Del Comité de Jueces

    Art. 22. En los juzgados de garantía en los que sirvan tres o más jueces y en cada tribunal de juicio oral en lo penal, habrá un comité de jueces, que estará integrado en la forma siguiente:
    En aquellos juzgados o tribunales compuestos por cinco jueces o menos, el comité de jueces se conformará por todos ellos.
    En aquellos juzgados o tribunales conformados por más de cinco jueces, el comité lo compondrán los cinco jueces que sean elegidos por la mayoría del tribunal, cada dos años.
    De entre los miembros del comité de jueces se elegirá al juez presidente, quien durará dos años en el cargo y podrá ser reelegido hasta por un nuevo período.
    Si se ausentare alguno de los miembros del comité de jueces o vacare el cargo por cualquier causa, será reemplazado, provisoria o definitivamente según el caso, por el juez que hubiere obtenido la más alta votación después de los que hubieren resultado electos y, en su defecto, por el juez más antiguo de los que no integraren el comité de jueces. En caso de ausencia o imposibilidad del juez presidente, será suplido en el cargo por el juez más antiguo si ella no superare los tres meses, o se procederá a una nueva elección para ese cargo si el impedimento excediere de ese plazo.
    Los acuerdos del comité de jueces se adoptarán por mayoría de votos; en caso de empate decidirá el voto del juez presidente.


    Art. 23. Al comité de jueces corresponderá:

    a) Aprobar el procedimiento objetivo y general a que se refieren los artículos 15 y 17, en su caso;
    b) Designar, de la terna que le presente el juez presidente, al administrador del tribunal;
    c) Suprimida.
    d) Resolver acerca de la remoción del administrador;
    e) Designar al personal del juzgado o tribunal, a propuesta en terna del administrador;
    f) Conocer de la apelación que se interpusiere en contra de la resolución del administrador que remueva al subadministrador, a los jefes de unidades o a los empleados del juzgado o tribunal;
    g) Decidir el proyecto de plan presupuestario anual que le presente el juez presidente, para ser propuesto a la Corporación Administrativa del Poder Judicial, y
    h) Conocer de todas las demás materias que señale la ley.

    En los juzgados de garantía en que se desempeñen uno o dos jueces, las atribuciones indicadas en las letras b), c), d) y f) corresponderán al Presidente de la Corte de Apelaciones respectiva. A su vez, las atribuciones previstas en los literales a), e), g) y h) quedarán radicadas en el juez que cumpla la función de juez presidente.

    Párrafo 4º

    Del Juez Presidente del Comité de Jueces

    Art. 24. Al juez presidente del comité de jueces le corresponderá velar por el adecuado funcionamiento del juzgado o tribunal.
    En el cumplimiento de esta función, tendrá los siguientes deberes y atribuciones:

    a) Presidir el comité de jueces;
    b) Relacionarse con la Corporación Administrativa del Poder Judicial en todas las materias relativas a la competencia de ésta;
    c) Proponer al comité de jueces el procedimiento objetivo y general a que se refieren los artículos 15 y 17;
    d) Elaborar anualmente una cuenta de la gestión jurisdiccional del juzgado;
    e) Aprobar los criterios de gestión administrativa que le proponga el administrador del tribunal y supervisar su ejecución;
    f) Aprobar la distribución del personal que le presente el administrador del tribunal;
    g) Calificar al personal, teniendo a la vista la evaluación que le presente el administrador del tribunal;
    h) Presentar al comité de jueces una terna para la designación del administrador del tribunal;
    i) Suprimida.
    j) Proponer al comité de jueces la remoción del administrador del tribunal.
    El desempeño de la función de juez presidente del comité de jueces del juzgado o tribunal podrá significar una reducción proporcional de su trabajo jurisdiccional, según determine el comité de jueces.
    Tratándose de los juzgados de garantía en los que se desempeñe un solo juez, éste tendrá las atribuciones del juez presidente, con excepción de las contempladas en las letras a) y c). Las atribuciones de las letras h) y j) las ejercerá el juez ante el Presidente de la Corte de Apelaciones respectiva.
    En aquellos juzgados de garantía conformados por dos jueces, las atribuciones del juez presidente, con las mismas excepciones señaladas en el inciso anterior, se radicarán anualmente en uno de ellos, empezando por el más antiguo.

    § 5. De la organización administrativa de los juzgados de garantía y de los tribunales de juicio oral en lo penal.
    Art. 25. Los juzgados de garantía y los tribunales de juicio oral en lo penal se organizarán en unidades administrativas para el cumplimiento eficaz y eficiente de las siguientes funciones:
    1.- Sala, que consistirá en la organización y asistencia a la realización de las audiencias.
    2.- Atención de público, destinada a otorgar una adecuada atención, orientación e información al público que concurra al juzgado o tribunal, especialmente a la víctima, al defensor y al imputado, recibir la información que éstos entreguen y manejar la correspondencia del juzgado o tribunal.
    3.- Servicios, que reunirá las labores de soporte técnico de la red computacional del juzgado o tribunal, de contabilidad y de apoyo a la actividad administrativa del juzgado o tribunal, y la coordinación y abastecimiento de todas las necesidades físicas y materiales para la realización de las audiencias.
    4.- Administración de causas, que consistirá en desarrollar toda la labor relativa a las notificaciones; al manejo de causas y registros del proceso penal en el juzgado o tribunal, incluidas las relativas al manejo de las fechas y salas para las audiencias; al archivo judicial básico, al ingreso y al número de rol de las causas nuevas; a la primera audiencia judicial de los detenidos; a la actualización diaria de la base de datos que contenga las causas del juzgado o tribunal, y a las estadísticas básicas del juzgado o tribunal.
    5.- Apoyo a testigos y peritos, destinada a brindar adecuada y rápida atención, información y orientación a los testigos y peritos citados a declarar en el transcurso de un juicio oral. Esta función existirá solamente en los tribunales de juicio oral en lo penal.


    Art. 26. Corresponderá a la Corporación Administrativa del Poder Judicial determinar, en la ocasión a que se refiere el inciso segundo del artículo 498, las unidades administrativas con que cada juzgado o tribunal contará para el cumplimiento de las funciones señaladas en el artículo anterior.
 La Ley 21527, Art. 56 N° 5, D.O. 12.01.2023 agregó un Artículo 26 BIS en esta ubicación que depende del siguiente evento para entrar en vigencia: Las modificaciones introducidas a la presente norma por la ley 21527, publicada el 12.01.2023, comenzarán a regir en forma gradual en plazos de 12, 24 y 36 meses desde su fecha de publicación, para las regiones que indica, conforme lo dispone su artículo primero transitorio.
Ver texto diferido
Ver modificatoria
 La Ley 21527, Art. 56 N° 6, D.O. 12.01.2023 agregó un Artículo 26 TER en esta ubicación que depende del siguiente evento para entrar en vigencia: Las modificaciones introducidas a la presente norma por la ley 21527, publicada el 12.01.2023, comenzarán a regir en forma gradual en plazos de 12, 24 y 36 meses desde su fecha de publicación, para las regiones que indica, conforme lo dispone su artículo primero transitorio.
Ver texto diferido
Ver modificatoria

    TITULO III

    De los Jueces de Letras

    Art. 27. Sin perjuicio de lo que se previene en los artículo 28 al 40, en cada comuna habrá, a lo menos, un juzgado de letras.
    Los juzgados de letras estarán conformados por uno o más jueces con competencia en un mismo territorio jurisdiccional; sin embargo, actuarán y resolverán unipersonalmente los asuntos sometidos a su conocimiento.
    Los nuevos juzgados que se instalen tendrán como territorio jurisdiccional la respectiva comuna y, en consecuencia, dejarán de ser competentes en esos territorios los juzgados que anteriormente tenían jurisdicción sobre dichas comunas.

    Art. 27 bis. Los juzgados de letras con competencia común integrados por dos jueces, tendrán la siguiente planta de personal: un administrador, un jefe de unidad, dos administrativos jefe, cinco administrativos 1º, dos administrativos 2º, un administrativo 3º, tres ayudantes de servicios y un auxiliar.
    Los juzgados de letras con competencia común integrados por tres jueces tendrán la siguiente planta de personal: un administrador, un jefe de unidad, dos administrativos jefe, cinco administrativos 1º, tres administrativos 2º, dos administrativos 3º y cuatro auxiliares.
    La planta de personal de los tribunales señalados en los incisos anteriores que tengan dentro de su competencia la resolución de asuntos de familia contarán, adicionalmente, con un consejero técnico.
    Los jueces y el personal directivo de estos juzgados tendrán los grados de la Escala de Sueldos Bases Mensuales del Escalafón del Personal Superior del Poder Judicial que se indican a continuación:
    a) Los jueces, el grado correspondiente según el asiento del tribunal.
    b) Los administradores de juzgados de letras de competencia común de capital de provincia y los de comuna o agrupación de comunas, grados VIII y IX del Escalafón Superior del Poder Judicial, respectivamente.
    c) Los jefes de unidad de juzgados de letras de competencia común de capital de provincia y los de comuna o agrupación de comunas, grados X y XI del Escalafón Superior del Poder Judicial, respectivamente.
    El personal de empleados de los juzgados de letras de competencia común con dos o tres jueces, tendrán los grados de la Escala de Sueldos Bases Mensuales del Personal del Poder Judicial, que a continuación se indican:
    a) Administrativos jefe de juzgados de letras de competencia común de capital de provincia y los de comuna o agrupación de comunas, grados XII y XIII del Escalafón de Empleados del Poder Judicial, respectivamente.
    b) Administrativos 1º de juzgados de letras de competencia común de capital de provincia y los de comuna o agrupación de comunas, grados XIII y XIV del Escalafón de Empleados del Poder Judicial, respectivamente.
    c) Administrativos 2º de juzgados de letras de competencia común de capital de provincia y los de comuna o agrupación de comunas, grados XIV y XV del Escalafón de Empleados del Poder Judicial, respectivamente.
    d) Administrativos 3º de juzgados de letras de competencia común de capital de provincia y los de comuna o agrupación de comunas, grados XV y XVI del Escalafón de Empleados del Poder Judicial, respectivamente.
    e) Ayudantes de servicios de juzgados de letras de competencia común de capital de provincia y los de comuna o agrupación de comunas, grados XVII y XVIII del Escalafón de Empleados del Poder Judicial, respectivamente.
    f) Auxiliares de juzgados de letras de competencia común de capital de provincia y los de comuna o agrupación de comunas, grado XVIII del Escalafón de Empleados del Poder Judicial.

    Art. 27 ter. En los juzgados de competencia común con dos o tres jueces, habrá un juez presidente del tribunal, cuyo cargo se radicará anualmente en cada uno de los jueces que lo integran comenzando por el más antiguo.
    Sus atribuciones y deberes son los siguientes:

    a) Velar por el adecuado funcionamiento del juzgado;
    b) Designar al personal del juzgado, a propuesta en terna del administrador;
    c) Relacionarse con la Corporación Administrativa del Poder Judicial en todas las materias relativas a la competencia de ésta;
    d) Decidir el proyecto de plan presupuestario anual para ser propuesto a la Corporación Administrativa del Poder Judicial;
    e) Elaborar anualmente una cuenta de la gestión jurisdiccional del juzgado;
    f) Aprobar los criterios de gestión administrativa que le proponga el administrador del tribunal y supervisar su ejecución;
    g) Aprobar la distribución del personal que le presente el administrador del tribunal;
    h) Aprobar, anualmente, un procedimiento objetivo y general de distribución de causas entre los jueces del tribunal;
    i) Calificar al personal, teniendo a la vista la evaluación que le presente el administrador del tribunal;
    j) Presentar al Presidente de la Corte de Apelaciones respectiva una terna para la designación del administrador del tribunal;
    k) Evaluar anualmente la gestión del administrador;
    l) Proponer al Presidente de la Corte de Apelaciones respectiva la remoción del administrador del tribunal, y
    m) Ejercer las demás atribuciones y deberes que determinen las leyes.

    Art. 27 quater. Los juzgados de letras de competencia común con dos o tres jueces se organizarán en las siguientes unidades administrativas para el cumplimiento eficaz y eficiente de las correspondientes funciones:
    a) Sala, que consistirá en la organización y asistencia a la realización de las audiencias.
    b) Atención a Público, destinada a otorgar una adecuada atención, orientación e información al público que concurra al tribunal y manejar la correspondencia y custodia del tribunal.
    c) Administración de Causas, que consistirá en desarrollar toda la labor relativa al manejo de causas y registros de los procesos en el juzgado, incluidas las relativas a las notificaciones, al manejo de las fechas y salas para las audiencias, al archivo judicial básico, al ingreso y al número de rol de las causas nuevas, a la actualización diaria de la base de datos que contenga las causas del juzgado y a las estadísticas básicas del mismo.
    d) Servicios, que reunirá las labores de soporte técnico de la red computacional del juzgado, de contabilidad y de apoyo a la actividad administrativa del mismo, y la coordinación y abastecimiento de todas las necesidades físicas y materiales que requiera el procedimiento.
    e) Cumplimiento, que desarrollará las gestiones necesarias para la adecuada y cabal ejecución de las resoluciones judiciales y demás títulos ejecutivos de competencia de estos tribunales.

    Art. 28. En la Primera Región, de Tarapacá, existirán los siguientes juzgados de letras:

    A.- JUZGADOS CIVILES:

    Tres juzgados con asiento en la comuna de Iquique, con competencia sobre las comunas de Iquique y Alto Hospicio.

    B.- JUZGADOS DE COMPETENCIA COMÚN:

    Un Juzgado con asiento en la comuna de Pozo Almonte, con tres jueces, con competencia sobre las comunas de Pica, Pozo Almonte, Huara, Colchane y Camiña.


    Art. 29. En la Segunda Región, de Antofagasta, existirán los siguientes juzgados de letras:

    A.- JUZGADOS CIVILES:

    Cuatro juzgados de letras en lo civil en la comuna de Antofagasta, con competencia sobre las comunas de Antofagasta y Sierra Gorda.

    B.- JUZGADOS DE COMPETENCIA COMUN:

    Un juzgado con asiento en la comuna de Tocopilla, con dos jueces, con competencia sobre la misma comuna;
    Un juzgado con asiento en la comuna de María Elena, con competencia sobre la misma comuna. Un juzgado con asiento en la comuna de Mejillones, con dos jueces, con competencia sobre la misma comuna;
    Tres juzgados con asiento en la comuna de Calama, con competencia sobre las comunas de la provincia de El Loa, y
    Un juzgado con asiento en la comuna de Taltal, con dos jueces, con competencia sobre la misma comuna.



    Art. 30. En la Tercera Región, de Atacama, existirán los siguientes juzgados de letras:

    A.- JUZGADOS CIVILES:

    Cuatro Juzgados con asiento en la comuna de Copiapó, con competencia sobre las comunas de Copiapó y Tierra Amarilla;

    B.- JUZGADOS DE COMPETENCIA COMÚN:

    Un Juzgado con asiento en la comuna de Chañaral, con dos jueces, con competencia sobre la misma comuna;
    Un Juzgado con asiento en la comuna de Diego de Almagro, con competencia sobre la misma comuna;
    Un Juzgado con asiento en la comuna de Caldera, con dos jueces, con competencia sobre la misma comuna;
    Un Juzgado con asiento en la comuna de Freirina, con competencia sobre las comunas de Freirina y Huasco, y
    Dos Juzgados con asiento en la comuna de Vallenar, con competencia sobre las comunas de Vallenar y Alto del Carmen.


    Art. 31. En la Cuarta Región, de Coquimbo, existirán los siguientes juzgados de letras:

    A.- JUZGADOS CIVILES:

    Tres Juzgados con asiento en la comuna de La Serena, con competencia sobre las comunas de La Serena y La Higuera;
    Tres Juzgados con asiento en la comuna de Coquimbo, con competencia sobre la misma comuna;

    B.- JUZGADOS DE COMPETENCIA COMÚN:

    Un Juzgado con asiento en la comuna de Vicuña, con dos jueces, con competencia sobre las comunas de Vicuña y Paihuano;
    Un Juzgado con asiento en la comuna de Andacollo, con competencia sobre la misma comuna;
    Tres Juzgados con asiento en la comuna de Ovalle, con competencia sobre las comunas de Ovalle, Río Hurtado, Monte Patria y Punitaqui;
    Un Juzgado con asiento en la comuna de Combarbalá, con competencia sobre la misma comuna;
    Un Juzgado con asiento en la comuna de Illapel, con dos jueces, con competencia sobre las comunas de Illapel y Salamanca, y
    Un Juzgado con asiento en la comuna de Los Vilos, con dos jueces, con competencia sobre las comunas de Los Vilos y Canela.


    Art. 32. En la Quinta Región, de Valparaíso, existirán los siguientes juzgados de letras que tendrán jurisdicción en los territorios que se indican:

    A.- JUZGADOS CIVILES:

    Cinco juzgados de letras en lo civil con asiento en la comuna de Valparaíso y competencia sobre las comunas de Valparaíso y Juan Fernández.
    Tres juzgados de letras en lo civil con asiento en la comuna de Viña del Mar y competencia sobre las comunas de Viña del Mar y Concón, los cuales tendrán la categoría de juzgados de asiento de Corte para todos los efectos legales.

    B.- JUZGADOS CON COMPETENCIA COMUN:

    Dos juzgados de letras con asiento en la comuna de Quilpué, con competencia sobre la misma comuna;
    Un juzgado de letras con asiento en la comuna de Villa Alemana, con dos jueces, con competencia sobre la misma comuna;
    Un juzgado de letras con asiento en la comuna de Casablanca, con competencia sobre las comunas de Casablanca, El Quisco y Algarrobo, de la Quinta Región y la comuna de Curacaví, de la Región Metropolitana;
    Un juzgado de letras con asiento en la comuna de La Ligua, con dos jueces, con competencia sobre las comunas de La Ligua, Cabildo, Zapallar y Papudo;
    Un juzgado de letras con asiento en la comuna de Petorca, con competencia sobre la misma comuna;
    Dos juzgados de letras con asiento en la comuna de Los Andes, con competencia sobre las comunas de la provincia de Los Andes;
    Un juzgado de letras con asiento en la comuna de San Felipe, con competencia sobre las comunas de San Felipe, Santa María, Panquehue, Llaillay y Catemu;
    Un juzgado de letras con asiento en la comuna de Putaendo, con competencia sobre la misma comuna;
    Dos juzgados de letras con asiento en la comuna de Quillota, con competencia sobre las comunas de Quillota y La Cruz;
    Un juzgado de letras con asiento en la comuna de Quintero, con tres jueces, con competencia sobre las comunas de Quintero y Puchuncaví;
    Un juzgado de letras con asiento en la comuna de Calera, con dos jueces, con competencia sobre las comunas de Calera, Nogales e Hijuelas;
    Un juzgado de letras con asiento en la comuna de Limache, con competencia sobre las comunas de Limache y Olmué;
    Dos juzgados de letras con asiento en la comuna de San Antonio, con competencia sobre las comunas de San Antonio, Cartagena, El Tabo y Santo Domingo, y
    Un juzgado de letras con asiento en Isla de Pascua, con competencia sobre la comuna de la provincia de Isla de Pascua.


    Art. 33. En la Sexta Región, del Libertador General Bernardo O'Higgins, existirán los siguientes juzgados de letras que tendrán competencia en los territorios que se indican:

    A.- JUZGADOS CIVILES:

    Dos juzgados de letras en lo civil con asiento en la comuna de Rancagua, con competencia sobre las comunas de Rancagua, Graneros, Mostazal, Codegua, Machalí, Coltauco, Doñihue, Coínco y Olivar.

    B.- JUZGADOS CON COMPETENCIA COMUN:

    Un juzgado con asiento en la comuna de Rengo, con dos jueces, con competencia sobre las comunas de Rengo, Requínoa, Malloa y Quinta de Tilcoco;
    Un juzgado con asiento en la comuna de San Vicente, con dos jueces, con competencia sobre las comunas de San Vicente y Pichidegua;
    Un juzgado con asiento en la comuna de Peumo, con dos jueces, con competencia sobre las comunas de Peumo y Las Cabras;
    Dos juzgados con asiento en la comuna de San Fernando, con competencia sobre las comunas de San Fernando, Chimbarongo, Placilla y Nancagua, conservando el Segundo Juzgado de Letras de San Fernando competencia especial en materia de menores;
    Un juzgado con asiento en la comuna de Santa Cruz, con competencia sobre las comunas de Santa Cruz, Chépica y Lolol.
    Un juzgado con asiento en la comuna de Pichilemu, con dos jueces, con competencia sobre la misma comuna.
    Un juzgado con asiento en la comuna de Litueche, con competencia sobre las comunas de Navidad, Litueche y La Estrella.
    Un juzgado con asiento en la comuna de Peralillo, con dos jueces, con competencia sobre las comunas de Marchihue, Paredones, Pumanque, Palmilla y Peralillo.


    Art. 34. En la Séptima Región, del Maule, existirán los siguientes juzgados de letras:

    A.- JUZGADOS CIVILES:

    Dos Juzgados con asiento en la comuna de Curicó, con competencia sobre las comunas de Curicó, Teno, Romeral y Rauco, y
    Cuatro Juzgados con asiento en la comuna de Talca, con competencia sobre las comunas de Talca, Pelarco, Río Claro, San Clemente, Maule, Pencahue y San Rafael;

    B.- JUZGADOS DE COMPETENCIA COMÚN:

    Un Juzgado con asiento en la comuna de Constitución, con dos jueces, con competencia sobre las comunas de Constitución y Empedrado;
    Un Juzgado con asiento en la comuna de Curepto, con competencia sobre la misma comuna;
    Un Juzgado con asiento en la comuna de Licantén, con competencia sobre las comunas de Licantén, Hualañé y Vichuquén;
    Un Juzgado con asiento en la comuna de Molina, con dos jueces, con competencia sobre las comunas de Molina y Sagrada Familia;
    Dos Juzgados con asiento en la comuna de Linares, con competencia sobre las comunas de Linares, Yerbas Buenas, Colbún y Longaví;
    Un Juzgado con asiento en la comuna de San Javier, con dos jueces, con competencia sobre las comunas de San Javier y Villa Alegre;
    Un Juzgado con asiento en la comuna de Cauquenes, con dos jueces, con competencia sobre la misma comuna;
    Un Juzgado con asiento en la comuna de Chanco, con competencia sobre las comunas de Chanco y Pelluhue, y
    Un Juzgado con asiento en la comuna de Parral, con competencia sobre las comunas de Parral y Retiro.


    Art. 35. En la Octava Región, del Bío Bío, existirán los siguientes juzgados de letras, que tendrán competencia en los territorios que se indican:

    A.- JUZGADOS CIVILES:

    Tres juzgados de letras en lo civil con asiento en la comuna de Concepción, con competencia sobre las comunas de Concepción, Penco, Hualqui, San Pedro de la Paz y Chiguayante, y
    Dos juzgados de letras en lo civil con asiento en la comuna de Talcahuano, con competencia sobre las comunas de Talcahuano y Hualpén, que tendrán la categoría de juzgados de asiento de Corte para todos los efectos legales.

    B.- JUZGADOS CON COMPETENCIA COMUN:

    Dos juzgados con asiento en la comuna de Los Angeles, con competencia sobre las comunas de Los Angeles, Quilleco y Antuco;
    Un juzgado con asiento en la comuna de Santa Bárbara, con dos jueces, con competencia sobre las comunas de Santa Bárbara, Quilaco y Alto Biobío;
    Un juzgado con asiento en la comuna de Mulchén, con dos jueces, con competencia sobre la comuna de Mulchén;
    Un juzgado con asiento en la comuna de Nacimiento, con dos jueces, con competencia sobre las comunas de Nacimiento y Negrete;
    Un juzgado con asiento en la comuna de Laja, con dos jueces, con competencia sobre las comunas de Laja y San Rosendo;
    Un juzgado con asiento en la comuna de Yumbel, con competencia sobre la misma comuna;
    Un juzgado con asiento en la comuna de Tomé, con competencia sobre la misma comuna;
    Un juzgado con asiento en la comuna de Florida, con competencia sobre la misma comuna;
    Un juzgado con asiento en la comuna de Santa Juana, con competencia sobre la misma comuna,
    Un juzgado con asiento en la comuna de Lota, con competencia sobre la misma comuna;
    Un juzgado con asiento en la comuna de Coronel, con competencia sobre la misma comuna;
    Un juzgado con asiento en la comuna de Lebu, con dos jueces, con competencia sobre las comunas de Lebu y Los Alamos;
    Un juzgado con asiento en la comuna de Arauco, con competencia sobre la misma comuna;
    Un juzgado con asiento en la comuna de Curanilahue, con dos jueces, con competencia sobre la misma comuna;
    Un juzgado con asiento en la comuna de Cañete, con dos jueces, con competencia sobre las comunas de Cañete, Contulmo y Tirúa, y
    Un juzgado con asiento en la comuna de Cabrero, con dos jueces, con competencia sobre la misma comuna.



    Art. 36. En la Novena Región, de la Araucanía, existirán los siguientes juzgados de letras:

    A.- JUZGADOS CIVILES:

    Tres juzgados en lo civil con asiento en la comuna de Temuco, con competencia sobre las comunas de Temuco, Vilcún, Melipeuco, Cunco, Freire y Padre Las Casas.

    B.- JUZGADOS CON COMPETENCIA COMUN:

    Un juzgado con asiento en la comuna de Angol, con competencia sobre las comunas de Angol y Renaico;
    Un juzgado con asiento en la comuna de Purén, con competencia sobre las comunas de Purén y Los Sauces;
    Un juzgado con asiento en la comuna de Collipulli, con dos jueces, con competencia sobre las comunas de Collipulli y Ercilla;
    Un juzgado con asiento en la comuna de Traiguén, con dos jueces, con competencia sobre las comunas de Traiguén y Lumaco;
    Un juzgado con asiento en la comuna de Victoria, con competencia sobre la misma comuna;
    Un juzgado con asiento en la comuna de Curacautín, con competencia sobre las comunas de Curacautín y Lonquimay;
    Un juzgado con asiento en la comuna de Toltén, con competencia sobre la misma comuna;
    Un juzgado con asiento en la comuna de Loncoche, con competencia sobre la misma comuna;
    Un juzgado con asiento en la comuna de Pitrufquén, con dos jueces, con competencia sobre las comunas de Pitrufquén y Gorbea;
    Un juzgado con asiento en la comuna de Villarrica, con dos jueces, con competencia sobre la misma comuna;
    Un juzgado con asiento en la comuna de Nueva Imperial, con dos jueces, con competencia sobre las comunas de Nueva Imperial, Cholchol y Teodoro Schmidt;
    Un juzgado con asiento en la comuna de Pucón, con dos jueces, con competencia sobre las comunas de Pucón y Curarrehue;
    Un juzgado con asiento en la comuna de Lautaro, con dos jueces, con competencia sobre las comunas de lautaro, Perquenco y Galvarino, y
    Un juzgado con asiento en la comuna de Carahue, con dos jueces, con competencia sobre las comunas de Carahue y Saavedra.


    Art. 37. En la Décima Región, de Los Lagos, existirán los siguientes juzgados de letras:

    A.- JUZGADOS CIVILES:

    Dos Juzgados con asiento en la comuna de Puerto Montt con competencia sobre las comunas de Puerto Montt y Cochamó;
    B.- JUZGADOS DE COMPETENCIA COMÚN:

    Tres Juzgados con asiento en la comuna de Osorno con competencia sobre las comunas de Osorno, San Pablo, Puyehue, Puerto Octay y San Juan de la Costa;
    Un Juzgado con asiento en la comuna de Río Negro, con competencia sobre las comunas de Río Negro y Purranque;
    Un Juzgado con asiento en la comuna de Puerto Varas, con competencia sobre las comunas de Puerto Varas, Llanquihue, Frutillar y Fresia;
    Un Juzgado con asiento en la comuna de Calbuco, con dos jueces, con competencia sobre la misma comuna;
    Un Juzgado con asiento en la comuna de Maullín, con competencia sobre la misma comuna;
    Un Juzgado con asiento en la comuna de Los Muermos, con competencia sobre la misma comuna;
    Un Juzgado con asiento en la comuna de Castro, con competencia sobre las comunas de Castro, Chonchi, Dalcahue, Puqueldón y Queilén;
    Un Juzgado con asiento en la comuna de Quellón, con dos jueces, con competencia sobre la misma comuna;
    Un Juzgado con asiento en la comuna de Ancud, con competencia sobre las comunas de Ancud y Quemchi. Este tribunal mantendrá su carácter de juzgado de capital de provincia, para todos los efectos legales, sin perjuicio de la calidad de juzgado de capital de provincia que corresponde al juzgado de Castro;
    Un Juzgado con asiento en la comuna de Quinchao, con competencia sobre las comunas de Quinchao y Curaco de Vélez;
    Un Juzgado con asiento en la comuna de Chaitén, con competencia sobre las comunas de Chaitén, Futaleufú y Palena, y
    Un Juzgado con asiento en la comuna de Hualaihué, con competencia sobre la misma comuna.




NOTA
    El Nº 10 del Art. 4 de la LEY 20252, publicada el 15.02.2008, dispuso la sustitución, en la letra A) del presente artículo, de la frase "Tres juzgados" por "Dos juzgados", sin embargo la referida frase no se encuentra en la letra A), por lo que no ha podido ser incorporada en este texto actualizado. Con todo, se hace presente la observación de que "Tres juzgados" es la frase con que se inicia la letra B.
    Art. 38. En la Décimo Primera Región de Aisén, del General Carlos Ibáñez del Campo, existirán los siguientes juzgados de letras:
    Un juzgado con asiento en la comuna de Coihaique, con competencia sobre las comunas de Coihaique y Río Ibáñez;
    Un juzgado con asiento en la comuna de Aisén, con dos jueces, con competencia sobre la misma comuna;
    Un juzgado con asiento en la comuna de Chile Chico, con competencia sobre la misma comuna,
    Un juzgado con asiento en la comuna de Cochrane, con competencia sobre las comunas de la provincia Capitán Prat, y
    Un juzgado con asiento en la comuna de Cisnes, con competencia sobre las comunas de Cisnes, Guaitecas y Lago Verde.


    Art. 39. En la Décima Segunda Región, de Magallanes y Antártica Chilena, existirán los siguientes juzgados de letras:

    A.- JUZGADOS CIVILES:

    Tres Juzgados con asiento en la comuna de Punta Arenas, con competencia sobre las comunas de la provincia de Magallanes;

    B.- JUZGADOS DE COMPETENCIA COMÚN:

    Un Juzgado con asiento en la comuna de Natales, con dos jueces, con competencia sobre las comunas de la provincia de Última Esperanza.
    Un Juzgado con asiento en la comuna de Porvenir, con competencia sobre las comunas de la provincia de Tierra del Fuego.
    Un Juzgado con asiento en la comuna de Cabo de Hornos, con competencia sobre las comunas de la Provincia de la Antártica Chilena.


    Art. 39 bis. En la Decimocuarta Región, de Los Ríos, existirán los siguientes juzgados de letras:

    A.- JUZGADOS CIVILES:

    Dos juzgados con asiento en la comuna de Valdivia, con competencia sobre las comunas de Valdivia y Corral;

    B.- JUZGADOS CON COMPETENCIA COMUN:

    Un juzgado con asiento en la comuna de Mariquina, con dos jueces, con competencia sobre las comunas de Mariquina, Máfil y Lanco;
    Un juzgado con asiento en la comuna de Los Lagos, con dos jueces, con competencia sobre las comunas de Los Lagos y Futrono;
    Un juzgado con asiento en la comuna de Panguipulli, con dos jueces, con competencia sobre la misma comuna;
    Un juzgado con asiento en la comuna de La Unión, con dos jueces, con competencia sobre la misma comuna;
    Un juzgado con asiento en la comuna de Paillaco, con dos jueces, con jurisdicción sobre la misma comuna, y
    Un juzgado con asiento en la comuna de Río Bueno, con dos jueces, con jurisdicción sobre las comunas de Río Bueno y Lago Ranco.




    Art. 39 ter.- En la Decimoquinta Región, de Arica y Parinacota, existirán los siguientes juzgados de letras:

    A.- JUZGADOS CIVILES:

    Tres juzgados con asiento en la comuna de Arica, con competencia sobre las comunas de las provincias de Arica y Parinacota.

    Artículo 39 quáter.- En la Región de Ñuble existirán los siguientes juzgados de letras, que tendrán competencia en los territorios que se indican:
    A.- JUZGADOS CIVILES:
    Dos juzgados de letras en lo civil, con asiento en la comuna de Chillán, con competencia sobre las comunas de Chillán, Pinto, Coihueco y Chillán Viejo.
    B.- JUZGADOS CON COMPETENCIA COMÚN:
    Un juzgado con asiento en la comuna de San Carlos, con dos jueces, con competencia sobre las comunas de San Carlos, Ñiquén, San Fabián y San Nicolás.
    Un juzgado con asiento en la comuna de Yungay, con dos jueces, con competencia sobre las comunas de Yungay, Pemuco, El Carmen y Tucapel.
    Un juzgado con asiento en la comuna de Bulnes, con dos jueces, con competencia sobre las comunas de Bulnes, Quillón y San Ignacio.
    Un juzgado con asiento en la comuna de Coelemu, con competencia sobre las comunas de Coelemu y Ránquil.
    Un juzgado con asiento en la comuna de Quirihue, con competencia sobre las comunas de Quirihue, Ninhue, Portezuelo, Treguaco y Cobquecura.


    Art. 40. En la Región Metropolitana de Santiago, existirán los siguientes juzgados de letras:

    A.- JUZGADOS CIVILES:

    Treinta juzgados de letras en lo civil, con asiento en la comuna de Santiago, con competencia sobre la provincia de Santiago, con excepción de las comunas de San Joaquín, La Granja, La Pintana, San Ramón, San Miguel, La Cisterna, El Bosque, Pedro Aguirre Cerda y Lo Espejo. Cualquiera fuere la comuna en que estos tribunales tengan su asiento, ellos tendrán la categoría de juzgados de asiento de Corte para todos los efectos legales;
    Cuatro juzgados de letras en lo civil, con competencia sobre las comunas de San Miguel, San Joaquín, La Granja, La Pintana, San Ramón, Pedro Aguirre Cerda, La Cisterna, El Bosque y Lo Espejo. Cualquiera fuere la comuna en que estos tribunales tengan su asiento, ellos tendrán la categoría de juzgados de asiento de Corte para todos los efectos legales.
    Un juzgado de letras en lo civil, con asiento en la comuna de Puente Alto, con competencia sobre las comunas de la provincia de Cordillera.

    B.- JUZGADOS CON COMPETENCIA COMUN:

    Dos juzgados con asiento en la comuna de San Bernardo, con competencia sobre las comunas de San Bernardo y Calera de Tango;
    Dos juzgados con asiento en la comuna de Talagante y competencia sobre las comunas de Talagante, El Monte e Isla de Maipo;
    Un juzgado con asiento en la comuna de Peñaflor, con competencia sobre las comunas de Peñaflor y Padre Hurtado;
    Un juzgado con asiento en la comuna de Melipilla, con dos jueces, con competencia sobre las comunas de la provincia de Melipilla, con excepción de Curacaví, y
    Dos juzgados con asiento en la comuna de Buin, con competencia sobre las comunas de Buin y Paine.
    Un juzgado con asiento en la comuna de Colina, con tres jueces, con competencia sobre las comunas de la Provincia de Chacabuco.


    Art. 41. Suprimido.

    Art. 42. Derogado.


    Art. 43. El Presidente de la República, previo informe favorable de la Corte de Apelaciones que corresponda, podrá fijar como territorio jurisdiccional exclusivo de uno o más de los jueces civiles de la Región Metropolitana de Santiago, una parte de la comuna o agrupación comunal respectiva, y en tal caso, autorizar el funcionamiento de estos Tribunales dentro de sus respectivos territorios jurisdiccionales.
    Los juzgados civiles de la Región Metropolitana de Santiago a los cuales se fije un territorio jurisdiccional exclusivo, podrán practicar, en los asuntos sometidos a su conocimiento, actuaciones en cualesquiera de las comunas que la integran.
    Con el acuerdo previo de la Corte de Apelaciones que corresponda, y por no más de una vez al año, el Presidente de la República podrá modificar los límites de la competencia territorial de los juzgados a que se refiere el inciso primero.


    Art. 44. Derogado.


    Art. 45. Los Jueces de Letras conocerán:

    1° En única instancia:

    a) De las causas civiles cuya cuantía no exceda de 10 Unidades Tributarias Mensuales;
    b) De las causas de comercio cuya cuantía no exceda de 10 Unidades Tributarias Mensuales, y
    c) Suprimido.

    2° En primera instancia:

    a) De las causas civiles y de comercio cuya cuantía exceda de 10 Unidades Tributarias Mensuales;
    b) De las causas de minas, cualquiera que sea su cuantía. Se entiende por causas de minas, aquellas en que se ventilan derechos regidos especialmente por el Código de Minería;
    c) De los actos judiciales no contenciosos, cualquiera sea su cuantía, salvo lo dispuesto en el artículo 494 del Código Civil;
    d) Derogado.
    e) Derogado.
    f) Derogado.
    g) De las causas civiles y de comercio cuya cuantía sea inferior a las señaladas en las letras a) y b), del N° 1 de este artículo, en que sean parte o tengan interés los Comandantes en Jefe del Ejército, de la Armada y de la Fuerza Aérea, el General Director de Carabineros, los Ministros de la Corte Suprema o de alguna Corte de Apelaciones, los Fiscales de estos tribunales, los jueces letrados, los párrocos y vicepárrocos, los cónsules generales, cónsules o vicecónsules de las naciones extranjeras reconocidas por el Presidente de la República, las corporaciones y fundaciones de derecho público o los establecimientos públicos de beneficencia y
    h) De las causas del trabajo y de familia cuyo conocimiento no corresponda a los Juzgados de Letras del Trabajo, de Cobranza Laboral y Previsional o de Familia, respectivamente.

    3° Suprimido.

    4° De los demás asuntos que otras leyes les encomienden.




NOTA
      El artículo 1° N° 6 de la ley 19708, publicada el 05.01.2001, deroga las letras d) y e), no obstante que fueron derogadas previamente por el artículo 11 de la ley 19665.

    Art. 46. Los jueces de letras que cumplan, además de sus funciones propias, las de juez de garantía, tendrán la competencia señalada en el artículo 14 de este Código.

    Art. 47. Tratándose de juzgados de letras que cuenten con un juez y un secretario, las Cortes de Apelaciones podrán ordenar que los jueces se aboquen de un modo exclusivo a la tramitación de una o más materias determinadas, de competencia de su tribunal, cuando hubiere retardo en el despacho de los asuntos sometidos al conocimiento del tribunal o cuando el mejor servicio judicial así lo exigiere.
    La Corporación Administrativa del Poder Judicial informará anualmente a las Cortes de Apelaciones y al Ministerio de Justicia respecto de la aplicación que hubiese tenido el sistema de funcionamiento extraordinario y de las disponibilidades presupuestarias para el año siguiente.

    Art. 47 A. Cuando se iniciare el funcionamiento extraordinario, se entenderá, para todos los efectos legales, que el juez falta en su despacho. En esa oportunidad, el secretario del mismo tribunal asumirá las demás funciones que le corresponden al juez titular, en calidad de suplente, y por el solo ministerio de la ley.
    Quien debiere cumplir las funciones del secretario del tribunal, de acuerdo a las reglas generales, las llevará a efecto respecto del juez titular y de quien lo supliere o reemplazare.
    Art. 47 B. Las atribuciones de las Cortes de Apelaciones previstas en el artículo 47 serán ejercidas por una sala integrada solamente por Ministros titulares.
    Art. 47 C. Tratándose de los tribunales de juicio oral en lo penal, las Cortes de Apelaciones podrán ejercer las potestades señaladas en el artículo 47, ordenando que uno o más de los jueces del tribunal se aboquen en forma exclusiva al conocimiento de las infracciones de los adolescentes a la ley penal, en calidad de jueces de garantía, cuando el mejor servicio judicial así lo exigiere.
    Art. 47 D.- En los Juzgados de Letras en lo Civil, en los Juzgados de Familia, en los Juzgados de Letras del Trabajo, en los Juzgados de Cobranza Laboral y Previsional, en el Juzgado de Letras de Familia, Garantía y Trabajo creado por el artículo 1º de la ley Nº 20.876, y en los Juzgados de Letras con competencia común, a solicitud del juez o del juez presidente, si es el caso, y previo informe de la Corporación Administrativa del Poder Judicial, las Cortes de Apelaciones podrán autorizar, por resolución fundada en razones de buen servicio con el fin de cautelar la eficiencia del sistema judicial para garantizar el acceso a la justicia o la vida o integridad de las personas, la adopción de un sistema de funcionamiento excepcional que habilite al tribunal a realizar de forma remota por videoconferencia las audiencias de su competencia en que no se rinda prueba testimonial, absolución de posiciones o declaración de partes o de peritos. Lo anterior no procederá respecto de las audiencias en materias penales que se realicen en los Juzgados de Letras con competencia común.
    La propuesta de funcionamiento excepcional será elaborada por el secretario o administrador del tribunal, y suscrita por el juez o juez presidente, según corresponda. Dicha propuesta tendrá una duración máxima de un año, la que se podrá prorrogar por una sola vez por el mismo período, sin necesidad de una nueva solicitud.
    El tribunal deberá solicitar a las partes una forma expedita de contacto a efectos de que coordine con ellas los aspectos logísticos necesarios, tales como número de teléfono o correo electrónico. Las partes deberán dar cumplimiento a esta exigencia hasta dos días antes de la realización de la audiencia respectiva. Si cualquiera de las partes no ofreciere oportunamente una forma expedita de contacto, o no fuere posible contactarla a través de los medios ofrecidos tras tres intentos, de lo cual se deberá dejar constancia, se entenderá que no ha comparecido a la audiencia.
    La constatación de la identidad de la parte que comparece de forma remota deberá efectuarse inmediatamente al inicio de la audiencia, de manera remota ante el ministro de fe o el funcionario que determine el tribunal respectivo, mediante la exhibición de su cédula de identidad o pasaporte, de lo que se dejará registro.
    De la audiencia realizada por vía remota mediante videoconferencia en los asuntos civiles y comerciales se levantará acta que consignará todo lo obrado en ella, la que deberá ser suscrita por las partes, el juez y los demás comparecientes, mediante firma electrónica simple o avanzada.
    Sin perjuicio de lo dispuesto en el inciso primero, cualquier persona legitimada a comparecer en la causa podrá solicitar, hasta dos días antes de la realización de la audiencia, que ésta se desarrolle de forma presencial, invocando razones graves que imposibiliten o dificulten su participación, o que por circunstancias particulares, quede en una situación de indefensión.
    La disponibilidad y correcto funcionamiento de los medios tecnológicos de las partes que comparezcan remotamente en dependencias ajenas al Poder Judicial será de su responsabilidad. Con todo, la parte podrá alegar entorpecimiento si el mal funcionamiento de los medios tecnológicos no fuera atribuible a ella. En caso de acoger dicho incidente, el tribunal fijará un nuevo día y hora para la continuación de la audiencia, sin que se pierda lo obrado con anterioridad a dicho mal funcionamiento. En la nueva audiencia que se fije, el tribunal velará por la igualdad de las partes en el ejercicio de sus derechos.
    La Corte Suprema regulará mediante auto acordado los criterios que las Cortes de Apelaciones deberán tener a la vista para aprobar este tipo de funcionamiento excepcional.



NOTA
      El inciso segundo del artículo duodécimo transitorio de la ley 21394, publicada el 30.11.2021, dispone que, durante el periodo de un año contado desde la entrada en vigencia señalada en el inciso primero de la citada norma transitoria, las disposiciones contenidas en el presente artículo regirán en los tiempos y territorios en que las disposiciones del artículo decimosexto transitorio no fueren aplicables, de conformidad a la extensión temporal o territorial que conforme a dicho artículo disponga la Corte Suprema.

    Art. 48. Los jueces de letras de comunas asiento de Corte conocerán en primera instancia de las causas de hacienda, cualquiera que sea su cuantía.
    No obstante lo dispuesto en el inciso anterior, en los juicios en que el Fisco obre como demandante, podrá éste ocurrir a los tribunales allí indicados o al del domicilio del demandado, cualquiera que sea la naturaleza de la acción deducida.
    Las mismas reglas se aplicarán a los asuntos no contenciosos en que el Fisco tenga interés.


    Art. 49. Derogado.


    TITULO IV

    De los Presidentes y Ministros de Corte como tribunales unipersonales


    Art. 50. Un Ministro de la Corte de Apelaciones respectiva, según el turno que ella fije, conocerá en primera instancia de los siguientes asuntos:

    1°) Derogado.
    2°) De las causas civiles en que sean parte o tengan interés el Presidente de la República, los ex Presidentes de la República, los Ministros de Estado, Senadores, Diputados, miembros de los Tribunales Superiores de Justicia, Contralor General de la República, Comandantes en Jefe de las Fuerzas Armadas, General Director de Carabineros de Chile, Director General de la Policía de Investigaciones de Chile, los Delegados Presidenciales Regionales, Delegados Presidenciales Provinciales, Gobernadores Regionales, los Agentes Diplomáticos chilenos, los Embajadores y los Ministros Diplomáticos acreditados con el Gobierno de la República o en tránsito por su territorio, los Arzobispos, los Obispos, los Vicarios Generales, los Provisores y los Vicarios Capitulares.
    La circunstancia de ser accionista de sociedades anónimas las personas designadas en este número, no se considerará como una causa suficiente para que un Ministro de la Corte de Apelaciones conozca en primera instancia de los juicios en que aquéllas tengan parte, debiendo éstos sujetarse en su conocimiento a las reglas generales.
    3°) Derogado.
    4°) De las demandas civiles que se entablen contra los jueces de letras para hacer efectiva la responsabilidad civil resultante del ejercicio de sus funciones ministeriales.
    5°) De los demás asuntos que otras leyes les encomienden.



    Art. 51. El Presidente de la Corte de Apelaciones de Santiago conocerá en primera instancia:

    1°) De las causas sobre amovilidad de los Ministros de la Corte Suprema; y
    2°) De las demandas civiles que se entablen contra uno o más miembros de la Corte Suprema o contra su fiscal judicial para hacer efectiva su responsabilidad por actos cometidos en el desempeño de sus funciones.



    Art. 52. Un Ministro de la Corte Suprema, designado por el Tribunal, conocerá en primera instancia:

    1°) De las causas a que se refiere el artículo 23, de la ley N° 12.033;
    2°) De los delitos de jurisdicción de los tribunales chilenos, cuando puedan afectar las relaciones internacionales de la República con otro Estado.
    3°) De la extradición pasiva.
    4°) De los demás asuntos que otras leyes le encomienden.



    Art. 53. El Presidente de la Corte Suprema conocerá en primera instancia:

    1°) De las causas sobre amovilidad de los Ministros de las Cortes de Apelaciones;
    2°) De las demandas civiles que se entablen contra uno o más miembros o fiscales judiciales de las Cortes de Apelaciones para hacer efectiva su responsabilidad por actos cometidos en el desempeño de sus funciones;
    3°) De las causas de presas y demás que deban juzgarse con arreglo al Derecho Internacional; y
    4°) De los demás asuntos que otras leyes entreguen a su conocimiento.

    En estas causas no procederán los recursos de casación en la forma ni en el fondo en contra de la sentencia dictada por la sala que conozca del recurso de apelación que se interpusiere en contra de la resolución del Presidente.



    TITULO V

    Las Cortes de Apelaciones


    § 1. Su organización y atribuciones


    Art. 54. Habrá en la República diecisiete Cortes de Apelaciones, las que tendrán su asiento en las siguientes comunas: Arica, Iquique, Antofagasta, Copiapó, La Serena, Valparaíso, Santiago, San Miguel, Rancagua, Talca, Chillán, Concepción, Temuco, Valdivia, Puerto Montt, Coihaique y Punta Arenas.



    Art. 55. El territorio jurisdiccional de las Cortes de Apelaciones será el siguiente:

    a) El de la Corte de Arica comprenderá la Decimoquinta Región de Arica y Parinacota;
    b) El de la Corte de Iquique comprenderá la Primera Región de Tarapacá;
    c) El de la Corte de Antofagasta comprenderá la Segunda Región de Antofagasta;
    d) El de la Corte de Copiapó comprenderá la Tercera Región de Atacama;
    e) El de la Corte de la Serena comprenderá la Cuarta Región de Coquimbo;
    f) El de la Corte de Valparaíso comprenderá la Quinta Región de Valparaíso;
    g) El de la Corte de Santiago comprenderá la parte de la Región Metropolitana de Santiago correspondiente a las provincias de Chacabuco y de Santiago, con exclusión de las comunas de Lo Espejo, San Miguel, San Joaquín, La Cisterna, San Ramón, La Granja, El Bosque, La Pintana y Pedro Aguirre Cerda;
    h) El de la Corte de San Miguel comprenderá la parte de la Región Metropolitana de Santiago correspondiente a las provincias de Cordillera, Maipo y Talagante; a la provincia de Melipilla; a las comunas de Lo Espejo, San Miguel, San Joaquín, La Cisterna, San Ramón, La Granja, El Bosque, La Pintana y Pedro Aguirre Cerda, de la provincia de Santiago;
    i) El de la Corte de Rancagua comprenderá la Sexta Región, del libertador General Bernardo O'Higgins;
    j) El de la Corte de Talca comprenderá el de la Séptima Región, del Maule;
    k) El de la Corte de Chillán comprenderá la Decimosexta Región, de Ñuble y la comuna de Tucapel, de la Provincia del Biobío de la Octava Región del Biobío;
    l) El de la Corte de Concepción comprenderá las provincias de Concepción, Arauco y Biobío, de la Región del Biobío, con excepción de la comuna de Tucapel;
    m) El de la Corte de Temuco comprenderá la Novena Región, de la Araucanía;
    n) El de la Corte de Valdivia comprenderá la Decimocuarta Región de Los Ríos, y la provincia de Osorno, de la Décima Región de Los Lagos;
    o) El de la Corte de Puerto Montt comprenderá las provincias de Llanquihue, Chiloé y Palena, de la Décima Región de Los Lagos;
    p) El de la Corte de Coihaique comprenderá la Décimo Primera Región de Aisén, del General Carlos Ibáñez del Campo, y
    q) El de la Corte de Punta Arenas comprenderá la Décimo Segunda Región de Magallanes y de la Antártica Chilena.




    Art. 56. Las Cortes de Apelaciones se compondrán del número de miembros que a continuación se indica:
    1º. Las Cortes de Apelaciones de Iquique, Copiapó, Chillán, Puerto Montt, Coihaique y Punta Arenas tendrán cuatro miembros;
    2º. Las Cortes de Apelaciones de Arica, Antofagasta, La Serena, Rancagua, Talca, Temuco y Valdivia tendrán siete miembros;
    3º. La Corte de Apelaciones de Valparaíso tendrá dieciséis miembros;
    4º. Las Cortes de Apelaciones de San Miguel y Concepción tendrán diecinueve miembros, y
    5º. La Corte de Apelaciones de Santiago tendrá treinta y cuatro miembros.



    Art. 57. Las Cortes de Apelaciones serán regidas por un Presidente. Sus funciones durarán un año contado del 1° de marzo y serán desempeñadas por los miembros del tribunal, turnándose cada uno por orden de antigüedad en la categoría correspondiente del escalafón.
    Los demás miembros de las Cortes de Apelaciones se llamarán Ministros y tendrán el rango y precedencia correspondientes a su antigüedad en la categoría correspondiente del escalafón.



    Art. 58. La Corte de Apelaciones de Santiago tendrá seis fiscales judiciales; la Corte de Apelaciones de San Miguel tendrá cuatro fiscales judiciales; las Cortes de Apelaciones de Valparaíso y Concepción tendrán tres fiscales judiciales; las Cortes de Apelaciones de Antofagasta, La Serena, Rancagua, Talca, Temuco y Valdivia tendrán dos fiscales judiciales. Las demás Cortes de Apelaciones tendrán un fiscal judicial cada una. El ejercicio de sus funciones será reglado por el tribunal como lo estime conveniente para el mejor servicio, con audiencia de estos funcionarios.


    Art. 59. Las Cortes de Apelaciones tendrán el número de relatores que a continuación se indica:

    1º. La Corte de Apelaciones de Chillán tendrá dos relatores;
    2º. Las Cortes de Apelaciones de Iquique, Copiapó, Puerto Montt, Coyhaique y Punta Arenas tendrán tres relatores;
    3º. Las Cortes de Apelaciones de Arica, Antofagasta, La Serena, Rancagua, Talca, Temuco y Valdivia tendrán cinco relatores;
    4º. Las Cortes de Apelaciones de Valparaíso y Concepción, tendrán once relatores;
    5º. La Corte de Apelaciones de San Miguel tendrá doce relatores, y
    6º. La Corte de Apelaciones de Santiago tendrá veintitrés relatores.



    Art. 60. Cada Corte de Apelaciones tendrá un secretario.
    La Corte de Apelaciones de San Miguel tendrá dos secretarios. La Corte de Apelaciones de Santiago tendrá tres secretarios. Cada tribunal reglará el ejercicio de las funciones de sus secretarios y distribuirá entre ellos el despacho de los asuntos que ingresen a la Corte, en la forma que estime más conveniente para el buen servicio.


    Art. 61. Las Cortes de Apelaciones de Arica, Antofagasta, La Serena, Rancagua, Talca, Temuco y Valdivia se dividirán en dos salas; la Corte de Apelaciones de Valparaíso, en cinco salas; las Cortes de Apelaciones de Concepción y San Miguel, en seis salas, y la Corte de Apelaciones de Santiago en diez salas. Cada una de las salas en que se dividan ordinariamente las Cortes de Apelaciones, tendrán tres ministros, a excepción de la primera sala que constará de cuatro. Para la constitución de las diversas salas en que se dividan las Cortes de Apelaciones para su funcionamiento ordinario, se sortearán anualmente los miembros del tribunal, con excepción de su Presidente, el que quedará incorporado a la Primera Sala, siendo facultativo para él integrarla. El sorteo correspondiente se efectuará el primer día hábil de diciembre del año anterior a aquel en que hayan de funcionar las salas en cada Corte de Apelaciones.
    No obstante, para los efectos de lo dispuesto en los incisos séptimo y noveno del artículo 66, las Cortes de Apelaciones designarán cada dos años, mediante auto acordado, a los miembros del tribunal que deberán integrar la sala a la que corresponda el conocimiento, en forma exclusiva o preferente, de los asuntos tributarios y aduaneros. Se preferirá para su integración a aquellos ministros que posean conocimientos especializados en estas materias, salvo en el caso del inciso séptimo del referido artículo, en el que los ministros deberán necesariamente poseer dichos conocimientos.
    Para la acreditación de los conocimientos especializados a que se refiere el inciso anterior, se deberá contar con cursos de perfeccionamiento o postgrado sobre la materia.



    Art. 62. Las Cortes de Apelaciones integradas por sus fiscales judiciales o con abogados integrantes, se dividirán en salas de tres miembros para el despacho de las causas, cuando hubiere retardo.
    Se entenderá que hay retardo cuando dividido el total de causas en estado de tabla y de las apelaciones que deban conocerse en cuenta, inclusive las criminales, por el número de salas, el cuociente fuere superior a ciento.
    Producido este caso y si no bastaren los relatores en propiedad, el tribunal designará por mayoría de votos los relatores interinos que estime conveniente, quienes gozarán durante el tiempo en que sirvieren de igual remuneración que los propietarios.



    Art. 63. Las Cortes de Apelaciones conocerán:

    1º En única instancia:

    a) De los recursos de casación en la forma que se interpongan en contra de las sentencias dictadas por los jueces de letras de su territorio jurisdiccional o por uno de sus ministros, y de las sentencias definitivas de primera instancia dictadas por jueces árbitros.
    b) De los recursos de nulidad interpuestos en contra de las sentencias definitivas dictadas por un tribunal con competencia en lo criminal, cuando corresponda de acuerdo a la ley procesal penal;
    c) De los recursos de queja que se deduzcan en contra de jueces de letras, jueces de policía local, jueces árbitros y órganos que ejerzan jurisdicción, dentro de su territorio jurisdiccional;
    d) De la extradición activa, y
    e) De las solicitudes que se formulen, de conformidad a la ley procesal, para declarar si concurren las circunstancias que habilitan a la autoridad requerida para negarse a proporcionar determinada información, siempre que la razón invocada no fuere que la publicidad pudiere afectar la seguridad nacional.

    2º En primera instancia:

    a) De los desafueros de las personas a quienes les fueren aplicables los incisos segundo, tercero y cuarto del artículo 58 de la Constitución Política;
    b) De los recursos de amparo y protección, y
    c) De los procesos por amovilidad que se entablen en contra de los jueces de letras, y
    d) De las querellas de capítulos.

    3º En segunda instancia:

    a) De las causas civiles, de familia y del trabajo y de los actos no contenciosos de que hayan conocido en primera los jueces de letras de su territorio jurisdiccional o uno de sus ministros, y
    b) De las apelaciones interpuestas en contra de las resoluciones dictadas por un juez de garantía.

    4º De las consultas de las sentencias civiles dictadas por los jueces de letras.

    5º De los demás asuntos que otras leyes les encomienden.



    Art. 64. La Corte de Santiago conocerá de los recursos de apelación y de casación en la forma que incidan en las causas de que haya conocido en primera instancia su Presidente.


    Art. 65. Suprimido.

    Art. 66. El conocimiento de todos los asuntos entregados a la competencia de las Cortes de Apelaciones pertenecerá a las salas en que estén divididas, a menos que la ley disponga expresamente que deban conocer de ellos en Pleno.
    Cada sala representa a la Corte en los asuntos de que conoce.
    En caso que ante una misma Corte de Apelaciones se encuentren pendientes distintos recursos de carácter jurisdiccional que incidan en una misma causa, cualesquiera sea su naturaleza, éstos deberán acumularse y verse conjunta y simultáneamente en una misma sala. La acumulación deberá hacerse de oficio, sin perjuicio del derecho de las partes a requerir el cumplimiento de esta norma. En caso que, además de haberse interpuesto recursos jurisdiccionales, se haya deducido recurso de queja, éste se acumulará a los recursos jurisdiccionales, y deberá resolverse conjuntamente con ellos.
    Corresponderá a todo el tribunal el ejercicio de las facultades disciplinarias, administrativas y económicas, sin perjuicio de que las salas puedan ejercer las primeras en los casos de los artículos 542 y 543 en los asuntos que estén conociendo. También corresponderá a todo el tribunal el conocimiento de los desafueros de los Diputados y de los Senadores y de los juicios de amovilidad en contra de los jueces de letras.
    No obstante lo dispuesto en el inciso anterior, los recursos de queja serán conocidos y fallados por las salas del tribunal, según la distribución que de ellos haga el Presidente; pero la aplicación de medidas disciplinarias corresponderá al tribunal pleno.
    La Corte de Apelaciones de Santiago conocerá en pleno de los recursos de apelación y casación en la forma, en su caso, que incidan en los juicios de amovilidad y en las demandas civiles contra los ministros y el fiscal judicial de la Corte Suprema.
    La Corte de Apelaciones de Santiago designará una de sus salas para que conozca exclusivamente de los asuntos tributarios y aduaneros que se promuevan. Dicha designación se efectuará mediante auto acordado que se dictará cada dos años.
    En las demás Cortes de Apelaciones, el Presidente designará una sala para que conozca en forma preferente de esta materia en uno o más días a la semana.
    El relator que se designare para las salas a que se hace referencia en los incisos precedentes, deberá contar con especialización en materias tributarias y aduaneras, la que deberá acreditarse preferentemente sobre la base de la participación en cursos de perfeccionamiento y postgrado u otra forma mediante la cual se demuestre tener conocimientos relevantes en dichas materias.



    Art. 67. Para el funcionamiento del tribunal pleno se requerirá, a lo menos, la concurrencia de la mayoría absoluta de los miembros de que se componga la Corte.
    Las salas no podrán funcionar sin la concurrencia de tres jueces como mínimum.


    Art. 68. Las Cortes de Apelaciones resolverán los asuntos en cuenta o previa vista de ellos, según corresponda.
    Art. 68 bis. Las Cortes de Apelaciones podrán autorizar, por resolución fundada en razones de buen servicio a fin de cautelar la eficiencia del sistema judicial para garantizar el acceso a la justicia o la vida o integridad de las personas, la adopción de un sistema de funcionamiento excepcional que las habilite a realizar la vista de las causas sometidas a su conocimiento en forma remota por videoconferencia. La propuesta de funcionamiento excepcional será elaborada por el presidente de la Corte respectiva y deberá ser aprobada por el pleno. Dicha propuesta tendrá una duración máxima de un año, la que se podrá prorrogar por una sola vez por el mismo período, sin necesidad de una nueva solicitud.
    En este caso, tendrá aplicación lo dispuesto en los artículos 223 y 223 bis del Código de Procedimiento Civil.
    Con todo, cualquiera de las partes podrá solicitar, hasta las 12:00 horas del día anterior a la vista de la causa, que esta se desarrolle de forma presencial, invocando razones graves que imposibiliten o dificulten su participación, o que por circunstancias particulares, quede en una situación de indefensión.
    La Corte Suprema regulará mediante auto acordado los criterios que las Cortes de Apelaciones deberán tener a la vista para aprobar este tipo de funcionamiento excepcional.



NOTA
      El inciso segundo del artículo duodécimo transitorio de la ley 21394, publicada el 30.11.2021, dispone que, durante el periodo de un año contado desde la entrada en vigencia señalada en el inciso primero de la citada norma transitoria, las disposiciones contenidas en el presente artículo regirán en los tiempos y territorios en que las disposiciones del artículo decimosexto transitorio no fueren aplicables, de conformidad a la extensión temporal o territorial que conforme a dicho artículo disponga la Corte Suprema.

    Art. 69. Los Presidentes de las Cortes de Apelaciones formarán el último día hábil de cada semana una tabla de los asuntos que verá el tribunal en la semana siguiente, que se encuentren en estado de relación. Se consideran expedientes en estado de relación aquellos que hayan sido previamente revisados y certificados al efecto por el relator que corresponda.
    En las Cortes de Apelaciones que consten de más de una sala se formarán tantas tablas cuantas sea el número de salas y se distribuirán entre ellas por sorteo, en audiencia pública. Sin perjuicio de lo anterior, los asuntos que según la materia deban ser conocidos por las salas a que se refieren los incisos séptimo y octavo del artículo 66, serán asignados a éstas por el Presidente del tribunal, quien lo determinará sin ulterior recurso.
    En las tablas deberá designarse un día de la semana para conocer las causas criminales y otro día distinto para conocer las causas de familia, sin perjuicio de la preferencia que la ley o el tribunal les acuerden.
    Sin embargo, los recursos de amparo y las apelaciones relativas a la libertad de los imputados u otras medidas cautelares personales en su contra serán de competencia de la sala que haya conocido por primera vez del recurso o de la apelación, o que hubiere sido designada para tal efecto, aunque no hubiere entrado a conocerlos.
    Serán agregados extraordinariamente a la tabla del día siguiente hábil al de su ingreso al tribunal, o el mismo día, en casos urgentes:

    1º Las apelaciones relativas a la prisión preventiva de los imputados u otras medidas cautelares personales en su contra;
    2º Los recursos de amparo, y
    3º Las demás que determinen las leyes.
   
    Serán agregados extraordinariamente a la tabla del día siguiente hábil al de su ingreso al tribunal, o el mismo día, en casos urgentes: 1° las apelaciones y consultas relativas a la libertad provisional de los inculpados y procesados; 2° los recursos de amparo; y 3° las demás que determinen las leyes.
    Se agregarán extraordinariamente, también, las apelaciones de las resoluciones relativas al auto de procesamiento señaladas en el inciso cuarto, en causas en que haya procesados privados de libertad. La agregación se hará a la tabla del día que determine el Presidente de la Corte, dentro del término de cinco días desde el ingreso de los autos a la Secretaría del Tribunal.



    Art. 70. La tramitación de los asuntos entregados a las Cortes de Apelaciones corresponderá, en aquellas que se compongan de más de una sala, a la primera.
    Para dictar las providencias de mera sustanciación bastará un solo ministro.
    Se entienden por providencias de mera sustanciación las que tienen por objeto dar curso progresivo a los autos, sin decidir ni prejuzgar ninguna cuestión debatida entre partes.
    Sin embargo, deberán dictarse por la sala respectiva las resoluciones de tramitación que procedan cuando ya estén conociendo de un asunto.



    Art. 71. La vista y conocimiento en cuenta de las causas y asuntos incidentales de las Cortes de Apelaciones, se regirán por las reglas de los Códigos de Procedimiento Civil y de Procedimiento Penal o Procesal Penal, según corresponda.




    § 2. Los Acuerdos de las Cortes de Apelaciones


    Art. 72. Las Cortes de Apelaciones deberán funcionar, para conocer y decidir los asuntos que les estén encomendados, con un número de miembros que no sea inferior al mínimum determinado en cada caso por la ley, y sus resoluciones se adoptarán por mayoría absoluta de votos conformes.
    Art. 73. Derogado.



    Art. 74. Si con ocasión de conocer alguna causa en materia criminal, se produce una dispersión de votos entre los miembros de la Corte, se seguirá las reglas señaladas para los tribunales de juicio oral en lo penal.


    Art. 75. No podrán tomar parte en ningún acuerdo los que no hubieren concurrido como jueces a la vista del negocio.