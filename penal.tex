CARLOS RIESCO.-M. E. BALLESTEROS.-RAMON C. BRISEÑO.


    EL PRESIDENTE DE LA REPÚBLICA.

    Santiago, noviembre 12 de 1874.

    Por cuanto el Congreso Nacional ha aprobado el siguiente

    CÓDIGO PENAL.

    LIBRO PRIMERO





    TÍTULO PRIMERO.

DE LOS DELITOS Y DE LAS CIRCUNSTANCIAS QUE EXIMEN DE RESPONSABILIDAD CRIMINAL, LA ATENÚAN O LA AGRAVAN.





    § I.

    De los delitos.



    ARTÍCULO 1.

    Es delito toda acción u omisión voluntaria penada por la ley.
    Las acciones u omisiones penadas por la ley se reputan siempre voluntarias, a no ser que conste lo contrario.

    El que cometiere delito será responsable de él e incurrirá en la pena que la ley señale, aunque el mal recaiga sobre persona distinta de aquella a quien se proponía ofender. En tal caso no se tomarán en consideración las circunstancias, no conocidas por el delincuente, que agravarían su responsabilidad; pero sí aquellas que la atenúen.


    ART. 2.

    Las acciones u omisiones que cometidas con dolo o malicia importarían un delito, constituyen cuasidelito si sólo hay culpa en el que las comete.



    ART. 3.

    Los delitos, atendida su gravedad, se dividen en crímenes, simples delitos y faltas y se califican de tales según la pena que les está asignada en la escala general del art. 21.


    ART. 4.

    La división de los delitos es aplicable a los cuasidelitos, que se califican y penan en los casos especiales que determina este Código.


    ART. 5.

    La ley penal chilena es obligatoria para todos los habitantes de la República, inclusos los extranjeros. Los delitos cometidos dentro del mar territorial o adyacente quedan sometidos a las prescripciones de este Código.


    ART. 6.

    Los crímenes o simples delitos perpetrados fuera del territorio de la República por chilenos o por extranjeros, no serán castigados en Chile sino en los casos determinados por la ley.


    ART. 7.

    Son punibles, no sólo el crimen o simple delito consumado, sino el frustrado y la tentativa.
    Hay crimen o simple delito frustrado cuando el delincuente pone de su parte todo lo necesario para que el crimen o simple delito se consume y esto no se verifica por causas independientes de su voluntad.
    Hay tentativa cuando el culpable da principio a la ejecución del crimen o simple delito por hechos directos, pero faltan uno o más para su complemento.



    ART. 8.

    La conspiración y proposición para cometer un crimen o un simple delito, sólo son punibles en los casos en que la ley las pena especialmente.
    La conspiración existe cuando dos o más personas se conciertan para la ejecución del crimen o simple delito.
    La proposición se verifica cuando el que ha resuelto cometer un crimen o un simple delito, propone su ejecución a otra u otras personas.
    Exime de toda pena por la conspiración o proposición para cometer un crimen o un simple delito, el desistimiento de la ejecución de éstos antes de principiar a ponerlos por obra y de iniciarse procedimiento judicial contra el culpable, con tal que denuncie a la autoridad pública el plan y sus circunstancias.




    ART. 9.

    Las faltas sólo se castigan cuando han sido consumadas.



    § II.

    De las circunstancias que eximen de responsabilidad criminal.


    ART. 10.

    Están exentos de responsabilidad criminal:

    1.° El loco o demente, a no ser que haya obrado en un intervalo lúcido, y el que, por cualquier causa independiente de su voluntad, se halla privado totalmente de razón.
    Inciso Derogado.
    Inciso Derogado.
    2.º El menor de dieciocho años. La responsabilidad de los menores de dieciocho años y mayores de catorce se regulará por lo dispuesto en la ley de responsabilidad penal juvenil.
    3.° Derogado.
    4.° El que obra en defensa de su persona o derechos, siempre que concurran las circunstancias siguientes:
    Primera.-Agresión Ilegítima.
    Segunda.- Necesidad racional del medio empleado para impedirla o repelerla.
    Tercera.-Falta de provocación suficiente por parte del que se defiende.
    Inciso Derogado.
    5.° El que obra en defensa de la persona o derechos de su cónyuge, de su conviviente civil, de sus parientes consanguíneos en toda la línea recta y en la colateral hasta el cuarto grado, de sus afines en toda la línea recta y en la colateral hasta el segundo grado, de sus padres o hijos, siempre que concurran la primera y segunda circunstancias prescritas en el número anterior, y la de que, en caso de haber precedido provocación de parte del acometido, no tuviere participación en ella el defensor.
    6.° El que obra en defensa de la persona y derechos de un extraño, siempre que concurran las circunstancias expresadas en el número anterior y la de que el defensor no sea impulsado por venganza, resentimiento u otro motivo ilegítimo.
    Se presumirá legalmente que concurren las circunstancias previstas en este número y en los números 4° y 5° precedentes, cualquiera que sea el daño que se ocasione al agresor, respecto de aquel que rechaza el escalamiento en los términos indicados en el número 1° del artículo 440 de este Código, en una casa, departamento u oficina habitados, o en sus dependencias o, si es de noche, en un local comercial o industrial y del que impida o trate de impedir la consumación de los delitos señalados en los artículos 141, 142, 361, 362, 365 bis, 390, 391, 433 y 436 de este Código.
    Se presumirá legalmente que concurren las circunstancias previstas en los números 4°, 5° y 6° de este artículo, respecto de las Fuerzas de Orden y Seguridad Pública, Gendarmería de Chile, las Fuerzas Armadas y los servicios bajo su dependencia, cuando éstas realicen funciones de orden público y seguridad pública interior; en dichos casos se entenderá que concurre el uso racional del medio empleado si, en razón de su cargo o con motivo u ocasión del cumplimiento de funciones de resguardo de orden público y seguridad pública interior, repele o impide una agresión que pueda afectar gravemente su integridad física o su vida o las de un tercero, empleando las armas o cualquier otro medio de defensa.
    Los numerales 4°, 5° y 6° se aplicarán respecto de los funcionarios de las Fuerzas de Orden y Seguridad Pública, Gendarmería de Chile, las Fuerzas Armadas y los servicios bajo su dependencia, cuando éstas realicen funciones de orden público y seguridad pública interior ante agresiones contra las personas. De afectarse exclusivamente bienes, procederá la aplicación del número 10° del presente artículo.
    Esta norma se utilizará con preferencia a lo establecido en el artículo 410 del Código de Justicia Militar.
    Respecto de lo dispuesto en los párrafos anteriores, los tribunales, según las circunstancias y si éstas demuestran que no había necesidad racional de usar el arma de servicio o armamento menos letal en toda la extensión que aparezca, deberán considerar esta circunstancia como atenuante de la responsabilidad y rebajar la pena en uno, dos o tres grados, salvo que concurra dolo.
    7.°  El que para evitar un mal ejecuta un hecho, que produzca daño en la propiedad ajena, siempre que concurran las circunstancias siguientes:
    Primera.-Realidad o peligro inminente del mal que se trata de evitar.
    Segunda.-Que sea mayor que el causado para evitarlo.
    Tercera.-Que no haya otro medio practicable y menos perjudicial para impedirlo.
    8.°  El que con ocasión de ejecutar un acto lícito, con la debida diligencia, causa un mal por mero accidente.
    9.°  El que obra violentado por una fuerza irresistible o impulsado por un miedo insuperable.
    10.° El que obra en cumplimiento de un deber o en el ejercicio legítimo de un derecho, autoridad, oficio o cargo.
    11.° El que obra para evitar un mal grave para su persona o derecho o los de un tercero, siempre que concurran las circunstancias siguientes:

    1ª. Actualidad o inminencia del mal que se trata de evitar.
    2ª. Que no exista otro medio practicable y menos perjudicial para evitarlo.
    3ª. Que el mal causado no sea sustancialmente superior al que se evita.
    4ª. Que el sacrificio del bien amenazado por el mal no pueda ser razonablemente exigido al que lo aparta de sí o, en su caso, a aquel de quien se lo aparta siempre que ello estuviese o pudiese estar en conocimiento del que actúa.
    12.° El que incurre en alguna omisión, hallándose impedido por causa legítima o insuperable.
    13.° El que cometiere un cuasidelito, salvo en los casos expresamente penados por la ley.






    § III.

    De las circunstancias que atenúan la responsabilidad criminal.


    ART. 11.

    Son circunstancias atenuantes:

    1.° Las expresadas en el artículo anterior, cuando no concurren todos los requisitos necesarios para eximir de responsabilidad en sus respectivos casos.
    2.° Derogado.
    3.° La de haber precedido inmediatamente de parte del ofendido, provocación o amenaza proporcionada al delito.
    4.° La de haberse ejecutado el hecho en vindicación próxima de una ofensa grave causada al autor, a su cónyuge, o su conviviente, a sus parientes legítimos por consanguinidad o afinidad en toda la línea recta y en la colateral hasta el segundo grado inclusive, a sus padres o hijos naturales o ilegítimos reconocidos.
    5.°  La de obrar por estímulos tan poderosos que naturalmente hayan producido arrebato y obcecación.
    6.°  Si la conducta anterior del delincuente ha sido irreprochable.
    7.°  Si ha procurado con celo reparar el mal causado o impedir sus ulteriores perniciosas consecuencias.
    8.°  Si pudiendo eludir la acción de la justicia por medio de la fuga u ocultándose, se ha denunciado y confesado el delito.
    9ª. Si se ha colaborado sustancialmente al esclarecimiento de los hechos.
    10.° El haber obrado por celo de la justicia.




    § IV.

    De las circunstancias que agravan la responsabilidad criminal.


    ART. 12.

    Son circunstancias agravantes:
    1.° Cometer el delito contra las personas con alevosía, entendiéndose que la hay cuando se obra a traición o sobre seguro.
    2.° Cometerlo mediante precio, recompensa o promesa.
    3.° Ejecutar el delito por medio de inundación, incendio, veneno u otro artificio que pueda ocasionar grandes estragos o dañar a otras personas.
    4.° Aumentar deliberadamente el mal del delito causando otros males innecesarios para su ejecución.
    5.° En los delitos contra las personas, obrar con premeditación conocida o emplear astucia, fraude o disfraz.
    6.° Abusar el delincuente de la superioridad de su sexo o de sus fuerzas, en términos que el ofendido no pudiera defenderse con probabilidades de repeler la ofensa.
    7.° Cometer el delito con abuso de confianza.
    8.° Prevalerse del carácter público que tenga el culpable.
    9.° Emplear medios o hacer que concurran circunstancias que añadan la ignominia a los efectos propios del hecho.
    10.° Cometer el delito con ocasión de incendio, naufragio, sedición, tumulto o conmoción popular u otra calamidad o desgracia.
    11.° Ejecutarlo con auxilio de gente armada o de personas que aseguren o proporcionen la impunidad.
    12.° Ejecutarlo de noche o en despoblado.
    El tribunal tomará o no en consideración esta circunstancia, según la naturaleza y accidentes del delito.
    13.° Ejecutarlo en desprecio o con ofensa de la autoridad pública o en el lugar en que se halle ejerciendo sus funciones.
    14.° Cometer el delito mientras cumple una condena o después de haberla quebrantado y dentro del plazo en que puede ser castigado por el quebrantamiento.
    15.° Haber sido condenado el culpable anteriormente por delitos a que la ley señale igual o mayor pena.
    16 ª Haber sido condenado el culpable anteriormente por delito de la misma especie.
    17.° Cometer el delito en lugar destinado al ejercicio de un culto permitido en la República.
    18.° Ejecutar el hecho con ofensa o desprecio del respeto que por la dignidad, autoridad, edad o sexo mereciere el ofendido, o en su morada, cuando él no haya provocado el suceso.
    19.° Ejecutarlo por medio de fractura o escalamiento de lugar cerrado.
    20.° Ejecutarlo portando armas de aquellas referidas en el artículo 132.
    21ª. Cometer el delito o participar en él motivado por la ideología, opinión política, religión o creencias de la víctima; la nación, raza, etnia o grupo social a que pertenezca; su sexo, orientación sexual, identidad de género, edad, filiación, apariencia personal o la enfermedad o discapacidad que padezca.
    22.° Cometer el delito contra una víctima menor de 18 años, un adulto mayor o una persona con discapacidad, en los términos de la ley N° 20.422, que establece normas sobre igualdad de oportunidades e inclusión social de personas con discapacidad.
    23ª. Ejecutar el hecho formando parte de una agrupación u organización de dos o más personas destinada a cometer crímenes o simples delitos, siempre que ésta o aquélla no constituya una asociación delictiva o criminal de que trata el Párrafo 10 del Título VI del Libro II, y ello ha facilitado la perpetración del delito o ha aumentado el peligro para la integridad física de la víctima, o haber ejecutado el hecho con violencia, intimidación o engaño.


      § V.

De las circunstancias que atenúan o agravan la responsabilidad criminal, según la naturaleza y accidentes del delito.



    ART. 13.

    Es circunstancia atenuante o agravante, según la naturaleza y accidentes del delito.
    Ser el agraviado cónyuge o conviviente civil, pariente por consanguinidad o afinidad en toda la línea recta y en la colateral hasta el segundo grado, padre o hijo del ofensor.



    TÍTULO SEGUNDO.

    DE LAS PERSONAS RESPONSABLES DE LOS DELITOS.


    ART. 14.

    Son responsables criminalmente de los delitos:
    1.° Los autores.
    2.° Los cómplices.
    3.° Los encubridores.


    ART. 15.

    Se consideran autores:

    1.° Los que toman parte en la ejecución del hecho, sea de una manera inmediata y directa; sea impidiendo o procurando impedir que se evite.
    2.° Los que fuerzan o inducen directamente a otro a ejecutarlo.
    3.° Los que, concertados para su ejecución, facilitan los medios con que se lleva a efecto el hecho o lo presencian sin tomar parte inmediata en él.


    ART. 16.

    Son cómplices los que, no hallándose comprendidos en el artículo anterior, cooperan a la ejecución del hecho por actos anteriores o simultáneos.


    ART. 17.

    Son encubridores los que con conocimiento de la perpetración de un crimen o de un simple delito o de los actos ejecutados para llevarlo a cabo, sin haber tenido participación en él como autores ni como cómplices, intervienen, con posterioridad a su ejecución, de alguno de los modos siguientes:

    1.° Aprovechándose por sí mismos o facilitando a los delincuentes medios para que se aprovechen de los efectos del crimen o simple delito.
    2.° Ocultando o inutilizando el cuerpo, los efectos o instrumentos del crimen o simple delito para impedir su descubrimiento.
    3.° Albergando, ocultando o proporcionando la fuga del culpable.
    4.° Acogiendo, receptando o protegiendo habitualmente a los malhechores, sabiendo que lo son, aun sin conocimiento de los crímenes o simples delitos determinados que hayan cometido, o facilitándoles los medios de reunirse u ocultar sus armas o efectos, o suministrándoles auxilios o noticias para que se guarden, precavan o salven.
    Están exentos de las penas impuestas a los encubridores los que lo sean de su cónyuge, de su conviviente civil, o de sus parientes por consanguinidad o afinidad en toda la línea recta y en la colateral hasta el segundo grado, de sus padres o hijos, con la sola excepción de los que se hallaren comprendidos en el número 1° de este artículo.


    TÍTULO TERCERO.

    DE LAS PENAS.



    § I.

    De las penas en general.


    ART. 18.

    Ningún delito se castigará con otra pena que la que le señale una ley promulgada con anterioridad a su perpetración.
    Si después de cometido el delito y antes de que se pronuncie sentencia de término, se promulgare otra ley que exima tal hecho de toda pena o le aplique una menos rigorosa, deberá arreglarse a ella su juzgamiento.
    Si la ley que exima el hecho de toda pena o le aplique una menos rigurosa se promulgare después de ejecutoriada la sentencia, sea que se haya cumplido o no la condena impuesta, el tribunal que hubiere pronunciado dicha sentencia, en primera o única instancia, deberá modificarla de oficio o a petición de parte.
    En ningún caso la aplicación de este artículo modificará las consecuencias de la sentencia primitiva en lo que diga relación con las indemnizaciones pagadas o cumplidas o las inhabilidades.


    ART. 19.

    El perdón de la parte ofendida no extingue la acción penal, salvo respecto de los delitos que no pueden ser perseguidos sin previa denuncia o consentimiento del agraviado.


    ART. 20.

    No se reputan penas, la restricción o privación de libertad de los detenidos o sometidos a prisión preventiva u otras medidas cautelares personales, la separación de los empleos públicos acordada por las autoridades en uso de sus atribuciones o por el tribunal durante el proceso o para instruirlo, ni las multas y demás correcciones que los superiores impongan a sus subordinados y administrados en uso de su jurisdicción disciplinal o atribuciones gubernativas.
    Tampoco se reputa pena el comiso de las ganancias provenientes del delito, ni cualquier forma de comiso sin condena prevista por la ley.



    § II.

    De la clasificación de las penas.


    ART. 21.

    Las penas que pueden imponerse con arreglo a este Código y sus diferentes clases, son las que comprende la siguiente:

    ESCALA GENERAL.

    PENAS DE CRÍMENES.
    Presidio perpetuo calificado.
    Presidio perpetuo.
    Reclusión perpetua.
    Presidio mayor.
    Reclusión mayor.
    Relegación perpetua.
    Confinamiento mayor.
    Extrañamiento mayor.
    Relegación mayor.
    Inhabilitación absoluta perpetua para cargos y oficios públicos, derechos políticos y profesiones titulares.
    Inhabilitación absoluta perpetua para cargos, empleos, oficios o profesiones ejercidos en ámbitos educacionales o que involucren una relación directa y habitual con personas menores de edad.
    Inhabilitación absoluta perpetua para cargos, empleos, oficios o profesiones ejercidos en ámbitos educacionales, de la salud o que involucren una relación directa y habitual con menores de dieciocho años de edad, adultos mayores o personas en situación de discapacidad.
    Inhabilitación absoluta perpetua para ejercer cargos, empleos, oficios o profesiones en empresas que contraten con órganos o empresas del Estado o con empresas o asociaciones en que éste tenga una participación mayoritaria; o en empresas que participen en concesiones otorgadas por el Estado o cuyo objeto sea la provisión de servicios de utilidad pública.
    Inhabilitación especial perpetua para algún cargo u oficio público o profesión titular.
    Inhabilitación absoluta temporal para ejercer cargos, empleos, oficios o profesiones en empresas que contraten con órganos o empresas del Estado o con empresas o asociaciones en que éste tenga una participación mayoritaria; o en empresas que participen en concesiones otorgadas por el Estado o cuyo objeto sea la provisión de servicios de utilidad pública.
    Inhabilitación absoluta temporal para cargos, empleos, oficios o profesiones ejercidos en ámbitos educacionales o que involucren una relación directa y habitual con personas menores de edad.
    Inhabilitación absoluta temporal para cargos, empleos, oficios o profesiones ejercidos en ámbitos educacionales, de la salud o que involucren una relación directa y habitual con menores de dieciocho años de edad, adultos mayores o personas en situación de discapacidad.
    Inhabilitación absoluta temporal para cargos y oficios públicos y profesiones titulares.
    Inhabilitación especial temporal para algún cargo u oficio público o profesión titular.

    PENAS DE SIMPLES DELITOS.

    Presidio menor.
    Reclusión menor.
    Confinamiento menor.
    Extrañamiento menor.
    Relegación menor.
    Destierro.
    Inhabilitación absoluta temporal para cargos, empleos, oficios o profesiones ejercidos en ámbitos educacionales o que involucren una relación directa y habitual con personas menores de edad.
    Inhabilitación absoluta temporal para cargos, empleos, oficios o profesiones ejercidos en ámbitos educacionales, de la salud o que involucren una relación directa y habitual con menores de dieciocho años de edad, adultos mayores o personas en situación de discapacidad.
    Inhabilitación absoluta temporal para ejercer cargos, empleos, oficios o profesiones en empresas que contraten con órganos o empresas del Estado o con empresas o asociaciones en que éste tenga una participación mayoritaria; o en empresas que participen en concesiones otorgadas por el Estado o cuyo objeto sea la provisión de servicios de utilidad pública.
    Inhabilitación especial temporal para emitir licencias médicas.
    Suspensión de cargo u oficio público o profesión titular.
    Inhabilidad perpetua para conducir vehículos a tracción mecánica o animal.
    Suspensión para conducir vehículos a tracción mecánica o animal.
    Inhabilidad absoluta perpetua para la tenencia de animales.

    PENAS DE LAS FALTAS.

    Prisión.
    Inhabilidad perpetua para conducir vehículos a tracción mecánica o animal.
    Suspensión para conducir vehículos a tracción mecánica o animal.

    PENAS COMUNES A LAS TRES CLASES ANTERIORES.

    Multa.
    Pérdida o comiso de los instrumentos o efectos del delito.

    PENAS ACCESORIAS DE LOS CRÍMENES Y SIMPLES DELITOS.

    ELIMINADA.
    Incomunicación con personas extrañas al establecimiento penal, en conformidad al Reglamento carcelario.

    Penas sustitutivas por vía de conversión de la multa
    Prestación de servicios en beneficio de la comunidad.








    ART. 22.

    Son penas accesorias las de suspensión e inhabilitación para cargos y oficios públicos, derechos políticos y profesiones titulares en los casos en que, no imponiéndolas especialmente la ley, ordena que otras penas las lleven consigo.



    ART. 23.

    La caución y la sujeción a la vigilancia de la autoridad podrán imponerse como penas accesorias o como medidas preventivas, en los casos especiales que determinen este Código y el de Procedimientos.


    ART. 24.

    Toda sentencia condenatoria en materia criminal lleva envuelta la obligación de pagar las costas, daños y perjuicios por parte de los autores, cómplices, encubridores y demás personas legalmente responsables.

    ART. 24 BIS

    Toda sentencia condenatoria en materia criminal lleva consigo el comiso de las ganancias provenientes del delito. Por el comiso de ganancias se priva a una persona de activos patrimoniales cuyo valor corresponda a la cuantía de las ganancias obtenidas a través del delito, o bien para o por perpetrarlo. Lo obtenido en virtud de lo señalado precedentemente será transferido al Fisco.
    Las ganancias obtenidas comprenden los frutos y las utilidades que se hubieren originado, cualquiera sea su naturaleza jurídica. Las ganancias comprenden también el equivalente a los costos evitados mediante el hecho ilícito.
    En la determinación del valor de las ganancias no se descontarán los gastos que hubieren sido necesarios para perpetrar el delito y obtenerlas.
    La acción para obtener el comiso de ganancias se sujetará a las reglas de la prescripción de la acción penal respectiva.
    Si un mismo bien pudiere ser objeto de comiso conforme a este artículo y conforme a otras disposiciones de este Código, sólo se aplicará lo dispuesto en este artículo.   


    ART. 24 TER.

    El comiso de ganancias también será impuesto a quien no ha intervenido en la perpetración del hecho, en cualquiera de las siguientes circunstancias:

    1ª. Si adquirió la ganancia como heredero o asignatario testamentario, a cualquier título gratuito o sin título válido, a menos que la haya adquirido del mismo modo de un tercero que no se encontrare en la misma circunstancia ni en las circunstancias que siguen.
    2ª. Si obtuvo la ganancia mediante el hecho ilícito y los intervinientes en la perpetración del hecho actuaron en su interés.
    3ª. Si adquirió la ganancia sabiendo o debiendo saber su procedencia ilícita al momento de la adquisición.
    4ª. Si se trata de una persona jurídica, que ha recibido la ganancia como aporte a su patrimonio.
   

   
    § III.

   
    De los límites, naturaleza y efectos de las penas.

    ART. 25.

    Las penas temporales mayores duran de cinco años y un día a veinte años, y las temporales menores de sesenta y un días a cinco años.
    Las de inhabilitación absoluta y especial temporales para cargos y oficios públicos y profesiones titulares duran de tres años y un día a diez años.
    La suspensión de cargo u oficio público o profesión titular, dura de sesenta y un días a tres años.
    Las penas de destierro y de sujeción a la vigilancia de la autoridad, de sesenta y un días a cinco años.
    La prisión dura de uno a sesenta días.
    La cuantía de la multa, tratándose de crímenes, no podrá exceder de treinta unidades tributarias mensuales; en los simples delitos, de veinte unidades tributarias mensuales, y en las faltas, de cuatro unidades tributarias mensuales; todo ello, sin perjuicio de que en determinadas infracciones, atendida su gravedad, se contemplen multas de cuantía superior.
    La expresión "unidad tributaria mensual" en cualquiera disposición de este Código, del Código de Procedimiento Penal y demás leyes penales especiales significa una unidad tributaria mensual vigente a la fecha de comisión del delito, y, tratándose de multas, ellas se deberán pagar en pesos, en el valor equivalente que tenga la unidad tributaria mensual al momento de su pago.
    Cuando la ley impone multas cuyo cómputo debe hacerse en relación a cantidades indeterminadas, nunca podrán aquéllas exceder de treinta unidades tributarias mensuales.
    En cuanto a la cuantía de la caución, se observarán las reglas establecidas para la multa, doblando las cantidades respectivamente, y su duración no podrá exceder del tiempo de la pena u obligación cuyo cumplimiento asegura, o de cinco años en los demás casos.
    INCISO SUPRIMIDO.




    ART. 26.

    La duración de las penas temporales empezará a contarse desde el día de la aprehensión del imputado.

    PENAS QUE LLEVAN CONSIGO OTRAS ACCESORIAS.


    ART. 27.

    Las penas de presidio, reclusión y relegación perpetuos, llevan consigo la de inhabilitación absoluta perpetua para cargos y oficios públicos y derechos políticos por el tiempo de la vida de los penados y la de sujeción a la vigilancia de la autoridad por el máximum que establece este Código.



    ART. 28.

    Las penas de presidio, reclusión, confinamiento, extrañamiento y relegación mayores, llevan consigo la de inhabilitación absoluta perpetua para cargos y oficios públicos y derechos políticos y la de inhabilitación absoluta para profesiones titulares mientras dure la condena.


    ART. 29.

    Las penas de presidio, reclusión, confinamiento, extrañamiento y relegación menores en sus grados máximos, llevan consigo la de inhabilitación absoluta perpetua para derechos políticos y la de inhabilitación absoluta para cargos y oficios públicos durante el tiempo de la condena.


    ART. 30.

    Las penas de presidio, reclusión, confinamiento, extrañamiento y relegación menores en sus grados medios y mínimos, y las de destierro y prisión, llevan consigo la de suspensión de cargo u oficio público durante el tiempo de la condena.



    ART. 31.

    Se impondrá el comiso de toda cosa que ha sido empleada como instrumento en la perpetración de un delito y sea especialmente apta para ser empleada delictivamente. Se entenderá que son especialmente aptas para ser utilizadas delictivamente, en todo caso, aquellas cosas cuya tenencia o porte se encuentra en general prohibida por la ley.
    El tribunal deberá decretar el comiso de cosas especialmente aptas para ser utilizadas delictivamente aun si el imputado resulta absuelto o sobreseído. Para ello bastará el establecimiento de su uso en un hecho delictivo. En este caso, el comiso será impuesto de conformidad con el procedimiento establecido en el Título III bis del Libro IV del Código Procesal Penal.
    El comiso de instrumentos especialmente aptos para ser utilizados delictivamente procederá aun respecto de terceros de buena fe y que tengan título para poseer la cosa, a menos que se establezca que el dueño no tuvo responsabilidad en el uso de la cosa por parte del hechor.
    Si el comiso afecta a un tercero de buena fe y que no tiene responsabilidad por el hecho, éste podrá solicitar indemnización al hechor.



    ART. 31 BIS.
    El comiso de una cosa que no sea especialmente apta para ser utilizada delictivamente y que ha servido de instrumento en la perpetración del hecho sólo será impuesto en la sentencia condenatoria y siempre que la cosa haya sido utilizada en la perpetración de un delito.
    Lo dispuesto en el inciso anterior no procederá respecto de terceros de buena fe. El tribunal prescindirá de su imposición cuando la privación de su propiedad le ocasione un perjuicio desproporcionado al afectado.



    ART. 31 TER.

    Se impondrá el comiso de toda cosa obtenida o producida a través de la perpetración del hecho.
    El comiso de los efectos del delito será decretado por el juez aun si el imputado resulta absuelto o sobreseído, siempre que se establezca que la cosa proviene de un hecho ilícito. En este caso, el comiso será impuesto de conformidad con el procedimiento establecido en el Título III bis del Libro IV del Código Procesal Penal.
    El comiso de los efectos del hecho no procederá respecto del tercero de buena fe.
    Tratándose de efectos de tenencia ilícita, el comiso procederá en todos los casos.
   

    NATURALEZA Y EFECTOS DE ALGUNAS PENAS.


    ART. 32.

    La pena de presidio sujeta al condenado a los trabajos prescritos por los reglamentos del respectivo establecimiento penal. Las de reclusión y prisión no le imponen trabajo alguno.


    ART. 32. BIS

    La imposición del presidio perpetuo calificado importa la privación de libertad del condenado de por vida, bajo un régimen especial de cumplimiento que se rige por las siguientes reglas:

    1.ª No se podrá conceder la libertad condicional sino una vez transcurridos cuarenta años de privación de libertad efectiva, debiendo en todo caso darse cumplimiento a las demás normas y requisitos que regulen su otorgamiento y revocación;

    2.ª El condenado no podrá ser favorecido con ninguno de los beneficios que contemple el reglamento de establecimientos penitenciarios, o cualquier otro cuerpo legal o reglamentario, que importe su puesta en libertad, aun en forma transitoria. Sin perjuicio de ello, podrá autorizarse su salida, con las medidas de seguridad que se requieran, cuando su cónyuge, su conviviente civil, o alguno de sus padres o hijos se encontrare en inminente riesgo de muerte o hubiere fallecido;

    3.ª No se favorecerá al condenado por las leyes que concedan amnistía ni indultos generales, salvo que se le hagan expresamente aplicables. Asimismo, sólo procederá a su respecto el indulto particular por razones de Estado o por el padecimiento de un estado de salud grave e irrecuperable, debidamente acreditado, que importe inminente riesgo de muerte o inutilidad física de tal magnitud que le impida valerse por sí mismo. En todo caso el beneficio del indulto deberá ser concedido de conformidad a las normas legales que lo regulen.




    ART. 33.

    Confinamiento es la expulsión del condenado del territorio de la República con residencia forzosa en un lugar determinado.





    ART. 34.

    Extrañamiento es la expulsión del condenado del territorio de la República al lugar de su elección.





    ART. 35.

    Relegación es la traslación del condenado a un punto habitado del territorio de la República con prohibición de salir de él, pero permaneciendo en libertad.



    ART. 36.

    Destierro es la expulsión del condenado de algún punto de la República.




    ART. 37.

    Para los efectos legales se reputan aflictivas todas las penas de crímenes y, respecto de las de simples delitos, las de presidio, reclusión, confinamiento, extrañamiento y relegación menores en sus grados máximos.


    ART. 38.

    La pena de inhabilitación absoluta perpetua para cargos y oficios públicos, derechos políticos y profesiones titulares, y la de inhabilitación absoluta temporal para cargos y oficios públicos Y profesiones titulares, producen:
    1.° La privación de todos los honores, cargos, empleos y oficios públicos y profesiones titulares de que estuviere en posesión el penado, aun cuando sean de elección popular.
    2.° La privación de todos los derechos políticos activos y pasivos y la incapacidad perpetua para obtenerlos.
    3.° La incapacidad para obtener los honores, cargos, empleos, oficios y profesiones mencionados, perpetuamente si la inhabilitación es perpetua y durante el tiempo de la condena si es temporal.
    4.° Derogado.




    ART. 39.

    Las penas de inhabilitación especial perpetua y temporal para algún cargo u oficio público o profesión titular, producen:
    1.° La privación del cargo, empleo, oficio o profesión sobre que recaen, y la de los honores anexos a él, perpetuamente si la inhabilitación es perpetua, y por el tiempo de la condena si es temporal.
    2.° La incapacidad para obtener dicho cargo, empleo, oficio o profesión u otros en la misma carrera, perpetuamente cuando la inhabilitación es perpetua, y por el tiempo de la condena cuando es temporal.

    ART. 39 bis.

    La pena de inhabilitación absoluta perpetua para cargos, empleos, oficios o profesiones ejercidos en ámbitos educacionales o que involucren una relación directa y habitual con personas menores de edad, prevista en el artículo 372, produce:
     
    1º La privación de todos los cargos, empleos, oficios y profesiones ejercidos en ámbitos educacionales o que involucren una relación directa y habitual con personas menores de edad que tenga el condenado.
    2º La incapacidad para obtener los cargos, empleos, oficios y profesiones mencionados perpetuamente.


    ART. 39 ter.

    La pena de inhabilitación absoluta perpetua o temporal para cargos, empleos, oficios o profesiones ejercidos en ámbitos educacionales, de la salud o que involucren una relación directa y habitual con menores de dieciocho años de edad, adultos mayores o personas en situación de discapacidad, prevista en el artículo 403 quáter de este código, produce:

    1º. La privación de todos los cargos, empleos, oficios y profesiones que tenga el condenado, ejercidos en ámbitos educacionales, de la salud o que involucren una relación directa y habitual con las personas mencionadas en el inciso primero de este artículo.
    2º. La incapacidad para obtener los cargos, empleos, oficios y profesiones mencionados, perpetuamente cuando la inhabilitación es perpetua, y por el tiempo de la condena cuando es temporal.

    La pena de inhabilitación absoluta temporal de que trata este artículo tiene una extensión de tres años y un día a diez años y es divisible en la misma forma que las penas de inhabilitación absoluta y especial temporales.




    ART. 39 quáter.- La pena de inhabilitación absoluta perpetua o temporal para ejercer cargos, empleos, oficios o profesiones en empresas que contraten con órganos o empresas del Estado o con empresas o asociaciones en que éste tenga una participación mayoritaria; o en empresas que participen en concesiones otorgadas por el Estado o cuyo objeto sea la provisión de servicios de utilidad pública, prevista en el artículo 251 quáter de este Código, produce:
     
    1º. La privación de todos los cargos, empleos, oficios y profesiones ejercidos en empresas que contraten con órganos o empresas del Estado o con empresas o asociaciones en que éste tenga una participación mayoritaria; o en empresas que participen en concesiones otorgadas por el Estado o cuyo objeto sea la provisión de servicios de utilidad pública.
    2º. La incapacidad para obtener los cargos, empleos, oficios y profesiones mencionados, perpetuamente cuando la inhabilitación es perpetua, y por el tiempo de la condena cuando es temporal.
     
    La pena de inhabilitación absoluta temporal de que trata este artículo tiene una extensión de tres años y un día a diez años y es divisible en la misma forma que las penas de inhabilitación absoluta y especial temporales.
    En este caso, ejecutoriada que sea la sentencia definitiva, el tribunal la comunicará a la Dirección de Compras y Contratación Pública. Dicha Dirección mantendrá un registro público actualizado de las personas naturales a las que se les haya impuesto esta pena.




    ART. 40.

    La suspensión de cargo y oficio público y profesión titular, inhabilita para su ejercicio durante el tiempo de la condena.
    La suspensión decretada durante el juicio, trae como consecuencia inmediata la privación de la mitad del sueldo al imputado, la cual sólo se le devolverá en el caso de pronunciarse sentencia absolutoria.
    La suspensión decretada por vía de pena, priva de todo sueldo al suspenso mientras ella dure.

    ART. 41.

    Cuando las penas de inhabilitación y suspensión recaigan en persona eclesiástica, sus efectos no se extenderán a los cargos, derechos y honores que tenga por la Iglesia. A los eclesiásticos incursos en tales penas y por todo el tiempo de su duración, no se les reconocerá en la República la jurisdicción eclesiástica y la cura de almas, ni podrán percibir rentas del tesoro nacional, salvo la congrua que fijará el tribunal.
    Esta disposición no comprende a los obispos en lo concerniente al ejercicio de la jurisdicción ordinaria que les corresponde.


    ART. 42.

    Los derechos políticos activos y pasivos a que se refieren los artículos anteriores, son: la capacidad para ser ciudadano elector, la capacidad para obtener cargos de elección popular y la capacidad para ser jurado.   

    El que ha sido privado de ellos sólo puede ser rehabilitado en su ejercicio en la forma prescrita por la Constitución.




    ART. 43.

    Cuando la inhabilitación para cargos y oficios públicos y profesiones titulares es pena accesoria, no la comprende el indulto de la pena principal, a menos que expresamente se haga extensivo a ella.


    ART. 44.

    El indulto de la pena de inhabilitación perpetua o temporal para cargos y oficios públicos y profesiones titulares, repone al penado en el ejercicio de estas últimas, pero no en los honores, cargos, empleos u oficios de que se le hubiere privado. El mismo efecto produce el cumplimiento de la condena a inhabilitación temporal.



    ART. 45.

    La sujeción a la vigilancia de la autoridad da al juez de la causa el derecho de determinar ciertos lugares en los cuales le será prohibido al penado presentarse después de haber cumplido su condena y de imponer a éste todas o algunas de las siguientes obligaciones:
    1.° La de declarar antes de ser puesto en libertad, el lugar en que se propone fijar su residencia.
    2.° La de recibir una boleta de viaje en que se le determine el itinerario que debe seguir, del cual no podrá apartarse, y la duración de su permanencia en cada lugar del tránsito.
    3.° La de presentarse dentro de las veinticuatro horas siguientes a su llegada, ante el funcionario designado en la boleta de viaje.
    4.° La de no poder cambiar de residencia sin haber dado aviso de ello, con tres días de anticipación, al mismo funcionario, quien le entregará la boleta de viaje primitiva visada para que se traslade a su nueva residencia.
    5.a La de adoptar oficio, arte, industria o profesión, si no tuviere medios propios y conocidos de subsistencia.


    ART. 46.

    La pena de caución produce en el penado la obligación de presentar un fiador abonado que responda o bien de que aquél no ejecutará el mal que se trata de precaver, o de que cumplirá su condena; obligándose a satisfacer, si causare el mal o quebrantare la condena, la cantidad que haya fijado el tribunal.
    Si el penado no presentare fiador, sufrirá una reclusión equivalente a la cuantía de la fianza, computándose un día por cada quinto de unidad tributaria mensual; pero sin poder en ningún caso exceder de seis meses.





    ART. 47.

    En todos los casos en que se imponga el pago de costas se entenderá comprender tanto las procesales como las personales y además los gastos ocasionados por el juicio y que no se incluyen en las costas. Estos gastos se fijarán por el tribunal, previa audiencia de las partes.



    ART. 48.

    Si los bienes del condenado no fueran bastantes para cubrir las responsabilidades pecuniarias, se satisfarán éstas en el orden siguiente:

    1. El comiso de las ganancias provenientes del delito o, en su caso, del valor equivalente a los efectos o instrumentos del delito.

    2. Las multas.

    3. Las costas procesales y el resarcimiento de los gastos ocasionados por el juicio.

    4. La reparación del daño causado e indemnización de perjuicios.

    5. Las costas personales.

    Si por aplicación de lo dispuesto en el inciso anterior no es posible satisfacer la indemnización de perjuicios derivada del delito por falta de bienes realizables, el perjudicado podrá ejercer la acción civil sobre los bienes decomisados para efectos del número 1, o el producto de su realización, siempre que exista una relación directa entre el perjuicio irrogado y las ganancias obtenidas. El Estado podrá excepcionarse del pago si demuestra la existencia de bienes realizables sobre los cuales puede hacerse efectiva la indemnización, o que ella no pudo ser satisfecha por negligencia del perjudicado.
    En caso de iniciarse un procedimiento concursal, estos créditos se graduarán considerándose la obligación de cumplir con el comiso como un crédito de la primera clase comprendido en el número 1 del artículo 2472 del Código Civil y los restantes como uno solo entre los que no gozan de preferencia. En este caso no se aplicará lo dispuesto en el inciso anterior.


    ART. 49.

    Si el sentenciado no tuviere bienes para satisfacer la multa podrá el tribunal imponer, por vía de sustitución, la pena de prestación de servicios en beneficio de la comunidad.
    Para proceder a esta sustitución se requerirá del acuerdo del condenado. En caso contrario, el tribunal impondrá, por vía de sustitución y apremio de la multa, la pena de reclusión, regulándose un día por cada tercio de unidad tributaria mensual, sin que ella pueda nunca exceder de seis meses.
    No se aplicará la pena sustitutiva señalada en el inciso primero ni se hará efectivo el apremio indicado en el inciso segundo, cuando, de los antecedentes expuestos por el condenado, apareciere la imposibilidad de cumplir la pena.
    Queda también exento de este apremio el condenado a reclusion menor en su grado máximo o a otra pena mas grave que deba cumplir efectivamente.



    Art. 49 bis.

    La pena de prestación de servicios en beneficio de la comunidad consiste en la realización de actividades no remuneradas a favor de ésta o en beneficio de personas en situación de precariedad, coordinadas por un delegado de Gendarmería de Chile.
    El trabajo en beneficio de la comunidad será facilitado por Gendarmería, pudiendo establecer los convenios que estime pertinentes para tal fin con organismos públicos y privados sin fines de lucro.
    Gendarmería de Chile y sus delegados, y los organismos públicos y privados que en virtud de los convenios a que se refiere el inciso anterior intervengan en la ejecución de esta sanción, deberán velar por que no se atente contra la dignidad del penado en la ejecución de estos servicios.


    Art. 49 ter.

    La pena de prestación de servicios en beneficio de la comunidad se regulará en ocho horas por cada tercio de unidad tributaria mensual, sin perjuicio de la conversión establecida en leyes especiales.
    Su duración diaria no podrá exceder de ocho horas.
    En cualquier momento el condenado podrá solicitar poner término a la prestación de servicios en beneficio de la comunidad previo pago de la multa, a la que se deberán abonar las horas trabajadas.



    Art. 49 quáter.

    En caso de decretarse la sanción de prestación de servicios en beneficio de la comunidad, el delegado de Gendarmería de Chile responsable de gestionar el cumplimiento informará al tribunal que dictó la sentencia, quien a su vez notificará al Ministerio Público, al defensor y al condenado, el tipo de servicio, el lugar donde deba realizarse y el calendario de su ejecución, dentro de los treinta días siguientes a la fecha en que la condena se encontrare firme o ejecutoriada.


    Art. 49 quinquies.

    En caso de incumplimiento de la pena de servicios en beneficio de la comunidad, el delegado deberá informar al tribunal que haya impuesto la sanción.
    El tribunal citará a una audiencia para resolver la mantención o la revocación de la pena.



    Art. 49 sexies.

    El juez podrá revocar la pena de servicios en beneficio de la comunidad cuando el condenado:
    a) No se presentare, injustificadamente, ante Gendarmería de Chile a cumplir la pena en el plazo que determine el juez, el que no podrá ser menor a tres ni superior a siete días;
    b) Se ausentare del trabajo durante al menos dos jornadas laborales. Si el penado faltare al trabajo por causa justificada no se entenderá dicha ausencia como abandono de la actividad;
    c) Su rendimiento en la ejecución de los servicios fuere sensiblemente inferior al mínimo exigible, a pesar de los requerimientos del responsable del centro de trabajo, o
    d) Se opusiere o incumpliere de forma reiterada y manifiesta las instrucciones que se le dieren por el responsable del centro de trabajo.
    En caso de revocar la pena de servicios en beneficio de la comunidad, el tribunal impondrá al condenado, por vía de sustitución y apremio de la multa originalmente impuesta, la pena de reclusión, regulándose un día por cada tercio de unidad tributaria mensual, sin que ella pueda nunca exceder de seis meses.
    Habiéndose decretado la revocación se abonará al tiempo de reclusión un día por cada ocho horas efectivamente trabajadas en beneficio de la comunidad.
    Si el tribunal no revocare la pena de servicios en beneficio de la comunidad podrá ordenar que el cumplimiento de la misma se lleve a cabo en un lugar distinto a aquel en el cual originalmente se estaba ejecutando; todo lo anterior sin perjuicio de la facultad prevista en el inciso tercero del artículo 49.



    § IV.

    De la aplicación de las penas.


    ART. 50.

    A los autores de delito se impondrá la pena que para éste se hallare señalada por la ley.
    Siempre que la ley designe la pena de un delito, se entiende que la impone al delito consumado.


    ART. 51.

    A los autores de crimen o simple delito frustrado y a los cómplices de crimen o simple delito consumado, se impondrá la pena inmediatamente inferior en grado a la señalada por la ley para el crimen o simple delito.


    ART. 52.

    A los autores de tentativa de crimen o simple delito, a los cómplices de crimen o simple delito frustrado y a los encubridores de crimen o simple delito consumado, se impondrá la pena inferior en dos grados a la que señala la ley para el crimen o simple delito.

    Exceptúanse de esta regla los encubridores comprendidos en el núm. 3.° del art. 17, en quienes concurra la circunstancia primera del mismo número, a los cuales se impondrá la pena de inhabilitación especial perpetua, si el delincuente encubierto fuere condenado por crimen y la de inhabilitación especial temporal en cualquiera de sus grados, si lo fuere por simple delito.

    También se exceptúan los encubridores comprendidos en el núm. 4.° del mismo art. 17, a quienes se aplicará la pena de presidio menor en cualquiera de sus grados.


    ART. 53.

    A los cómplices de tentativa de crimen o simple delito y a los encubridores de crimen o simple delito frustrado, se impondrá la pena inferior en tres grados a la que señala la ley para el crimen o simple delito.


    ART. 54.

    A los encubridores de tentativa de crimen o simple delito, se impondrá la pena inferior en cuatro grados a la señalada para el crimen o simple delito.


    ART. 55.

    Las disposiciones generales contenidas en los cuatro artículos precedentes no tienen lugar en los casos en que el delito frustrado, la tentativa, la complicidad o el encubrimiento se hallan especialmente penados por la ley.


    ART. 56.

    Las penas divisibles constan de tres grados, mínimo, medio y máximo, cuya extensión se determina en la siguiente:

    TABLA DEMOSTRATIVA

  



    ART. 57.

    Cada grado de una pena divisible constituye pena distinta.


    ART. 58.

    En los casos en que la ley señala una pena compuesta de dos o más distintas, cada una de éstas forma un grado de penalidad, la más leve de ellas el mínimo y la más grave el máximo.



    ART. 59.

    Para determinar las penas que deben imponerse según los arts. 51, 52, 53 y 54: 1.° a los autores de crimen o simple delito frustrado; 2.° a los autores de tentativa de crimen o simple delito, cómplices de crimen o simple delito frustrado y encubridores de crimen o simple delito consumado; 3.° a los cómplices de tentativa de crimen o simple delito y encubridores de crimen o simple delito frustrado, y 4.° a los encubridores de tentativa de crimen o simple delito, el tribunal tomará por base las siguientes escalas graduales:

                    ESCALA NUMERO 1
    Grados.
    1° Presidio perpetuo calificado.
    2° Presidio o reclusión perpetuos.
    3° Presidio o reclusión mayores en sus grados máximos.
    4° Presidio o reclusión mayores en sus grados medios.
    5° Presidio o reclusión mayores en sus grados mínimos.
    6° Presidio o reclusión menores en sus grados máximos.
    7° Presidio o reclusión menores en sus grados medios.
    8° Presidio o reclusión menores en sus grados mínimos.
    9° Prisión en su grado máximo.
    10. Prisión en su grado medio.
    11. Prisión en su grado mínimo.

                    ESCALA NUMERO 2
    Grados.
    1° Relegación perpetua.
    2° Relegación mayor en su grado máximo.
    3° Relegación en su grado medio.
    4° Relegación mayor en su grado mínimo.
    5° Relegación menor en su grado máximo.
    6° Relegación menor en su grado medio.
    7° Relegación menor en su grado mínimo.
    8° Destierro en su grado máximo.
    9° Destierro en su grado medio.
    10. Destierro en su grado mínimo.

                  ESCALA NUMERO 3
    Grados.
    1° Confinamiento o extrañamiento mayores en sus grados máximos.
    2° Confinamiento o extrañamiento mayores en sus grados medios.
    3° Confinamiento o extrañamiento mayores en sus grados mínimos.
    4° Confinamiento o extrañamiento menores en sus grados máximos.
    5° Confinamiento o extrañamiento menores en sus grados medios.
    6° Confinamiento o extrañamiento menores en sus grados mínimos.
    7° Destierro en su grado máximo.
    8° Destierro en su grado medio.
    9° Destierro en su grado mínimo.

                  ESCALA NUMERO 4
    Grados.
    1° Inhabilitación absoluta perpetua.
    2° Inhabilitación absoluta temporal en su grado máximo.
    3° Inhabilitación absoluta temporal en su grado medio.
    4° Inhabilitación absoluta temporal en su grado mínimo.
    5° Suspensión en su grado máximo.
    6° Suspensión en su grado medio.
    7° Suspensión en su grado mínimo.

                    ESCALA NUMERO 5
    Grados.
    1° Inhabilitación especial perpetua.
    2° Inhabilitación especial temporal en su grado máximo.
    3° Inhabilitación especial temporal en su grado medio.
    4° Inhabilitación especial temporal en su grado mínimo.
    5° Suspensión en su grado máximo.
    6° Suspensión en su grado medio.
    7° Suspensión en su grado mínimo.


    ART. 60.

    La multa se considera como la pena inmediatamente inferior a la última en todas las escalas graduales.
    Para fijar su cuantía respectiva se adoptará la base establecida en el art. 25, y en cuanto a su aplicación a cada caso especial se observará lo que prescribe el art. 70.
    El producto de las multas, cauciones y comisos derivados de faltas y contravenciones, se aplicará a fondos de la Municipalidad correspondiente al territorio donde ellas se cometieron.


    ART. 61.

    La designación de las penas que corresponde aplicar en los diversos casos a que se refiere el art. 59, se hará con sujeción a las siguientes reglas:

    1.° Si la pena señalada al delito es una indivisible o un solo grado de otra divisible, corresponde a los autores de crimen o simple delito frustrado y a los cómplices de crimen o simple delito consumado la inmediatamente inferior en grado.
    Para determinar las que deben aplicarse a los demás responsables relacionados en el art. 59, se bajará sucesivamente un grado en la escala correspondiente respecto de los comprendidos en cada uno de sus números, siguiendo el orden que en ese artículo se establece.
    2.° Cuando la pena que se señala al delito consta de dos o más grados, sea que los compongan dos penas indivisibles, diversos grados de penas divisibles o bien una o dos indivisibles y uno o más grados de otra divisible, a los autores de crimen o simple delito frustrado y a los cómplices de crimen o simple delito consumado corresponde la inmediatamente inferior en grado al mínimo de los designados por la ley.
    Para determinar las que deben aplicarse a los demás responsables se observará lo prescrito en la regla anterior.
    3.° Si se designan para un delito penas alternativas, sea que se hallen comprendidas en la misma escala o en dos o más distintas, no estará obligado el tribunal a imponer a todos los responsables las de la misma naturaleza.
    4.° Cuando se señalan al delito copulativamente penas comprendidas en distintas escalas o se agrega la multa a las de la misma escala, se aplicarán unas y otras, con sujeción a las reglas 1.° y 2.°, a todos los responsables; pero cuando una de dichas penas se impone al autor de crimen o simple delito por circunstancias peculiares a él que no concurren en los demás, no se hará extensiva a éstos.
    5.°  Si al poner en práctica las reglas precedentes no resultare pena que imponer por falta de grados inferiores o por no ser aplicables las de inhabilitación o suspensión, se impondrá siempre la multa.

    APLICACION PRACTICA DE LAS REGLAS ANTERIORES 




NOTA
      El Art. 1° de la ley 19450, publicada el 18.03.1996, modificada por la ley 19501, publicada el 15.05.1997, dispuso la sustitución de las escalas de multas establecidas en sueldos vitales en el Código Penal por otras expresadas en unidades tributarias mensuales o fracción de unidad tributaria mensual, de acuerdo con la tabla de conversión que establece.

    ART. 62.

    Las circunstancias atenuantes o agravantes se tomarán en consideración para disminuir o aumentar la pena en los casos y conforme a las reglas que se prescriben en los artículos siguientes.



    ART. 63.

    No producen el efecto de aumentar la pena las circunstancias agravantes que por sí mismas constituyen un delito especialmente penado por la ley, o que ésta haya expresado al describirlo y penarlo.
    Tampoco lo producen aquellas circunstancias agravantes de tal manera inherentes al delito que sin la concurrencia de ellas no puede cometerse.


    ART. 64.

    Las circunstancias atenuantes o agravantes que consistan en la disposición moral del delincuente, en sus relaciones particulares con el ofendido o en otra causa personal, servirán para atenuar o agravar la responsabilidad de sólo aquellos autores, cómplices o encubridores en quienes concurran.
    Las que consistan en la ejecución material del hecho o en los medios empleados para realizarlo, servirán para atenuar o agravar la responsabilidad únicamente de los que tuvieren conocimiento de ellas antes o en el momento de la acción o de su cooperación para el delito.




    ART. 65.

    Cuando la ley señala una sola pena indivisible, la aplicará el tribunal sin consideración a las circunstancias agravantes que concurran en el hecho. Pero si hay dos o más circunstancias atenuantes y no concurre ninguna agravante, podrá aplicar la pena inmediatamente inferior en uno o dos grados.




    ART. 66.

    Si la ley señala una pena compuesta de dos indivisibles y no acompañan al hecho circunstancias atenuantes ni agravantes, puede el tribunal imponerla en cualquiera de sus grados.
    Cuando solo concurre alguna circunstancia atenuante, debe aplicarla en su grado mínimo, y si habiendo una circunstancia agravante, no concurre ninguna atenuante, la impondrá en su grado máximo.
    Siendo dos o más las circunstancias atenuantes sin que concurra ninguna agravante, podrá imponer la pena inferior, en uno o dos grados al mínimo de los señalados por la ley, según sea el número y entidad de dichas circunstancias.
    Si concurrieren circunstancias atenuantes y agravantes, las compensará racionalmente el tribunal para la aplicación de la pena, graduando el valor de unas y otras.



    ART. 67.

    Cuando la pena señalada al delito es un grado de una divisible y no concurren circunstancias atenuantes ni agravantes en el hecho, el tribunal puede recorrer toda su extensión al aplicarla.
    Si concurre sólo una circunstancia atenuante o sólo una agravante, la aplicará en el primer caso en su mínimum y en el segundo en su máximum.
    Para determinar en tales casos el mínimum y el máximum de la pena, se divide por mitad el período de su duración: la más alta de estas partes formará el máximum y la más baja el mínimum.
    Siendo dos o más las circunstancias atenuantes y no habiendo ninguna agravante, podrá el tribunal imponer la inferior en uno o dos grados, según sea el número y entidad de dichas circunstancias.
    Si hay dos o más circunstancias agravantes y ninguna atenuante, puede aplicar la pena superior en un grado.
    En el caso de concurrir circunstancias atenuantes y agravantes, se hará su compensación racional para la aplicación de la pena, graduando el valor de unas y otras.




    ART. 68.

    Cuando la pena señalada por la ley consta de dos o más grados, bien sea que los formen una o dos penas indivisibles y uno o más grados de otra divisible, o diversos grados de penas divisibles, el tribunal al aplicarla podrá recorrer toda su extensión, si no concurren en el hecho circunstancias atenuantes ni agravantes.
    Habiendo una sola circunstancia atenuante o una sola circunstancia agravante, no aplicará en el primer caso el grado máximo ni en el segundo el mínimo.
    Si son dos o más las circunstancias atenuantes y no hay ninguna agravante, el tribunal podrá imponer la pena inferior en uno, dos o tres grados al mínimo de los señalados por la ley, según sea el número y entidad de dichas circunstancias.
    Cuando, no concurriendo circunstancias atenuantes, hay dos o más agravantes, podrá imponer la inmediatamente superior en grado al máximo de los designados por la ley.
    Concurriendo circunstancias atenuantes y agravantes, se observará lo prescrito en los artículos anteriores para casos análogos.



    ART. 68 BIS.

    Sin perjuicio de lo dispuesto en los cuatro artículos anteriores, cuando sólo concurra una atenuante muy calificada el Tribunal podrá imponer la pena inferior en un grado al mínimo de la señalada al delito.

    ART. 69.

    Dentro de los límites de cada grado el tribunal determinará la cuantía de la pena en atención al número y entidad de las circunstancias atenuantes y agravantes y a la mayor o menor extensión del mal producido por el delito, teniendo en especial consideración la circunstancia de ser la víctima un menor de 18 años, un adulto mayor, según lo dispuesto por la ley Nº 19.828, o una persona con discapacidad en los términos de la ley Nº 20.422.



    ART. 69 BIS.

    Sin perjuicio de lo dispuesto en el artículo anterior, en los delitos contra las personas, en el caso que concurra alguna de las circunstancias agravantes del número 22º del artículo 12, la pena se determinará excluyendo el grado mínimo si es compuesta, o el mínimum si consta de un solo grado.


    ART. 70.

    En la aplicación de las multas el tribunal podrá recorrer toda la extensión en que la ley le permite imponerlas, consultando para determinar en cada caso su cuantía, no solo las circunstancias atenuantes y agravantes del hecho, sino principalmente el caudal o facultades del culpable. Asimismo, en casos calificados, de no concurrir agravantes y considerando las circunstancias anteriores, el juez podrá imponer una multa inferior al monto señalado en la ley, lo que deberá fundamentar en la sentencia.
    Tanto en la sentencia como en su ejecución el Tribunal podrá, atendidas las circunstancias, autorizar al afectado para pagar las multas por parcialidades, dentro de un límite que no exceda del plazo de un año. El no pago de una sola de las parcialidades, hará exigible el total de la multa adeudada.




    ART. 71.

    Cuando no concurran todos los requisitos que se exigen en el caso del núm. 8.° del art. 10 para eximir de responsabilidad, se observará lo dispuesto en el art. 490.



    ART.72.

    Cuando el delito sea cometido con la intervención de una o más personas menores de dieciocho años de edad y mayores de catorce, se excluirá el mínimum o el grado mínimo de la pena señalada, según corresponda, respecto de los imputados mayores de edad que hubieren participado en él.
    Asimismo, se aumentará en un grado la pena al mayor de dieciocho años de edad cuando el crimen o simple delito sea cometido o perpetrado con la intervención de una o más personas menores de catorce años de edad.
    El consentimiento dado por el menor de dieciocho años no eximirá al mayor de esta edad de la aplicación de las reglas previstas en los incisos precedentes.



    ART. 73.

    Se aplicará asimismo la pena inferior en uno, dos o tres grados al mínimo de los señalados por la ley, cuando el hecho no fuere del todo excusable por falta de alguno de los requisitos que se exigen para eximir de responsabilidad criminal en los respectivos casos de que trata el art. 10, siempre que concurra el mayor número de ellos, imponiéndola en el grado que el tribunal estime correspondiente, atendido el número entidad de los requisitos que falten o concurran.
    Esta disposición se entiende sin perjuicio de la contenida en el art. 71.


    ART. 74.

    Al culpable de dos o más delitos se le impondrán todas las penas correspondientes a las diversas infracciones.
    El sentenciado cumplirá todas sus condenas simultáneamente, siendo posible. Cuando no lo fuere, o si de ello hubiere de resultar ilusoria alguna de las penas, las sufrirá en orden sucesivo, principiando por las más graves o sea las más altas en la escala respectiva, excepto las de confinamiento, extrañamiento, relegación y destierro, las cuales se ejecutarán después de haber cumplido cualquiera otra pena de las comprendidas en la escala gradual núm. 1.



    ART. 75.

    La disposición del artículo anterior no es aplicable en el caso de que un solo hecho constituya dos o más delitos, o cuando uno de ellos sea el medio necesario para cometer el otro.
    En estos casos solo se impondrá la pena mayor asignada al delito más grave.




    ART. 76.

    Siempre que el tribunal imponga una pena que lleve consigo otras por disposición de la ley, según lo prescrito en el § III de este título, condenará también al acusado expresamente en estas últimas.



    ART. 77.

    En los casos en que la ley señala una pena inferior o superior en uno o más grados a otra determinada, la pena inferior o superior se tomará de la escala gradual en que se halle comprendida la pena determinada.
    Si no hubiere pena superior en la escala gradual respectiva, se impondrá el presidio perpetuo. Sin embargo, cuando se tratare de la escala número 1 prevista en el artículo 59, se impondrá el presidio perpetuo calificado.
    Faltando pena inferior se aplicará siempre la multa.
    Cuando sea preciso elevar las inhabilitaciones absolutas o especiales perpetuas a grados superiores, se agravarán con la reclusión menor en su grado medio.   




    ART. 78.

    Siempre que sea necesario determinar la correspondencia entre las penas de este Código y las impuestas con anterioridad a su vigencia, se hará tomando en cuenta la naturaleza de éstas y el período de su duración. Así por ejemplo, cuatro años de presidio o de penitenciaria equivalen a presidio menor en su grado máximo.

    ART. 78 bis.-
    La circunstancia de que un hecho constitutivo de delito pueda asimismo dar lugar a una o más sanciones o medidas de las establecidas en el artículo 20 no obsta a la imposición de las penas que procedan.
    Con todo, el monto de la pena de multa pagada será abonado a la multa no constitutiva de pena que se imponga al condenado por el mismo hecho. Si el condenado hubiere pagado una multa no constitutiva de pena como consecuencia del mismo hecho, el monto pagado será abonado a la pena de multa impuesta.     
    La extensión de la suspensión o inhabilitación impuesta al condenado como consecuencia adicional a la pena será deducida de la extensión de la suspensión o inhabilitación de la misma naturaleza que fuere impuesta como sanción administrativa o disciplinaria. Si el condenado hubiere sido sometido a una suspensión o inhabilitación como sanción administrativa o disciplinaria, la extensión de ésta será deducida de la suspensión o inhabilitación de la misma naturaleza que se le impusiere.

    § V.
   
    De la ejecución de las penas y de su cumplimiento.


    ART. 79.

    No podrá ejecutarse pena alguna sino en virtud de sentencia ejecutoriada.


    ART. 80.

    Tampoco puede ser ejecutada pena alguna en otra forma que la prescrita por la ley, ni con otras circunstancias o accidentes que los expresados en su texto.
    Se observará también además de lo que dispone la ley, lo que se determine en los reglamentos especiales para el gobierno de los establecimientos en que deben cumplirse las penas, acerca de los castigos disciplinarios, de la naturaleza, tiempo y demás circunstancias de los trabajos, de las relaciones de los penados con otras personas, de los socorros que pueden recibir y del régimen alimenticio.
    En los reglamentos sólo podrán imponerse como castigos disciplinarios, el encierro en celda solitaria e incomunicación con personas extrañas al establecimiento penal por un tiempo que no exceda de un mes, u otros de menor gravedad.
    La repetición de estas medidas deberá comunicarse antes de su aplicación al juez del lugar de reclusión, quien sólo podrá autorizarla por resolución fundada y adoptando las medidas para resguardar la seguridad e integridad, del detenido o preso.






    Artículo 81.- Si después de cometido el delito cayere el delincuente en estado de locura o demencia, se observarán las reglas establecidas en el Código de Procedimiento Penal.



    ART. 82. Derogado.-
    ART. 83. Derogado.-
    ART. 84. Derogado.-
    ART. 85. Derogado.-
    ART. 86.

    Los condenados a penas privativas de libertad cumplirán sus condenas en la clase de establecimientos carcelarios que corresponda en conformidad al Reglamento respectivo.

    ART. 87.

    Los menores de veintiún años y las mujeres cumplirán sus condenas en establecimientos especiales. En los lugares donde éstos no existan, permanecerán en los establecimientos carcelarios comunes, convenientemente separados de los condenado adultos y varones, respectivamente.



    ART. 88.

    El producto del trabajo de los condenados a presidio será destinado:
    1.° A indemnizar al establecimiento de los gastos que ocasionen.
    2.° A proporcionarles alguna ventaja o alivio durante su detención, si lo merecieren.
    3.° A hacer efectiva la responsabilidad civil de aquellos proveniente del delito.
    4.° A formarles un fondo de reserva que se les entregará a su salida del establecimiento penal.



    ART. 89.

    Los condenados a reclusión y prisión son libres para ocuparse, en beneficio propio, en trabajos de su elección, siempre que sean compatibles con la disciplina reglamentaria del establecimiento penal; pero si afectándoles las responsabilidades de las reglas 1.° y 3.° del artículo anterior, carecieren de los medios necesarios para llenar los compromisos que ellas les imponen o no tuvieren oficio o modo de vivir conocido y honesto, estarán sujetos forzosamente a los trabajos del establecimiento hasta hacer efectivas con su producto aquellas responsabilidades y procurarse la subsistencia.

    ART. 89 bis.

    El Ministro de Justicia podrá disponer, de acuerdo con los tratados internacionales vigentes sobre la materia y ratificados por Chile, o sobre la base del principio de reciprocidad, que los extranjeros condenados por alguno de los delitos contemplados en los artículos 411 bis, 411 ter, 411 quáter y 411 quinquies, cumplan en el país de su nacionalidad las penas privativas de libertad que les hubieren sido impuestas.   



    TÍTULO CUART0.

    DE LAS PENAS EN QUE INCURREN LOS QUE QUEBRANTAN LAS SENTENCIAS Y LOS QUE DURANTE UNA CONDENA DELINQUEN DE NUEVO.



    § I.

    De las penas en que Incurren los que quebrantan las sentencias.


    ART. 90.

    Los sentenciados que quebrantaren su condena serán castigados con las penas que respectivamente se designan en los números siguientes:

    1.° Los condenados a presidio, reclusión o prisión sufrirán la pena de incomunicación con personas extrañas al establecimiento penal por un tiempo que, atendidas las circunstancias, podrá extenderse hasta tres meses, quedando durante el mismo tiempo sujetos al régimen más estricto del establecimiento.
    2° Los reincidentes en el quebrantamiento de tales condenas, a más de las penas de la regla anterior, sufrirán la pena de incomunicación con personas extrañas al establecimiento penal por un término prudencial, atendidas las circunstancias, que no podrá exceder de seis meses.
    3.° Derogado.
    4.° Los condenados a confinamiento, extrañamiento, relegación o destierro, sufrirán las penas de presidio, reclusión o prisión, según las reglas siguientes:
    Primera.-El condenado a relegación perpetua sufrirá la de presidio mayor en su grado medio.
    Segunda.-El condenado a confinamiento o extrañamiento sufrirá la de presidio, por la mitad del tiempo que le falte por cumplir de la pena primitiva.
    Tercera.-El condenado a relegación temporal o a destierro sufrirá la de reclusión o prisión por la mitad del tiempo que le falte por cumplir de la pena primitiva.
    5.° El inhabilitado para cargos y oficios públicos, derechos políticos y profesiones titulares o para cargos, oficios o profesiones ejercidos en ámbitos educacionales, de la salud o que involucren una relación directa y habitual con menores de dieciocho años de edad o para la tenencia de animales, adultos mayores o personas en situación de discapacidad, que los ejerciere, cuando el hecho no constituya un delito especial, sufrirá la pena de reclusión menor en su grado mínimo o multa de seis a veinte unidades tributarias mensuales.
    En caso de reincidencia se doblará esta pena.
    6.° El suspenso de cargo u oficio público o profesión titular que los ejerciere, sufrirá un recargo por igual tiempo al de su primitiva condena.
    En caso de reincidencia sufrirá la pena de reclusión menor en su grado mínimo o multa de seis a veinte unidades tributarias mensuales.
    7.° El sometido a la vigilancia de la autoridad, que faltare a las reglas que debe observar, sufrirá la pena de reclusión menor en sus grados mínimo a medio.
    8.° El condenado en proceso por crimen o simple delito a la pena de retiro o suspensión del carnet, permiso o autorización que lo faculta para conducir vehículos o embarcaciones, o a la sanción de inhabilidad perpetua para conducirlos, sufrirá la pena de presidio menor en su grado mínimo.


    § II.

De las penas en que incurren los que durante una condena delinquen de nuevo.


    ART. 91.

    Los que después de haber sido condenados por sentencia ejecutoriada cometieren algún crimen o simple delito durante el tiempo de su condena, bien sea mientras la cumplen o después de haberla quebrantado, sufrirán la pena que la ley señala al nuevo crimen o simple delito que cometieren, debiendo cumplir esta condena y la primitiva por el orden que el tribunal prefije en la sentencia, de conformidad con las reglas prescritas en el art. 74 para el caso de imponerse varias penas al mismo delincuente. 
    Cuando en el caso de este artículo el nuevo crimen debiere penarse con presidio o reclusión perpetuos y el delincuente se hallare cumpliendo alguna de estas penas, podrá imponérsele la de presidio perpetuo calificado. Si el nuevo crimen o simple delito tuviere señalada una pena menor, se agravará la pena perpetua con una o más de las penas accesorias indicadas, a arbitrio del Tribunal, que podrán imponerse hasta por el máximo del tiempo que permite el artículo 25.
    En el caso de que el nuevo crimen deba penarse con relegación perpetua y el delincuente se halle cumpliendo la misma pena, se le impondrá la de presidio mayor en su grado medio, dándose por terminada la de relegación.
    Cuando la pena que mereciere el nuevo crimen o simple delito fuere otra menor, se observará lo prescrito en el acápite primero del presente artículo.

    ART. 92.

    Si el nuevo delito se cometiere después de haberse impuesto una condena, habrá que distinguir tres casos:
    1.° Cuando es de la misma especie que el anterior.
    2.° Cuando es de distinta especie y el culpable ha sido condenado ya por dos o más delitos a que la ley señala igual o mayor pena.
    3.° Cuando siendo de distinta especie, el delincuente sólo ha sido condenado una vez por delito a que la ley señala igual o mayor pena, o más de una vez por delito cuya pena sea menor.
    En los dos primeros casos el hecho se considera revestido de circunstancia agravante, atendido a lo que disponen los núms. 15 y 16 del art. 12, y en el último no se tomarán en cuenta para aumentar la pena los delitos anteriores.





    TÍTULO QUINTO.

    DE LA EXTINCIÓN DE LA RESPONSABILIDAD PENAL.


    ART. 93.

    La responsabilidad penal se extingue:

    1.° Por la muerte del responsable, siempre en cuanto a las penas personales, y respecto de las pecuniarias sólo cuando a su fallecimiento no se hubiere dictado sentencia ejecutoriada.
    2.° Por el cumplimiento de la condena.
    3.° Por amnistía, la cual extingue por completo la pena y todos sus efectos.
    4.° Por indulto.
    La gracia de indulto sólo remite o conmuta la pena; pero no quita al favorecido el carácter de condenado para los efectos de la reincidencia o nuevo delinquimiento y demás que determinan las leyes.
    5.° Por el perdón del ofendido cuando la pena se haya impuesto por delitos respecto de los cuales la ley sólo concede acción privada.
    6.° Por la prescripción de la acción penal.
    7.° Por la prescripción de la pena.


    ART. 94.

    La acción penal prescribe:
    Respecto de los crímenes a que la ley impone pena de presidio, reclusión o relegación perpetuos, en quince años.
    Respecto de los demás crímenes, en diez años.
    Respecto de los simples delitos, en cinco años.
    Respecto de las faltas, en seis meses.
    Cuando la pena señalada al delito sea compuesta, se estará a la privativa de libertad, para la aplicación de las reglas comprendidas en los tres primeros acápites de este artículo; si no se impusieren penas privativas de libertad, se estará a la mayor.
    Las reglas precedentes se entienden sin perjuicio de las prescripciones de corto tiempo que establece este Código para delitos determinados.





    ART. 94 bis.

    No prescribirá la acción penal respecto de los crímenes y simples delitos descritos y sancionados en los artículos 141, inciso final, y 142, inciso final, ambos en relación con la violación; los artículos 150 B y 150 E, ambos en relación con los artículos 361, 362 y 365 bis; los artículos 361, 362, 363, 365 bis, 366, 366 bis, 366 quáter, 367, 367 ter, 367 quáter, 367 septies; el artículo 411 quáter en relación con la explotación sexual; y el artículo 433, N° 1, en relación con la violación, cuando al momento de la perpetración del hecho la víctima fuere menor de edad.
    En caso de que el delito previsto en el inciso primero del artículo 366 se cometiere contra mayores de edad, la prescripción de la acción penal será de diez años.



    ART. 95.

    El término de la prescripción empieza a correr desde el día en que se hubiere cometido el delito.



    ART. 96.
   
    Esta prescripción se interrumpe, perdiéndose el tiempo trascurrido, siempre que el delincuente comete nuevamente crimen o simple delito, y se suspende desde que el procedimiento se dirige contra él; pero si se paraliza su prosecución por tres años o se termina sin condenarle, continúa la prescripción como si no se hubiere interrumpido.


    ART. 97.

    Las penas impuestas por sentencia ejecutoria prescriben:
    La de presidio, reclusión y relegación perpetuos, en quince años.
    Las demás penas de crímenes, en diez años.
    Las penas de simples delitos, en cinco años.
    Las de faltas, en seis meses.




    ART. 98.

    El tiempo de la prescripción comenzará a correr desde la fecha de la sentencia de término o desde el quebrantamiento de la condena, si hubiere ésta principiado a cumplirse.


    ART. 99.

    Esta prescripción se interrumpe quedando sin efecto el tiempo trascurrido, cuando el condenado, durante ella, cometiere nuevamente crimen o simple delito, sin perjuicio de que comience a correr otra vez.



    ART. 100.

    Cuando el responsable se ausentare del territorio de la República sólo podrá prescribir la acción penal o la pena contando por uno cada dos días de ausencia, para el cómputo de los años.
    Para los efectos de aplicar la prescripción de la acción penal o de la pena, no se entenderán ausentes del territorio nacional los que hubieren estado sujetos a prohibición o impedimento de ingreso al país por decisión de la autoridad política o administrativa, por el tiempo que les hubiere afectado tal prohibición o impedimento.


    ART. 101.

    Tanto la prescripción de la acción penal como la de la pena corren a favor y en contra de toda clase de personas.


    ART. 102.

    La prescripción será declarada de oficio por el tribunal aún cuando el imputado o acusado no la alegue, con tal que se halle presente en el juicio.


    ART. 103.

    Si el responsable se presentare o fuere habido antes de completar el tiempo de la prescripción de la acción penal o de la pena, pero habiendo ya trascurrido la mitad del que se exige, en sus respectivos casos, para tales prescripciones, deberá el tribunal considerar el hecho como revestido de dos o más circunstancias atenuantes muy calificadas y de ninguna agravante y aplicar las reglas de los arts. 65, 66, 67 y 68, sea en la imposición de la pena, sea para disminuir la ya impuesta.

    Esta regla no se aplica a las prescripciones de las faltas y especiales de corto tiempo.

    ART. 104.

    Las circunstancias agravantes comprendidas en los núms. 15 y 16 del art. 12, no se tomarán en cuenta tratándose de crímenes, después de diez años, a contar desde la fecha en que tuvo lugar el hecho, ni después de cinco, en los casos de simples delitos.


    ART. 105.

    Las inhabilidades legales provenientes de crimen o simple delito sólo durarán el tiempo requerido para prescribir la pena, computado de la manera que se dispone en los arts. 98, 99 y 100. Esta regla no es aplicable a las inhabilidades para el ejercicio de los derechos políticos.
    La prescripción de la responsabilidad civil proveniente de delito, se rige por el Código civil.




    LIBRO SEGUNDO.

    CRÍMENES Y SIMPLES DELITOS Y SUS PENAS.





    TÍTULO PRIMERO.

    CRÍMENES Y SIMPLES DELITOS CONTRA LA SEGURIDAD EXTERIOR Y SOBERANÍA DEL ESTADO.




    ART. 106.

    Todo el que dentro del territorio de la República conspirare contra su seguridad exterior para inducir a una potencia extranjera a hacer la guerra a Chile, será castigado por presidio mayor en su grado máximo a presidio perpetuo. Si se han seguido hostilidades bélicas, la pena podrá elevarse hasta el presidio perpetuo calificado.
    Las prescripciones de este artículo se aplican a los chilenos, aún cuando la conspiración haya tenido lugar fuera del territorio de la República.

    ART. 107.

    El chileno que militare contra su patria bajo banderas enemigas, será castigado con presidio mayor en su grado medio a presidio perpetuo.



    ART. 108.

    Todo individuo que, sin proceder a nombre y con la autorización de una potencia extranjera hiciere armas contra Chile amenazando la independencia o integridad de su territorio, sufrirá la pena de presidio mayor en su grado máximo a presidio perpetuo.



    ART. 109.

    Será castigado con la pena de presidio mayor en su grado máximo a presidio perpetuo:

    El que facilitare al enemigo la entrada en el territorio de la República.
    El que le entregare ciudades, puertos, plazas, fortalezas, puestos, almacenes, buques, dineros u otros objetos pertenecientes al Estado, de reconocida utilidad para el progreso de la guerra.
    El que le suministrare auxilio de hombres, dinero, víveres, armas, municiones, vestuarios, carros, caballerías, embarcaciones u otros objetos conocidamente útiles al enemigo.
    El que favoreciere el progreso de las armas enemigas en el territorio de la República o contra las fuerzas chilenas de mar y tierra, corrompiendo la fidelidad de los oficiales, soldados, marineros u otros ciudadanos hacia el Estado.
    El que suministrare al enemigo planos de fortificaciones, arsenales, puertos o radas.
    El que le revelare el secreto de una negociación o de una expedición.
    El que ocultare o hiciere ocultar a los espías o soldados del enemigo enviados a la descubierta.
    El que como práctico dirigiere el ejército o armada enemigos.
    El que diere maliciosamente falso rumbo o falsas noticias al ejército o armada de la República.
    El proveedor que maliciosamente faltare a su deber con grave daño del ejército o armada.
    El que impidiere que las tropas de la República reciban auxilios de caudales, armas, municiones de boca o de guerra, equipos o embarcaciones, o los planos, instrucciones o noticias convenientes para el mejor progreso de la guerra.
    El que por cualquier medio hubiere incendiado algunos objetos con intención de favorecer al enemigo.
    En los casos de este artículo si el delincuente fuero funcionario público, agente o comisionado del Gobierno de la República, que hubiere abusado de la autoridad, documentos o noticias que tuviere por razón de su cargo, la pena será la de presidio perpetuo.





    ART. 110.

    Con la pena de presidio mayor en su grado medio a presidio perpetuo, se castigarán los crímenes enumerados en el artículo anterior cuando ellos se cometieren respecto de los aliados de la República que obran contra el enemigo común.



    ART. 111.

    En los casos de los cinco artículos precedentes el delito frustrado se castiga como si fuera consumado, la tentativa con la pena inferior en un grado a la señalada para el delito, la conspiración con la inferior en dos grados y la proposición con la de presidio menor en cualquiera de sus grados.


    ART. 112.

    Todo individuo que hubiere mantenido con los ciudadanos o súbditos de una potencia enemiga correspondencia que, sin tener en mira alguno de los crímenes enumerados en el art. 109, ha dado por resultado suministrar al enemigo noticias perjudiciales a la situación militar de Chile o de sus aliados, que obran contra el enemigo común, sufrirá la pena de presidio menor en cualquiera de sus grados.
    La misma pena se aplicará cuando la correspondencia fuere en cifras que no permitan apreciar su contenido.
    Si las noticias son comunicadas por un empleado público, que tiene conocimiento de ellas en razón de su empleo, la pena será presidio mayor en su grado medio.



    ART. 113.

    El que violare tregua o armisticio acordado entre la República y otra nación enemiga o entre sus fuerzas beligerantes de mar o tierra, sufrirá la pena de presidio menor en su grado medio.


    ART. 114.

    El que sin autorización legítima levantare tropas en el territorio de la República o destinare buques al corso, cualquiera que sea el objeto que se proponga o la nación a que intente hostilizar, será castigado con presidio mayor en su grado mínimo y multa de veintiuna a treinta unidades tributarias mensuales.






    ART. 115.

    El que violare la neutralidad de la República, comerciando con los beligerantes en artículos declarados de contrabando de guerra en los respectivos decretos o proclamas de neutralidad, será penado con presidio menor en su grado medio.
    Si un empleado público fuere autor o cómplice en este delito, se le castigará con presidio menor en su grado máximo.


    ART. 116.

    El ciudadano o súbdito de una nación con quien Chile está en guerra, que violare los decretos de internación o expulsión del territorio de la República, expedidos por el Gobierno respecto de los ciudadanos o súbditos de dicha nación, sufrirá la pena de reclusión menor en su grado medio; no pudiendo ésta en ningún caso, extenderse más allá de la duración de la guerra que motivó aquellas medidas.


    ART. 117.

    El chileno culpable de tentativa para pasar a país enemigo cuando lo hubiere prohibido el Gobierno, será castigado con la pena de reclusión menor en su grado mínimo.


    ART. 118.

    El que ejecutare en la República cualesquiera órdenes o disposiciones de un Gobierno extranjero, que ofendan la independencia o seguridad del Estado, incurrirá en la pena de extrañamiento menor en sus grados mínimo a medio.


    ART. 119.

    Si un empleado público, abusando de su oficio, cometiere cualquiera de los simples delitos de que se trata en el artículo anterior, se le impondrá además de la pena señalada en él, la de inhabilitación absoluta temporal para cargos y oficios públicos en su grado mínimo.


    ART. 120.

    El que violare la inmunidad personal o el domicilio del representante de una potencia extranjera, será castigado con reclusión menor en su grado mínimo, a menos que tal violación importe un delito que tenga señalada pena mayor, debiendo en tal caso ser considerada aquélla como circunstancia agravante.


    TÍTULO SEGUNDO.

    CRÍMENES Y SIMPLES DELITOS CONTRA LA SEGURIDAD INTERIOR DEL ESTADO.








    ART. 121.

    Los que se alzaren a mano armada contra el Gobierno legalmente constituido con el objeto de promover la guerra civil, de cambiar la Constitución del Estado o su forma de gobierno, de privar de sus funciones o impedir que entren en el ejercicio de ellas al Presidente de la República o al que haga sus veces, a los miembros del Congreso Nacional o de los Tribunales Superiores de Justicia, sufrirán la pena de reclusión mayor, o bien la de confinamiento mayor o la de extrañamiento mayor, en cualesquiera de sus grados.


    ART. 122.

    Los que induciendo a los alzados, hubieren promovido o sostuvieren la sublevación y los caudillos principales de ésta, serán castigados con las mismas penas del artículo anterior, aplicadas en sus grados máximos.


    ART. 123.

    Los que tocaren o mandaren tocar campanas u otro instrumento cualquiera para excitar al pueblo al alzamiento y los que, con igual fin, dirigieren discursos a la muchedumbre o le repartieren impresos, si la sublevación llega a consumarse, serán castigados con la pena de reclusión menor o de extrañamiento menor en sus grados medios, a no ser que merezcan la calificación de promovedores.


    ART. 124.

    Los que sin cometer los crímenes enumerados en el art. 121, pero con el propósito de ejecutarlos, sedujeren tropas, usurparen el mando de ellas, de un buque de guerra, de una plaza fuerte, de un puesto de guardia, de un puerto o de una ciudad, o retuvieren contra la orden del Gobierno un mando político o militar cualquiera, sufrirán la pena de reclusion mayor o de confinamiento mayor en sus grados medios.



NOTA
      El artículo 25 de la Ley N° 6.026, sobre Seguridad Interior el Estado, M. Interior, publicada el 12.02.1937, derogó el Decreto Ley N° 672 de 1925, norma que en su artículo 1° había incorporado los incisos segundo, tercero y cuarto al presente artículo. El texto oficial de la Editorial Jurídica no contempla tales incisos, razón por la cual se han eliminado en este texto actualizado.

    ART. 125.

    En los crímenes de que tratan los arts. 121, 122 y 124, la conspiración se pena con extrañamiento mayor en su grado medio y la proposición con extrañamiento menor en su grado medio.


    ART. 126.

    Los que se alzaren públicamente con el propósito de impedir la promulgación o la ejecución de las leyes, la libre celebración de una elección popular, de coartar el ejercicio de sus atribuciones o la ejecución de sus providencias a cualquiera de los Poderes Constitucionales, de arrancarles resoluciones por medio de la fuerza o de ejercer actos de odio o de venganza en la persona o bienes de alguna autoridad o de sus agentes o en las pertenencias del Estado o de alguna corporación pública, sufrirán la pena de reclusión menor o bien la de confinamiento menor o de extrañamiento menor en cualesquiera de sus grados.


    ART. 127.

    Las prescripciones de los arts. 122, 123, 124 y 125 tienen aplicación respecto de los simples delitos de que trata el artículo precedente, siendo las penas respectivamente inferiores en un grado a las que en dichos artículos se establecen.


    ART. 128.

    Luego que se manifieste la sublevación, la autoridad intimará hasta dos veces a los sublevados que inmediatamente se disuelvan y retiren, dejando pasar entre una y otra intimación el tiempo necesario para ello.
    Si los sublevados no se retiraren inmediatamente después de la segunda intimación, la autoridad hará uso de la fuerza pública para disolverlos.
    No serán necesarias respectivamente, la primera o la segunda intimación, desde el momento en que los sublevados ejecuten actos de violencia.


    ART. 129.

    Cuando los sublevados se disolvieren o sometieren a la autoridad legítima antes de las intimaciones o a consecuencia de ellas sin haber ejecutado actos de violencia, quedarán exentos de toda pena.
    Los instigadores, promovedores y sostenedores de la sublevación, en el caso del presente artículo, serán castigados con una pena inferior en uno o dos grados a la que les hubiera correspondido consumado el delito.


    ART. 130.

    En el caso de que la sublevación no llegare a agravarse hasta el punto de embarazar de una manera sensible el ejercicio de la autoridad pública, serán juzgados los sublevados con arreglo a lo que se previene en el inciso final del artículo anterior.


    ART. 131.

    Los delitos particulares cometidos en una sublevación o con motivo de ella, serán castigados respectivamente, con las penas designadas para ellos, no obstante lo dispuesto en el art. 129.
    Si no pueden descubrirse los autores, serán considerados y penados como cómplices de tales delitos los jefes principales o subalternos de los sublevados, que hallándose en la posibilidad de impedirlos, no lo hubieren hecho.


    ART. 132.

    Cuando en las sublevaciones de que trata este título se supone uso de armas, se comprenderá bajo esta palabra toda máquina, instrumento, utensilio u objeto cortante, punzante o contundente que se haya tomado para matar, herir o golpear, aún cuando no se haya hecho uso de él.


    ART. 133.

    Los que por astucia o por cualquier otro medio, pero sin alzarse contra el Gobierno, cometieren alguno de los crímenes o simples delitos de que tratan los arts. 121 y 126, serán penados con reclusión o relegación menores en cualquiera de sus grados, salvo lo dispuesto en el art. 137 respecto de los delitos que conciernen al ejercicio de los derechos políticos.


    ART. 134.

    Los empleados públicos que debiendo resistir la sublevación por razón de su oficio, no lo hubieren hecho por todos los medios que estuvieren a sus alcances, sufrirán la pena de inhabilitación absoluta temporal para cargos y oficios públicos en cualquiera de sus grados.



    ART. 135

    Los empleados que continuaren funcionando bajo las órdenes de los sublevados o que sin haberles admitido la renuncia de su empleo, lo abandonaren cuando haya peligro de alzamiento, incurrirán en la pena de inhabilitación absoluta temporal para cargos y oficios públicos en sus grados medio a máximo.


    ART. 136.

    Los que aceptaren cargos o empleos de los sublevados, serán castigados con inhabilitación absoluta temporal para cargos y oficios públicos en su grado mínimo y multa de once a veinte unidades tributarias mensuales.
   




NOTA
      El artículo 25 de la Ley N° 6.026, sobre Seguridad Interior el Estado, M. Interior, publicada el 12.02.1937, derogó el Decreto Ley N° 672 de 1925, norma que en su artículo 2° agregaba, a continuación del presente artículo, lo siguiente: "El que de hecho o palabra hiciere objeto de mofa o de desprecio a la bandera o himno nacional de la República, será castigado con pena de prisión, en cualquiera de sus grados, y multas de veinte a mil pesos.". Sin embargo, no lo contempla el texto oficial de la Editorial Jurídica, razón por la cual se ha eliminado en este texto actualizado.

    TÍTULO TERCERO.

    DE LOS CRÍMENES Y SIMPLES DELITOS QUE AFECTAN LOS DERECHOS GARANTIDOS POR LA CONSTITUCIÓN.





    § I.

De los delitos relativos al ejercicio de los derechos políticos y a la libertad de imprenta.





    ART. 137.

    Los delitos relativos al libre ejercicio del sufragio y a la libertad de emitir opiniones por la prensa, se clasifican y penan respectivamente por las leyes de elecciones y de imprenta.


    § II.

    De los crímenes y simples delitos relativos al ejercicio de los cultos permitidos en la República.





    ART. 138.

    Todo el que por medio de violencia o amenazas hubiere impedido a uno o más individuos el ejercicio de un culto permitido en la República, será castigado con reclusión menor en su grado mínimo.



    ART. 139.

    Sufrirán la pena de reclusión menor en su grado mínimo y multa de seis a diez unidades tributarias mensuales:
    1° Los que con tumulto o desorden hubieren impedido, retardado o interrumpido el ejercicio de un culto que se practicaba en lugar destinado a él o que sirve habitualmente para celebrarlo, o en las ceremonias públicas de ese mismo culto.
    2° Los que con acciones, palabras o amenazas ultrajaren los objetos de un culto, sea en los lugares destinados a él o que sirven habitualmente para su ejercicio, sea en las ceremonias públicas de ese mismo culto.
    3.° Los que con acciones, palabras o amenazas ultrajaren, al ministro de un culto en el ejercicio de su ministerio.





    ART. 140. 

    Cuando en el caso del núm. 3.° del artículo precedente, la injuria fuere de hecho, poniendo manos violentas sobre la persona del ministro, el delincuente sufrirá las penas de reclusión menor en sus grados mínimo a medio y multa de seis a diez unidades tributarias mensuales.
    Si los golpes causaren al ofendido algunas de las lesiones a que se refiere el art. 399, la pena será presidio menor en su grado medio; cuando las lesiones fueren de las comprendidas en el núm. 2.° del art. 397, se castigarán con presidio menor en su grado máximo; si fueren de las que relaciona el núm. 1.° de dicho artículo, con presidio mayor en su grado medio, y cuando de las lesiones resultare la muerte del paciente, se impondrá al ofensor la pena de presidio mayor en su grado máximo a presidio perpetuo.








    § III.

    Crímenes y simples delitos contra la libertad y seguridad, cometidos por particulares.





    ART. 141.

    El que sin derecho encerrare o detuviere a otro privándole de su libertad, comete el delito de secuestro y será castigado con la pena de presidio o reclusión menor en su grado máximo.
    En la misma pena incurrirá el que proporcionare lugar para la ejecución del delito.
    Si se ejecutare para obtener un rescate o imponer exigencias o arrancar decisiones, o si el encierro o detención se prolongare por más de 24 horas, será castigado con la pena de presidio mayor en su grado mínimo a medio.
    Si en cualesquiera de los casos anteriores, el encierro o la detención se prolongare por más de quince días o si de ello resultare un daño grave en la persona o intereses del secuestrado, la pena será presidio mayor en su grado medio a máximo.
    El que con motivo u ocasión del secuestro cometiere además homicidio, violación o algunas de las lesiones comprendidas en los artículos 395, 396 y 397 N° 1, en la persona del ofendido, será castigado con presidio perpetuo a presidio perpetuo calificado.






    ART. 142.

    La sustracción de un menor de 18 años será castigada:

    1.- Con presidio mayor en su grado máximo a presidio perpetuo, si se ejecutare para obtener un rescate, imponer exigencias, arrancar decisiones o si resultare un grave daño en la persona del menor.
    2.- Con presidio mayor en su grado medio a máximo en los demás casos.
    Si con motivo u ocasión de la sustracción se cometiere alguno de los delitos indicados en el inciso final del artículo anterior, se aplicará la pena que en él se señala.


    ART. 142 bis.

    Si los partícipes en los delitos de secuestro de una persona o de sustracción de un menor, antes de cumplir cualquiera de las condiciones exigidas por los secuestradores para devolver a la victima, la devolvieren libre de todo daño, la pena asignada al delito se rebajará en dos grados. Si la devolución se realiza después de cumplida alguna de las condiciones, el juez podrá rebajar la pena en un grado a la señalada en los dos artículos anteriores.


    ART. 143.

    El que fuera de los casos permitidos por la ley, aprehendiere a una persona para presentarla a la autoridad, sufrirá la pena de reclusión menor en su grado mínimo o multa de seis a diez unidades tributarias mensuales.






    ART. 144.

    El que entrare en morada ajena contra la voluntad de su morador, será castigado con reclusión menor en su grado mínimo o multa de seis a diez unidades tributarias mensuales.
    Si el hecho se ejecutare con violencia o intimidación, el tribunal podrá aplicar la reclusión menor hasta en su grado medio y elevar la multa hasta quince sueldos vitales.




    ART. 145.

    La disposición del artículo anterior no es aplicable al que entra en la morada ajena para evitar un mal grave a sí mismo, a los moradores o a un tercero, ni al que lo hace para prestar algún auxilio a la humanidad o a la justicia.
    Tampoco tiene aplicación respecto de los cafés, tabernas, posadas y demás casas públicas, mientras estuvieren abiertos y no se usare de violencia inmotivada.


    ART. 146.

    El que abriere o registrare la correspondencia o los papeles de otro sin su voluntad, sufrirá la pena de reclusión menor en su grado medio si divulgare o se aprovechare de los secretos que ellos contienen, y en el caso contrario la de reclusión menor en su grado mínimo.
    Esta disposición no es aplicable entre cónyuges, convivientes civiles, ni a los padres, guardadores o quienes hagan sus veces, en cuanto a los papeles o cartas de sus hijos o menores que se hallen bajo su dependencia.
    Tampoco es aplicable a aquellas personas a quienes por leyes o reglamentos especiales, les es lícito instruirse de correspondencia ajena.




    ART. 147.

    El que bajo cualquier pretexto, impusiere a otros contribuciones o les exigiere, sin título para ello, servicios personales, incurrirá en las penas de reclusión menor en sus grados mínimo a medio y multa de once a veinte unidades tributarias mensuales.







    § IV

    De la tortura, otros tratos crueles, inhumanos o degradantes, y de otros agravios inferidos por funcionarios públicos a los derechos garantidos por la Constitución.








    ART. 148.

    Todo empleado público que ilegal y arbitrariamente desterrare, arrestare o detuviere a una persona, sufrirá la pena de reclusión menor y suspensión del empleo en sus grados mínimos a medios.
    Si el arresto o detención excediere de treinta días, las penas serán reclusión menor y suspensión en sus grados máximos.


    ART. 149.

    Serán castigados con las penas de reclusión menor y suspensión en sus grados mínimos a medios:
    1.° Los que encargados de un establecimiento penal, recibieren en él a un individuo en calidad de preso o detenido sin haberse llenado los requisitos prevenidos por la ley.
    2.° Los que habiendo recibido a una persona en clase de detenida, no dieren parte al tribunal competente dentro de las veinte y cuatro horas siguientes.
    3.° Los que impidieren comunicarse a los detenidos con el juez que conoce de su causa y a los rematados con los magistrados encargados de visitar los respectivos establecimientos penales.
    4.° Los encargados de los lugares de detención que se negaren a trasmitir al tribunal, a requisición del preso, copia del decreto de prisión, o a reclamar para que se dé dicha copia, o a dar ellos mismos un certificado de hallarse preso aquel individuo.
    5.° Los que teniendo a su cargo la policía administrativa o judicial y sabedores de cualquiera detención arbitraria, no la hicieren cesar, teniendo facultad para ello, o en caso contrario dejaren de dar parte a la autoridad superior competente.
    6.° Los que habiendo hecho arrestar a un individuo no dieren parte al tribunal competente dentro de las cuarenta y ocho horas, poniendo al arrestado a su disposición.
    En los casos a que se refieren los núms. 2.°, 5.° y 6.° de este artículo, los culpables incurrirán respectivamente en las penas del artículo anterior, si pasaren más de tres días sin cumplir con las obligaciones cuya ejecución se castiga en tales números.




    ART. 150.

    Sufrirá las penas de presidio o reclusión menores y la accesoria que corresponda:

    1º. El que incomunicare a una persona privada de libertad o usare con ella de un rigor innecesario, y
    2º. El que arbitrariamente hiciere arrestar o detener en otros lugares que los establecidos por la ley.
   
    Al que, sin revestir la calidad de empleado público, participare en la comisión de estos delitos, se le impondrá la pena de presidio o reclusión menor en su grado mínimo a medio.

    ART. 150 A.

    El empleado público que, abusando de su cargo o sus funciones, aplicare, ordenare o consintiere en que se aplique tortura, será penado con presidio mayor en su grado mínimo. Igual sanción se impondrá al empleado público que, conociendo de la ocurrencia de estas conductas, no impidiere o no hiciere cesar la aplicación de tortura, teniendo la facultad o autoridad necesaria para ello o estando en posición para hacerlo.
    La misma pena se aplicará al particular que, en el ejercicio de funciones públicas, o a instigación de un empleado público, o con el consentimiento o aquiescencia de éste, ejecutare los actos a que se refiere este artículo.
    Se entenderá por tortura todo acto por el cual se inflija intencionalmente a una persona dolores o sufrimientos graves, ya sean físicos, sexuales o psíquicos, con el fin de obtener de ella o de un tercero información, declaración o una confesión, de castigarla por un acto que haya cometido, o se le impute haber cometido, o de intimidar o coaccionar a esa persona, o en razón de una discriminación fundada en motivos tales como la ideología, la opinión política, la religión o creencias de la víctima; la nación, la raza, la etnia o el grupo social al que pertenezca; el sexo, la orientación sexual, la identidad de género, la edad, la filiación, la apariencia personal, el estado de salud o la situación de discapacidad.
    Se entenderá también por tortura la aplicación intencional de métodos tendientes a anular la personalidad de la víctima, o a disminuir su voluntad o su capacidad de discernimiento o decisión, con alguno de los fines referidos en el inciso precedente. Esta conducta se sancionará con la pena de presidio menor en su grado máximo.
    No se considerarán como tortura las molestias o penalidades que sean consecuencia únicamente de sanciones legales, o que sean inherentes o incidentales a éstas, ni las derivadas de un acto legítimo de autoridad.

    ART. 150 B.

    Si con ocasión de la tortura se cometiere además:
    1° Homicidio, se aplicará la pena de presidio mayor en su grado máximo a presidio perpetuo calificado.
    2° Alguno de los delitos previstos en los artículos 361, 362, 365 bis, 395, 396 o 397, número 1°, la pena será de presidio mayor en su grado máximo a presidio perpetuo.
    3° Alguno de los cuasidelitos a que se refiere el artículo 490, número 1°, la pena será de presidio mayor en su grado medio.

    ART. 150 C.-

    En los casos previstos en los artículos 150 A y 150 B se excluirá el mínimum o el grado mínimo de la pena señalada, según corresponda, al que torture a otro que se encuentre, legítima o ilegítimamente, privado de libertad, o en cualquier caso bajo su cuidado, custodia o control.

    ART. 150 D.-

    El empleado público que, en incumplimiento de los reglamentos respectivos actúe abusando de su cargo o que en el ejercicio de sus funciones, aplique, ordene o consienta en que se apliquen apremios ilegítimos u otros tratos crueles, inhumanos o degradantes, que no alcancen por su gravedad a constituir tortura, será castigado con las penas de presidio menor en sus grados medio a máximo y la accesoria correspondiente. Igual sanción se impondrá al empleado público que, conociendo de la ocurrencia de estas conductas, no impida o no haga cesar la aplicación de los apremios o de los otros tratos, teniendo la facultad o autoridad necesaria para ello y estando en posición para hacerlo.
    Si la conducta descrita en el inciso precedente se comete en contra de una persona menor de edad o en situación de vulnerabilidad por discapacidad, enfermedad o vejez, la pena se aumentará en un grado.
    No se considerarán como apremios ilegítimos u otros tratos crueles, inhumanos o degradantes las molestias o penalidades que sean consecuencia únicamente de sanciones legales, o que sean inherentes o incidentales a éstas, ni las derivadas de un acto legítimo de autoridad.
    Sin perjuicio de lo dispuesto en los incisos anteriores, si los hechos constituyeren algún delito o delitos de mayor gravedad, se estará a la pena señalada para ellos.


    ART. 150 E.-

    Si con ocasión de los apremios ilegítimos u otros tratos crueles, inhumanos o degradantes se cometiere además:
    1° Homicidio, se aplicará la pena de presidio mayor en su grado máximo a presidio perpetuo.
    2° Alguno de los delitos previstos en los artículos 361, 362, 365 bis, 395, 396 o 397, número 1°, la pena será de presidio mayor en su grado medio.
    3° Alguno de los cuasidelitos a que se refiere el artículo 490, número 1°, la pena será de presidio menor en su grado máximo a presidio mayor en su grado mínimo.

    ART. 150 F.-

    La misma pena se aplicará al particular que, en el ejercicio de funciones públicas, o a instigación de un empleado público, o con el consentimiento o aquiescencia de éste, ejecutare los actos a que se refieren los artículos 150 D o 150 E.

    ART. 151.

    El empleado público que en el arresto o formación de causa contra un senador, un diputado u otro funcionario, violare las prerrogativas que la ley les acuerda, incurrirá en la pena de reclusión menor o suspensión en cualesquiera de sus grados.


    ART. 152.

    Los empleados públicos que arrogándose facultades judiciales, impusieren algún castigo equivalente a pena corporal, incurrirán:
    1.° En inhabilitación absoluta temporal para cargos y oficios públicos en cualquiera de sus grados, si el castigo impuesto fuere equivalente a pena de crimen.
    2.° En la misma inhabilitación en sus grados mínimo a medio, cuando fuere equivalente a pena de simple delito.
    3.° En suspensión de cargo u oficio en cualquiera de sus grados, si fuere equivalente a pena de falta.



    ART. 153.

    Si el castigo arbitrariamente impuesto se hubiere ejecutado en todo o en parte, además de las penas del artículo anterior se aplicará al empleado culpable la de presidio o reclusión menores o mayores en cualquiera de sus grados, atendidas las circunstancias y naturaleza del castigo ejecutado.
    Cuando no hubiere tenido efecto por revocación espontánea del mismo empleado antes de ser intimado al penado, no incurrirá aquél en responsabilidad.


    ART. 154.

    Si la pena arbitrariamente impuesta fuere pecuniaria, el empleado culpable será castigado:
    1.° Con inhabilitación absoluta temporal para cargos y oficios públicos en sus grados mínimo a medio y multa del tanto al triple de la pena impuesta, cuando ésta se hubiere ejecutado.
    2° Con suspensión de cargo u oficio en su grado mínimo y multa de la mitad al tanto, si la pena no se hubiere ejecutado.   
    Cuando no hubiere tenido efecto por revocación voluntaria del empleado antes de intimarse al penado, no incurrirá aquél en responsabilidad.



    ART. 155.

    El empleado público que abusando de su oficio, allanare un templo o la casa de cualquiera persona o hiciere registro en sus papeles, a no ser en los casos y forma que prescriben las leyes, será castigado con la pena de reclusión menor en sus grados mínimo a medio o con la de suspensión en cualquiera de sus grados.


    ART. 156.

    Los empleados en el servicio de correos y telégrafos u otros que prevaliéndose de su autoridad interceptaren o abrieren la correspondencia o facilitaren a tercero su apertura o supresión, sufrirán la pena de reclusión menor en su grado mínimo y, si se aprovecharen de los secretos que contiene o los divulgaren, las penas serán reclusión menor en cualquiera de sus grados y multa de once a veinte unidades tributarias mensuales.
    En los casos de retardo doloso en el envío o entrega de la correspondencia epistolar o de partes telegráficos, la pena será reclusión menor en su grado mínimo.





    ART. 157.

    Todo empleado público que sin un decreto de autoridad competente, deducido de la ley que autoriza la exacción de una contribución o de un servicio personal, los exigiere bajo cualquier pretexto, será penado con inhabilitación absoluta temporal para cargos y oficios públicos en cualquiera de sus grados y multa de once a veinte unidades tributarias mensuales.
    Si la exacción de la contribución o servicio personal se hiciere con ánimo de lucro, el empleado culpable será sancionado conforme a lo dispuesto en los párrafos 2 u 8 del Título IX, según corresponda.


    Art. 158.

    Sufrirá la pena de suspensión en sus grados mínimo a medio, si gozare de renta, y la de reclusión menor en su grado mínimo o multa de once a veinte unidades tributarias mensuales, cuando prestare servicios gratuitos, el empleado público que arbitrariamente:
    1.° Derogado.
    2.° Prohibiere un trabajo o industria que no se oponga a la ley, a las buenas costumbres, seguridad y salubridad públicas.
    3.° Prohibiere o impidiere una reunión o manifestación pacífica y legal o la mandare disolver o suspender.
    4.° Impidiere a un habitante de la República permanecer en cualquier punto de ella, trasladarse de uno a otro o salir de su territorio, en los casos que la ley no lo prohíba; concurrir a una reunión o manifestación pacífica y legal; formar parte de cualquier asociación lícita, o hacer uso del derecho de petición que le garantiza la ley.
    5.° Privare a otro de la propiedad exclusiva de su descubrimiento o producción, o divulgare los secretos del invento, que hubiere conocido por razón de su empleo.
    6.° Expropiare a otro de sus bienes o le perturbare en su posesión, a no ser en los casos que permite la ley.







    ART. 159.

    Si en los casos de los artículos anteriores de este párrafo, aquél a quien se atribuyere responsabilidad justificare que ha obrado por orden de sus superiores a quienes debe obediencia disciplinaria, las penas señaladas en dichos artículos se aplicarán sólo a los superiores que hayan dado la orden.


    ART. 160.

    Si un empleado público acusado de haber ordenado, autorizado o facilitado alguno de los actos de que se trata en el presente párrafo, pretende que la orden lo ha sido arrancada por sorpresa, será obligado, revocando desde luego tal orden para hacer cesar el acto, a denunciar al culpable; en caso de no denunciarlo, responderá personalmente.




    ART. 161.

    Cuando para llevar a efecto alguno de los delitos enunciados, se hubiere falsificado o supuesto la firma de un funcionario público, los autores y los que maliciosa o fraudulentamente hubieren usado de la falsificación o suposición, serán castigados con presidio menor en su grado máximo.

    § 5. De los delitos contra el respeto y protección a la vida privada y pública de la persona y su familia.

    ART. 161 - A.
    Se castigará con la pena de reclusión menor en cualquiera de sus grados y multa de 50 a 500 Unidades Tributarias Mensuales al que, en recintos particulares o lugares que no sean de libre acceso al público, sin autorización del afectado y por cualquier medio, capte, intercepte, grabe o reproduzca conversaciones o comunicaciones de carácter privado; sustraiga, fotografíe, fotocopie o reproduzca documentos o instrumentos de carácter privado; o capte, grabe, filme o fotografíe imágenes o hechos de carácter privado que se produzcan, realicen, ocurran o existan en recintos particulares o lugares que no sean de libre acceso al público.
    Igual pena se aplicará a quien difunda las conversaciones, comunicaciones, documentos, instrumentos, imágenes y hechos a que se refiere el inciso anterior.
    En caso de ser una misma la persona que los haya obtenido y divulgado, se aplicarán a ésta las penas de reclusión menor en su grado máximo y multa de 100 a 500 Unidades Tributarias Mensuales.
    Esta disposición no es aplicable a aquellas personas que, en virtud de ley o de autorización judicial, estén o sean autorizadas para ejecutar las acciones descri tas.
    ART. 161-B.
    Se castigará con la pena de reclusión menor en su grado máximo y multa de 100 a 500 Unidades Tributarias Mensuales, al que pretenda obtener la entrega de dinero o bienes o la realización de cualquier conducta que no sea jurídicamente obligatoria, mediante cualquiera de los actos señalados en el artículo precedente. En el evento que se exija la ejecución de un acto o hecho que sea constitutivo de delito, la pena de reclusión se aplicará aumentada en un grado.


    ART. 161-C.

    Se castigará con la pena de presidio menor en su grado mínimo y multa de cinco a diez unidades tributarias mensuales, al que en lugares públicos o de libre acceso público y que por cualquier medio capte, grabe, filme o fotografíe imágenes, videos o cualquier registro audiovisual, de los genitales u otra parte íntima del cuerpo de otra persona con fines de significación sexual y sin su consentimiento.
    Se impondrá la misma pena de presidio menor en su grado mínimo y multa de diez a veinte unidades tributarias mensuales, al que difunda dichas imágenes, videos o registro audiovisual a que se refiere el inciso anterior.
    En caso de ser una misma la persona que los haya obtenido y divulgado, se aplicarán a ésta, la pena de presidio menor en su grado mínimo a medio y multa de veinte a treinta unidades tributarias mensuales.




    TÍTULO CUARTO.

DE LOS CRÍMENES Y SIMPLES DELITOS CONTRA LA FE PÚBLICA, DE LAS FALSIFICACIONES, DEL FALSO TESTIMONIO Y DEL PERJURIO.





    § I.

    De la moneda falsa.





    ART. 162.

    El que sin autorización fabricare moneda que tenga curso legal en la República, aunque sea de la misma materia, peso y ley que la legítima, sufrirá las penas de reclusión menor en su grado mínimo y multa de seis a diez unidades tributarias mensuales.
    Cuando el peso o la ley fueren inferiores a los legales, las penas serán presidio menor en su grado medio y multa de seis a quince unidades tributarias mensuales.






    ART. 163.

    El que falsificare moneda de oro o plata que tenga curso legal, empleando otras sustancias diversas, será castigado con presidio menor en sus grados medio a máximo y multa de once a veinte unidades tributarias mensuales.
    Si la moneda falsificada fuere de vellón, las penas serán presidio menor en sus grados mínimo a medio y multa de seis a diez unidades tributarias mensuales.





    ART. 164.

    El que cercenare moneda de oro o plata de curso legal, sufrirá las penas de presidio menor en sus grados mínimo a medio y multa de seis a diez unidades tributarias mensuales.








    ART. 165.

    El que falsificare moneda que no tenga curso legal en la República, será castigado con presidio menor en su grado medio y multa de seis a quince unidades tributarias mensuales, si la moneda falsificada fuere de oro o plata, y con presidio menor en su grado mínimo y multa de seis a diez unidades tributarias mensuales, cuando fuere de vellón.




    ART. 166.

    El que cercenare moneda de oro o plata que no tenga curso legal en la República, sufrirá las penas de presidio menor en su grado mínimo y multa de seis a diez unidades tributarias mensuales.








    ART. 167.
    El que de concierto con los falsificadores o cercenadores, tomare parte en la emisión o introducción a la República de la moneda falsificada o cercenada, será castigado con las mismas penas que por la falsificación o cercenamiento corresponderían a aquéllos según los artículos anteriores.


    ART. 168.

    El que, sin ser culpable de la participación a que se refiere el artículo precedente, se hubiere procurado a sabiendas moneda falsificada o cercenada y la pusiere en circulación, sufrirá las penas de presidio menor en sus grados mínimo a medio y multa de once a veinte unidades tributarias mensuales.








    ART. 169.

    La tentativa respecto de cualquiera de los delitos de que tratan los artículos precedentes, será castigada con el mínimum de las penas establecidas en ellos para el delito consumado.


    ART. 170.

    El que habiendo recibido de buena fe moneda falsa o cercenada, la circulare después de constarle su falsedad o cercenamiento, sufrirá la pena de reclusión menor en su grado mínimo o multa seis a diez unidades tributarias mensuales, si el valor de la moneda circulada subiere una unidad tributaria mensual.
    Inciso Derogado.





    ART. 171.

    Si la falsificación o cercenamiento fueren tan ostensibles que cualquiera pueda notarlos y conocerlos a la simple vista, los que fabricaren, cercenaren, expendieren, introdujeren o circularen la moneda así falsificada o cercenada podrán ser castigados como responsables de estafas y otros engaños, con las penas que se establecen en el título respectivo.

    § II.

De la falsificación de documentos de crédito del Estado, de las Municipalidades, de los establecimientos públicos, sociedades anónimas o bancos de emisión legalmente autorizados.





    ART. 172.

    El que falsificare bonos emitidos por el Estado, cupones de intereses correspondientes a estos bonos, billetes de banco al portador, cuya emisión estuviere autorizada por una ley de la República, será castigado con las penas de presidio menor en su grado máximo a presidio mayor en su grado mínimo y multa veintiuna a veinticinco unidades tributarias mensuales.









    ART. 173.

    El que falsificare obligaciones al portador de la deuda pública de un país extranjero, cupones de intereses correspondientes a estos títulos o billetes de banco al portador, cuya emisión estuviere autorizada por una ley de ese país extranjero, sufrirá las penas de presidio menor en su grado medio y multa de seis a diez unidades tributarias mensuales.








    ART. 174

    El que falsificare acciones o promesas de acciones de sociedades anónimas, obligaciones u otros títulos legalmente emitidos por las municipalidades o establecimientos públicos de cualquiera denominación, o cupones de intereses o de dividendos correspondientes a estos diversos títulos, será castigado con presidio menor en sus grados medio a máximo y multa de once a veinte unidades tributarias mensuales, si la emisión hubiere tenido lugar en Chile, y con presidio menor en su grado medio y multa de seis a diez unidades tributarias mensuales, cuando hubiere tenido lugar en el extranjero.




    ART. 175.

    La misma pena que correspondería al falsificador se impondrá al que de concierto con él tomare parte en la emisión o introducción a la República de los bonos, acciones, obligaciones, billetes o cupones falsificados.


    ART. 176.

    El que sin ser culpable de la participación a que se refiere el artículo anterior, se hubiere procurado a sabiendas y emitido esos bonos, acciones, obligaciones, billetes o cupones falsificados, sufrirá las penas de presidio menor en sus grados mínimo a medio y multa de once a veinte unidades tributarias mensuales.








    ART. 177.

    La tentativa para la falsificación, emisión o introducción de tales títulos, se castigará con el mínimum, de las penas señaladas al delito consumado.




    ART. 178.

    El que habiendo adquirido de buena fe los títulos falsos de que trata este párrafo, los circulare después, constándole su falsedad, sufrirá la pena de reclusión menor en su grado mínimo o multa de seis a diez unidades tributarias mensuales, si subiere de medio sueldo vital el valor del título circulado.
    Cuando no exceda de esta suma, estimándose el acto mera falta, se penará como tal.





    ART. 179.

    Si la falsificación fuere tan grosera y ostensible que cualquiera pueda notarla y conocerla a la simple vista, los que falsificaren, expendieren, introdujeren o circularen los títulos así falsificados, podrán ser castigados como responsables de estafas y otros engaños con las penas que se establecen en el título respectivo.

    § III.

De la falsificación de sellos, punzones, matrices, marcas, papel sellado, timbres, estampillas, etc.





    ART. 180.

    El que falsificare el sello del Estado o hiciere uso del sello falso, sufrirá la pena de presidio mayor en su grado medio.


    ART. 181.

    El que falsificare punzones, cuños o cuadrados destinados a la fabricación de moneda; punzones, matrices, clisés, planchas o cualesquiera otros objetos que sirvan para la fabricación de bonos, acciones, obligaciones, cupones de intereses o de dividendos, o billetes de banco cuya emisión haya sido autorizada por la ley; timbres, planchas o cualesquiera otros objetos destinados a la fabricación de papel sellado o estampillas, o el que hiciere uso de estos sellos o planchas falsos, será castigado con presidio mayor en sus grados mínimo a medio y multa de veintiuna a treinta unidades tributarias mensuales.







    ART. 182.

    El que de concierto con los falsificadores tomare parte en la emisión del papel sellado o estampillas falsificados, sufrirá las penas de presidio mayor en su grado mínimo y multa de veintiuna a veinticinco unidades tributarias mensuales.








    ART. 183.

    El que sin ser culpable de la participación a que se refiere el artículo anterior, se hubiere procurado a sabiendas papel sellado o estampillas falsos y los emitiere introdujere en la República, será castigado con presidio menor en sus grados mínimo a medio y multa de once a veinte unidades tributarias mensuales.
    Las penas serán presidio menor en su grado mínimo multa de seis a diez unidades tributarias mensuales, si habiéndose procurado a sabiendas papel sellado o estampillas falsos, se hubiere hecho uso de ellos.



    ART. 184.

    Cuando la falsificación fuere tan mal ejecutada que cualquiera pueda notarla y conocerla a la simple vista, los que la hubieren efectuado y los que expendieren o introdujeren el papel sellado o las estampillas así falsificados, podrán ser castigados como responsables de estafas y otros engaños con las penas que se establecen en el Título respectivo.


    ART. 185.

    El que falsificare boletas para el trasporte de personas o cosas, o para reuniones o espectáculos públicos, con el propósito de usarlas o de circularlas fraudulentamente, y el que a sabiendas de que son falsificadas las usare o circulare; el que falsificare el sello, timbre o marca de una autoridad cualquiera, de un establecimiento privado de banco, de industria o de comercio, o de un particular, o hiciere uso de los sellos, timbres o marcas falsos, sufrirá la pena de presidio menor en cualquiera de sus grados y multa de once a veinte unidades tributarias mensuales.






    ART. 186.

    El que habiéndose procurado indebidamente los verdaderos sellos, timbres, punzones, matrices o marcas que tengan alguno de los destinos expresados en los arts. 180 y 181, hiciere de ellos una aplicación o uso perjudicial a los derechos e intereses del Estado, de una autoridad cualquiera o de un particular, será castigado con presidio menor en cualquiera de sus grados y multa de once a veinte unidades tributarias mensuales.








    ART. 187.

    El que falsificare los sellos, timbres, punzones, matrices o marcas, que tengan alguno de los destinos expresados en los arts. 180 y 181 y que pertenezcan a países extranjeros, o el que hiciere uso de dichos sellos, timbres, punzones, matrices o marcas falsos, sufrirá las penas de presidio menor en sus grados mínimo a medio y multa de seis a diez unidades tributarias mensuales.








    ART. 188.

    Las penas serán presidio menor en sus grados mínimo a medio y multa de once a veinte unidades tributarias mensuales, cuando habiéndose procurado indebidamente los verdaderos sellos, timbres, punzones, matrices o marcas, se hubiere hecho de ellos en Chile una aplicación o uso perjudicial a los derechos e intereses de esos países, de una autoridad cualquiera o de un particular.




      ART. 189.
      El que hiciere desaparecer de estampillas de correos u otras adhesivas, o de boletas para el transporte de personas o cosas, la marca que indica que ya han servido, con el fin de utilizarlas, y el que, a sabiendas espendiere o usare estampillas o boletas de las cuales se ha hecho desaparecer dicha marca, siempre que en uno y otro caso el valor de tales estampillas o boletas exceda de una unidad tributaria mensual, será castigado con reclusión menor en su grado mínimo o multa de seis a diez unidades tributarias mensuales.






    ART. 190.

    El que hiciere poner sobre objetos fabricados el nombre de un fabricante que no sea autor de tales objetos, o la razón comercial de una fábrica que no sea la de la verdadera fabricación, sufrirá las penas de presidio menor en sus grados mínimo a medio y multa de seis a diez unidades tributarias mensuales.
    Las mismas penas se aplicarán a todo mercader, comisionista o vendedor que a sabiendas hubiere puesto en venta o circulación objetos marcados con nombres supuestos o alterados.





    ART. 191.

    La tentativa para cualquiera de los delitos enumerados en los artículos precedentes de este párrafo, será castigada con el mínimum de las penas señaladas para el delito consumado.


    ART. 192.

    Quedan exentos de pena los culpables de los delitos castigados por los arts. 162, 163, 165, 167, 172, 173, 174, 175, 180, 181 y 182 siempre que, antes de haberse hecho uso de los objetos falsificados, sin ser descubiertos y no habiéndose iniciado procedimiento, alguno en su contra, se delataren a la autoridad, revelándole las circunstancias del delito.


    § IV.

De la falsificación de documentos públicos o auténticos.



    ART. 193.

    Será castigado con presidio menor en su grado máximo a presidio mayor en su grado mínimo el empleado público que, abusando de su oficio, cometiere falsedad:
    1.° Contrahaciendo o fingiendo letra, firma o rúbrica.
    2.° Suponiendo en un acto la intervención de personas que no la han tenido.
    3.° Atribuyendo a los que han intervenido en él declaraciones o manifestaciones diferentes de las que hubieren hecho.
    4.° Faltando a la verdad en la narración de hechos sustanciales.
    5.° Alterando las fechas verdaderas.
    6.° Haciendo en documento verdadero cualquiera alteración o intercalación que varíe su sentido.
    7.° Dando copia en forma fehaciente de un documento supuesto, o manifestando en ella cosa contraria o diferente de la que contenga el verdadero original.
    8.° Ocultando en perjuicio del Estado o de un particular cualquier documento oficial.


    ART. 194.

    El particular que cometiere en documento público o auténtico alguna de las falsedades designadas en el artículo anterior, sufrirá la pena de presidio menor en sus grados medio a máximo.


    ART. 195.

    El encargado o empleado de una oficina telegráfica que cometiere falsedad en el ejercicio de sus funciones, forjando o falsificando partes telegráficos, será castigado con presidio menor en su grado medio.


    ART. 196.

    El que maliciosamente hiciere uso del instrumento o parte falso, será castigado como si fuere autor de la falsedad.


    § V.

    De la falsificación de instrumentos privados.



    ART. 197.

    El que, con perjuicio de tercero, cometiere en instrumento privado alguna de las falsedades designadas en el art. 193, sufrirá las penas de presidio menor en cualquiera de sus grados y multa de once a quince unidades tributarias mensuales, o sólo la primera de ellas según las circunstancias.
    Si tales falsedades se hubieren cometido en letras de cambio u otra clase de documentos mercantiles, se castigará a los culpables con presidio menor en su grado máximo y multa de dieciséis a veinte unidades tributarias mensuales, o sólo con la primera de estas penas atendidas las circunstancias.





    ART. 198.

    El que maliciosamente hiciere uso de los instrumentos falsos a que se refiere el artículo anterior, será castigado como si fuera autor de la falsedad.


    § VI.

    De la falsificación de pasaportes, portes de armas y certificados.


    ART. 199.

    El empleado público que expidiere un pasaporte o porte de armas bajo nombre supuesto o lo diere en blanco, sufrirá las penas de reclusión menor en sus grados mínimo a medio e inhabilitación absoluta temporal para cargos y oficios públicos en los mismos grados.


    ART. 200.

    El que hiciere un pasaporte o porte de armas falso, será castigado con reclusión menor en su grado medio y multa de seis a diez unidades tributarias mensuales.
    Las mismas penas se impondrán al que en un pasaporte o porte de armas verdadero mudare el nombre de la persona a cuyo favor se halle expedido, o el de la autoridad que lo expidió, o que altere en él alguna otra circunstancia esencial.





    ART. 201.

    El que hiciere uso del pasaporte o porte de armas falso a que se refiere el artículo anterior, incurrirá en una multa de seis a diez unidades tributarias mensuales.
    La misma pena se impondrá al que hiciere uso de un pasaporte o porte de armas verdadero expedido a favor de otra persona.






    ART. 202.

    El facultativo que librare certificación falsa de enfermedad o lesión con el fin de eximir a una persona de algún servicio público, será castigado con reclusión menor en sus grados mínimo a medio y multa de seis a diez unidades tributarias mensuales.
    El que incurra en las falsedades del artículo 193 en el otorgamiento, obtención o tramitación de licencias médicas o declaraciones de invalidez será sancionado con las penas de reclusión menor en sus grados mínimo a medio y multa de veinticinco a doscientas cincuenta unidades tributarias mensuales.
    Si el que cometiere la conducta señalada en el inciso anterior fuere un facultativo se castigará con las mismas penas y una multa de cincuenta a quinientas unidades tributarias mensuales. Asimismo, el tribunal deberá aplicar la pena de inhabilitación especial temporal para emitir licencias médicas durante el tiempo de la condena.
    En caso de reincidencia, la pena privativa de libertad se aumentará en un grado y se aplicará multa de setenta y cinco a setecientas cincuenta unidades tributarias mensuales.









    ART. 203.

    El empleado público que librare certificación falsa de méritos o servicios, de buena conducta, de pobreza, o de otras circunstancias semejantes de recomendación, incurrirá en una multa de seis a diez unidades tributarias mensuales.









    ART. 204.

    El que falsificare un documento de la clase designada en los dos artículos anteriores, será castigado con reclusión menor en su grado mínimo y multa de seis a diez unidades tributarias mensuales.
    Esta disposición es aplicable al que maliciosamente usare, con el mismo fin, de los documentos falsos.





    ART. 205.

    El que falsificare certificados de funcionarios públicos que puedan comprometer intereses públicos o privados, sufrirá la pena de reclusión menor en su grado medio.
    Si el certificado ha sido falsificado bajo el nombre de un particular, la pena será reclusión menor en su grado mínimo.


    § VII.

    De las falsedades vertidas en el proceso y del perjurio.




    ART. 206.
    El testigo, perito o intérprete que ante un tribunal faltare a la verdad en su declaración, informe o traducción, será castigado con la pena de presidio menor en sus grados mínimo a medio y multa de seis a veinte unidades tributarias mensuales, si se tratare de proceso civil o por falta, y con presidio menor en su grado medio a máximo y multa de veinte a treinta unidades tributarias mensuales, si se tratare de proceso penal por crimen o simple delito.

    Tratándose de peritos e intérpretes, sufrirán además la pena de suspensión de profesión titular durante el tiempo de la condena.

    Si la conducta se realizare contra el imputado o acusado en proceso por crimen o simple delito, la pena se impondrá en el grado máximo.

    Están exentos de responsabilidad penal por las conductas sancionadas en este artículo quienes se encuentren amparados por cualquiera de los supuestos a que se refiere el artículo 305 del Código Procesal Penal.


    ART. 207.
    El que, a sabiendas, presentare ante un tribunal a los testigos, peritos o intérpretes a que se refiere el artículo precedente, u otros medios de prueba falsos o adulterados, será castigado con la pena de presidio menor en su grado mínimo a medio y multa de seis a veinte unidades tributarias mensuales, si se tratare de proceso civil o por falta, y con presidio menor en su grado medio a máximo y multa de veinte a treinta unidades tributarias mensuales, si se tratare de proceso penal por crimen o simple delito.

    Los abogados que incurrieren en la conducta descrita sufrirán, además, la pena de suspensión de profesión titular durante el tiempo de la condena.

    Tratándose de un fiscal del Ministerio Público, la pena será de presidio menor en su grado máximo a presidio mayor en su grado mínimo.

    En todo caso, si la conducta se realizare contra el imputado o acusado en proceso por crimen o simple delito, la pena se impondrá en el grado máximo.


    ART. 208.
    La retractación oportuna de quien hubiere incurrido en alguna de las conductas previstas en los dos artículos precedentes constituirá circunstancia atenuante muy calificada, en los términos del artículo 68 bis de este Código.

    Retractación oportuna es aquélla que tiene lugar ante el juez en condiciones de tiempo y forma adecuados para ser considerada por el tribunal que debe resolver la causa.

    En todo caso, la retractación oportuna eximirá de responsabilidad penal en casos calificados, cuando su importancia para el esclarecimiento de los hechos y la gravedad de los potenciales efectos de su omisión así lo justificaren.


    ART. 209.

    El falso testimonio en causa civil, será castigado con presidio menor en su grado medio y multa de once a veinte unidades tributarias mensuales.
    Si el valor de la demanda no excediere de cuatro sueldos vitales, las penas serán presidio menor en su grado mínimo y multa de seis a diez unidades tributarias mensuales.









    ART. 210.

    El que ante la autoridad o sus agentes perjurare o diere falso testimonio en materia que no sea contenciosa, sufrirá las penas de presidio menor en sus grados mínimo a medio y multa de seis a diez unidades tributarias mensuales.

    Inciso Eliminado.


    ART. 211.

    El que maliciosamente presentare una denuncia por la cual se impute falsamente a otra persona un hecho determinado constitutivo de delito, infracción administrativa o infracción disciplinaria será sancionado:
     
    1. Con la pena de presidio menor en su grado medio y multa de once a veinte unidades tributarias mensuales si el hecho imputado fuere constitutivo de crimen.
     
    2. Con la pena de presidio menor en sus grados mínimo a medio y multa de seis a diez unidades tributarias mensuales si el hecho imputado fuere constitutivo de simple delito o de infracción administrativa.
     
    3. Con la pena de presidio menor en su grado mínimo y multa de una a cinco unidades tributarias mensuales si el hecho imputado fuere constitutivo de falta o fuere de aquellos que diere lugar a una infracción disciplinaria.
     
    Para los efectos del inciso anterior, se entenderá también que denuncia el que presenta querella o formula acusación particular en un proceso penal.


    ART. 211 BIS.

    Para efectos de lo dispuesto en el artículo precedente se entenderá que constituyen infracción administrativa los hechos por los que la administración o los tribunales que no ejercen jurisdicción en lo penal pueden imponer multas u otras sanciones privativas o restrictivas de derechos patrimoniales o civiles, e infracción disciplinaria los hechos por los que se imponen sanciones por la contravención de las normas que regulan el correcto ejercicio de los cargos y funciones públicos.

     
    ART. 211 TER.

    La retractación oportuna de quien hubiere incurrido en alguna de las conductas previstas en el artículo 211 constituirá una atenuante muy calificada en los términos del artículo 68 bis.
     
    Para estos efectos, la retractación es oportuna:
     
    1. Tratándose de un hecho constitutivo de crimen, simple delito o falta, antes de que se adopte una medida judicial que afecte los derechos de una persona y antes del término del procedimiento.

    2. Tratándose de una infracción administrativa o de un proceso que pudiere dar lugar a una infracción disciplinaria, antes de que se formulen cargos contra la persona afectada.
     
    En todo caso, la retractación oportuna eximirá de responsabilidad penal en casos calificados, cuando su importancia para el esclarecimiento de los hechos y la gravedad de los potenciales efectos de su omisión así lo justifiquen.



    ART. 212.
    El que fuera de los casos previstos en los artículos precedentes faltare a la verdad en declaración prestada bajo juramento o promesa exigida por ley, será castigado con la pena de prisión en cualquiera de sus grados o multa de una a cuatro unidades tributarias mensuales.

    § VIII.

    Del ejercicio ilegal de una profesión y de la usurpación de funciones o nombres.



    Artículo 213° El que se fingiere autoridad, funcionario público o titular de una profesión que, por disposición de la ley, requiera título o el cumplimiento de determinados requisitos, y ejerciere actos propios de dichos cargos profesiones, será penado con presidio menor en sus grados mínimo a medio y multa de seis a veinte unidades tributarias mensuales.
    El mero fingimiento de esos cargos o profesiones será sancionado como tentativa del delito que establece el inciso anterior.




    Artículo 214° El que usurpare el nombre de otro será castigado con presidio menor en su grado mínimo, sin perjuicio de la pena que pudiere corresponderle a consecuencia del daño que en su fama o intereses ocasionare a la persona cuyo nombre ha usurpado.

    Artículo 215°  Suprimido.



    TÍTULO QUINTO.

DE LOS CRÍMENES Y SIMPLES DELITOS COMETIDOS POR EMPLEADOS PÚBLICOS EN EL DESEMPEÑO DE SUS CARGOS.


    § I.

    Anticipación y prolongación indebida de funciones públicas. Derogado.





    ART. 216. Suprimido.
    ART. 217. Suprimido.
    ART. 218. Suprimido.
    ART. 219. Suprimido.

    § II.

    Nombramientos ilegales.

    ART. 220.

    El empleado público que a sabiendas designare en un cargo público a persona que se encuentre afecta a inhabilidad legal que le impida ejercerlo, será sancionado con la pena de inhabilitación especial temporal en cualquiera de sus grados y multa de cinco a diez unidades tributarias mensuales.

    § III.

    Usurpación de atribuciones.


    ART. 221.

    El empleado público que dictare reglamentos o disposiciones generales excediendo maliciosamente sus atribuciones, será castigado con suspensión del empleo en su grado medio.


    ART. 222. El empleado del orden judicial que se arrogare atribuciones propias de las autoridades administrativas o impidiere a éstas el ejercicio legítimo de las suyas, sufrirá la pena de suspensión del empleo en su grado medio.
    En la misma pena incurrirá todo empleado del orden administrativo que se arrogare atribuciones judiciales o impidiere la ejecución de una providencia dictada por tribunal competente.
    Las disposiciones de este artículo sólo se harán efectivas cuando entablada la competencia y resuelta por la autoridad correspondiente, los empleados administrativos o judiciales continuaren procediendo indebidamente.



    § IV.

    Prevaricación.



    ART. 223.

    Los miembros de los tribunales de justicia colegiados o unipersonales y los fiscales judiciales, sufrirán las penas de inhabilitación absoluta perpetua para cargos y oficios públicos, derechos políticos y profesiones titulares y la de presidio o reclusión menores en cualesquiera de sus grados:

    1° Cuando a sabiendas fallaren contra ley expresa y vigente en causa criminal o civil.
    2° Suprimido.
    3° Cuando ejerciendo las funciones de su empleo o valiéndose del poder que éste les da, seduzcan o soliciten a persona imputada o que litigue ante ellos.






    ART. 224.

    Sufrirán las penas de inhabilitación absoluta temporal para cargos y oficios públicos en cualquiera de sus grados y la de presidio o reclusión menores en sus grados mínimos a medios:
    1° Cuando por negligencia o ignorancia inexcusables dictaren sentencia manifiestamente injusta en causa criminal.
    2° Cuando a sabiendas contravinieren a las leyes que reglan la sustanciación de los juicios, en términos de producir nulidad en todo o en parte sustancial.
    3° Cuando maliciosamente nieguen o retarden la administración de justicia y el auxilio o protección que legalmente se les pida.
    4° Cuando maliciosamente omitan decretar la prisión de alguna persona, habiendo motivo legal para ello, o no lleven a efecto la decretada, pudiendo hacerlo.
    5° Cuando maliciosamente retuvieren en calidad de preso a un individuo que debiera ser puesto en libertad con arreglo a la ley.
    6° Cuando revelen los secretos del juicio o den auxilio o consejo a cualquiera de las partes interesadas en él, en perjuicio de la contraria.
    7° Cuando con manifiesta implicancia, que les sea conocida y sin haberla hecho saber previamente a las partes, fallaren en causa criminal o civil.

    ART. 225.

    Incurrirán en las penas de suspensión de cargo o empleo en cualquiera de sus grados y multa de once a veinte unidades tributarias mensuales o solo en esta última, cuando por negligencia o ignorancia inexcusables:

    1.° Dictaren sentencia manifiestamente injusta en causa civil.
    2° Contravinieren a las leyes que reglan la sustanciación de los juicios en términos de producir nulidad en todo o en parte sustancial.
    3º Negaren o retardaren la administración de justicia y el auxilio o protección que legalmente se les pida.
    4° Omitieren decretar la prisión de alguna persona, habiendo motivo legal para ello, o no llevaren a efecto la decretada, pudiendo hacerlo.
    5º Retuvieren preso por más de cuarenta y ocho horas a un individuo que debiera ser puesto en libertad con arreglo a la ley.







    ART. 226.

    En las mismas penas incurrirán cuando no cumplan las órdenes que legalmente se les comuniquen por las autoridades superiores competentes, a menos de ser evidentemente contrarias a las leyes, o que haya motivo fundado para dudar de su autenticidad, o que aparezca que se han obtenido por engaño o se tema con razón que de su ejecución resulten graves males que el superior no pudo prever.
    En estos casos el tribunal, suspendiendo el cumplimiento de la orden, representará inmediatamente a la autoridad superior las razones de la suspensión, y si ésta insistiere, le dará cumplimiento, libertándose así de responsabilidad, que recaerá sobre el que la mandó cumplir.


    ART. 227.

    Se aplicarán respectivamente las penas determinadas en los artículos precedentes:

    1.° A las personas que, desempeñando por ministerio de la ley los cargos de miembros de los tribunales de justicia colegiados o unipersonales, fueren condenadas por alguno de los crímenes o simples delitos enumerados en dichos artículos.
    2.º A los subdelegados e inspectores que incurrieren en iguales infracciones.
    3.° A los compromisarios, peritos y otras personas que, ejerciendo atribuciones análogas, derivadas de la ley, del tribunal o del nombramiento de las partes, se hallaren en idénticos casos.

    ART. 228.

    El que, desempeñando un empleo público no perteneciente al orden judicial, dictare a sabiendas providencia o resolución manifiestamente injusta en negocio contencioso-administrativo o meramente administrativo, incurrirá en las penas de suspensión del empleo en su grado medio y multa de once a quince unidades tributarias mensuales.
    Si la resolución o providencia manifiestamente injusta la diere por negligencia o ignorancia inexcusables, las penas serán suspensión en su grado mínimo y multa de seis a diez unidades tributarias mensuales.





    ART. 229.

    Sufrirán las penas de suspensión de empleo en su grado medio y multa de seis a diez unidades tributarias mensuales los funcionarios a que se refiere el artículo anterior, que, por malicia o negligencia inexcusables y faltando a las obligaciones de su oficio, no procedieren a la persecución o aprehensión de los delincuentes después de requerimiento o denuncia formal hecha por escrito.





    ART. 230.

    Si no tuviere renta el funcionario que debe ser penado con suspensión o inhabilitación para cargos o empleos públicos, se le aplicará además de estas penas la de reclusión menor en cualquiera de sus grados o multa de once a veinte unidades tributarias mensuales, según los casos.









    ART. 231.

    El abogado o procurador que con abuso malicioso de su oficio, perjudicare a su cliente o descubriere sus secretos, será castigado según la gravedad del perjuicio que causare, con la pena de suspensión en su grado mínimo a inhabilitación especial perpetua para el cargo o profesión y multa de once a veinte unidades tributarias mensuales.








    Art. 232.

    El abogado que, teniendo la defensa actual de un pleito, patrocinare a la vez a la parte contraria en el mismo negocio, sufrirá las penas de inhabilitación especial perpetua para el ejercicio de la profesión y multa de once a veinte unidades tributarias mensuales.








    § V.

    Malversación de caudales públicos.



    Artículo 233.- El empleado público que, teniendo a su cargo caudales o efectos públicos o de particulares en depósito, consignación o secuestro, los substrajere o consintiere que otro los substraiga, será castigado:
    1.º Con presidio menor en sus grados medio a máximo, si la substracción excediere de una unidad tributaria mensual y no pasare de cuatro unidades tributarias mensuales.
    2.º Con presidio menor en su grado máximo a presidio mayor en su grado mínimo, si excediere de cuatro unidades tributarias mensuales y no pasare de cuarenta unidades tributarias mensuales.
    3.º Con presidio mayor en sus grados mínimo a medio, si excediere de cuarenta unidades tributarias mensuales.
    En todos los casos, con las penas de multa del doble de lo substraído y de inhabilitación absoluta temporal en su grado medio a inhabilitación absoluta perpetua para cargos y oficios públicos.





    ART. 234.

    El empleado público que, por abandono o negligencia inexcusables, diere ocasión a que se efectúe por otra persona la sustracción de caudales o efectos públicos o de particulares de que se trata en los tres números del artículo anterior, incurrirá en la pena de suspensión en cualquiera de sus grados, quedando además obligado a la devolución de la cantidad o efectos sustraídos.



    ART. 235.

    El empleado que, con daño o entorpecimiento del servicio público, aplicare a usos propios o ajenos los caudales o efectos puestos a su cargo, sufrirá las penas de inhabilitación especial temporal para el cargo u oficio en su grado medio y multa de la mitad al tanto de la cantidad que hubiere sustraído.
    No verificado el reintegro, se le aplicarán las penas señaladas en el art. 233.
    Si el uso indebido de los fondos fuere sin daño ni entorpecimiento del servicio público, las penas serán suspensión del empleo en su grado medio y multa de la mitad de la cantidad sustraída, sin perjuicio del reintegro.




    ART. 236.

    El empleado público que arbitrariamente diere a los caudales o efectos que administre una aplicación pública diferente de aquella a que estuvieren destinados, será castigado con la pena de suspensión del empleo en su grado medio, si de ello resultare daño o entorpecimiento para el servicio u objeto en que debían emplearse, y con la misma en su grado mínimo, si no resultare daño o entorpecimiento.


    ART. 237.

    El empleado público que, debiendo hacer un pago como tenedor de fondos del Estado, rehusare hacerlo sin causa bastante, sufrirá la pena de suspensión del empleo en sus grados mínimo a medio.
    Esta disposición es aplicable al empleado público que, requerido por orden de autoridad competente, rehusare hacer entrega de una cosa puesta bajo su custodia o administración.



    ART. 238.

    Las disposiciones de este párrafo son estensivas al que se halle encargado por cualquier concepto de fondos, rentas o efectos municipales o pertenecientes a un establecimiento público de instruccion o beneficencia.
    En los delitos a que se refiere este párrafo, se aplicará el máximo del grado cuando el valor de lo malversado excediere de cuatrocientas unidades tributarias mensuales, siempre que la pena señalada al delito conste de uno solo en conformidad a lo establecido en el inciso tercero del artículo 67 de este Código. Si la pena consta de dos o más grados, se impondrá el grado máximo.




    § VI.

    Fraudes y exacciones ilegales.



    ART. 239.

    El empleado público que en las operaciones en que interviniere por razón de su cargo, defraudare o consintiere que se defraude al Estado, a las municipalidades o a los establecimientos públicos de instrucción o de beneficencia, sea originándoles pérdida o privándoles de un lucro legítimo, incurrirá en la pena de presidio menor en sus grados medio a máximo.
    En aquellos casos en que el monto de lo defraudado excediere de cuarenta unidades tributarias mensuales, se impondrá la pena de presidio menor en su grado máximo a presidio mayor en su grado mínimo.
    Si la defraudación excediere de cuatrocientas unidades tributarias mensuales se aplicará la pena de presidio mayor en sus grados mínimo a medio.
    En todo caso, se aplicarán las penas de multa de la mitad al tanto del perjuicio causado e inhabilitación absoluta temporal para cargos, empleos u oficios públicos en sus grados medio a máximo.


    ART. 240.

    Será sancionado con la pena de reclusión menor en sus grados medio a máximo, inhabilitación absoluta temporal para cargos, empleos u oficios públicos en sus grados medio a máximo y multa de la mitad al tanto del valor del interés que hubiere tomado en el negocio:
   
    1° El empleado público que directa o indirectamente se interesare en cualquier negociación, actuación, contrato, operación o gestión en la cual hubiere de intervenir en razón de su cargo.
    2º El árbitro o el liquidador comercial que directa o indirectamente se interesare en cualquier negociación, actuación, contrato, operación o gestión en la cual hubiere de intervenir en relación con los bienes, cosas o intereses patrimoniales cuya adjudicación, partición o administración estuviere a su cargo.
    3° El veedor o liquidador en un procedimiento concursal que directa o indirectamente se interesare en cualquier negociación, actuación, contrato, operación o gestión en la cual hubiere de intervenir en relación con los bienes o intereses patrimoniales cuya salvaguardia o promoción le corresponda.
    En este caso se aplicará lo dispuesto en el artículo 465 de este Código.
    4° El perito que directa o indirectamente se interesare en cualquier negociación, actuación, contrato, operación o gestión en la cual hubiere de intervenir en relación con los bienes o cosas cuya tasación le corresponda.
    5° El guardador o albacea que directa o indirectamente se interesare en cualquier negociación, actuación, contrato, operación o gestión en la cual hubiere de intervenir en relación con el patrimonio de los pupilos y las testamentarías a su cargo, incumpliendo las condiciones establecidas en la ley.
    6º El que tenga a su cargo la salvaguardia o la gestión de todo o parte del patrimonio de otra persona que estuviere impedida de administrarlo, que directa o indirectamente se interesare en cualquier negociación, actuación, contrato, operación o gestión en la cual hubiere de intervenir en relación con ese patrimonio, incumpliendo las condiciones establecidas en la ley.
    7° El director o gerente de una sociedad anónima abierta o especial que directa o indirectamente se interesare en cualquier negociación, actuación, contrato, operación o gestión que involucre a la sociedad, incumpliendo las condiciones establecidas por la ley, así como toda persona a quien le sean aplicables las normas que en materia de deberes se establecen para los directores o gerentes de estas sociedades.
   
    Las mismas penas se impondrán a las personas mencionadas en los números 1 a 6 del inciso precedente si, en las mismas circunstancias, dieren o dejaren tomar interés, debiendo impedirlo, a su cónyuge o conviviente civil, a un pariente en cualquier grado de la línea recta o hasta en el tercer grado inclusive de la línea colateral, sea por consanguinidad o afinidad.
    Lo mismo valdrá en caso de que alguna de las personas mencionadas en los números 1 a 6 del inciso primero, en las mismas circunstancias, diere o dejare tomar interés, debiendo impedirlo, a terceros asociados con ella o con las personas indicadas en el inciso precedente, o a sociedades, asociaciones o empresas en las que ella misma, dichos terceros o esas personas ejerzan su administración en cualquier forma o tengan interés social, el cual deberá ser superior al diez por ciento si la sociedad fuere anónima.
    Tratándose de una sociedad anónima abierta o especial, las mismas penas referidas en el inciso primero se aplicarán al director o gerente que diere o dejare tomar interés a personas consideradas por la ley como partes relacionadas.

    ART. 240 bis.


    Las penas establecidas en el artículo precedente serán también aplicadas al empleado público que, interesándose directa o indirectamente en cualquier clase de contrato u operación en que deba intervenir otro empleado público, ejerciere influencia en éste para obtener una decisión favorable a sus intereses.
    Las mismas penas se impondrán al empleado público que, para dar interés a cualquiera de las personas expresadas en los incisos segundo y final del artículo precedente en cualquier clase de contrato u operación en que deba intervenir otro empleado público, ejerciere influencia en él para obtener una decisión favorable a esos intereses.
    En los casos a que se refiere este artículo el juez podrá imponer la pena de inhabilitación absoluta perpetua para cargos u oficios públicos.


    ART. 241.

    El empleado público que directa o indirectamente exigiere mayores derechos de los que le están señalados por razón de su cargo, o un beneficio económico para sí o un tercero para ejecutar o por haber ejecutado un acto propio de su cargo en razón del cual no le están señalados derechos, será sancionado con reclusión menor en su grado máximo a reclusión mayor en su grado mínimo, salvo que el hecho sea constitutivo de un delito que merezca mayor pena, caso en el cual se aplicará sólo la pena asignada por la ley a éste. En todo caso se impondrán, además, las penas de inhabilitación absoluta temporal para cargos u oficios públicos en sus grados medio a máximo y multa del duplo al cuádruplo de los derechos o del beneficio obtenido.


   
    ART. 241 bis.

    El empleado público que durante el ejercicio de su cargo obtenga un incremento patrimonial relevante e injustificado, será sancionado con multa equivalente al monto del incremento patrimonial indebido y con la pena de inhabilitación absoluta temporal para el ejercicio de cargos y oficios públicos en sus grados mínimo a medio.
    Lo dispuesto en el inciso precedente no se aplicará si la conducta que dio origen al incremento patrimonial indebido constituye por sí misma alguno de los delitos descritos en el presente Título, caso en el cual se impondrán las penas asignadas al respectivo delito.
    La prueba del enriquecimiento injustificado a que se refiere este artículo será siempre de cargo del Ministerio Público.
    Si el proceso penal se inicia por denuncia o querella y el empleado público es absuelto del delito establecido en este artículo o se dicta en su favor sobreseimiento definitivo por alguna de las causales establecidas en las letras a) o b) del artículo 250 del Código Procesal Penal, tendrá derecho a obtener del querellante o denunciante la indemnización de los perjuicios por los daños materiales y morales que haya sufrido, sin perjuicio de la responsabilidad criminal de estos últimos por el delito del artículo 211 de este Código.


    § VII.

    Infidelidad en la custodia de documentos.


    ART. 242.

    El eclesiástico o empleado público que sustraiga o destruya documentos o papeles que le estuvieren confiados por razón de su cargo, será castigado:
    1.° Con las penas de reclusión menor en su grado máximo y multa de veintiuna a veinticinco unidades tributarias mensuales, siempre que del hecho resulte grave daño de la causa pública o de tercero.
    2.° Con reclusión menor en sus grados mínimo a medio y multa de once a veinte unidades tributarias mensuales, cuando no concurrieren las circunstancias expresadas en el número anterior.





    ART. 243.

    El empleado público que, teniendo a su cargo la custodia de papeles o efectos sellados por la autoridad, quebrantare los sellos o consintiere en su quebrantamiento, sufrirá las penas de reclusión menor en su grados mínimo a medio y multa de once a quince unidades tributarias mensuales.
    El guardián que por su negligencia diere lugar al delito, será castigado con reclusión menor en su grado mínimo o multa de seis a diez unidades tributarias mensuales.





    ART. 244.

    El empleado público que abriere o consintiere que se abran, sin la autorización competente, papeles o documentos cerrados cuya custodia le estuviere confiada, incurrirá en las penas de reclusión menor en su grado mínimo y multa de seis a diez unidades tributarias mensuales.








    ART. 245.

    Las penas designadas en los tres artículos anteriores son aplicables a los particulares encargados accidentalmente del despacho o custodia de documentos o papeles, por comisión del Gobierno o de los funcionarios a quienes hubieren sido confiados aquéllos en razón de su oficio, y que dieren el encargo ejerciendo sus atribuciones.


    § VIII.

    Violación de secretos.





    ART. 246.

    El empleado público que revelare los secretos de que tenga conocimiento por razón de su oficio o entregare indebidamente papeles o copia de papeles que tenga a su cargo y no deban ser publicados, incurrirá en las penas de suspensión del empleo en sus grados mínimo a medio o multa de seis a veinte unidades tributarias mensuales, o bien en ambas conjuntamente.

    Si de la revelación o entrega resultare grave daño para la causa pública, las penas serán reclusión mayor en cualquiera de sus grados y multa de veintiuno a treinta unidades tributarias mensuales.

    Las penas señaladas en los incisos anteriores se aplicarán, según corresponda, al empleado público que indebidamente anticipare en cualquier forma el conocimiento de documentos, actos o papeles que tenga a su cargo y que deban ser publicados.

    ART. 246 BIS.

    El funcionario público que revelare o consintiere que otro tomare conocimiento de uno o más hechos ventilados en un procedimiento judicial o administrativo sancionatorio o disciplinario en el cual le hubiere correspondido intervenir bajo un deber de reserva será sancionado con la pena de reclusión menor en cualquiera de sus grados y multa de diez a treinta unidades tributarias mensuales.
    Si la información a que se refiere el inciso anterior fuere la de la identidad del denunciante, la pena será de reclusión menor en sus grados medio a máximo y multa de veinte a treinta unidades tributarias mensuales.



    ART. 247.

    El empleado público que, sabiendo por razón de su cargo los secretos de un particular, los descubriere con perjuicio de éste, incurrirá en las penas de reclusión menor en sus grados mínimo a medio y multa de seis a diez unidades tributarias mensuales.

    Las mismas penas se aplicarán a los que, ejerciendo alguna de las profesiones que requieren título, revelen los secretos que por razón de ella se les hubieren confiado.




    ART. 247 bis.

    El empleado público que, haciendo uso de un secreto o información concreta reservada, de que tenga conocimiento en razón de su cargo, obtuviere un beneficio económico para sí o para un tercero, será castigado con la pena privativa de libertad del artículo anterior y multa del tanto al triplo del beneficio obtenido.
    Con las mismas penas serán castigados los que, ejerciendo alguna de las profesiones que requieren título, obtuvieren un beneficio económico para sí o para un tercero haciendo uso de los secretos que por razón de su profesión se les hubiere confiado. Tratándose de un abogado, si el hecho perjudicare a su cliente, se impondrán además las penas privativas de derechos señaladas en el artículo 231.


    § IX.

    Cohecho.


    ART. 248.

    El empleado público que en razón de su cargo solicitare o aceptare un beneficio económico o de otra naturaleza al que no tiene derecho, para sí o para un tercero, será sancionado con la pena de reclusión menor en su grado medio, inhabilitación absoluta para cargos u oficios públicos temporal en su grado mínimo y multa del tanto del beneficio solicitado o aceptado. Si el beneficio fuere de naturaleza distinta a la económica, la multa será de veinticinco a doscientos cincuenta unidades tributarias mensuales.
    El empleado público que solicitare o aceptare recibir mayores derechos de los que le están señalados por razón de su cargo, o un beneficio económico o de otra naturaleza, para sí o un tercero, para ejecutar o por haber ejecutado un acto propio de su cargo en razón del cual no le están señalados derechos, será sancionado con la pena de reclusión menor en sus grados medio a máximo, inhabilitación absoluta temporal para cargos u oficios públicos en su grado medio y multa del tanto al duplo de los derechos o del beneficio solicitados o aceptados. Si el beneficio fuere de naturaleza distinta a la económica, la multa será de cincuenta a quinientas unidades tributarias mensuales.





    ART. 248 bis.

    El empleado público que solicitare o aceptare recibir un beneficio económico o de otra naturaleza, para sí o un tercero para omitir o por haber omitido un acto debido propio de su cargo, o para ejecutar o por haber ejecutado un acto con infracción a los deberes de su cargo, será sancionado con la pena de reclusión menor en su grado máximo a reclusión mayor en su grado mínimo y, además, con las penas de inhabilitación absoluta temporal para cargos u oficios públicos en su grado máximo y multa del duplo al cuádruplo del provecho solicitado o aceptado. Si el beneficio fuere de naturaleza distinta a la económica, la multa será de cien a mil unidades tributarias mensuales.
    Si la infracción al deber del cargo consistiere en ejercer influencia en otro empleado público con el fin de obtener de éste una decisión que pueda generar un provecho para un tercero interesado, se impondrá la pena de inhabilitación absoluta para cargos u oficios públicos, perpetua, además de las penas de reclusión y multa establecidas en el inciso precedente.




    ART. 249.

    El empleado público que solicitare o aceptare recibir un beneficio económico o de otra naturaleza, para sí o para un tercero para cometer alguno de los crímenes o simples delitos expresados en este Título, o en el párrafo 4 del Título III, será sancionado con las penas de reclusión menor en su grado máximo a reclusión mayor en su grado mínimo, de inhabilitación absoluta perpetua para cargos u oficios públicos y multa del cuádruplo del provecho solicitado o aceptado. Si el beneficio fuere de naturaleza distinta de la económica, la multa será de ciento cincuenta a mil quinientas unidades tributarias mensuales.

    Las penas previstas en el inciso anterior se aplicarán sin perjuicio de las que además corresponda imponer por la comisión del crimen o simple delito de que se trate.




    ART. 250.

    El que diere, ofreciere o consintiere en dar a un empleado público un beneficio económico o de otra naturaleza, en provecho de éste o de un tercero, en razón del cargo del empleado en los términos del inciso primero del artículo 248, o para que realice las acciones o incurra en las omisiones señaladas en los artículos 248, inciso segundo, 248 bis y 249, o por haberlas realizado o haber incurrido en ellas, será castigado con las mismas penas de multa e inhabilitación establecidas en dichas disposiciones.
    Tratándose del beneficio dado, ofrecido o consentido en razón del cargo del empleado público en los términos del inciso primero del artículo 248, el sobornante será sancionado, además, con la pena de reclusión menor en su grado medio, en el caso del beneficio dado u ofrecido, o de reclusión menor en su grado mínimo, en el caso del beneficio consentido.
    Tratándose del beneficio dado, ofrecido o consentido en relación con las acciones u omisiones del inciso segundo del artículo 248, el sobornante será sancionado, además, con la pena de reclusión menor en sus grados medio a máximo, en el caso del beneficio dado u ofrecido, o de reclusión menor en sus grados mínimo a medio, en el caso del beneficio consentido.
    Tratándose del beneficio dado, ofrecido o consentido en relación con las acciones u omisiones señaladas en el artículo 248 bis, el sobornante será sancionado, además, con pena de reclusión menor en su grado máximo a reclusión mayor en su grado mínimo, en el caso del beneficio dado u ofrecido, o de reclusión menor en sus grados medio a máximo, en el caso del beneficio consentido.
    Tratándose del beneficio dado, ofrecido o consentido en relación con los crímenes o simples delitos señalados en el artículo 249, el sobornante será sancionado, además, con pena de reclusión menor en su grado máximo a reclusión mayor en su grado mínimo, en el caso del beneficio dado u ofrecido, o con reclusión menor en sus grados medio a máximo, en el caso del beneficio consentido. Las penas previstas en este inciso se aplicarán sin perjuicio de las que además corresponda imponer por la comisión del crimen o simple delito de que se trate.




    ART. 250 bis.

    En los casos en que el delito previsto en el artículo anterior tuviere por objeto la realización u omisión de una actuación de las señaladas en los artículos 248 ó 248 bis que mediare en causa criminal a favor del imputado, y fuere cometido por su cónyuge o su conviviente civil, por alguno de sus ascendientes o descendientes consanguíneos o afines, por un colateral consanguíneo o afín hasta el segundo grado inclusive, o por persona ligada a él por adopción, sólo se impondrá al responsable la multa que corresponda conforme las disposiciones antes mencionadas.


    ART. 250 bis A. Derogado
    ART. 250 bis B. Derogado
    ART. 251.

    Los bienes recibidos por el empleado público caerán siempre en comiso.
    Inciso Suprimido.





    § 9 bis. Cohecho a Funcionarios Públicos Extranjeros


    ART. 251 bis.

    El que, con el propósito de obtener o mantener para sí o para un tercero cualquier negocio o ventaja en el ámbito de cualesquiera transacciones internacionales o de una actividad económica desempeñada en el extranjero, ofreciere, prometiere, diere o consintiere en dar a un funcionario público extranjero un beneficio económico o de otra naturaleza en provecho de éste o de un tercero, en razón del cargo del funcionario, o para que omita o ejecute, o por haber omitido o ejecutado, un acto propio de su cargo o con infracción a los deberes de su cargo, será sancionado con la pena de reclusión menor en su grado máximo a reclusión mayor en su grado mínimo y, además, con multa del duplo al cuádruplo del beneficio ofrecido, prometido, dado o solicitado, e inhabilitación absoluta temporal para cargos u oficios públicos en su grado máximo. Si el beneficio fuere de naturaleza distinta de la económica, la multa será de cien a mil unidades tributarias mensuales.
    Los bienes recibidos por el funcionario público caerán siempre en comiso.


    Artículo 251 ter.- Para los efectos de lo dispuesto en el artículo anterior, se considera funcionario público extranjero toda persona que tenga un cargo legislativo, administrativo o judicial en un país extranjero, haya sido nombrada o elegida, así como cualquier persona que ejerza una función pública para un país extranjero, sea dentro de un organismo público o de una empresa pública. También se entenderá que inviste la referida calidad cualquier funcionario o agente de una organización pública internacional.


    §9 ter. Normas comunes a los Párrafos anteriores






   
    ART. 251 quáter.

    El que cometiere cualquiera de los delitos previstos en los dos Párrafos anteriores será condenado, además, a la pena de inhabilitación absoluta, perpetua o temporal, en cualquiera de sus grados, para ejercer cargos, empleos, oficios o profesiones en empresas que contraten con órganos o empresas del Estado o con empresas o asociaciones en que éste tenga una participación mayoritaria; o en empresas que participen en concesiones otorgadas por el Estado o cuyo objeto sea la provisión de servicios de utilidad pública.

   
    ART. 251 quinquies.

    En el caso de los delitos previstos en los artículos 241, 248, 248 bis y 249, se excluirá el mínimum o el grado mínimo de las penas señaladas, según corresponda, respecto de todos sus responsables, en los siguientes casos:
   
    1° Cuando hayan sido cometidos por un empleado público que desempeñe un cargo de elección popular, de exclusiva confianza de éstos, de alta dirección pública del primer nivel jerárquico o por un fiscal del Ministerio Público o por cualquiera que, perteneciendo o no al orden judicial, ejerza jurisdicción; por los Comandantes en Jefe del Ejército, de la Armada, de la Fuerza Aérea, o por el General Director de Carabineros o el Director General de la Policía de Investigaciones, o
    2° Cuando hayan sido cometidos por un empleado público con ocasión de su intervención en cualquiera de los siguientes procesos:
   
    a) La designación de una persona en un cargo o función pública;
    b) Un procedimiento de adquisición, contratación o concesión que supere las mil unidades tributarias mensuales en que participe un órgano o empresa del Estado, o una empresa o asociación en que éste tenga una participación mayoritaria; o en el cumplimiento o la ejecución de los contratos o concesiones que se suscriban o autoricen en el marco de dichos procedimientos;
    c) El otorgamiento de permisos o autorizaciones para el desarrollo de actividades económicas por parte de personas naturales cuyos ingresos anuales sean superiores a dos mil cuatrocientas unidades de fomento; o jurídicas con o sin fines de lucro, cuyos ingresos anuales sean superiores a cien mil unidades de fomento, o
    d) La fiscalización de actividades económicas desarrolladas por personas naturales cuyos ingresos anuales sean superiores a dos mil cuatrocientas unidades de fomento; o jurídicas con o sin fines de lucro, cuyos ingresos anuales sean superiores a cien mil unidades de fomento.
   
    Para los efectos de este artículo, se determinará el valor de la unidad de fomento considerando el vigente a la fecha de comisión del delito.

   
    ART. 251 sexies.

    No será constitutivo de los delitos contemplados en los artículos 248, 250, incisos segundo y tercero, y 251 bis aceptar, dar u ofrecer donativos oficiales o protocolares, o aquellos de escaso valor económico que autoriza la costumbre como manifestaciones de cortesía y buena educación.
    Lo dispuesto en el inciso anterior no se aplicará respecto del delito contemplado en el artículo 251 bis cuando se ofreciere, prometiere, diere o consintiere en dar a un funcionario público extranjero un beneficio, para que omita o ejecute, o por haber omitido o ejecutado un acto con infracción a los deberes de su cargo.

    ART. 251 septies.-

    En los delitos contemplados en los artículos 248; 250, incisos segundo y tercero, y 251 bis, cuyo beneficio económico o de otra naturaleza provenga de personas naturales condenadas por alguna de las conductas punibles contempladas en las leyes números 19.366, 19.913 y 20.000, la pena deberá ser aumentada en dos grados. Igual agravante se impondrá en el caso de que el beneficio económico o de otra naturaleza provenga de personas jurídicas, cuando cualquiera de sus representantes legales o administradores, y socios en el caso de las sociedades que no sean anónimas, se encuentren en alguna de dichas situaciones.

    § X.

    Resistencia y desobediencia.





    ART. 252.

    El empleado público que se negare abiertamente a obedecer las órdenes de sus superiores en asuntos del servicio, será penado con inhabilitación especial perpetua para el cargo u oficio.
    En la misma pena incurrirá cuando habiendo suspendido con cualquier motivo la ejecución de órdenes de sus superiores, las desobedeciere después que éstos hubieren desaprobado la suspensión.
    En uno y otro caso, si el empleado no fuere retribuido, la pena será reclusión menor en cualquiera de sus grados o multa de once a veinte unidades tributarias mensuales.







    § XI.

    Denegación de auxilio y abandono de destino.





    ART. 253.

    El empleado público del orden civil o militar que requerido por autoridad competente, no prestare, en el ejercicio de su ministerio, la debida cooperación para la administración de justicia u otro servicio público, será penado con suspensión del empleo en sus grados mínimo a medio y multa de seis a diez unidades tributarias mensuales.
    Si de su omisión resultare grave daño a la causa pública o a un tercero, las penas serán inhabilitación especial perpetua para el cargo u oficio y multa de once a veinte unidades tributarias mensuales.






    ART. 254.

    El empleado que sin renunciar su destino lo abandonare, sufrirá la pena de suspensión en su grado mínimo a inhabilitación especial temporal para el cargo u oficio en su grado medio y multa de seis a diez unidades tributarias mensuales.
    Si renunciado el destino y antes de trascurrir un plazo prudencial en que haya podido ser reemplazado por el superior respectivo, lo abandonare con daño de la causa pública, las penas serán multa de seis a diez unidades tributarias mensuales e inhabilitación especial temporal para el cargo u oficio en su grado medio.
    Las penas establecidas en los dos incisos anteriores se aplicarán respectivamente al que abandonare un cargo concejil sin alegar excusa legítima, y al que después de haber alegado tal excusa, pero antes de trascurrir un plazo prudencial en que haya podido ser reemplazado, hace el abandono ocasionando daño a la causa pública.
    Las disposiciones de este artículo han de entenderse sin perjuicio de lo establecido en el 135.




    § XII.

    Abusos contra particulares.

    ART. 255.

    El empleado público que, desempeñando un acto del servicio, cometiere cualquier vejación injusta contra las personas será castigado con la pena de reclusión menor en su grado mínimo, salvo que el hecho sea constitutivo de un delito de mayor gravedad, caso en el cual se aplicará sólo la pena asignada por la ley a éste.
    Si la conducta descrita en el inciso precedente se cometiere en contra de una persona menor de edad o en situación de vulnerabilidad por discapacidad, enfermedad o vejez; o en contra de una persona que se encuentre bajo el cuidado, custodia o control del empleado público, la pena se aumentará en un grado.
    No se considerarán como vejaciones injustas las molestias o penalidades que sean consecuencia únicamente de sanciones legales, o que sean inherentes o incidentales a éstas, ni las derivadas de un acto legítimo de autoridad.

    ART. 256.

    El empleado público del orden administrativo que maliciosamente retardare o negare a los particulares la protección o servicio que deba dispensarles en conformidad a las leyes y reglamentos será castigado con las penas de suspensión del empleo en cualquiera de sus grados y multa de once a veinte unidades tributarias mensuales.



    ART. 257.

    El empleado público que arbitrariamente rehusare dar certificación o testimonio, o impidiere la presentación o el curso de una solicitud, será penado con multa de seis a diez unidades tributarias mensuales.
    Si el testimonio, certificación o solicitud versaren sobre un abuso cometido por el mismo empleado, la multa será de once a veinte unidades tributarias mensuales.





    ART. 258.

    El empleado público que solicitare a persona que tenga pretensiones pendientes de su resolución, será castigado con la pena de inhabilitación especial temporal para el cargo u oficio en su grado medio.



    ART. 259.

    El empleado que solicitare a persona sujeta a su guarda por razón de su cargo, sufrirá la pena de reclusión menor en cualquiera de sus grados e inhabilitación especial temporal para el cargo u oficio en su grado medio.
    Si la persona solicitada fuere cónyuge, conviviente, descendiente, ascendiente o colateral hasta el segundo grado de quien estuviere bajo la guarda del solicitante, las penas serán reclusión menor en sus grados medio a máximo e inhabilitación especial perpetua para el cargo u oficio.


    § XIII.

    Disposición general.





    ART. 260
   
    Para los efectos de este Título y del Párrafo IV del Título III, se reputa empleado todo el que desempeñe un cargo o función pública, sea en la Administración Central o en instituciones o empresas semifiscales, municipales, autónomas u organismos creados por el Estado o dependientes de él, aunque no sean de nombramiento del Jefe de la República ni reciban sueldo del Estado. No obstará a esta calificación el que el cargo sea de elección popular.





    ART. 260 bis.

    En los delitos contemplados en los Párrafos 5, 6, 9 y 9 bis de este Título el plazo de prescripción de la acción penal empezará a correr desde que el empleado público que intervino en ellos cesare en su cargo o función. Sin embargo, si el empleado, dentro de los seis meses que siguen al cese de su cargo o función, asumiere uno nuevo con facultades de dirección, supervigilancia o control respecto del anterior, el plazo de prescripción empezará a correr desde que cesare en este último.




    ART. 260 ter.

    Será circunstancia agravante de los delitos contemplados en los Párrafos 5, 6, 9 y 9 bis el hecho de que los responsables hayan actuado formando parte de una agrupación u organización de dos o más personas destinada a cometer dichos hechos punibles, siempre que ésta o aquélla no constituyere una asociación ilícita de que trata el Párrafo 10 del Título VI del Libro Segundo.




    ART. 260 quáter.

    Será circunstancia atenuante de responsabilidad penal de los delitos contemplados en los Párrafos 5, 6, 9 y 9 bis la cooperación eficaz que conduzca al esclarecimiento de los hechos investigados o permita la identificación de sus responsables, o sirva para prevenir o impedir la perpetración o consumación de estos delitos, o facilite el comiso de los bienes, instrumentos, efectos o productos del delito. En estos casos, el tribunal podrá reducir la pena hasta en dos grados.
    Se entiende por cooperación eficaz el suministro de datos o informaciones precisos, verídicos y comprobables, que contribuyan necesariamente a los fines señalados en el inciso anterior.
    El Ministerio Público deberá expresar, en la formalización de la investigación o en su escrito de acusación, si la cooperación prestada por el imputado ha sido eficaz a los fines señalados en el inciso primero.
    La reducción de pena se determinará con posterioridad a la individualización de la sanción penal según las circunstancias atenuantes o agravantes comunes que concurran; o de su compensación, de acuerdo con las reglas generales.
    La circunstancia atenuante prevista en este artículo no se aplicará a los empleados públicos que desempeñen un cargo de elección popular o de exclusiva confianza de éstos, o de alta dirección pública del primer nivel jerárquico; a los que sean fiscales del Ministerio Público; ni a aquellos que, perteneciendo o no al orden judicial, ejerzan jurisdicción.




    TÍTULO SEXTO.

DE LOS CRÍMENES Y SIMPLES DELITOS CONTRA EL ORDEN Y LA SEGURIDAD PÚBLICOS COMETIDOS POR PARTICULARES.




    § I.

    Atentados contra la autoridad.








    ART. 261.

    Cometen atentado contra la autoridad:
    1.° Los que sin alzarse públicamente emplean fuerza o intimidación para alguno de los objetos señalados en los arts. 121 y 126.
    2.° Los que acometen o resisten con violencia, emplean fuerza o intimidación contra la autoridad pública o sus agentes, carabineros, funcionarios de la Policía de Investigaciones o de Gendarmería de Chile, cuando aquélla o éstos ejercieron funciones de su cargo.


    ART. 262.

    Los atentados a que se refiere el artículo anterior serán castigados con la pena de reclusión menor en su grado medio o multa de once a quince unidades tributarias mensuales, siempre que concurra alguna de las circunstancias siguientes:
    1° Si la agresión se verifica a mano armada.
    2° Si los delincuentes pusieren manos en la autoridad o en las personas que acudieren a su auxilio.
    3° Si por consecuencia de la coacción la autoridad hubiere accedido a las exigencias de los delincuentes.
    Sin estas circunstancias la pena será reclusión menor en su grado mínimo o multa de seis a diez unidades tributarias mensuales.
    Para determinar si la agresión se verifica a mano armada se estará a lo dispuesto en el art. 132.



    ART. 263. Derogado.
    ART. 264.

    El que amenace durante las sesiones de los cuerpos colegisladores o en las audiencias de los tribunales de justicia a algún diputado o senador o a un miembro de dichos tribunales, o a un senador o diputado por las opiniones manifestadas en el Congreso, o a un miembro de un tribunal de justicia por los fallos que hubiere pronunciado o a los ministros de Estado u otra autoridad en el ejercicio de sus cargos, será castigado con reclusión menor en cualquiera de sus grados.
    El que perturbe gravemente el orden de las sesiones de los cuerpos colegisladores o de las audiencias de los tribunales de justicia, u ocasionare tumulto o exaltare al desorden en el despacho de una autoridad o corporación pública hasta el punto de impedir sus actos, será castigado con la pena de reclusión menor en su grado mínimo y multa de seis a diez unidades tributarias mensuales, o sólo esta última.


    ART. 265. Derogado.

    ART. 266.

    Para todos los efectos de las disposiciones penales respecto de los que cometen atentado contra la autoridad o funcionarios públicos, se entiende que ejercen aquélla constantemente los ministros de Estado y las autoridades de funciones permanentes o llamadas a ejercerlas en todo caso y circunstancias.
    Entiéndese también ofendida la autoridad en ejercicio de sus funciones cuando tuviere lugar el atentado con ocasión de ellas o por razón de su cargo.



    ART. 267.

    El que con violencia o fraude impidiere ejercer sus funciones a un miembro del Congreso, de los tribunales superiores de justicia o del Consejo de Estado, sufrirá las penas de reclusión menor en su grado mínimo y multa de once a veinte unidades tributarias mensuales.








    ART. 268. Derogado.
    ART. 268 bis.

    El que diere falsa alarma de incendio, emergencia o calamidad pública a los Cuerpos de Bomberos u otros servicios de utilidad pública, incurrirá en la pena de reclusión menor en su grado mínimo.

1 bis. Atentados y amenazas contra fiscales del Ministerio Público y defensores penales públicos


    ART. 268 ter.-

    El que mate a un fiscal del Ministerio Público o a un defensor penal público en razón del ejercicio de sus funciones, será castigado con la pena de presidio mayor en su grado máximo a presidio perpetuo calificado.

    ART. 268 quáter.-

    El que hiera, golpee o maltrate de obra a un fiscal del Ministerio Público o a un defensor penal público en razón del ejercicio de sus funciones, será castigado:
    1º. Con la pena de presidio mayor en su grado medio, si de rezsultas de las lesiones el ofendido queda demente, inútil para el trabajo, impotente, impedido de algún miembro importante o notablemente deforme.
    2º. Con presidio menor en su grado máximo a presidio mayor en su grado mínimo, si las lesiones producen al ofendido enfermedad o incapacidad para el trabajo por más de treinta días.
    3º. Con presidio menor en grado medio a máximo, si le causa lesiones menos graves.
    4º. Con reclusión menor en su grado mínimo y multa de once a veinte unidades tributarias mensuales, o sólo esta última, si le ocasiona lesiones leves o no se produce daño alguno.

    ART. 268 quinquies.-

    El que amenazare a un fiscal del Ministerio Público o a un defensor penal público en los términos de los artículos 296 y 297 de este Código, en razón del ejercicio de sus funciones, será castigado con el máximo de la pena o el grado máximo de las penas previstas en dichos artículos, según correspondiere.
    § I ter.

    Retenciones o toma de control de vehículo de transporte público de pasajeros. 

    ART. 268 sexies.-

    Los que mediante violencia o intimidación retuvieren o tomaren el control de un vehículo de transporte público de pasajeros serán sancionados con la pena de presidio mayor en su grado mínimo, sin perjuicio de las penas que correspondan por los otros delitos cometidos con ocasión del hecho. En este último caso, todas las penas se impondrán conjuntamente, en la forma prescrita por el artículo 74 de este Código.
    Si el hecho consistiere en la apropiación del vehículo no tendrá lugar lo previsto en el inciso precedente y, en su lugar, se impondrán las penas de los delitos establecidos en el artículo 433 y en el inciso primero del artículo 436, ambos de este Código, según correspondiere, con exclusión de su grado mínimo.
    Sin perjuicio de lo dispuesto en los incisos anteriores, si los hechos constituyeren algún delito o delitos de mayor gravedad, se estará a la pena señalada para ellos.
    § II.

    Otros desordenes públicos.








    ART. 268 septies.-

    El que, sin estar autorizado, interrumpiere completamente la libre circulación de personas o vehículos en la vía pública, mediante violencia o intimidación en las personas o la instalación de obstáculos levantados en la misma con objetos diversos, será sancionado con la pena de presidio menor en su grado mínimo. Idéntica pena se impondrá a los que, sin mediar accidente o desperfecto mecánico, interpusieren sus vehículos en la vía, en términos tales de hacer imposible la circulación de otros por esta.
    Será castigado con la pena de presidio menor en su grado mínimo a medio el que lanzare a personas o vehículos que se encontraren en la vía pública instrumentos, utensilios u objetos cortantes, punzantes o contundentes potencialmente aptos para causar la muerte o producir lesiones corporales, a menos que el hecho constituya un delito más grave. El tribunal, al momento de determinar la pena, tendrá especialmente en consideración la peligrosidad del instrumento, utensilio u objeto lanzado.
    Si alguno de los hechos previstos en este artículo constituyere un delito más grave, se aplicará la pena señalada a este, sin atención a su grado mínimo o mínimum, según los respectivos casos.



    ART. 269.

    Fuera de los casos sancionados en el Párrafo anterior y en el artículo 268 septies, los que turbaren gravemente la tranquilidad pública para causar injuria u otro mal a alguna persona particular o con cualquier otro fin reprobado, incurrirán en la pena de reclusión menor en su grado mínimo, sin perjuicio de las que les correspondan por el daño u ofensa causados.

    Incurrirá en la pena de presidio menor, en su grado mínimo a medio, el que impidiere o dificultare la actuación del personal de los Cuerpos de Bomberos u otros servicios de utilidad pública, destinada a combatir un siniestro u otra calamidad o desgracia que constituya peligro para la seguridad de las personas.




    § II bis. De la obstrucción a la investigación.




    ART. 269 bis.

    El que, a sabiendas, obstaculice gravemente el esclarecimiento de un hecho punible o la determinación de sus responsables, mediante la aportación de antecedentes falsos que condujeren al Ministerio Público a realizar u omitir actuaciones de la investigación, será sancionado con la pena de presidio menor en su grado mínimo y multa de dos a doce unidades tributarias mensuales.

    La pena prevista en el inciso precedente se aumentará en un grado si los antecedentes falsos aportados condujeren al Ministerio Público a solicitar medidas cautelares o a deducir una acusación infundada.

    El abogado que incurriere en las conductas descritas en los incisos anteriores será castigado, además, con la pena de suspensión de profesión titular durante el tiempo de la condena.

    La retractación oportuna de quien hubiere incurrido en las conductas de que trata el presente artículo constituirá circunstancia atenuante. Tratándose de las situaciones a que se refiere el inciso segundo, la atenuante se considerará como muy calificada, en los términos del artículo 68 bis.

    Se entiende por retractación oportuna aquélla que se produjere en condiciones de tiempo y forma adecuados para ser considerada por el tribunal que debiere resolver alguna medida solicitada en virtud de los antecedentes falsos aportados o, en su caso, aquélla que tuviere lugar durante la vigencia de la medida cautelar decretada en virtud de los antecedentes falsos aportados y que condujere a su alzamiento o, en su caso, la que ocurra antes del pronunciamiento de la sentencia o de la decisión de absolución o condena, según corresponda.

    Estarán exentas de las penas que establece este artículo las personas a que se refieren el inciso final del artículo 17 de este Código y el artículo 302 del Código Procesal Penal.



    ART. 269 ter.

    El funcionario policial, el fiscal del Ministerio Público, o el abogado asistente del fiscal, en su caso, que a sabiendas ocultare, alterare o destruyere cualquier antecedente, objeto o documento que permita establecer la existencia o inexistencia de un delito, la participación punible en él de alguna persona o su inocencia, o que pueda servir para la determinación de la pena, será castigado con presidio menor en cualquiera de sus grados e inhabilitación especial perpetua para el cargo.




    § III.

    De la rotura de sellos.





    ART. 270.

    Los que hubieren roto intencionalmente los sellos puestos por orden de la autoridad pública, serán castigados con reclusión menor en su grado mínimo y multa de seis a diez unidades tributarias mensuales.

    Las penas serán reclusión menor en su grado medio y multa de seis a quince unidades tributarias mensuales cuando los sellos rotos estaban colocados sobre papeles o efectos de un individuo acusado o condenado por crimen.



    ART. 271.

    Si la rotura de los sellos ha sido ejecutada con violencia contra las personas, el culpable sufrirá las penas de reclusión menor en su grado máximo y multa de once a veinte unidades tributarias mensuales.








    § IV.

    De los embarazos puestos a la ejecución de los trabajos públicos.





    ART. 272.

    El que por vías de hecho se hubiere opuesto, sin motivo justificado, a la ejecución de trabajos públicos ordenados o permitidos por autoridad competente, será castigado con reclusión menor en su grado mínimo o multa de once a veinte unidades tributarias mensuales.








    § V.

    Crímenes y simples delitos de los proveedores.





    ART. 273.

    Las personas encargadas de provisiones, empresas o administraciones por cuenta del ejército o de la armada, o sus agentes que voluntariamente hubieren faltado a sus compromisos embarazando el servicio que tuvieren a su cargo con daño grave e inevitable de la causa pública, sufrirán las penas de reclusión mayor en su grado mínimo y multa de veintiuna a treinta unidades tributarias mensuales.






    ART. 274.

    Si ha habido fraude en la naturaleza, calidad o cantidad de los objetos o mano de obra, o de las cosas suministradas, con daño grave e inevitable de la causa pública, los culpables sufrirán las penas de presidio mayor en cualquiera de sus grados y multa de veintiuna a treinta unidades tributarias mensuales.






    § VI.

    De las infracciones de las leyes y reglamentos referentes a loterías, casas de juego y de préstamo sobre prendas.





    ART. 275.

    Es lotería toda operación ofrecida al público y destinada a procurar ganancia por medio de la suerte.


    ART. 276.

    Los autores, empresarios, administradores, comisionados o agentes de loterías no autorizadas legalmente, incurrirán en la multa de once a veinte unidades tributarias mensuales y perderán los objetos muebles puestos en lotería.
    Si los objetos puestos en lotería fueren inmuebles, la pena será multa de veintiuna a treinta unidades tributarias mensuales.
    En caso de reincidencia se les aplicará además la reclusión menor en su grado mínimo.







    ART. 277.

    Los banqueros, dueños, administradores o agentes de casas de juego de suerte, envite o azar, serán castigados con reclusión menor en cualquiera de sus grados y multa de once a veinte unidades tributarias mensuales.








    ART. 278.

    Los que concurrieren a jugar a las casas referidas, sufrirán la pena de reclusión menor en su grado mínimo o multa de once a veinte unidades tributarias mensuales.










    ART. 279.

    El dinero o efectos puestos en juego y los instrumentos, objetos y útiles destinados a él caerán siempre en comiso.


    ART. 280.

    El que sin autorización legal estableciere casas de préstamo sobre prendas, sueldos o salarios, sufrirá las penas de reclusión menor en su grado mínimo, multa de once a veinte unidades tributarias mensuales, y comiso de las cantidades prestadas, hasta la suma de veintiuna a treinta unidades tributarias mensuales.







    ART. 281.

    Los que habiendo obtenido autorización no llevaren libros con la debida formalidad, asentando en ellos, sin claros ni entre renglones, las cantidades prestadas, los plazos e intereses, los nombres y domicilio de los que las reciban, la naturaleza, calidad y valor de los objetos dados en prenda y las demás circunstancias que exijan los reglamentos que deberá dictar el Presidente de la República, incurrirán en las penas de multa de seis a diez unidades tributarias mensuales y comiso de las cantidades prestadas, hasta diez unidades tributarias mensuales.
    Las mismas penas se impondrán a los que no hagan la enajenación de las prendas con arreglo a las leyes y reglamentos.





    ART. 282.

    El prestamista que no diere resguardo de la prenda o seguridad recibida, será castigado con una multa del duplo al quíntuplo de su valor y la cantidad que hubiere prestado caerá en comiso.


    Art. 283.

    El prestamista que hiciere préstamos de la clase indicada en los artículos precedentes a una persona manifiestamente incapaz para contratar por su edad o falta de discernimiento, será castigado con las mismas penas del artículo anterior.



    § VII.

    Crímenes y simples delitos relativos a la industria, al comercio y a las subastas públicas.





    ART. 284.

    El que sin el consentimiento de su legítimo poseedor accediere a un secreto comercial mediante intromisión indebida con el propósito de revelarlo o aprovecharse económicamente de él será castigado con presidio o reclusión menor en su grado medio.
    Para efectos de lo dispuesto en el inciso anterior, se entenderá por intromisión:

    1. El ingreso a dependencias de la empresa o la captación visual o sonora mediante dispositivos técnicos de lo que tuviere lugar al interior de dependencias de la empresa, siempre que ello no fuere perceptible desde su exterior sin la utilización de dispositivos técnicos como los empleados en la captación o sin recurrir a escalamiento o a algún otro modo de vencimiento de un obstáculo a la percepción.

    2. La captación visual o sonora mediante dispositivos técnicos del contenido de la comunicación que dos o más personas mantuvieren de la ejecución de una acción o del desarrollo de una situación por parte de una persona cuando los involucrados tuvieren una expectativa legítima de no estar siendo vistos, escuchados, filmados o grabados, manifestada en las circunstancias de la comunicación, la acción o la situación y que ésta concerniere a la empresa.

    3. El acceso a un sistema informático sin autorización o excediendo la autorización que se posea y superando barreras técnicas o medidas tecnológicas de seguridad.

    La pena señalada en el inciso primero se impondrá también al que sin el consentimiento de su legítimo poseedor reprodujere la fijación en cualquier formato de información constitutiva de un secreto comercial con el propósito de revelarlo o aprovecharse económicamente de él.

    El que, habiendo perpetrado cualquiera de los hechos previstos en los incisos anteriores, sin el consentimiento de su legítimo poseedor revelare o consintiere en que otro accediere al secreto comercial será sancionado con la pena de presidio o reclusión menor en su grado máximo.
    ART. 284 bis.-

    Será castigado con presidio o reclusión menor en su grado medio el que sin el consentimiento de su legítimo poseedor revelare o consintiere que otra persona accediere a un secreto comercial que hubiere conocido:

    1. Bajo un deber de confidencialidad con ocasión del ejercicio de un cargo o una función pública o de una profesión cuyo título se encontrare legalmente reconocido y siempre que el deber de confidencialidad profesional estuviere fundado en la ley o en un reglamento, o en las reglas que definen su correcto ejercicio.

    2. En razón o a consecuencia de una relación contractual o laboral con la empresa afectada o con otra que le haya prestado servicios.
    ART. 284 ter.-

    El que sin el consentimiento de su legítimo poseedor se aprovechare económicamente de un secreto comercial que hubiere conocido en alguna de las circunstancias previstas en los incisos primero o segundo del artículo 284 o en el artículo 284 bis, o sabiendo que su conocimiento del secreto proviene de alguno de esos hechos, será sancionado con presidio o reclusión menor en su grado máximo.
    ART. 284 quáter.-

    Sin perjuicio de las penas previstas en los artículos precedentes, cuando el delito se cometa con ocasión del ejercicio de una de las profesiones a que se refiere el artículo 284 bis se impondrá, además, la pena accesoria de suspensión o inhabilitación del ejercicio de su profesión.
    La pena y su duración serán determinadas atendiendo a la pena principal impuesta conforme a las reglas previstas por los artículos 29 y 30 para la inhabilitación o suspensión de cargo u oficio público.
    ART. 284 quinquies.-

    No incurre en el delito previsto en los artículos 284 bis y 284 ter quien en el ejercicio de su profesión, oficio, trabajo o actividad económica usa la experiencia y las competencias legítimamente adquiridas en conocimiento lícito de un secreto comercial.
    ART. 284 sexies.-

    Para efectos de lo dispuesto en los artículos precedentes, se entenderá por secreto comercial la información que reúna los requisitos exigidos por la ley de propiedad industrial.

    ART. 285.

    El que por medios fraudulentos alterare el precio de bienes o servicios sufrirá las penas de presidio o reclusión menor en sus grados medio a máximo.


    ART. 286.

    Se impondrá la pena de presidio o reclusión menor en su grado máximo a presidio o reclusión mayor en su grado mínimo cuando el fraude expresado en el artículo anterior recayere sobre el precio de bienes o servicios de primera necesidad o de consumo masivo.


    ART. 287.

    Los que emplearen amenaza o cualquier otro medio fraudulento para alejar a los postores en una subasta pública con el fin de alterar el precio del remate, serán castigados con una multa de diez al cincuenta por ciento del valor de la cosa subastada; a no merecer mayor pena por la amenaza u otro medio ilícito que emplearen.



    §7° bis. De la corrupción entre particulares






   
    ART. 287 bis.

    El director, administrador, mandatario o empleado de una empresa que solicitare o aceptare recibir un beneficio económico o de otra naturaleza, para sí o un tercero, para favorecer o por haber favorecido en el ejercicio de sus labores la contratación con un oferente sobre otro será sancionado con la pena de reclusión menor en su grado medio y multa del tanto al duplo del beneficio solicitado o aceptado. Si el beneficio fuere de naturaleza distinta de la económica, la multa será de cincuenta a quinientas unidades tributarias mensuales.


   
    ART. 287 ter.

    El que diere, ofreciere o consintiere en dar a un director, administrador, mandatario o empleado de una empresa un beneficio económico o de otra naturaleza, para sí o un tercero, para que favorezca o por haber favorecido la contratación con un oferente por sobre otro será castigado con la pena de reclusión menor en su grado medio, en el caso del beneficio dado u ofrecido, o de reclusión menor en su grado mínimo, en el caso del beneficio consentido. Además, se le sancionará con las penas de multa señaladas en el artículo precedente.




    § VIII.

    De las infracciones de las leyes y reglamentos relativos a las armas prohibidas.










    ART. 288.

    El que fabricare, vendiere o distribuyere armas absolutamente prohibidas por la ley o por los reglamentos generales que dicte el Presidente de la República, sufrirá la pena de reclusión menor en su grado mínimo o multa de seis a diez unidades tributarias mensuales.









NOTA
      El artículo 24 de la ley 17798, publicada el 21.10.1972, deroga parcialmente este artículo, sólo en cuanto se refiere a armas de fuego, explosivos y demás elementos contemplados en la referida ley.
    ART. 288 bis.

    El que portare armas cortantes o punzantes en recintos de expendio de bebidas alcohólicas que deban consumirse en el mismo local, sufrirá la pena de presidio menor en su grado mínimo o multa de 1 a 4 UTM.
    Igual sanción se aplicará al que en espectáculos públicos, en establecimientos de enseñanza o en vías o espacios públicos en áreas urbanas portare dichas armas, cuando no pueda justificar razonablemente su porte.

    ART. 288 ter.

    El que, en el contexto de reuniones en lugares de uso público, porte injustificadamente combustible apto para cometer atentados contra las personas o para ocasionar daño en las cosas, será sancionado con presidio menor en su grado mínimo.

    § IX.

    Delitos relativos a la salud animal y vegetal.


    Artículo 289° El que de propósito y sin permiso de la autoridad competente propagare una enfermedad animal o una plaga vegetal, será penado con presidio menor en su grado medio a máximo.
    Si la propagación se produjere por negligencia inexcusable del tenedor o encargado de las especies animales o vegetales afectadas por la enfermedad o plaga o del funcionario a cargo del respectivo control sanitario, la pena será de presidio menor en su grado mínimo a medio.
    Si la enfermedad o plaga propagada fuere de aquellas declaradas susceptibles de causar grave daño a la economía nacional, se aplicará la pena asignada al delito correspondiente en su grado máximo.
    El reglamento determinará las enfermedades y plagas a que se refiere el inciso anterior.




    Artículo 290.- Si la propagación de las enfermedades a que se refiere este párrafo se originare con motivo u ocasión de la introducción ilícita al país de animales o especies vegetales, la pena asignada al delito correspondiente podrá aumentarse en un grado.





    Artículo 291.- Los que propagaren indebidamente organismos, productos, elementos o agentes químicos, virales, bacteriológicos, radiactivos, o de cualquier otro orden que por su naturaleza sean susceptibles de poner en peligro la salud animal o vegetal, o el abastecimiento de la población, serán penados con presidio menor en su grado máximo.


    ART. 291 BIS.

    El que cometiere actos de maltrato o crueldad con animales será castigado con la pena de presidio menor en sus grados mínimo a medio y multa de dos a treinta unidades tributarias mensuales, o sólo con esta última.
    Si como resultado de una acción u omisión se causare al animal daño, la pena será presidio menor en sus grados mínimo a medio y multa de diez a treinta unidades tributarias mensuales, además de la accesoria de inhabilidad absoluta perpetua para la tenencia de cualquier tipo de animales.
    Si como resultado de las referidas acción u omisión se causaren lesiones que menoscaben gravemente la integridad física o provocaren la muerte del animal se impondrá la pena de presidio menor en su grado medio y multa de veinte a treinta unidades tributarias mensuales, además de la accesoria de inhabilidad absoluta perpetua para la tenencia de animales.



    Artículo 291 ter.- Para los efectos del artículo anterior se entenderá por acto de maltrato o crueldad con animales toda acción u omisión, ocasional o reiterada, que injustificadamente causare daño, dolor o sufrimiento al animal.


    § 10. De las asociaciones delictivas y criminales



    ART. 292.

    Quien sea parte en una asociación delictiva será sancionado con presidio menor en su grado mínimo a medio.
    La pena será de presidio menor en su grado máximo si la participación consiste en cumplir funciones de jefatura, ejercer mando en ella, financiarla o proveerle recursos o medios, o en haberla fundado.
    Se entenderá por asociación delictiva toda organización formada por tres o más personas, con acción sostenida en el tiempo, que tenga entre sus fines la perpetración de simples delitos.


    ART. 293.

    Quien sea parte en una asociación criminal será sancionado con presidio menor en su grado máximo.
    La pena será presidio mayor en su grado mínimo si la participación consiste en cumplir funciones de jefatura, ejercer mando en ella, financiarla o proveerle recursos o medios, o en haberla fundado.
    Se entenderá por asociación criminal toda organización formada por tres o más personas, con acción sostenida en el tiempo, que tenga entre sus fines la perpetración de hechos constitutivos de crímenes.
    Si la asociación tiene entre sus fines la perpetración de crímenes y simples delitos se aplicarán las sanciones dispuestas en el inciso primero.


    ART. 293 BIS.

    Será sancionado con presidio menor en su grado máximo el que, en un proceso por asociación delictiva o criminal:

    a) Amenace a otro con el objeto de que preste una declaración o un testimonio falso.

    b) Amenace o constriña a otro a que omita prestar declaración o testimonio, a que produzca o presente antecedentes o pruebas falsas, o a que omita producir o presentar antecedentes o pruebas relevantes.

    c) Ofrezca o entregue a otro un beneficio económico o de otra naturaleza para que preste una declaración o testimonio falso o para que omita declarar o testificar.

    d) Ofrezca o entregue a otro un beneficio económico o de otra naturaleza con el objeto de que produzca o presente antecedentes o pruebas falsas u omita producir o presentar antecedentes o pruebas relevantes.


    ART. 294.

    Las penas de los artículos 292 y 293 se impondrán sin perjuicio de las que correspondan por los crímenes o simples delitos cometidos con motivo u ocasión de tales actividades.
    Cuando la asociación se ha formado a través de una persona jurídica, se impondrá, además, como consecuencia accesoria de la pena impuesta a los responsables individuales, la disolución o cancelación de la personalidad jurídica.
    En todo caso se impondrá el comiso de ganancias, de conformidad con el artículo 24 bis. Asimismo, caerán en comiso todos los activos vinculados a la actividad en cuyo contexto se haya perpetrado el delito, a menos que se acredite su origen lícito.
    El comiso de ganancias será impuesto en conformidad con los procedimientos establecidos por la ley.


    ART. 294 BIS.

    Se impondrá asimismo el comiso de las ganancias obtenidas por una organización delictiva o criminal, en los términos del artículo anterior, si se dicta:

    1. Sobreseimiento temporal conforme a las letras b) y c) del inciso primero, y el inciso segundo del artículo 252 del Código Procesal Penal.

    2. Sentencia absolutoria fundada en la falta de convicción a que se refiere el artículo 340 del Código Procesal Penal o sobreseimiento definitivo fundado en la letra b) del artículo 250 del mismo Código.

    3. Sobreseimiento definitivo o sentencia absolutoria fundados en la concurrencia de circunstancias eximentes de responsabilidad que no excluyen la ilicitud del hecho.

    4. Sobreseimiento definitivo o sentencia absolutoria fundados en haberse extinguido la responsabilidad penal o en haber sobrevenido un hecho que, con arreglo a la ley, pone fin a esa responsabilidad.

    El comiso de ganancias sin condena previa también será impuesto respecto de aquellas personas que no han intervenido en la realización del hecho ilícito que se encontraren en cualquiera de las circunstancias señaladas en el artículo 24 ter.
    El comiso de ganancias sin condena previa será impuesto de conformidad al procedimiento especial previsto en el Título III bis del Libro IV del Código Procesal Penal.
    La acción para obtener el comiso de ganancias en virtud de este artículo prescribirá en el plazo de cuatro años, contado desde que ha transcurrido el plazo de prescripción de la acción penal respectiva.


    ART. 294 TER.

    Cuando la cosa usada como instrumento por una organización delictiva o criminal o que resulte de dichos delitos sea dinero o haya sido enajenada, perdida u ocultada, el juez deberá imponer comiso sustitutivo por un valor equivalente.
    El comiso por valor equivalente sólo procederá como consecuencia adicional a la pena. En la determinación del valor equivalente de la cosa a ser decomisada no podrán descontarse los gastos que han sido necesarios para perpetrar el hecho. El valor equivalente se extenderá, asimismo, a los frutos o utilidades de los efectos del hecho.
    El Ministerio Público deberá solicitar la aplicación del comiso por valor equivalente en la oportunidad procesal prevista para solicitar el comiso de ganancias, y la discusión sobre el monto del valor equivalente tendrá lugar en la oportunidad procesal prevista para la determinación de la magnitud del comiso de ganancias.


    ART. 295.

    El tribunal prescindirá de las penas señaladas en los artículos 292 y 293 o impondrá la pena inferior en uno o dos grados al integrante que:

    1. Antes de tener lugar alguno de los hechos cuya perpetración constituye el fin o la actividad de la asociación, revele a la autoridad la existencia de la asociación, sus planes y propósitos o la identidad de sus miembros.

    2. Haya o no intervenido en la perpetración de los delitos que constituyen el fin o la actividad de la asociación o que corresponden a medios de los que ella se vale, revele a la autoridad la existencia de la asociación, sus planes y propósitos o la identidad de sus miembros de tal modo que a juicio del tribunal la autoridad haya estado en condiciones de disolverla antes de la perpetración de hechos ulteriores.



    § XI.

    De las amenazas de atentado contra las personas y propiedades.





    ART. 296.

    El que amenazare seriamente a otro con causar a él mismo o a su familia, en su persona, honra o propiedad, un mal que constituya delito, siempre que por los antecedentes aparezca verosímil la consumación del hecho, será castigado:
    1º. Con presidio menor en sus grados medio a máximo, si hubiere hecho la amenaza exigiendo una cantidad o imponiendo ilegítimamente cualquiera otra condición y el culpable hubiere conseguido su propósito.
    2° Con presidio menor en sus grados mínimo a medio, si hecha la amenaza bajo condición el culpable no hubiere conseguido su propósito,
    3.° Con presidio menor en su grado mínimo, si la amenaza no fuere condicional; a no ser que merezca mayor pena el hecho consumado, caso en el cual se impondrá ésta.
    Cuando las amenazas se hicieren por escrito o por medio de emisario, éstas se estimarán como circunstancias agravantes.
    Para los efectos de este artículo se entiende por familia el cónyuge, los parientes en la línea recta de consanguinidad o afinidad legítima, los padres e hijos naturales y la descendencia legítima de éstos, los hijos ilegítimos reconocidos y los colaterales hasta el tercer grado de consanguinidad o afinidad legítimas.



    ART. 297.

    Las amenazas de un mal que no constituya delito hechas en la forma expresada en los números 1º o 2º del artículo anterior, serán castigadas con la pena de reclusión menor en sus grados mínimo a medio.



    ART. 297. BIS

    Cuando las amenazas se hicieren contra los profesionales y funcionarios de los establecimientos de salud, públicos o privados, o contra los profesionales, funcionarios y manipuladores de alimentos de establecimientos educacionales, públicos o privados, al interior de sus dependencias o mientras éstos se encontraren en el ejercicio de sus funciones o en razón, con motivo u ocasión de ellas, se impondrá el grado máximo o el máximum de las penas previstas en los dos artículos anteriores en sus respectivos casos.


    ART. 298.

    En los casos de los tres artículos precedentes se podrá condenar además al amenazador a dar caución de no ofender al amenazado, y en su defecto a la pena de sujeción a la vigilancia de la autoridad.




    § XII.

    De la evasión de los detenidos y el ingreso de los elementos que se señalan a los recintos penitenciarios.








    ART. 299.

    El empleado público culpable de connivencia en la evasión de un preso o detenido cuya conducción o custodia le estuviere confiada, será castigado:

    1.° En el caso de que el fugitivo se halle condenado por ejecutoria a alguna pena, con la inferior en dos grados y la de inhabilitación especial perpetua para el cargo u oficio.
    2.° Con la pena inferior en tres grados a la señalada por la ley si al fugitivo no se le hubiere condenado por sentencia ejecutoriada, y con la de inhabilitación especial temporal para el cargo u oficio en su grado medio.

    ART. 300.

    El particular que, encargado de la conducción o custodia de un preso o detenido, se hallare en alguno de los casos del artículo precedente, será castigado con las penas inmediatamente inferiores en grado a las señaladas para el empleado público.


    ART. 301.

    Los que extrajeron de las cárceles o de establecimientos penales a alguna persona presa o detenida en ellos o le proporcionare la evasión, serán castigados con las penas señaladas en el art. 299, según el caso respectivo, si emplearen la violencia o el soborno, y con las inferiores en un grado cuando se valieren de otros medios.
    Si fuera de dichos establecimientos se verificare la sustracción o se facilitare la fuga de los presos o detenidos violentando o sorprendiendo a los encargados de conducirlos o custodiarlos, se aplicarán respectivamente las penas inferiores en un grado a las señaladas en el inciso precedente.


    ART. 302.

    Cuando la evasión o fuga de los presos o detenidos se efectuare por descuido culpable de los guardianes, se aplicará a éstos una pena inferior en un grado a la que les correspondería en caso de connivencia según los artículos anteriores.



    ART. 303.

    Si los fugados fueron dos o más, se tomará como base para fijar la pena de los procesados a quienes se refiere este párrafo, la mayor de las que estuvieren sufriendo o merecieren aquéllos.



    ART. 304.

    Cuando empleando las reglas anteriores para aplicar la pena, no pudiera ésta determinarse por falta de grados inferiores o por no ser aplicables las de inhabilitación y suspensión, se impondrá la última que contenga la respectiva escala gradual.


    ART. 304 bis.

    El que sin estar legal o reglamentariamente autorizado al efecto ingresare, intentare o permitiere ingresar por cualquier medio a un establecimiento penitenciario intercomunicadores, teléfonos, partes de ellos, chips telefónicos u otros elementos tecnológicos que permitan comunicarse con el exterior, será sancionado con la pena de presidio menor en su grado mínimo a medio.
    Si las conductas a que se refiere el inciso anterior fueren perpetradas por un abogado, procurador o empleado público, la pena no se aplicará en su grado mínimo y, además, conllevará desde suspensión en su grado mínimo a inhabilitación absoluta temporal en cualquiera de sus grados para el ejercicio de la profesión y del cargo u oficio, respectivamente.
    Si la conducta descrita en el inciso primero fuere cometida por el empleado público para facilitar la perpetración de alguno de los crímenes o simples delitos previstos en el artículo 27 letra a) de la ley N° 19.913, artículos 1, 2, 3 y 4 de la ley N° 20.000, y en los artículos 141, 142, 268 ter, 391, 438, 467 y 468 del presente Código, se aumentará la pena del inciso primero en un grado y además conllevará la inhabilitación absoluta perpetua para cargos u oficios públicos.


    ART. 304 ter.

    El que, encontrándose privado de libertad en un establecimiento penitenciario, tuviere en su poder cualquiera de los elementos señalados en el artículo anterior, sin estar legal o reglamentariamente autorizado al efecto, será sancionado con la pena de presidio menor en su grado mínimo a medio.
    El funcionario público que, teniendo conocimiento de la existencia no autorizada al interior de un establecimiento penitenciario de cualquiera de los elementos señalados en el artículo anterior, omitiere denunciar el hecho a la autoridad competente, será sancionado con presidio menor en su grado mínimo y suspensión en su grado mínimo a inhabilitación absoluta temporal en cualquiera de sus grados para el ejercicio del cargo u oficio.
    Está exento de responsabilidad penal el abogado defensor de quien tuviere en su poder los elementos a que se refiere el artículo anterior, y que omitiere denunciar este hecho.
    Lo dispuesto en el presente artículo se entiende sin perjuicio de las sanciones que fueren procedentes conforme al Reglamento de Establecimientos Penitenciarios, contenido en el decreto Nº 518, promulgado y publicado el año 1998, del Ministerio de Justicia.

    § 13. Atentados contra el medio ambiente



    ART. 305.-

    Será sancionado con presidio o reclusión menor en sus grados mínimo a medio el que sin haber sometido su actividad a una evaluación de impacto ambiental a sabiendas de estar obligado a ello:

    1. Vierta sustancias contaminantes en aguas marítimas o continentales.

    2. Extraiga aguas continentales, sean superficiales o subterráneas, o aguas marítimas.

    3. Vierta o deposite sustancias contaminantes en el suelo o subsuelo, continental o marítimo.

    4. Vierta tierras u otros sólidos en humedales.

    5. Extraiga componentes del suelo o subsuelo.

    6. Libere sustancias contaminantes al aire.

    La pena será de presidio o reclusión menor en sus grados medio a máximo si el infractor perpetra el hecho estando obligado a someter su actividad a un estudio de impacto ambiental.


    ART. 306.-

    Las penas señaladas en el inciso primero del artículo anterior serán aplicables al que, contando con autorización para verter, liberar o extraer cualquiera de las sustancias o elementos mencionados en los números 1 a 6 del artículo 305, incurra en cualquiera de los hechos allí previstos, contraviniendo una norma de emisión o de calidad ambiental, incumpliendo las medidas establecidas en un plan de prevención, de descontaminación o de manejo ambiental, incumpliendo una resolución de calificación ambiental, o cualquier condición asociada al otorgamiento de la autorización, y siempre que el infractor hubiere sido sancionado administrativamente en, al menos, dos procedimientos sancionatorios distintos, por infracciones graves o gravísimas, dentro de los diez años anteriores al hecho punible y cometidas en relación con una misma unidad sometida a control de la autoridad.


    ART. 307.-

    Las penas señaladas en el inciso primero del artículo 305 serán también aplicables al que, contando con autorización para extraer aguas continentales, superficiales o subterráneas, las extraiga infringiendo las reglas de su distribución y aprovechamiento en cualquiera de las siguientes circunstancias:

    1. Habiéndose establecido por la autoridad la reducción temporal del ejercicio de esos derechos de aprovechamiento.

    2. En una zona que haya sido declarada zona de prohibición para nuevas explotaciones acuíferas, haya sido decretada área de restricción del sector hidrogeológico, que se haya declarado a su respecto el agotamiento de las fuentes naturales de aguas o se la haya declarado zona de escasez hídrica.


    ART. 308.-

    El que, vertiendo, depositando o liberando sustancias contaminantes, o extrayendo aguas o componentes del suelo o subsuelo, afectare gravemente las aguas marítimas o continentales, superficiales o subterráneas, el suelo o el subsuelo, fuere continental o marítimo, o el aire, o bien la salud animal o vegetal, la existencia de recursos hídricos o el abastecimiento de agua potable, o que afectare gravemente humedales vertiendo en ellos tierras u otros sólidos, será sancionado:

    1. Con la pena de presidio o reclusión mayor en su grado mínimo, si la afectación grave fuere perpetrada concurriendo las circunstancias previstas en los artículos 305, 306 o 307.

    2. Con la pena de presidio o reclusión menor en su grado máximo a presidio mayor en su grado mínimo en los casos no comprendidos en el número precedente, y siempre que no estuviere autorizado para ello.


    ART. 309.-

    El que por imprudencia temeraria o por mera imprudencia o negligencia con infracción de los reglamentos incurriere en los hechos señalados en el artículo anterior, será sancionado:

    1. Con la pena de presidio o reclusión menor en su grado máximo, si la afectación grave fuere perpetrada concurriendo las circunstancias previstas en los artículos 305, 306 o 307.

    2. Con la pena de presidio o reclusión menor en cualquiera de sus grados en los casos no comprendidos en el número precedente.


    ART. 310.-

    El que afectare gravemente uno o más de los componentes ambientales de una reserva de región virgen, un parque nacional, un monumento natural, una reserva nacional o un humedal de importancia internacional, será sancionado con presidio o reclusión mayor en su grado mínimo.
    La misma pena se impondrá al que, infringiendo una resolución de calificación ambiental o sin haber sometido su actividad a una evaluación de impacto ambiental estando obligado a ello, afectare gravemente un glaciar.     
    La pena será de presidio o reclusión menor en su grado máximo si cualquiera de los hechos señalados en los incisos anteriores fuere perpetrado por imprudencia temeraria o por mera imprudencia o negligencia con infracción de los reglamentos.


    ART. 310 bis.-

    Para los efectos de los tres artículos precedentes se entenderá por afectación grave de uno o más componentes ambientales el cambio adverso producido en alguno de ellos, siempre que concurra alguna de las siguientes circunstancias:

    1. Tener una extensión espacial de relevancia, según las características ecológicas o geográficas de la zona afectada.

    2. Tener efectos prolongados en el tiempo.

    3. Ser irreparable o difícilmente reparable.

    4. Alcanzar a un conjunto significativo de especies, según las características de la zona afectada.

    5. Incidir en especies categorizadas como extintas, extintas en grado silvestre, en peligro crítico o en peligro o vulnerables.

    6. Poner en serio riesgo de grave daño la salud de una o más personas.

    7. Afectar significativamente los servicios o funciones ecosistémicas del elemento o componente ambiental.

    Tratándose de los hechos previstos en el número 1 del artículo 308 y en los incisos primero y segundo del artículo 310, si la afectación grave causa un daño irreversible a un ecosistema, se impondrá el máximum de las penas a ellos señaladas.


    ART. 310 ter.-

    Además de las penas señaladas en las disposiciones de este Párrafo, el tribunal impondrá la pena de multa:

    1. De ciento veinte a sesenta mil unidades tributarias mensuales, si la pena máxima señalada fuere inferior a la de presidio o reclusión menor en su grado máximo.

    2. De doce mil a noventa mil unidades tributarias mensuales, si la pena mínima señalada fuere inferior a la de presidio o reclusión menor en su grado máximo.

    3. De veinticuatro mil a ciento veinte mil unidades tributarias mensuales, si la pena mínima señalada fuere igual o superior a la de presidio o reclusión menor en su grado máximo.

    El monto de la pena de multa pagada será abonado a la sanción de multa no constitutiva de pena que le fuere impuesta por el mismo hecho. Si el condenado hubiere pagado una multa no constitutiva de pena por el mismo hecho, el monto pagado será abonado a la pena de multa impuesta.


    ART. 311.-

    Tratándose de los hechos previstos en los artículos 305, 306 o 307, la pena sólo será la multa de ciento veinte a doce mil unidades tributarias mensuales cuando:

    1. La cantidad vertida, liberada o extraída en exceso no supere en forma significativa el límite permitido o autorizado, atendidas las características de la sustancia y la condición del medio ambiente que pudieren verse afectadas por el exceso y, además,

    2. El infractor hubiere obrado con diligencia para restablecer las emisiones o extracciones al valor permitido o autorizado y para evitar las consecuencias dañinas del hecho.

    El tribunal podrá imponer una multa inferior a la señalada, desde una unidad tributaria mensual, cuando el hecho fuere perpetrado extrayendo aguas continentales, superficiales o subterráneas, se cumpliere la condición señalada en el número 1 y la extracción hubiere estado destinada a las bebidas y usos domésticos de subsistencia.


    ART. 311 bis.-

    Tratándose de los hechos previstos en el artículo 310, el tribunal impondrá al condenado como pena accesoria la prohibición perpetua de ingresar al área afectada, y podrá extenderla mediante resolución fundada a otras áreas de las señaladas en dicho artículo que exhiban características ecosistémicas similares.
    El tribunal podrá autorizar el ingreso al área con el único objeto de recorrer un trayecto entre dos lugares ubicados fuera de ella, cuando no hubiere vías alternativas disponibles.

   
    ART. 311 ter.-

    Fuera de los casos señalados en el artículo 310, el tribunal podrá apreciar la concurrencia de una atenuante muy calificada conforme al artículo 68 bis cuando el hechor repare el daño ambiental causado por el hecho.


    ART. 311 quáter.-

    Las penas previstas en las disposiciones de este Párrafo para los atentados contra el medio ambiente perpetrados extrayendo aguas continentales, superficiales o subterráneas, serán impuestas sin perjuicio de la aplicación de las penas que correspondan por el delito de usurpación.


    ART. 311 quinquies.-

    Cuando la persona obligada por las normas ambientales o el infractor a que se refieren las disposiciones de este Párrafo fuere una persona jurídica, se entenderá que esa calidad concurre respecto de quienes hubieren intervenido por ella en el hecho punible.


    ART. 311 sexies.-

    Para efectos de lo dispuesto en este Párrafo, cuenta con la autorización correspondiente quien la tiene en el momento del hecho, aun cuando ella sea posteriormente declarada inválida.
    No vale como autorización la que hubiere sido obtenida mediante engaño, coacción o cohecho, ni aquella que la persona autorizada sabe que es o ha devenido manifiestamente improcedente.
    La declaración administrativa de no estar obligado a someter la actividad a una evaluación de impacto ambiental exime de responsabilidad conforme al artículo 305, a menos que concurran las circunstancias señaladas en el inciso precedente.


    ART. 312.-

    Si con ocasión de la investigación o el juicio por los hechos previstos en las disposiciones del presente Párrafo, el tribunal estimare procedente la imposición al imputado o condenado de condiciones destinadas a evitar o reparar el daño ambiental, consultará a los organismos técnicos competentes. Si las impusiere, oficiará a la autoridad reguladora pertinente para la fiscalización de su cumplimiento, y ésta última quedará obligada a informar al tribunal. La autoridad requerida podrá ejercer todas las competencias fiscalizadoras establecidas por la ley para tal efecto, y quedará obligada a informar al tribunal.

    § XIV.

    Crímenes y Simples Delitos contra la Salud Pública.


    ART. 313.

    El que, sin hallarse competentemente autorizado, elaborare sustancias o productos nocivos a la salud o traficare en ellos, estando prohibidos su fabricación o tráfico, será castigado con reclusión menor en su grado medio y multa de ciento a quinientos pesos.


    Artículo 313° a. El que, careciendo de título profesional competente o de la autorización legalmente exigible para el ejercicio profesional, ejerciere actos propios de la respectiva profesión de médico-cirujano, dentista, químico-farmacéutico, bioquímico u otra de características análogas, relativa a la ciencia y arte de precaver y curar las enfermedades del cuerpo humano, aunque sea a título gratuito, será penado con presidio menor en su grado medio y multa de seis a veinte unidades tributarias mensuales.
    Para estos efectos se entenderá que ejercen actos propios de dichas profesiones:
    1.- El que se atribuya la respectiva calidad;
    2.- El que ofrezca tales servicios públicamente por cualquier medio de propaganda o publicidad;
    3.- El que habitualmente realizare diagnósticos, prescribiere tratamientos o llevare a cabo operaciones o DO intervenciones curativas de aquellas cuya ejecución exige los conocimientos o las técnicas propios de tales profesiones.
    Las disposiciones de este artículo no se aplicarán en ningún caso a quienes prestaren auxilios cuando no fuere posible obtener oportuna atención profesional.
    En las mismas penas incurrirá el que prestare su nombre para amparar el ejercicio profesional de un tercero no autorizado para el mismo.



    Artículo 313° b. El que, estando legalmente habilitado para el ejercicio de una profesión médica o auxiliar de ella ofreciere abusando de la credulidad del público, la prevención o curación de enfermedades o defectos por fórmulas ocultas o sistemas infalibles, será penado con presidio menor en sus grados mínimo a medio y multa de seis a veinte unidades tributarias mensuales.






    Artículo 313° c Las penas señaladas en los artículos precedentes se impondrán sin perjuicio de las que correspondieren por la muerte, lesiones u otras consecuencias punibles que eventualmente resultaren de la comisión de tales delitos.

    Artículo 313° d. El que fabricare o a sabiendas expendiere a cualquier título sustancias medicinales deterioradas o adulteradas en su especie, cantidad, calidad o proporciones, de modo que sean peligrosas para la salud por su nocividad o por el menoscabo de sus propiedades curativas, será penado con presidio menor en sus grados medio a máximo y multa de seis a cincuenta unidades tributarias mensuales.
    Si la fabricación o expendio fueren clandestinos, ello se considerará como circunstancia de agravante.



    Artículo 314° El que, a cualquier título, expendiere otras sustancias peligrosas para la salud, distintas de las señaladas en el artículo anterior, contraviniendo las disposiciones legales o reglamentarias establecidas en consideración a la peligrosidad de dichas sustancias, será penado con presidio menor en sus grados mínimo a medio y multa de seis a veinte unidades tributarias mensuales.






    Artículo 315° El que envenenare o infectare comestibles, aguas u otras bebidas destinadas al consumo público, en términos de poder provocar la muerte o grave daño para la salud, y el que a sabiendas los vendiere o distribuyere, serán penados con presidio mayor en su grado mínimo y multa de veintiuna a cincuenta unidades tributarias mensuales.
    El que efectuare otras adulteraciones en dichas sustancias destinadas al consumo público, de modo que sean peligrosas para la salud por su nocividad o por el menoscabo apreciable de sus propiedades alimenticias, y el que a sabiendas las vendiere o distribuyere, serán penados con presidio menor en su grado máximo y multa de seis a cincuenta unidades tributarias mensuales.
    Para los efectos de este artículo, se presumirá que la situación de vender o distribuir establecida en los incisos precedentes se configura por el hecho de tener a la venta en un lugar público los artículos alimenticios a que éstos se refieren. La clandestinidad en la venta o distribución y la publicidad de alguno de estos productos constituirán circunstancias agravantes.
    Se presume que son destinados al consumo público los comestibles, aguas u otras bebidas elaborados para ser ingeridos por un grupo de personas indeterminadas.
    Los delitos previstos en los incisos anteriores y los correspondientes cuasidelitos a que se refiere el inciso 2° del artículo 317°, sólo podrán perseguirse criminalmente previa denuncia o querella del Ministerio Público o del Director General del Servicio Nacional de Salud o de su delegado, siempre que aquellos no hayan causado la muerte o grave daño para la salud de alguna persona. En los demás, los correspondientes procesos criminales quedarán sometidos a las normas de de las causas que se siguen de oficio.
    No será aplicable al Ministerio Público ni a los funcionarios del Servicio Nacional de Salud respecto de estos delitos, lo dispuesto en los N.os 1 y 3 del artículo 84, respectivamente, del Código de Procedimiento Penal.






    Artículo 316° El que diseminare gérmenes patógenos con el propósito de producir una enfermedad, será penado con presidio mayor en su grado mínimo y multa de veintiuna a treinta unidades tributarias mensuales.


    Artículo 317° Si a consecuencia de cualquiera de los delitos señalados en los cuatro artículos precedentes, se produjere la muerte o enfermedad grave de alguna persona, las penas corporales se elevarán en uno o dos grados, según la naturaleza y número de tales consecuencias, y la multa podrá elevarse hasta el doble del máximo señalado en cada caso.
    Si alguno de tales hechos punibles se cometiere por imprudencia temeraria o por mera negligencia con infracción de los reglamentos respectivos, las penas serán de presidio menor en su grado mínimo o multa de seis a veinte unidades tributarias mensuales.






    Art 318.

    El que pusiere en peligro la salud pública por infracción de las reglas higiénicas o de salubridad, debidamente publicadas por la autoridad, en tiempo de catástrofe, epidemia o contagio, será penado con presidio menor en su grado mínimo a medio o multa de seis a doscientas unidades tributarias mensuales.
    Será circunstancia agravante de este delito cometerlo mediante la convocatoria a espectáculos, celebraciones o festividades prohibidas por la autoridad sanitaria en tiempo de catástrofe, pandemia o contagio.
    En los casos en que el Ministerio Público solicite únicamente la pena de multa de seis unidades tributarias mensuales, se procederá en cualquier momento conforme a las reglas generales del procedimiento monitorio, siendo aplicable lo previsto en el artículo 398 del Código Procesal Penal. Tratándose de multas superiores se procederá de acuerdo con las normas que regulan el procedimiento simplificado.







    ART. 318 bis.

    El que, en tiempo de pandemia, epidemia o contagio, genere, a sabiendas, riesgo de propagación de agentes patológicos con infracción de una orden de la autoridad sanitaria, será sancionado con la pena de presidio menor en su grado medio a máximo, y multa de veinticinco a doscientas cincuenta unidades tributarias mensuales.


    ART. 318 ter.

    El que, a sabiendas y teniendo autoridad para disponer el trabajo de un subordinado, le ordene concurrir al lugar de desempeño de sus labores cuando éste sea distinto de su domicilio o residencia, y el trabajador se encuentre en cuarentena o aislamiento sanitario obligatorio decretado por la autoridad sanitaria, será castigado con presidio menor en sus grados mínimo a medio y una multa de diez a doscientas unidades tributarias mensuales por cada trabajador al que se le hubiere ordenado concurrir.

    ART. 319 a).    Derogado.

    ART. 319 b).    Derogado.
    ART. 319 c).    Derogado.

    ART. 319 d).    Derogado.

    ART. 319 e).    Derogado.

    ART. 319 f).    Derogado.

    ART. 319 g).    Derogado.


    § XV.

    De la infracción de las leyes o reglamentos sobre inhumaciones y exhumaciones.





    ART. 320.

    El que practicare o hiciere practicar una inhumación contraviniendo a lo dispuesto por las leyes o reglamentos respecto al tiempo, sitio y demás formalidades prescritas para las inhumaciones, incurrirá en las penas de reclusión menor en su grado mínimo y multa de seis a diez unidades tributarias mensuales.








    ART. 321.  Derogado.-



    ART. 322.

    El que exhumare o trasladare los restos humanos con infracción de los reglamentos y demás disposiciones de sanidad, sufrirá las penas de reclusión menor en su grado mínimo y multa de seis a diez unidades tributarias mensuales.








    § XV bis.

    Del ultraje de cadáver y sepultura.


     
    ART. 322 bis.

    Ultraje de cadáver. Será sancionado con la pena de reclusión menor en su grado medio, el que en menosprecio de la memoria de quien hubiere muerto:
     
    1° Exhumare total o parcialmente sus restos humanos;
    2° Sustrajere sus restos humanos de quien los tuviere legítimamente, o
    3° Manipulare sus restos humanos o sus cenizas, o realizare sobre cualquiera de ellos actos que los afectaren considerablemente.
     
    Para efectos de este artículo se entenderá que la acción no se realiza en menosprecio de la memoria de quien hubiere muerto si quien la realiza obra con autorización y respetando las reglas de la profesión respectiva o los estándares aceptados en la prestación de servicios mortuorios.
     
    ART. 322 ter.

    Ultraje de sepultura. Será sancionado con la pena de reclusión menor en su grado medio, el que, en menosprecio de la memoria de quien hubiere muerto, profanare su sepultura.

    § XVI.

Crímenes y simples delitos relativos a los ferrocarriles, telégrafos y conductores de correspondencia.





    ART. 323.

    El que destruyere o descompusiere una vía férrea o colocare en ella obstáculos que puedan producir el descarrilamiento, o tratare de producirlo de cualquiera otra manera, será castigado con presidio menor en sus grados mínimo a medio.


    ART. 324.

    Si a virtud de la destrucción, descompostura u obstáculos puestos o por cualquier otro acto ejecutado se verificare el descarrilamiento, la pena será presidio menor en sus grados medio a máximo.


    ART. 325.

    Cuando a consecuencia del accidente producido por los actos relacionados en el artículo anterior, se causaren lesiones u otros daños a las personas, se aplicará al culpable la pena correspondiente al daño causado, siempre que fuere mayor que la señalada en el artículo anterior; en el caso contrario se le impondrá el grado máximo de ésta.



    ART. 326.

    Si el accidente ocasionare la muerte de alguna persona, el culpable sufrirá la pena señalada al homicidio voluntario ejecutado con alevosía, en su grado máximo.



    ART. 327.

    El autor de los hechos que hubieren producido el accidente no sólo es obligado a reparar los daños que la empresa del ferrocarril experimentare, sino también los que sufran los particulares que se encontraban en el tren o que trasportaban por él objetos muebles o semovientes.



    ART. 328.

    La amenaza hecha de palabra o por escrito, de cometer alguno de los delitos previstos en el art. 323, será castigada con reclusión menor en su grado mínimo o con multa de once a veinte unidades tributarias mensuales.








    ART. 329.

    El que por ignorancia culpable, imprudencia o descuido, o por inobservancia de los reglamentos del camino, que deba conocer, causare involuntariamente accidentes que ocasionen lesión o daño a alguna persona, sufrirá las penas de reclusión menor en su grado mínimo y multa de seis a diez unidades tributarias mensuales.
    Cuando el accidente ocasionare la muerte a una persona, la pena será reclusión menor en cualquiera de sus grados.
    Las disposiciones de este artículo son también aplicables a los empresarios, directores o empleados de la línea.





    Art. 330.

    El maquinista, conductor o guarda-frenos que abandonare su puesto o se embriagare durante su servicio, será castigado con presidio menor en su grado mínimo y multa de seis a diez unidades tributarias mensuales.
    Si a consecuencia del abandono del puesto o de la embriaguez ocurrieren accidentes que causaren lesiones a alguna persona, las penas serán presidio menor en su grado medio y multa de once a quince unidades tributarias mensuales.
    Cuando de tales accidentes resultare la muerte de algún individuo, se impondrán al culpable las penas de presidio menor en su grado máximo y multa de dieciséis a veinte unidades tributarias mensuales.








    ART. 331.

    En el caso de abandono intencional por causar daño a alguna de las personas que iban en los trenes, se aplicarán al maquinista, conductor o guarda-frenos, según los casos y aumentadas en un grado, las penas que señalan los arts. 323, 324 Y 325.



    ART. 332.

    Las penas que establecen los tres artículos precedentes se aplicarán respectivamente a cualquier otro empleado en el servicio del camino que teniendo un cargo que desempeñar, lo abandonare o ejerciere mal con peligro de la seguridad del tráfico.



    ART. 333.

    El que por imprudencia rompiere los postes o alambres de una línea telegráfica establecida o en construcción, o ejecutare actos que interrumpan el servicio de los telégrafos, será penado con multa de seis a diez unidades tributarias mensuales.









    ART. 334.

    El que intencionalmente interrumpiere la comunicación telegráfica o causare daño a una línea en construcción rompiendo los alambres o postes, inutilizando los aparatos de trasmisión o por cualquier otro medio, sufrirá las penas de presidio menor en su grado mínimo y multa de seis a diez unidades tributarias mensuales.








    ART. 335.

    Los que en casos de motín, insurrección, guerra exterior u otra calamidad pública, rompieren los alambres o postes, destruyeren las máquinas o aparatos telegráficos, se apoderaren con violencia o amenazas de las oficinas, o empleando los mismos medios impidieren de cualquier modo la correspondencia telegráfica entre los depositarios de la autoridad pública, o se opusieren con fuerza o violencia al restablecimiento de una línea telegráfica, serán castigados con presidio menor en cualquiera de sus grados y multa de once a veinte unidades tributarias mensuales.









    ART. 336.

    Los autores del daño estarán siempre obligados a indemnizar los costos que demanden las reparaciones o el restablecimiento de las líneas deterioradas o destruidas.



    ART. 337.

    El empleado de una oficina telegráfica que divulgare el contenido de un mensaje sin autorización expresa de la persona que lo dirige o a quien es dirigido, incurrirá en una multa de seis a diez unidades tributarias mensuales y deberá indemnizar los perjuicios provenientes de la divulgación.
    Las mismas penas se impondrán al empleado que, por descuido culpable, no trasmitiere fielmente un mensaje telegráfico y, si en la trasmisión infiel hubiere mala fe, se estará a lo dispuesto en el art. 195.






    ART. 338.

    El empleado que habiendo trasmitido órdenes encaminadas a la persecución o aprehensión de delincuentes o para que se practiquen diligencias dirigidas a una averiguación judicial o gubernativa, trasmitiere avisos o prevenciones que hagan ilusorias dichas órdenes, incurrirá en la pena de reclusión menor en su grado medio.
    Igual pena se aplicará cuando maliciosamente frustrare las medidas de la autoridad en tales casos, con una trasmisión o traducción infiel.



    ART. 339.
   
    En el momento de motín o asonada es prohibido a toda oficina telegráfica:
    1°. Trasmitir o tolerar que se trasmitan mensajes dirigidos a fomentar o favorecer el desorden.
    2°  Dar aviso de la marcha que siguen los sucesos y tumultos, si no es a la autoridad o con asentimiento de ésta.
    3.° Instruir del movimiento de tropas o de las medidas tomadas para combatir la insurrección o desorden.
    4.º Comunicar toda noticia cuyo objeto sea frustrar las providencias tomadas para restablecer la tranquilidad interior.
    La infracción de cualquiera de estas prohibiciones sujeta al infractor a las penas de reclusión menor en su grado medio y multa de seis a diez unidades tributarias mensuales; sin perjuicio de ser castigado como instigador o como cómplice del motín o asonada, siempre que los hechos dieren mérito para considerarlo tal.






    ART. 340.

    Cuando en una oficina telegráfica so reincidiere en las infracciones de que habla el artículo precedente, podrá la autoridad superior inmediata prohibir el uso del telégrafo o someterlo a su dirección o inspección mientras duren las circunstancias extraordinarias de motín, sedición, etc.


    ART. 341.

    El que acometiere a un conductor de correspondencia pública para interceptarla o detenerla o para apoderarse de ella o de cualquier modo inutilizarla, será castigado con presidio menor en sus grados medio a máximo, si interviniere violencia. Si no interviniere violencia, con presidio menor en sus grados mínimo a medio.
    Lo cual no obsta para que se aplique la pena correspondiente al delito cometido en la persona del conductor o en la sustracción de la correspondencia, siempre que fuere mayor.



   
    TÍTULO SÉPTIMO

    CRÍMENES Y DELITOS CONTRA EL ORDEN DE LAS FAMILIAS, CONTRA LA MORALIDAD PÚBLICA Y CONTRA LA INTEGRIDAD SEXUAL.









    § I.

    Aborto.


    ART. 342.

    El que maliciosamente causare un aborto será castigado:
    1.° Con la pena de presidio mayor en su grado mínimo, si ejerciere violencia en la persona de la mujer embarazada.
    2.° Con la de presidio menor en su grado máximo, si, aunque no la ejerza, obrare sin consentimiento de la mujer.
    3.° Con la de presidio menor en su grado medio, si la mujer consintiere.


    ART. 343.

    Será castigado con presidio menor en sus grados mínimo a medio, el que con violencias ocasionare un aborto, aun cuando no haya tenido propósito de causarlo, con tal que el estado de embarazo de la mujer sea notorio o le constare al hechor.


    ART. 344.

    La mujer que, fuera de los casos permitidos por la ley, causare su aborto o consintiere que otra persona se lo cause, será castigada con presidio menor en su grado máximo.
    Si lo hiciere por ocultar su deshonra, incurrirá en la pena de presidio menor en su grado medio.


    ART. 345.

    El facultativo que, abusando de su oficio, causare el aborto o cooperare a él, incurrirá respectivamente en las penas señaladas en el art. 342, aumentadas en un grado.


    § II.

    Abandono de niños y personas desvalidas.





    ART. 346.

    El que abandonare en un lugar no solitario a un niño menor de siete años, será castigado con presidio menor en su grado mínimo.


    ART. 347.

    Si el abandono se hiciere por los padres legítimos o ilegítimos o por personas que tuvieren al niño bajo su cuidado, la pena será presidio menor en su grado máximo, cuando el que lo abandona reside a menos de cinco quilómetros de un pueblo o lugar en que hubiere casa de expósitos, y presidio menor en su grado medio en los demás casos.


    ART. 348.

    Si a consecuencia del abandono resultaron lesiones graves o la muerte del niño, se impondrá al que lo efectuare la pena de presidio mayor en su grado mínimo, cuando fuere alguna de las personas comprendidas en el artículo anterior, y la de presidio menor en su grado máximo en el caso contrario.
    Lo dispuesto en este artículo y en los dos precedentes no se aplica al abandono hecho en casas de expósitos.



    ART. 349.

    El que abandonare en un lugar solitario a un niño menor de diez años, será castigado con presidio menor en su grado medio.


    Art. 350.

    La pena será presidio mayor en su grado mínimo cuando el que abandona es alguno de los relacionados en el art. 347.


    ART. 351.

    Si del abandono en un lugar solitario resultaren lesiones graves o la muerte del niño, se impondrá al que lo ejecuta la pena de presidio mayor en su grado medio, cuando fuere alguna de las personas a que se refiere el artículo precedente, y la de presidio mayor en su grado mínimo en el caso contrario.


    ART. 352.

    El que abandonare a su cónyuge o a un ascendiente o descendiente, legítimo o ilegítimo, enfermo o imposibilitado, si el abandonado sufriere lesiones graves o muriere a consecuencia del abandono, será castigado con presidio mayor en su grado mínimo.


    § III.

    Crímenes y simples delitos contra el estado civil de las personas.




    ART. 353.

    La suposición de parto y la sustitución de un niño por otro, serán castigadas con las penas de presidio menor en cualquiera de sus grados y multa de veintiuna a veinticinco unidades tributarias mensuales.









    ART. 354.

    El que usurpare el estado civil de otro, sufrirá la pena de presidio menor en sus grados medio a máximo y multa de once a veinte unidades tributarias mensuales.

    Las mismas penas se impondrán al que sustrajere, ocultare o expusiere a un hijo legítimo o ilegítimo con ánimo verdadero o presunto de hacerle perder su estado civil.







    ART. 355.

    El que hallándose encargado de la persona de un menor no lo presentare, reclamándolo sus padres, guardadores o la autoridad, a petición de sus demás parientes o de oficio, ni diere explicaciones satisfactorias acerca de su desaparición, sufrirá la pena de presidio menor en su grado medio.


    ART. 356.

    El que teniendo a su cargo la crianza o educación de un menor de diez años, lo entregare a un establecimiento público o a otra persona, sin la anuencia de la que se lo hubiere confiado o de la autoridad en su defecto, y de ello resultare perjuicio gravo, será castigado con reclusión menor en su grado medio y multa de seis a diez unidades tributarias mensuales.









    ART. 357.

    El que indujere a un menor de edad, pero mayor de diez años, a que abandone la casa de sus padres, guardadores o encargados de su persona, sufrirá las penas de reclusión menor en cualquiera de sus grados y multa de once a veinte unidades tributarias mensuales.








    § IV.

    Del rapto. Derogado.




    ART. 358.    Derogado.

    ART. 359.    Derogado.

    ART. 360.    Derogado.



    § V.

    De la violación.




    ART. 361.

    La violación será castigada con la pena de presidio mayor en su grado mínimo a medio.
    Comete violación el que accede carnalmente, por vía vaginal, anal o bucal, a una persona mayor de catorce años, en alguno de los casos siguientes:

    1º Cuando se usa de fuerza o intimidación.
    2º Cuando la víctima se halla privada de sentido, o cuando se aprovecha su incapacidad para oponerse.
    3º Cuando se abusa de la enajenación o trastorno mental de la víctima.

    ART. 362.

    El que accediere carnalmente, por vía vaginal, anal o bucal, a una persona menor de catorce años, será castigado con presidio mayor en sus grados medio a máximo, aunque no concurra circunstancia alguna de las enumeradas en el artículo anterior.

    § VI.

Del estupro y otros delitos sexuales.


    ART. 363.

    Será castigado con presidio menor en su grado máximo a presidio mayor en su grado mínimo, el que accediere carnalmente, por vía vaginal, anal o bucal, a una persona menor de edad pero mayor de catorce años, concurriendo cualquiera de las circunstancias siguientes:

    1º Cuando se abusa de una anomalía o perturbación mental, aun transitoria, de la víctima, que por su menor entidad no sea constitutiva de enajenación o trastorno.
    2º Cuando se abusa de una relación de dependencia de la víctima, como en los casos en que el agresor está encargado de su custodia, educación o cuidado, o tiene con ella una relación laboral.
    3º Cuando se abusa del grave desamparo en que se encuentra la víctima.
    4º Cuando se engaña a la víctima abusando de su inexperiencia o ignorancia sexual.
    ART. 364.    Derogado.

    Art. 365.    Derogado.


    ART. 365 bis.

    Si la acción sexual consistiere en la introducción de objetos de cualquier índole, por vía vaginal, anal o bucal, o se utilizaren animales en ello, será castigada:

    1.- con presidio mayor en su grado mínimo a medio, si concurre cualquiera de las circunstancias enumeradas en el artículo 361;
    2.- con presidio mayor en cualquiera de sus grados, si la víctima fuere menor de catorce años, y 3.- con presidio menor en su grado máximo a presidio mayor en su grado mínimo, si concurre alguna de las circunstancias enumeradas en el artículo 363 y la víctima es menor de edad, pero mayor de catorce años.
    ART. 366.

    El que abusivamente realizare una acción sexual distinta del acceso carnal con una persona mayor de catorce años, será castigado con presidio menor en su grado máximo, cuando el abuso consistiere en la concurrencia de alguna de las circunstancias enumeradas en el artículo 361.

    Igual pena se aplicará cuando el abuso consistiere en la concurrencia de alguna de las circunstancias enumeradas en el artículo 363, siempre que la víctima fuere mayor de catorce y menor de dieciocho años.

    Se aplicará la pena de presidio menor en su grado mínimo a medio, cuando el abuso consistiere en el empleo de sorpresa u otra maniobra que no suponga consentimiento de la víctima, siempre que ésta sea mayor de catorce años.

    ART. 366 bis.

    El que realizare una acción sexual distinta del acceso carnal con una persona menor de catorce años, será castigado con la pena de presidio menor en su grado máximo a presidio mayor en su grado mínimo.

    ART. 366 ter.


    Para los efectos de los tres artículos anteriores, se entenderá por acción sexual cualquier acto de significación sexual y de relevancia realizado mediante contacto corporal con la víctima, o que haya afectado los genitales, el ano o la boca de la víctima, aun cuando no hubiere contacto corporal con ella.

    ART. 366 quáter.

    El que, sin realizar una acción sexual en los términos anteriores, para procurar su excitación sexual o la excitación sexual de otro, realizare acciones de significación sexual ante una persona menor de catorce años, será castigado con presidio menor en su grado medio a máximo.

    Si se determinare a una persona menor de catorce años a realizar acciones de significación sexual delante suyo o de otro, o se la hiciere ver o escuchar material pornográfico o de explotación sexual o presenciar espectáculos del mismo carácter, la pena será presidio menor en su grado máximo.

    Será sancionado con la misma pena del inciso precedente al que determinare a una persona menor de catorce años a enviar, entregar o exhibir:
     
    a) Imágenes o grabaciones en que se representaren acciones de significación sexual de su persona o de otro menor de catorce años de edad.
    b) Imágenes o grabaciones de sus genitales o los de otra persona menor de catorce años.

    Quien realice alguna de las conductas descritas en los incisos anteriores con una persona menor de edad pero mayor de catorce años, concurriendo cualquiera de las circunstancias del numerando 1º del artículo 361 o de las enumeradas en el artículo 363 o mediante amenazas en los términos de los artículos 296 y 297, tendrá las mismas penas señaladas en los incisos anteriores.

    Las penas señaladas en el presente artículo se aplicarán también cuando los delitos descritos en él sean cometidos a distancia, mediante cualquier medio electrónico.

    Si en la comisión de cualquiera de los delitos descritos en este artículo, el autor falseare su identidad o edad, se aumentará la pena aplicable en un grado.



    § 6 bis. Explotación sexual comercial y material pornográfico de niños, niñas y adolescentes.

    ART. 366 quinquies. Derogado.




    ART. 367.

    El que promoviere o facilitare la explotación sexual de una persona menor de dieciocho años sufrirá la pena de presidio mayor en su grado mínimo.
    Si se perpetrare el hecho explotándola en razón de su dependencia personal o económica o si concurriere habitualidad, la pena será de presidio mayor en cualquiera de sus grados y multa de treinta y una a treinta y cinco unidades tributarias mensuales.
    Para efectos de lo dispuesto en el inciso primero, se entenderá por explotación sexual la utilización de una persona menor de dieciocho años para la realización de una acción sexual o de una acción de significación sexual con ella a cambio de cualquier tipo de retribución hacia la víctima o un tercero.

    ART. 367 bis.    Derogado.


    ART. 367 ter.

    El que obtuviere la realización de una acción sexual de una persona menor de dieciocho años a cambio de cualquier tipo de retribución hacia la víctima o un tercero, será castigado con presidio menor en su grado máximo.


    ART. 367 quáter.

    El que comercializare, importare, exportare, distribuyere, difundiere o exhibiere material pornográfico o de explotación sexual, cualquiera sea su soporte, en cuya elaboración hayan sido utilizadas personas menores de dieciocho años, será sancionado con la pena de presidio menor en su grado máximo.
    Con la misma pena señalada en el inciso anterior será sancionado el que participare en la producción de dicho material pornográfico o de explotación sexual.
    El que maliciosamente almacenare o adquiriere material pornográfico o de explotación sexual, cualquiera sea su soporte, en cuya elaboración hayan sido utilizadas personas menores de dieciocho años, será castigado con presidio menor en su grado medio.
    Para los efectos de este artículo, se entenderá por material pornográfico o de explotación sexual en cuya elaboración hubieren sido utilizadas personas menores de dieciocho años, toda representación de éstos dedicados a actividades sexuales explícitas, reales o simuladas, o toda representación de sus partes genitales con fines primordialmente sexuales, o toda representación de dichos menores en que se emplee su voz o imagen, con los mismos fines.


    ART. 367 quinquies.

    Las conductas de comercialización, distribución, difusión y exhibición, señaladas en el artículo anterior, se entenderán cometidas en Chile cuando se realicen a través de un sistema de telecomunicaciones al que se tenga acceso desde territorio nacional.


    ART. 367 sexies.

    Lo dispuesto en este párrafo no será aplicable si el hecho fuere constitutivo de un delito sancionado con igual o mayor pena por alguna disposición de los párrafos 5 o 6 del Título VII del Libro Segundo, en cuyo caso el ánimo de lucro, la entrega o promesa de entrega de dinero o especies susceptibles de valoración pecuniaria serán considerados como una sola circunstancia agravante.


    ART. 367 septies.

    El que usando dispositivos técnicos transmitiere la imagen o sonido de una situación o interacción que permitiere presenciar, observar o escuchar la realización de una acción sexual o de una acción de significación sexual, por parte de una persona menor de dieciocho años, será sancionado con presidio menor en su grado máximo.


    ART. 367 octies.
   
    Para los efectos de determinar la reincidencia de la circunstancia 16 del artículo 12, en los delitos sancionados en este párrafo, se considerarán también las sentencias firmes dictadas en un Estado extranjero, aun cuando la pena impuesta no haya sido cumplida.


    § VII.

    Disposiciones comunes a los tres párrafos anteriores.




    ART. 368.

    Si los delitos previstos en los tres párrafos anteriores hubieren sido cometidos por autoridad pública, ministro de un culto religioso, guardador, maestro, empleado o encargado por cualquier título o causa de la educación, guarda, curación o cuidado del ofendido, se impondrá al responsable la pena señalada al delito con exclusión de su grado mínimo, si ella consta de dos o más grados, o de su mitad inferior, si la pena es un grado de una divisible. La misma regla se aplicará a quien hubiere cometido los mencionados delitos en contra de un menor de edad con ocasión de las funciones que desarrolle, aun en forma esporádica, en recintos educacionales, y al que los cometa con ocasión del servicio de transporte escolar que preste a cualquier título.
    Exceptúanse los casos en que el delito sea de aquellos que la ley describe y pena expresando las circunstancias de usarse fuerza o intimidación, abusarse de una relación de dependencia de la víctima o abusarse de autoridad o confianza.





    ART. 368 bis.

    Sin perjuicio de lo dispuesto en el artículo 63, en los delitos señalados en los tres párrafos anteriores, serán circunstancias agravantes las siguientes:

    1º La 1ª del artículo 12.
    2º Ser dos o más los autores del delito.




    ART. 368 bis A.

    La circunstancia atenuante señalada en el N° 7 del artículo 11 no podrá aplicarse tratándose de los delitos previstos en los artículos 141, inciso final; 142, inciso final; 150 A, 150 D, 361, 362, 363, 365 bis; 366, incisos primero y segundo, 366 bis, 366 quáter, 367 y 367 ter, 372 bis, 411 quáter cuando se cometa con fines de explotación sexual, y 433, número 1, en relación con la violación.


    ART. 368 ter.

    Cuando, en la comisión de los delitos señalados en los artículos 366 quáter, 367, 367 ter, 367 quáter o 367 septies se utilizaren establecimientos o locales, a sabiendas de su propietario o encargado, o no pudiendo éste menos que saberlo, podrá decretarse en la sentencia su clausura definitiva.

    Asimismo, durante el proceso judicial respectivo, podrá decretarse, como medida cautelar, la clausura temporal de dichos establecimientos o locales.



    ART. 369.

    No se puede proceder por causa de los delitos previstos en los artículos 361 a 366 quáter, sin que, a lo menos, se haya denunciado el hecho a la justicia, al Ministerio Público o a la policía por la persona ofendida o por su representante legal.

    Si la persona ofendida no pudiere libremente hacer por sí misma la denuncia, ni tuviere representante legal, o si, teniéndolo, estuviere imposibilitado o implicado en el delito, podrá procederse de oficio por el Ministerio Público, que también estará facultado para deducir las acciones civiles a que se refiere el artículo 370. Sin perjuicio de lo anterior, cualquier persona que tome conocimiento del hecho podrá denunciarlo.

    Con todo, tratándose de víctimas menores de edad, se estará a lo dispuesto en el artículo 369 quinquies de este Código y en el inciso segundo del artículo 53 del Código Procesal Penal.

    En caso de que un cónyuge o conviviente cometa alguno de los delitos establecidos en los tres párrafos anteriores en contra de aquél con quien hace vida común, se podrá poner término al proceso a requerimiento del ofendido, a menos que el juez, por motivos fundados, no acepte.




    ART. 369 bis. Derogado.



    ART. 369 bis A.

    Tratándose de los delitos previstos en los artículos 141, inciso final; 142, inciso final; 150 A, 150 D, 361, 362, 363, 365 bis; 366, incisos primero y segundo, 366 bis, 366 quáter, 367 y 367 ter, 372 bis, 411 quáter cuando se cometa con fines de explotación sexual, y 433, número 1, en relación con la violación, para la determinación de la cuantía de la pena en los términos dispuestos en el artículo 69, el tribunal tendrá en especial consideración la afectación psíquica o mental de la víctima para la calificación de la extensión del mal producido por el delito.


    ART. 369 ter.

    Cuando existieren sospechas fundadas de que una persona hubiere cometido o preparado la comisión de alguno de los delitos previstos en los artículos 367, 367 ter, 367 quáter, incisos primero y segundo, y 367 septies, y la investigación lo hiciere imprescindible, el tribunal, a petición del Ministerio Público, podrá autorizar la interceptación o grabación de las telecomunicaciones de esa persona. La captación, grabación y registro subrepticio de imágenes o sonidos en lugares cerrados o que no sean de libre acceso al público, podrá ser autorizada por el juez, a solicitud del fiscal cuando existan fundadas sospechas basadas en hechos determinados y graves que lo hagan imprescindible para el esclarecimiento de los hechos. En lo demás, se estará íntegramente a lo dispuesto en los artículos 222 a 225 del Código Procesal Penal.

    Igualmente, bajo los mismos supuestos previstos en el inciso precedente, podrá el tribunal, a petición del Ministerio Público, autorizar la intervención de agentes encubiertos. Mediando igual autorización y con el objeto exclusivo de facilitar la labor de estos agentes, los organismos policiales pertinentes podrán mantener un registro reservado de producciones del carácter investigado. Asimismo, podrán tener lugar entregas vigiladas de material respecto de la investigación de hechos que se instigaren o materializaren a través del intercambio de dichos elementos, en cualquier soporte.

    La actuación de los agentes encubiertos y las entregas vigiladas serán plenamente aplicables al caso en que la actuación de los agentes o el traslado o circulación de producciones se desarrolle a través de un sistema de telecomunicaciones.

    Los agentes encubiertos, el secreto de sus actuaciones, registros o documentos y las entregas vigiladas se regirán por las disposiciones del Párrafo 3° bis del Título I del Libro II del Código Procesal Penal.

    ART. 369 quáter. Suprimido.


    ART. 369 quinquies.
    Tratándose de los delitos establecidos en los artículos 141, inciso final, y 142, inciso final, ambos en relación con la violación; los artículos 150 B y 150 E, ambos en relación con los artículos 361, 362 y 365 bis; los artículos 361, 362, 363, 365 bis, 366, 366 bis, 366 quáter, 367, 367 ter, 367 quáter y 367 septies; el artículo 411 quáter en relación con la explotación sexual; y el artículo 433, N° 1, en relación con la violación, perpetrados en contra de una víctima menor de edad, se considerarán delitos de acción pública previa instancia particular y se regirán por lo dispuesto en el artículo 54 del Código Procesal Penal desde que el ofendido por el delito haya cumplido los dieciocho años de edad, si no se ha ejercido antes la acción penal.

    ART. 370.

    Además de la indemnización que corresponda conforme a las reglas generales, el condenado por los delitos previstos en los artículos 361 a 366 bis será obligado a dar alimentos cuando proceda de acuerdo a las normas del Código Civil.

    ART. 370 bis.

    El que fuere condenado por alguno de los delitos a que se refieren los tres párrafos anteriores cometido en la persona de un menor del que sea pariente, quedará privado de la patria potestad si la tuviere o inhabilitado para obtenerla si no la tuviere y, además, de todos los derechos que por el ministerio de la ley se le confirieren respecto de la persona y bienes del ofendido, de sus ascendientes y descendientes. El juez así lo declarará en la sentencia, decretará la emancipación del menor si correspondiere, y ordenará dejar constancia de ello mediante subinscripción practicada al margen de la inscripción de nacimiento del menor. Además, si el condenado es una de las personas llamadas por ley a dar su autorización para que la víctima salga del país, se prescindirá en lo sucesivo de aquélla.
    El pariente condenado conservará, en cambio, todas las obligaciones legales cuyo cumplimiento vaya en beneficio de la víctima o de sus descendientes.




    ART. 371.

    Los ascendientes, guardadores, maestros y cualesquiera personas que con abuso de autoridad o encargo, cooperaren como cómplices a la perpetración de los delitos comprendidos en los tres párrafos anteriores, serán penados como autores.
    Los maestros o encargados en cualquier manera de la educación o dirección de la juventud, serán además condenados a inhabilitación especial perpetua para el cargo u oficio.



    ART. 372.

    Los comprendidos en el artículo anterior y cualesquiera otros condenados por la comisión de los delitos previstos en los tres párrafos anteriores en contra de un menor de edad, serán también condenados a las penas de interdicción del derecho de ejercer la guarda y ser oídos como parientes en los casos que la ley designa, y de sujeción a la vigilancia de la autoridad durante los diez años siguientes al cumplimiento de la pena principal. Esta sujeción consistirá en informar a Carabineros cada tres meses su domicilio actual. El incumplimiento de esta obligación configurará la conducta establecida en el artículo 496 Nº 1 de este Código.

    El que cometiere cualquiera de los delitos previstos en los artículos 361, 362, 363, 365 bis, 366, 366 bis, 366 quáter, 367, 367 ter, 367 quáter, 367 septies y 372 bis en contra de un menor de edad será condenado, además, a la pena de inhabilitación absoluta perpetua para cargos, empleos, oficios o profesiones ejercidos en ámbitos educacionales o que involucren una relación directa y habitual con personas menores de edad. La misma pena se aplicará a quien cometiere cualquiera de los delitos establecidos en los artículos 142 y 433 Nº 1º, cuando alguna de las víctimas hubiere sufrido violación y fuere menor de edad.

    En los casos del inciso anterior, los fiscales del Ministerio Público, de conformidad con lo dispuesto en el literal g) del artículo 259 del Código Procesal Penal, deberán solicitar la pena de inhabilitación cuando formularen acusación, y el tribunal en caso de dictar sentencia condenatoria deberá imponerla de forma específica, de conformidad con lo dispuesto en el artículo 348 del Código Procesal Penal. Si la sentencia condenatoria no cumpliere con esta exigencia, el fiscal siempre deberá deducir recurso en conformidad a la ley.



    ART. 372 BIS.

    El que, con ocasión de violación, cometiere además homicidio en la persona de la víctima, será castigado con presidio perpetuo a presidio perpetuo calificado.
    Si el autor del delito descrito en el inciso anterior es un hombre y la víctima una mujer, el delito tendrá el nombre de violación con femicidio.

    ART. 372 TER.

    En los delitos contemplados en los artículos 141, inciso final; 142, inciso final; 150 A; 150 D; 361; 362; 363; 365 bis; 366; 366 bis; 366 quáter; 367; 367 ter; 372 bis; 411 quáter; cuando se cometan con fines de explotación sexual, y 433, número 1, en relación con la violación, el juez podrá en cualquier etapa de la investigación o del procedimiento, y aun antes de la formalización, a petición de parte, o de oficio por razones fundadas, disponer las medidas de protección de la víctima y su familia que estime convenientes, tales como la sujeción del imputado a la vigilancia de una persona o institución determinada, las que informarán periódicamente al tribunal; la prohibición de visitar el domicilio, el lugar de trabajo o el establecimiento educacional de la víctima; la prohibición de aproximarse a la víctima o a su familia, la prohibición de tomar contacto con la víctima o con su familia, y, en su caso, la obligación de abandonar el hogar que compartiere con la víctima.
     


    § VIII.

    De los ultrajes públicos a las buenas costumbres.





    ART. 373.

    Los que de cualquier modo ofendieren el pudor o las buenas costumbres con hechos de grave escándalo o trascendencia, no comprendidos expresamente en otros artículos de este Código, sufrirán la pena de reclusión menor en sus grados mínimo a medio.


    ART. 374.

    El que vendiere, distribuyere o exhibiere canciones, folletos u otros escritos, impresos o no, figuras o estampas contrarios a las buenas costumbres, será condenado a las penas de reclusión menor en su grado mínimo o multa de once a veinte unidades tributarias mensuales.
    En las mismas penas incurrirá el autor del manuscrito, de la figura o de la estampa o el que los hubiere reproducido por un procedimiento cualquiera que no sea la imprenta.

    La sentencia condenatoria por este delito ordenará la destrucción total o parcial, según proceda, de los impresos o de las grabaciones sonoras o audiovisuales de cualquier tipo que sean objeto de comiso.
    ART. 374 bis. Derogado.


    ART. 374 ter. Derogado.


    § IX. Del incesto.
    ART. 375.

    El que, conociendo las relaciones que lo ligan, cometiere incesto con un ascendiente o descendiente por consanguinidad o con un hermano consanguíneo, será castigado con reclusión menor en sus grados mínimo a medio.

    ART. 376. Derogado.
    ART. 377. Derogado.
    ART. 378. Derogado.
    ART. 379. Derogado.
    ART. 380. Derogado.
    ART. 381. Derogado.



    Celebración de matrimonios ilegales.





    ART. 382.

    El que contrajere matrimonio estando casado válidamente, será castigado con reclusión menor en su grado máximo.
    En igual pena incurrirá el que contrajere matrimonio estando ordenado in sacris o ligado con voto solemne de castidad.

    ART. 383.

    El que engañare a una persona simulando la celebración de matrimonio con ella, sufrirá la pena de reclusión menor en sus grados medio a máximo.



    ART. 384.

    El que por sorpresa o engaño hiciere intervenir al funcionario que debe autorizar su matrimonio sin haber observado las prescripciones que la ley exige para su celebración, aun cuando el matrimonio sea válido, sufrirá la pena de reclusión menor en su grado mínimo.
    Si lo hiciere intervenir con violencia o intimidación, la pena será reclusión menor en sus grados medio a máximo.

    ART. 385. Derogado.
    ART. 386. Derogado.
    ART. 387. Derogado.
    ART. 388.

    El oficial civil que autorice o inscriba un matrimonio prohibido por la ley o en que no se hayan cumplido las formalidades que ella exige para su celebración o inscripción, sufrirá las penas de relegación menor en su grado medio y multa de seis a diez unidades tributarias mensuales. Igual multa se aplicará al ministro de culto que autorice un matrimonio prohibido por la ley.
    El ministro de culto que, con perjuicio de tercero, cometiere falsedad en el acta o en el certificado de matrimonio religioso destinados a producir efectos civiles, sufrirá las penas de presidio menor en cualquiera de sus grados.


    ART. 389.

    El tercero que impidiere la inscripción, ante un oficial civil, de un matrimonio religioso celebrado ante una entidad autorizada para tal efecto por la Ley de Matrimonio Civil, será castigado con la pena de presidio menor en su grado mínimo o multa de seis a diez unidades tributarias mensuales.


    TÍTULO OCTAVO.
   
    CRÍMENES Y SIMPLES DELITOS CONTRA LAS PERSONAS.
   





    § I.

    Del parricidio.






    ART. 390.

    El que, conociendo las relaciones que los ligan, mate a su padre, madre o hijo, a cualquier otro de sus ascendientes o descendientes o a quien es o ha sido su cónyuge o su conviviente, será castigado, como parricida, con la pena de presidio mayor en su grado máximo a presidio perpetuo calificado.

    §1 bis.

    Del femicidio
     


    Artículo 390 bis.- El hombre que matare a una mujer que es o ha sido su cónyuge o conviviente, o con quien tiene o ha tenido un hijo en común, será sancionado con la pena de presidio mayor en su grado máximo a presidio perpetuo calificado.
    La misma pena se impondrá al hombre que matare a una mujer en razón de tener o haber tenido con ella una relación de pareja de carácter sentimental o sexual sin convivencia.
     
    Artículo 390 ter.- El hombre que matare a una mujer en razón de su género será sancionado con la pena de presidio mayor en su grado máximo a presidio perpetuo.
    Se considerará que existe razón de género cuando la muerte se produzca en alguna de las siguientes circunstancias:
     
    1.- Ser consecuencia de la negativa a establecer con el autor una relación de carácter sentimental o sexual.
    2.- Ser consecuencia de que la víctima ejerza o haya ejercido la prostitución, u otra ocupación u oficio de carácter sexual.
    3.- Haberse cometido el delito tras haber ejercido contra la víctima cualquier forma de violencia sexual, sin perjuicio de lo dispuesto en el artículo 372 bis.
    4.- Haberse realizado con motivo de la orientación sexual, identidad de género o expresión de género de la víctima.
    5.- Haberse cometido en cualquier tipo de situación en la que se den circunstancias de manifiesta subordinación por las relaciones desiguales de poder entre el agresor y la víctima, o motivada por una evidente intención de discriminación.
    Artículo 390 quáter.- Son circunstancias agravantes de responsabilidad penal para el delito de femicidio, las siguientes:
     
    1. Encontrarse la víctima embarazada.
    2. Ser la víctima una niña o una adolescente menor de dieciocho años de edad, una mujer adulta mayor o una mujer en situación de discapacidad en los términos de la ley N° 20.422.
    3. Ejecutarlo en presencia de ascendientes o descendientes de la víctima.
    4. Ejecutarlo en el contexto de violencia física o psicológica habitual del hechor contra la víctima.
    Artículo 390 quinquies.- Tratándose del delito de femicidio, el juez no podrá aplicar la circunstancia atenuante de responsabilidad penal prevista en el N° 5 del artículo 11.
    ART. 390 SEXIES.

    El que con ocasión de hechos previos constitutivos de violencia de género, cometidos por éste en contra de la víctima, causare el suicidio de una mujer, será sancionado con la pena de presidio menor en su grado máximo a presidio mayor en su grado mínimo como autor de suicidio femicida.
    Se entenderá por violencia de género cualquier acción u omisión basada en el género, que causare muerte, daño o sufrimiento físico, sexual o psicológico a la mujer, donde quiera que esto ocurra, especialmente aquellas circunstancias establecidas en el artículo 390 ter.


    §1 ter.

    Del homicidio


    ART.391.

    El que mate a otro y no esté comprendido en los artículos 390, 390 bis y 390 ter, será penado:

    1.° Con presidio mayor en su grado máximo a presidio perpetuo, si ejecutare el homicidio con alguna de las circunstancias siguientes:

    Primera.- Con alevosía.
    Segunda.- Por premio o promesa remuneratoria, o por beneficio económico o de otra naturaleza en provecho propio o de un tercero.
    Tercera.- Por medio de veneno.
    Cuarta.-  Con ensañamiento, aumentando deliberada e inhumanamente el dolor al ofendido.
    Quinta.-  Con premeditación conocida.

    2.º Con presidio mayor en su grado medio a máximo en cualquier otro caso.



    ART. 391 bis.-

    El que conspire para cometer el delito de homicidio calificado previsto en los términos del artículo 391 N° 1°, circunstancia segunda, será castigado con la pena de presidio menor en su grado máximo.
    Si la conducta descrita en el inciso precedente se comete en contra de un juez con competencia en lo penal, de un fiscal del Ministerio Público, de un defensor penal público, de un funcionario de Carabineros de Chile, de la Policía de Investigaciones de Chile o de Gendarmería de Chile, en razón del ejercicio de sus funciones, será castigado con la pena de presidio menor en su grado máximo a presidio mayor en su grado mínimo.

    ART. 392.

    Cometiéndose un homicidio en riña o pelea y no constando el autor de la muerte, pero sí los que causaron lesiones graves al occiso, se impondrá a todos éstos la pena de presidio menor en su grado máximo.
    Si no constare tampoco quienes causaron lesiones graves al ofendido, se impondrá a todos los que hubieren ejercido violencia en su persona la de presidio menor en su grado medio.



    ART. 393.

    El que con conocimiento de causa prestare auxilio a otro para que se suicide, sufrirá la pena de presidio menor en sus grados medio a máximo, si se efectúa la muerte.


    ART. 393 bis.

    Quien induzca a otra persona a cometer suicidio será sancionado con la pena de presidio menor en sus grados mínimo a medio. Si por tal circunstancia se produjera la muerte, la pena será de presidio menor en sus grados medio a máximo.
    Si la inducción al suicidio y la consecuente muerte de la víctima, se produce con ocasión de concurrir cualesquiera de las circunstancias establecidas en el artículo 390 ter, será castigado con la pena de presidio menor en su grado máximo a presidio mayor en su grado mínimo.



    § II.

    Del infanticidio.


    ART. 394.

    Cometen infanticidio el padre, la madre o los demás ascendientes legítimos o ilegítimos que dentro de las cuarenta y ocho horas después del parto, matan al hijo o descendiente, y serán penados con presidio mayor en sus grados mínimo a medio.


    § III.

    Lesiones corporales.


    ART. 395.

    El que maliciosamente castrare a otro será castigado con presidio mayor en sus grados mínimo a medio.



    ART. 396.

    Cualquiera otra mutilación de un miembro importante que deje al paciente en la imposibilidad de valerse por sí mismo o de ejecutar las funciones naturales que antes ejecutaba, hecha también con malicia, será penada con presidio menor en su grado máximo a presidio mayor en su grado mínimo.
    En los casos de mutilaciones de miembros menos importantes, como un dedo o una oreja, la pena será presidio menor en sus grados mínimo a medio.


    Art. 397.

    El que hiriere, golpeare o maltratare de obra a otro, será castigado como responsable de lesiones graves:

    1.° Con la pena de presidio mayor en su grado mínimo, si de resultas de las lesiones queda el ofendido demente, inútil para el trabajo, impotente, impedido de algún miembro importante o notablemente deforme.
    2.° Con la de presidio menor en su grado medio, si las lesiones produjeren al ofendido enfermedad o incapacidad para el trabajo por más de treinta días.


    ART. 398.

    Las penas del artículo anterior son aplicables respectivamente al que causare a otro alguna lesión grave, ya sea administrándole a sabiendas sustancias o bebidas nocivas o abusando de su credulidad o flaqueza de espíritu.



    ART. 399.

    Las lesiones no comprendidas en los artículos precedentes se reputan menos graves, y serán penadas con relegación o presidio menores en sus grados mínimos o con multa de once a veinte unidades tributarias mensuales.








    ART. 400.
    Si los hechos a que se refieren los artículos anteriores de este párrafo se ejecutan en contra de alguna de las personas que menciona el artículo 5º de la Ley sobre Violencia Intrafamiliar, o con cualquiera de las circunstancias Segunda, Tercera o Cuarta del número 1º del artículo 391 de este Código, las penas se aumentarán en un grado.
    Asimismo, si los hechos a que se refieren los artículos anteriores de este párrafo se ejecutan en contra de un menor de dieciocho años de edad, adulto mayor o persona en situación de discapacidad, por quienes tengan encomendado su cuidado, la pena señalada para el delito se aumentará en un grado.
    De la misma forma, si los hechos a que se refieren el numeral 2° del artículo 397 y el artículo 399 se ejecutaren en contra de miembros de los Cuerpos de Bomberos en ejercicio de sus funciones, la pena señalada para el delito se aumentará en un grado.




    ART. 401.

    Las lesiones menos graves inferidas a guardadores, sacerdotes o personas constituidas en dignidad o autoridad pública, serán castigadas siempre con presidio o relegación menores en sus grados mínimos a medios.




    ART. 401 BIS

    Las lesiones inferidas a los profesionales y funcionarios de los establecimientos de salud, públicos o privados, o contra los profesionales, funcionarios y manipuladores de alimentos de establecimientos educacionales, públicos o privados, al interior de sus dependencias o mientras éstos se encontraren en el ejercicio de sus funciones o en razón, con motivo u ocasión de ellas, serán sancionadas:
    1. Con presidio mayor en sus grados mínimo a medio en los casos del número 1° del artículo 397.
    2. Con presidio menor en su grado máximo en los casos del número 2° del artículo 397.
    3. Con presidio menor en su grado medio en los casos del artículo 399.
    4. Con presidio menor en su grado mínimo si las lesiones que se causaren fueren leves.
    En los casos en que se maltratare corporalmente de manera relevante a las personas señaladas en el inciso anterior, la pena será de prisión en su grado máximo y multa de una a cuatro unidades tributarias mensuales.



    ART. 402.

    Si resultaren lesiones graves de una riña o pelea y no constare su autor, pero sí los que causaron lesiones menos graves, se impondrán a todos éstos las penas inmediatamente inferiores en grado a las que les hubieran correspondido por aquellas lesiones.
    No constando tampoco los que causaron lesiones menos graves, se impondrán las penas inferiores en dos grados a los que aparezca que hicieron uso en la riña o pelea de armas que pudieron causar esas lesiones graves.


    ART. 403.

    Cuando sólo hubieren resultado lesiones menos graves sin conocerse a los autores de ellas, pero sí a los que hicieron uso de armas capaces de producirlas, se impondrá a todos éstos las penas inmediatamente inferiores en grado a las que les hubieran correspondido por tales lesiones.
    En los casos de este artículo y del anterior, se estará a lo dispuesto en el 304 para la aplicación de la pena.


     
    § III. bis.

    Del maltrato a menores de dieciocho años de edad, adultos mayores o personas en situación de discapacidad.




    ART. 403 bis.-

    El que, de manera relevante, maltratare corporalmente a un niño, niña o adolescente menor de dieciocho años, a una persona adulta mayor o a una persona en situación de discapacidad en los términos de la ley N° 20.422 será sancionado con prisión en cualquiera de sus grados o multa de una a cuatro unidades tributarias mensuales, salvo que el hecho sea constitutivo de un delito de mayor gravedad.
    El que teniendo un deber especial de cuidado o protección respecto de alguna de las personas referidas en el inciso primero, la maltratare corporalmente de manera relevante o no impidiere su maltrato debiendo hacerlo, será castigado con la pena de presidio menor en su grado mínimo, salvo que el hecho fuere constitutivo de un delito de mayor gravedad, caso en el cual se aplicará sólo la pena asignada por la ley a éste.

    ART. 403 ter.-

    El que sometiere a una de las personas referidas en los incisos primero y segundo del artículo 403 bis a un trato degradante, menoscabando gravemente su dignidad, será sancionado con la pena de presidio menor en su grado mínimo.

    ART. 403 quáter.-

    El que cometiere cualquiera de los delitos contemplados en los párrafos 1, 3 y 3 bis del título VIII del libro II de este código, en contra de un menor de dieciocho años de edad, adulto mayor o persona en situación de discapacidad, además será condenado a la pena de inhabilitación absoluta temporal para ejercer los cargos contemplados en el artículo 39 ter, en cualquiera de sus grados. En caso de reincidencia en delitos de la misma especie, el juez podrá imponer la inhabilitación absoluta con el carácter de perpetua.

    ART. 403 quinquies.-

    Las condenas dictadas en virtud del artículo anterior deberán inscribirse en la respectiva sección del Registro General de Condenas, establecido en el decreto ley N° 645, de 1925, del Ministerio de Justicia, sobre el Registro Seccional de Inhabilitaciones.



    ART. 403 sexies.-

    Además de las penas establecidas en los artículos anteriores, el juez podrá decretar, como pena accesoria, la asistencia a programas de rehabilitación para maltratadores o el cumplimiento de un servicio comunitario por el plazo que prudencialmente determine, el cual no podrá exceder de sesenta días, debiendo las instituciones respectivas dar cuenta sobre el cumplimiento efectivo de dichas penas ante el tribunal.
    Asimismo, el juez podrá decretar, como penas o medidas accesorias, la prohibición de acercarse a la víctima o a su domicilio, lugar de cuidado, trabajo o estudio, así como a cualquier otro lugar al que ésta concurra o visite habitualmente; también, la prohibición de porte y tenencia y, en su caso, el comiso de armas de fuego; y, además, la asistencia obligatoria a programas de tratamiento para la rehabilitación del consumo problemático de drogas o alcohol, si ello corresponde.

    ART. 403 septies.-

    Los delitos contemplados en este párrafo serán de acción penal pública.


    § IV.

    Del duelo.


    ART. 404.

    La provocación a duelo será castigada con reclusión menor en su grado mínimo.



    ART. 405.

    En igual pena incurrirá el que denostare o públicamente desacreditare a otro por haber rehusado un duelo.


    ART. 406.

    El que matare en duelo a su adversario sufrirá la pena de reclusión mayor en su grado mínimo.
    Si le causare las lesiones señaladas en el núm. 1.° del art. 397, será castigado con reclusión menor en su grado máximo.
    Cuando las lesiones fueren de las relacionadas en el núm. 2.° de dicho art. 397, la pena será reclusión menor en sus grados mínimo a medio.
    En los demás casos se impondrá a los combatientes reclusión menor en su grado mínimo o multa de once a veinte unidades tributarias mensuales.








    ART. 407.

    El que incitare a otro a provocar o aceptar un duelo, será castigado respectivamente con las penas señaladas en el artículo anterior, si el duelo se lleva a efecto.



    ART. 408.

    Los padrinos de un duelo que se lleve a efecto incurrirán en la pena de reclusión menor en su grado mínimo; pero si ellos lo hubieron concertado a muerte o con ventaja conocida de alguno de los combatientes, la pena será reclusión menor en su grado máximo.



    ART. 409.

    Se impondrán las penas generales de este Código para los casos de homicidio y lesiones:
    1.° Si el duelo se hubiere verificado sin la asistencia de padrinos.
    2º ° Cuando se provocare o diere causa a un desafío proponiéndose un interés pecuniario o un objeto inmoral.
    3.° Al combatiente que faltare a las condiciones esenciales concertadas por los padrinos.


    § V.

    Disposiciones comunes a los párrafos 1, 1 bis, 1 ter, 3 y 4 de este Título



    ART. 410.

    En los casos de homicidio o lesiones a que se refieren los párrafos 1, 1 bis, 1 ter, 3 y 4 del presente título, el ofensor, a más de las penas que en ellos se establecen, quedará obligado:
    1.° A suministrar alimentos a la familia del occiso.
    2.° A pagar la curación del demente o imposibilitado para el trabajo y a dar alimentos a él y a su familia.
    3.° A pagar la curación del ofendido en los demás casos de lesiones y a dar alimentos a él y a su familia mientras dure la imposibilidad para el trabajo ocasionada por tales lesiones.
    Los alimentos serán siempre congruos tratándose del ofendido, y la obligación de darlos cesa si éste tiene bienes suficientes con que atender a su cómoda subsistencia y para suministrarlos a su familia en los casos y en la forma que determina el Código Civil.




    ART. 411.

    Para los efectos del artículo anterior se entiende por familia todas las personas que tienen derecho a pedir alimentos al ofendido.

    § V bis.

    De los delitos de tráfico ilícito de migrantes y trata de personas


    ART. 411 bis.-

    Tráfico de migrantes. El que con ánimo de lucro facilite o promueva la entrada ilegal al país de una persona que no sea nacional o residente, será castigado con reclusión menor en su grado medio a máximo y multa de cincuenta a cien unidades tributarias mensuales.
    La pena señalada en el inciso anterior se aplicará en su grado máximo si se pusiere en peligro la integridad física o salud del afectado.
    Si se pusiere en peligro la vida del afectado o si éste fuere menor de edad, la pena señalada en el inciso anterior se aumentará en un grado.
    Las mismas penas de los incisos anteriores, junto con la de inhabilitación absoluta temporal para cargos u oficios públicos en su grado máximo, se impondrá si el hecho fuere ejecutado, aun sin ánimo de lucro, por un funcionario público en el desempeño de su cargo o abusando de él. Para estos efectos se estará a lo dispuesto en el artículo 260.
    Por entrada ilegal se entenderá el paso de fronteras sin haber cumplido los requisitos necesarios para entrar legalmente a Chile.

    ART. 411 ter.-

    El que promoviere o facilitare la entrada o salida del país de personas para que ejerzan la prostitución en el territorio nacional o en el extranjero, será castigado con la pena de reclusión menor en su grado máximo y multa de veinte unidades tributarias mensuales.
    ART. 411 quáter.-

    El que mediante violencia, intimidación, coacción, engaño, abuso de poder, aprovechamiento de una situación de vulnerabilidad o de dependencia de la víctima, o la concesión o recepción de pagos u otros beneficios para obtener el consentimiento de una persona que tenga autoridad sobre otra capte, traslade, acoja o reciba personas para que sean objeto de alguna forma de explotación sexual, incluyendo la pornografía, trabajos o servicios forzados, servidumbre o esclavitud o prácticas análogas a ésta, o extracción de órganos, será castigado con la pena de reclusión mayor en cualquiera de sus grados y multa de cincuenta a cien unidades tributarias mensuales.
    Si la víctima fuere menor de edad, aun cuando no concurriere violencia, intimidación, coacción, engaño, abuso de poder, aprovechamiento de una situación de vulnerabilidad o de dependencia de la víctima, o la concesión o recepción de pagos u otros beneficios para obtener el consentimiento de una persona que tenga autoridad sobre otra, se impondrán las penas de reclusión mayor en sus grados medio a máximo y multa de cincuenta a cien unidades tributarias mensuales.
    El que promueva, facilite o financie la ejecución de las conductas descritas en este artículo será sancionado como autor del delito.


    ART. 411 quinquies.-

    Los que se asociaren u organizaren con el objeto de cometer alguno de los delitos de este párrafo serán sancionados, por este solo hecho, conforme a lo dispuesto en los artículos 292 y siguientes de este Código.

    ART. 411 sexies.-

    El tribunal podrá reducir la pena en dos grados al imputado o acusado que prestare cooperación eficaz que conduzca al esclarecimiento de los hechos investigados o permita la identificación de sus responsables, o que sirva para prevenir o impedir la perpetración o consumación de igual o mayor gravedad.
    Se entiende por cooperación eficaz el suministro de datos o informaciones precisas, verídicas y comprobables, que contribuyan necesariamente a los fines señalados en el inciso primero.
    Si con ocasión de la investigación de otro hecho constitutivo de delito, el fiscal correspondiente necesita tomar conocimiento de los antecedentes proporcionados por el cooperador eficaz, deberá solicitarlos fundadamente. El fiscal requirente, para los efectos de efectuar la diligencia, deberá realizarla en presencia del fiscal ante quien se prestó la cooperación, debiendo este último previamente calificar su conveniencia. El superior jerárquico común dirimirá cualquier dificultad que surja con ocasión de dicha petición y de su cumplimiento.
    La reducción de pena se determinará con posterioridad a la individualización de la sanción penal según las circunstancias atenuantes o agravantes comunes que concurran; o de su compensación, de acuerdo con las reglas generales.


    ART. 411 septies.-

    Para los efectos de determinar la reincidencia del artículo 12, circunstancia 16ª en los delitos sancionados en este párrafo, se considerarán también las sentencias firmes dictadas en un Estado extranjero, aun cuando la pena impuesta no haya sido cumplida.


    Artículo 411 octies.-

    Previa autorización del juez de garantía competente, el fiscal podrá autorizar, en las investigaciones por los delitos previstos en el presente párrafo, que funcionarios policiales se desempeñen como agentes encubiertos y, a propuesta de dichos funcionarios, que determinados informantes de esos servicios actúen en esa calidad.
    Cuando existan fundadas sospechas de que una persona ha cometido o preparado la comisión de alguno de los delitos indicados en este Párrafo y la investigación lo haga imprescindible, el tribunal, a petición del Ministerio Público, podrá autorizar la interceptación o grabación de las telecomunicaciones de esa persona. La captación, grabación y registro subrepticio de imágenes o sonidos en lugares cerrados o que no sean de libre acceso al público, podrá ser autorizada por el juez, a solicitud del fiscal, cuando existan fundadas sospechas basadas en hechos determinados y graves que lo hagan imprescindible para el esclarecimiento de los hechos. En lo demás, se estará íntegramente a lo dispuesto en los artículos 222 a 225 del Código Procesal Penal.
    Igualmente, cuando la investigación lo haga imprescindible, el tribunal, a petición del Ministerio Público, podrá autorizar la utilización de otra u otras de las diligencias especiales de investigación reguladas en el Párrafo 3° bis del Título I del Libro II del Código Procesal Penal.
    En todo aquello no regulado por este artículo los agentes encubiertos e informantes se regirán por las disposiciones respectivas del Párrafo 3° bis del Título I del Libro II del Código Procesal Penal.

   
    § VI

    De la calumnia.


    ART. 412.

    Es calumnia la imputación de un delito determinado pero falso y que pueda actualmente perseguirse de oficio.

    ART. 413.

    La calumnia propagada por escrito y con publicidad será castigada:
    1.° Con las penas de reclusión menor en su grado medio y multa de once a veinte unidades tributarias mensuales, cuando se imputare
un crimen.
    2.° Con las de reclusión menor en su grado mínimo y multa de seis a diez unidades tributarias mensuales, si se imputare un simple delito.




    ART. 414.

    No propagándose la calumnia con publicidad y por escrito, será castigada:
    1° Con las penas de reclusión menor en su grado mínimo y multa de seis a quince unidades tributarias mensuales, cuando se imputare un crimen.
    2.° Con las de reclusión menor en su grado mínimo y multa de seis a diez unidades tributarias mensuales, si se imputare un simple delito.




    ART. 415.

    El acusado de calumnia quedará exento de toda pena probando el hecho criminal que hubiere imputado.
    La sentencia en que se declare la calumnia, si el ofendido lo pidiere, se publicará por una vez a costa del calumniante en los periódicos que aquél designare, no excediendo de tres.



    § VII

    De las injurias.


    ART. 416.

    Es injuria toda expresión proferida o acción ejecutada en deshonra, descrédito o menosprecio de otra persona.


    ART. 417.

    Son injurias graves:
    1.° La imputación de un crimen o simple delito de los que no dan lugar a procedimiento de oficio.
    2° La imputación de un crimen o simple delito penado o prescrito.
    3.° La de un vicio o falta de moralidad cuyas consecuencias puedan perjudicar considerablemente la fama, crédito o intereses del agraviado.
    4.° Las injurias que por su naturaleza, ocasión o circunstancias fueren tenidas en el concepto público por afrentosas.
    5.° Las que racionalmente merezcan la calificación de graves atendido el estado, dignidad y circunstancias del ofendido y del ofensor.



    ART. 418.

    Las injurias graves hechas por escrito y con publicidad, serán castigadas con las penas de reclusión menor en sus grados mínimo a medio y multa de once a veinte unidades tributarias mensuales.
    No concurriendo aquellas circunstancias, las penas serán reclusión menor en su grado mínimo y multa de seis a diez unidades tributarias mensuales.






    Art. 419.

    Las injurias leves se castigarán con las penas de reclusión menor en su grado mínimo y multa de seis a diez unidades tributarias mensuales, cuando fueren hechas por escrito y con publicidad. No concurriendo estas circunstancias se penarán como faltas.





    ART. 420.

    Al acusado de injuria no se admitirá prueba sobre la verdad de las imputaciones, sino cuando éstas fueren dirigidas contra empleados públicos sobre hechos concernientes al ejercicio de su cargo.
    En este caso será absuelto el acusado si probare la verdad de las imputaciones.



    § VIII.

    Disposiciones comunes a los dos párrafos anteriores.








    ART. 421.

    Se comete el delito de calumnia o injuria no sólo manifiestamente, sino por medio de alegorías, caricaturas, emblemas o alusiones.


    ART. 422.

    La calumnia y la injuria se reputan hechas por escrito y con publicidad cuando se propagaren por medio de carteles o pasquines fijados en los sitios públicos; por papeles impresos, no sujetos a la ley de imprenta, litografías, grabados o manuscritos comunicados a más de cinco personas, o por alegorías, caricaturas, emblemas o alusiones reproducidos por medio de la litografía, el grabado, la fotografía u otro procedimiento cualquiera.



    ART. 423.

    El acusado de calumnia o injuria encubierta o equívoca que rehusare dar en juicio explicaciones satisfactorias acerca de ella, será castigado con las penas de los delitos de calumnia o injuria manifiesta.

    ART. 424. Derogado.

    ART. 425.

    Respecto de las calumnias o injurias publicadas por medio de periódicos extranjeros, podrán ser acusados los que, desde el territorio de la República, hubieren enviado los artículos o dado orden para su inserción, o contribuido a la introducción o expendición de esos periódicos en Chile con ánimo manifiesto de propagar la calumnia o injuria.
    ART. 426.

    La calumnia o injuria causada en juicio se juzgará disciplinariamente por el tribunal que conoce de la causa; sin perjuicio del derecho del ofendido para deducir, una vez que el proceso haya concluido, la acción penal correspondiente.


    ART. 427

    Las expresiones que puedan estimarse calumniosas o injuriosas, consignadas en un documento oficial, no destinado a la publicidad, sobre asuntos del servicio público, no dan derecho para acusar criminalmente al que las consignó.


    ART. 428.

    El condenado por calumnia o injuria puede ser relevado de la pena impuesta mediante perdón del acusador; pero la remisión no producirá efecto respecto de la multa una vez que ésta haya sido satisfecha.

    La calumnia o injuria se entenderá tácitamente remitida cuando hubieren mediado actos positivos que, en concepto del tribunal, importen reconciliación o abandono de la acción.

    ART. 429. Derogado.

    ART. 430.

    En el caso de calumnias o injurias recíprocas, se observarán las reglas siguientes:

    1.° Si las más graves de las calumnias o injurias recíprocamente inferidas merecieren igual pena, el tribunal las dará todas por compensadas.
    2.° Cuando la más grave de las calumnias o injurias imputadas por una de las partes, tuviere señalado mayor castigo que la más grave de las imputadas por la otra, al imponer la pena correspondiente a aquélla se rebajará la asignada para ésta.




    Art. 431.

    La acción de calumnia o injuria prescribe en un año, contado desde que el ofendido tuvo o pudo racionalmente tener conocimiento de la ofensa.

    La misma regla se observará respecto de las demás personas enumeradas en el artículo 108 del Código Procesal Penal; pero el tiempo trascurrido desde que el ofendido tuvo o pudo tener conocimiento de la ofensa hasta su muerte, se tomará en cuenta al computarse el año durante el cual pueden ejercitar esta acción las personas comprendidas en dicho artículo.

    No podrá entablarse acción de calumnia o injuria después de cinco años, contados desde que se cometió el delito. Pero si la calumnia o injuria hubiere sido causada en juicio, este plazo no obstará al cómputo del año durante el cual se podrá ejercer la acción.


    TÍTULO NOVENO.

    CRÍMENES Y SIMPLES DELITOS CONTRA LA PROPIEDAD.








    § I.

    De la apropiación de las cosas muebles ajenas contra la voluntad de su dueño.





    ART. 432.

    El que sin la voluntad de su dueño y con ánimo de lucrarse se apropia cosa mueble ajena usando de violencia o intimidación en las personas o de fuerza en las cosas, comete robo; si faltan la violencia, la intimidación y la fuerza, el delito se califica de hurto.
   


NOTA
      El N° 7 de la ley 11183, publicada el 10.06.1956, modifica el presente artículo, en el sentido de derogar su inciso segundo.

    § II.

    Del robo con violencia o intimidación en las personas.




    ART. 433.

    El culpable de robo con violencia o intimidación en las personas, sea que la violencia o la intimidación tenga lugar antes del robo para facilitar su ejecución, en el acto de cometerlo o después de cometido para favorecer su impunidad, será castigado:
    1°. Con presidio mayor en su grado máximo a presidio perpetuo calificado cuando, con motivo u ocasión del robo, se cometiere, además, homicidio o violación.
    2°. Con presidio mayor en su grado máximo a presidio perpetuo cuando, con motivo u ocasión del robo, se cometiere alguna de las lesiones comprendidas en los artículos 395, 396 y 397, número 1°.
    3°. Con presidio mayor en su grado medio a máximo cuando se cometieren lesiones de las que trata el número 2° del artículo 397 o cuando las víctimas fueren retenidas bajo rescate o por un lapso mayor a aquel que resulte necesario para la comisión del delito.

    ART. 434.

    Los que cometieren actos de piratería serán castigados con la pena de presidio mayor en su grado mínimo a presidio perpetuo.





    ART. 435.    Derogado.

    ART. 436.

    Fuera de los casos previstos en los artículos precedentes, los robos ejecutados con violencia o intimidación en las personas, serán penados con presidio mayor en sus grados mínimo a máximo, cualquiera que sea el valor de las especies sustraídas.
    Se considerará como robo y se castigará con la pena de presidio menor en sus grados medio a máximo, la apropiación de dinero u otras especies que los ofendidos lleven consigo, cuando se proceda por sorpresa o aparentando riñas en lugares de concurrencia o haciendo otras maniobras dirigidas a causar agolpamiento o confusión.
    También será considerado robo, y se sancionará con la pena de presidio menor en su grado máximo, la apropiación de vehículos motorizados, siempre que se valga de la sorpresa, de la distracción de la víctima o se genere por parte del autor cualquier maniobra distractora cuyo objeto sea que la víctima abandone el vehículo para facilitar su apropiación, en ambos casos, en el momento en que ésta se apreste a ingresar o hacer abandono de un lugar habitado, destinado a la habitación o sus dependencias, o su lugar de trabajo, salvo en aquellos casos en que medie violencia o intimidación, en los que se aplicará lo dispuesto en el inciso primero.


    ART. 437.    Derogado.


    ART. 438.

    El que para obtener un provecho patrimonial para sí o para un tercero constriña a otro con violencia o intimidación a suscribir, otorgar o entregar un instrumento público o privado que importe una obligación estimable en dinero, o a ejecutar, omitir o tolerar cualquier otra acción que importe una disposición patrimonial en perjuicio suyo o de un tercero, será castigado con las penas respectivamente señaladas en este párrafo para el culpable de robo.

    ART. 439.

    Para los efectos del presente párrafo se estimarán por violencia o intimidación en las personas los malos tratamientos de obra, las amenazas ya para hacer que se entreguen o manifiesten las cosas, ya para impedir la resistencia u oposición a que se quiten, o cualquier otro acto que pueda intimidar o forzar a la manifestación o entrega. Hará también violencia el que para obtener la entrega o manifestación alegare orden falsa de alguna autoridad, o la diere por sí fingiéndose ministro de justicia o funcionario público. Por su parte, hará también intimidación el que para apropiarse u obtener la entrega o manifestación de un vehículo motorizado o de las cosas ubicadas dentro del mismo, fracture sus vidrios, encontrándose personas en su interior; o amenace la integridad de niños que se encuentren al interior del vehículo, sin perjuicio de la prueba que se pudiere presentar en contrario.



    § III.

    Del robo con fuerza en las cosas.


    ART. 440.

    El culpable de robo con fuerza en las cosas efectuado en lugar habitado o destinado a la habitación o en sus dependencias, sufrirá la pena de presidio mayor en su grado mínimo si cometiere el delito:

    1.º Con escalamiento, entendiéndose que lo hay cuando se entra por vía no destinada al efecto, por forado o con rompimiento de pared o techos, o fractura de puertas o ventanas.
    2.º Haciendo uso de llaves falsas, o verdadera que hubiere sido sustraída, de ganzúas u otros instrumentos semejantes para entrar en el lugar del robo.
    3.º A Introduciéndose en el lugar del robo mediante la seducción de algún doméstico, o a favor de nombres supuestos o simulación de autoridad.
    4.° Eliminado.

    ART. 441.    Derogado.


    ART. 442.

    El robo en lugar no habitado, se castigará con presidio menor en sus grados medio a máximo, siempre que concurra alguna de las circunstancias siguientes:
    1.º A Escalamiento.
    2.º Fractura de puertas interiores, armarios, arcas u otra clase de muebles u objetos cerrados o sellados.
    3.º Haber hecho uso de llaves falsas, o verdadera que se hubiere sustraído, de ganzúas u otros instrumentos semejantes para entrar en el lugar del robo o abrir los muebles cerrados.


   
    ART. 443.

    Con la misma pena señalada en el artículo anterior se castigará el robo de cosas que se encuentren en bienes nacionales de uso público, en sitio no destinado a la habitación o en el interior de vehículos motorizados, si el autor hace uso de llaves falsas o verdaderas que se hayan substraído, de ganzúas u otros instrumentos semejantes o si se procede, mediante fractura de puertas, vidrios, cierros, candados u otros dispositivos de protección o si se utilizan medios de tracción.
    Si el delito a que se refiere el inciso precedente recayere sobre un vehículo motorizado, se impondrá la pena de presidio menor en su grado máximo.
    Se considerará robo y se castigará con la pena del inciso precedente la apropiación de un vehículo motorizado mediante la generación de cualquier maniobra distractora cuyo objeto sea que la víctima abandone el vehículo, fuera de los casos a los que se refiere el artículo 436.
    Si con ocasión de alguna de las conductas señaladas en el inciso primero, se produce la interrupción o interferencia del suministro de un servicio público o domiciliario, tales como electricidad, gas, agua, alcantarillado, colectores de aguas lluvia o telefonía, la pena se aplicará en su grado máximo.


    ART. 443 bis.

    El robo con fuerza de cajeros automáticos, dispensadores o contenedores de dinero, o del dinero y valores contenidos en ellos, será sancionado con la pena de presidio menor en su grado máximo. Para los efectos del presente artículo se entenderá que hay fuerza en las cosas si se ha procedido con alguno de los medios señalados en el artículo 440, Nos 1° y 2°; si se ha fracturado, destruido o dañado el cajero automático o dispensador o sus dispositivos de protección o sujeción mediante el uso de instrumentos contundentes o cortantes de cualquier tipo, incluyendo el empleo de medios químicos; o si se utilizan medios de tracción.

    ART. 444.

    Se presume autor de tentativa de robo al que se introdujere con forado, fractura, escalamiento, uso de llave falsa o de llave verdadera sustraída o de ganzúa en algún aposento, casa, edificio habitado o destinado a la habitación o en sus dependencias.


    ART. 445.

    El que fabricare, expendiere o tuviere en su poder llaves falsas, ganzúas u otros instrumentos destinados conocidamente para efectuar el delito de robo y no diere descargo suficiente sobre su fabricación, expendición, adquisición o conservación, será castigado con presidio menor en su grado mínimo.


    § IV.

    Del hurto.


    Artículo 446.- Los autores de hurto serán castigados:
    1.º Con presidio menor en sus grados medio a máximo y multa de once a quince unidades tributarias mensuales, si el valor de la cosa hurtada excediera de cuarenta unidades tributarias mensuales.
    2.º Con presidio menor en su grado medio y multa de seis a diez unidades tributarias mensuales, si el valor excediere de cuatro unidades tributarias mensuales y no pasare de cuarenta unidades tributarias mensuales.
    3.º Con presidio menor en su grado mínimo y multa de cinco unidades tributarias mensuales, si excediere de media unidad tributaria mensual y no pasare de cuatro unidades tributarias mensules.
    Si el valor de la cosa hurtada excediere de cuatrocientas unidades tributarias mensuales, se aplicará la pena de presidio menor en su grado máximo y multa de veintiuna a treinta unidades tributarias mensuales.


    ART. 447.

    En los casos del artículo anterior podrá aplicarse la pena inmediatamente superior en grado:
   
    1.° Si el hurto se cometiere por dependiente, criado o sirviente asalariado, bien sea en la casa en que sirve o bien en aquella a que lo hubiere llevado su amo o patrón.
    2.º Cuando se cometiere por obrero, oficial o aprendiz en la casa, taller o almacén de su maestro o de la persona para quien trabaja, o por individuo que trabaja habitualmente en la casa donde hubiere hurtado.
    3.° Si se cometiere por el posadero, fondista u otra persona que hospede gentes en cosas que hubieren llevado a la posada o fonda.
    4.° Cuando se cometiere por patrón o comandante de buque, lanchero, conductor o bodeguero de tren, guarda almacenes, carruajero, carretero o arriero en cosas que se hayan puesto en su buque, carro, bodega, etc.



    ART. 447 bis.-

    El hurto de cosas que forman parte de redes de suministro de servicios públicos o domiciliarios, tales como electricidad, gas, agua, alcantarillado, colectores de aguas lluvia o telefonía, será castigado con presidio menor en sus grados medio a máximo.
    Si con ocasión de alguna de las conductas señaladas en este artículo se produce la interrupción o interferencia del servicio, la pena se aplicará en su grado máximo.


    ART. 448.

    El que, hallándose una especie mueble, al parecer perdida, cuyo valor exceda de una unidad tributaria mensual, no la entregare a la autoridad o a su dueño, siempre que le conste quién sea éste, por hechos coexistentes posteriores al hallazgo, será castigado con presidio menor en su grado mínimo y multa de cinco unidades tributarias mensuales.

    También será castigado con presidio menor en su grado mínimo y multa de cinco unidades tributarias mensuales, el que hallare especies, al parecer perdidas o abandonadas, a consecuencia de naufragio, inundación, incendio, terremoto, accidente en ferrocarril u otra causa análoga, cuyo valor exceda la cantidad mencionada en el inciso anterior, y no las entregare a los dueños o a la autoridad en su defecto.
    § IV bis. Del Abigeato



    ART. 448 bis.
   
    El que robe o hurte uno o más caballos o bestias de silla o carga, o especies de ganado mayor, menor o colmenas, comete abigeato y será castigado con las penas señaladas en los Párrafos 2, 3 y 4.
    Asimismo, se considerará autor del delito de abigeato al que sin el consentimiento de quienes pueden disponer del ganado:

    1°. Altere o elimine marcas o señales en animales ajenos.

    2°. Marque, señale, contramarque o contraseñale animales ajenos.

    3°. Expida o porte certificados falsos para obtener guías o formularios o haga conducir animales ajenos sin estar debidamente autorizado.


    ART. 448 ter.
    Una vez determinada la pena que correspondería a los autores, cómplices y encubridores de abigeato sin el requisito de tratarse de la substracción de animales y considerando las circunstancias modificatorias de responsabilidad penal concurrentes, el juez deberá aumentarla en un grado y aplicará, en todo caso, la pena de comiso en los términos del artículo 31 de este Código.

    Cuando las especies substraídas tengan un valor que exceda las cinco unidades tributarias mensuales, se aplicará, además, la accesoria de multa de setenta y cinco a cien unidades tributarias mensuales.

    Si la pena consta de dos o más grados, el aumento establecido en el inciso primero se hará después de determinar la pena que habría correspondido al imputado, con prescindencia del requisito de tratarse de la substracción de animales.

    Será castigado como autor de abigeato el que beneficie o destruya una especie para apropiarse de toda ella o de alguna de sus partes.

    La regla del inciso primero de este artículo se observará también en los casos previstos en el artículo 448, si se trata de animales comprendidos en el artículo anterior.

    ART. 448 quáter.

    Se castigará como autor de abigeato a aquel en cuyo poder se encuentren animales o partes de los mismos referidos en este Párrafo, cuando no pueda justificar su adquisición o legítima tenencia y, del mismo modo, al que sea habido en predio ajeno, arreando, transportando, manteniendo cautivas, inmovilizadas o maniatadas dichas especies animales. El porte de armas, herramientas o utensilios comúnmente empleados para el faenamiento de animales por quien no diere descargo suficiente de su tenencia, se castigará de conformidad a lo establecido en el artículo 445.

    Las marcas registradas, señales conocidas, dispositivos de identificación individual oficial registrados ante el Servicio Agrícola y Ganadero u otras de carácter electrónico o tecnológico puestas sobre el animal, constituyen presunción de dominio a favor del dueño de la marca o señal.

    Para los efectos previstos en el inciso primero, en los casos de traslado de animales o de partes de los mismos, realizado en vehículos de transporte de carga, Carabineros de Chile deberá exigir, además del formulario de movimiento animal, la boleta, factura o guía de despacho correspondiente, a efectos de acreditar el dominio, posesión o legítima tenencia de las especies. Ante la imposibilidad de acreditar dicho dominio, posesión o legítima tenencia, según corresponda, por carecer de los mencionados documentos o por negarse a su exhibición, los funcionarios policiales se incautarán de las especies, sus partes y del medio de transporte, dando aviso a la fiscalía correspondiente para el inicio de la investigación que proceda, al Servicio de Impuestos Internos ante un eventual delito tributario, a la autoridad sanitaria competente para que instruya sumario sanitario y al Servicio Agrícola y Ganadero para determinar la eventual existencia de infracciones a la normativa agropecuaria.
   
    Ante la sospecha o la comisión de los delitos a que se refiere este párrafo, el Ministerio Público podrá, en lo pertinente, autorizar la correspondiente investigación bajo la técnica de entrega vigilada en los términos regulados en el Párrafo 3° bis del Título I del Libro II del Código Procesal Penal.

    ART. 448 quinquies.
    El que se apropie de las plumas, pelos, crines, cerdas, lanas o cualquier elemento del pelaje de animales ajenos, por cualquier medio que ello se realice, será castigado con presidio menor en sus grados mínimo a medio.

    ART. 448 sexies.

    Los vehículos motorizados o de otra clase, las herramientas y los instrumentos utilizados en la comisión del delito de abigeato, caerán en comiso.

    Durante el curso del procedimiento dichos bienes serán incautados de conformidad a las reglas generales, sin perjuicio del derecho establecido en el artículo 189 del Código Procesal Penal.

    § IV ter.
   
    De la sustracción de madera


     
    ART. 448 septies.

    El que robe o hurte troncos o trozas de madera comete el delito de sustracción de madera y será sancionado con las penas señaladas en los Párrafos II, III y IV del presente Título. Cuando la madera sustraída tenga un valor que exceda las 10 unidades tributarias mensuales se aplicará además la accesoria de multa de 75 a 100 unidades tributarias mensuales.
    Si la madera sustraída tiene un valor superior a las 50 unidades tributarias mensuales o si la sustracción obedece a un proceder sistemático u organizado, se podrán aplicar las técnicas especiales de investigación previstas en el artículo 226 bis del Código Procesal Penal.
    Los vehículos motorizados o de otra clase, las herramientas y los instrumentos utilizados en la comisión del delito, caerán en comiso.
     
    ART. 448 octies.

    Se castigará como autor de sustracción de madera, con las penas previstas en el artículo 446, a quien en cuyo poder se encuentren troncos o trozas de madera, cuando no pueda justificar su adquisición, su legítima tenencia o su labor en dichas faenas o actividades conexas destinadas a la tala de árboles y, del mismo modo, al que sea habido en predio ajeno, en idénticas faenas o actividades, sin consentimiento de su propietario ni autorización de tala.
    Asimismo, será sancionado con la pena de presidio menor en sus grados medio a máximo quien falsifique o maliciosamente haga uso de documentos falsos para obtener guías o formularios con miras a trasladar o comercializar madera de manera ilícita.
    § V.

    Disposiciones comunes a los cuatro Párrafos anteriores.





   
    ART. 449.

    Para determinar la pena de los delitos comprendidos en los Párrafos 1 a 4 ter, con excepción de aquellos contemplados en los artículos 448, inciso primero, y 448 quinquies, y del artículo 456 bis A, no se considerará lo establecido en los artículos 65 a 69 y se aplicarán las reglas que a continuación se señalan:
    1ª. Dentro del límite del grado o grados señalados por la ley como pena al delito, el tribunal determinará la cuantía de la pena en atención al número y entidad de las circunstancias atenuantes y agravantes concurrentes, así como a la mayor o menor extensión del mal causado, fundamentándolo en su sentencia.
    2ª. Tratándose de condenados reincidentes en los términos de las circunstancias agravantes de los numerales 15 y 16 del artículo 12, el tribunal deberá, para los efectos de lo señalado en la regla anterior, excluir el grado mínimo de la pena si ésta es compuesta, o el mínimum si consta de un solo grado.


    ART. 449 bis.-

    Será circunstancia agravante de los delitos contemplados en los Párrafos 1, 2, 3, 4, 4 bis y 4 ter de este Título, y del descrito en el artículo 456 bis A, el hecho de que el imputado haya actuado formando parte de una agrupación u organización de dos o más personas destinada a cometer dichos hechos punibles, siempre que ésta o aquélla no constituyere una asociación ilícita de que trata el Párrafo 10 del Título VI del Libro Segundo.





    ART. 449 ter.

    Cuando los delitos sancionados en los Párrafos 3 y 4 de este Título se perpetraren con ocasión de calamidad pública o alteración del orden público, sea que se actúe en grupo o individualmente pero amparado en este, se aumentará la pena privativa de libertad respectiva en un grado.
    Tratándose de la conducta sancionada en el inciso primero del artículo 436, y concurriendo las circunstancias descritas en el inciso anterior, se aplicará la pena privativa de libertad respectiva, con exclusión de su grado mínimo.




    ART. 449 quáter.

    Se aplicará en todo caso la regla 2ª del artículo 449, aun cuando el responsable no sea reincidente, si los delitos señalados en dicho artículo se cometen en circunstancias tales que contribuyan a la sustracción o destrucción de todo o la mayor parte de aquello que había o se guardaba en algún establecimiento de comercio o industrial o del propio establecimiento. En estos casos el hecho se denominará saqueo.
    Si el responsable fuere reincidente en los términos de las circunstancias agravantes de los numerales 15 y 16 del artículo 12, el juez podrá considerar suficiente fundamento esta circunstancia para la imposición del máximo de la pena resultante. 



    ART. 450.

    Los delitos a que se refiere al Párrafo 2 y el artículo 440 del Párrafo 3 de este Título se castigarán como consumados desde que se encuentren en grado de tentativa.
    La misma regla se aplicará a los delitos sancionados en los Párrafos 3, 4, 4 bis y 4 ter de este Título cuando se cometieren con las circunstancias señaladas en el inciso primero de los artículos 449 ter o 449 quáter.
   




    ART. 450 bis.
    En el robo con violencia o intimidación en las personas no procederá la atenuante de responsabilidad penal contenida en el artículo 11, N° 7.

    Artículo 451.- En los casos de reiteración de hurtos, aunque se trate de faltas, a una misma persona, o a distintas personas en una misma casa, establecimiento de comercio, centro comercial, feria, recinto o lugar el tribunal calificará el ilícito y hará la regulación de la pena tomando por base el importe total de los objetos sustraídos, y la impondrá al delincuente en su grado superior.
    Esta regla es sin perjuicio de lo dispuesto en el art. 447.


    ART. 452.

    El que después de haber sido condenado por robo o hurto cometiere cualquiera de estos delitos, además de las penas que le correspondan por el hecho o hechos en que hubiere reincidido, el tribunal podrá imponerle la de sujeción a la vigilancia de la autoridad dentro de los límites fijados en el art. 25.



    ART. 453.

    Cuando se reunieren en un hecho varias de las circunstancias a que se señala pena diversa según los párrafos precedentes, se aplicará la de las circunstancias que en aquel caso particular la merezcan más grave, pudiendo el tribunal aumentarla en un grado.


    ART. 454.
    Se presumirá autor del robo o hurto de una cosa aquel en cuyo poder se encuentre, salvo que justifique su legítima adquisición o que la prueba de su irreprochable conducta anterior establezca una presunción en contrario.


    ART. 455.

    Cuando del proceso no resulte probado el valor de la cosa sustraída ni pudiere estimarse por peritos u otro arbitrio legal, el tribunal hará su regulación prudencialmente.



    ART. 455 bis.

    Si en el momento de producirse el robo o hurto de un vehículo motorizado, se encontrare en su interior un infante o una persona que no pudiere abandonar el vehículo por sus propios medios, y el autor del robo o hurto inicia la conducción del mismo, se aplicará la pena de presidio mayor en sus grados medio a máximo.



    ART. 456.

    Si antes de perseguir al responsable o antes de decretar su prisión devolviere voluntariamente la cosa robada o hurtada, no hallándose comprendido en los casos de los arts. 433 y 434, se le aplicará la pena inmediatamente inferior en grado a la señalada para el delito.
    ART. 456 BIS.

    En los delitos de robo y hurto serán circunstancias agravantes las siguientes:
    1°) Ejecutar el delito en sitios faltos de vigilancia policial, obscuros, solitarios, sin tránsito habitual o que por cualquiera otra condición favorezcan la impunidad.
    2°) Ser la víctima niño, anciano, inválido o persona en manifiesto estado de inferioridad física;
    3°) Ejecutar el delito usando un vehículo motorizado sin placa patente delantera, trasera o ambas; o con cualquiera oculta o con vidrios oscuros o polarizados, en contravención a la ley N° 18.290, de Tránsito; o en el que se haya utilizado cualquier otra práctica, técnica, intervención, herramienta, dispositivo o condición que favorezca su impunidad;
    4°) Ejercer la violencia en las personas que intervengan en defensa de la víctima, salvo que este hecho importe otro delito; y
    5°) Actuar con personas exentas de responsabilidad criminal, según el número 1.o del artículo 10.
    Las circunstancias agravantes de los números 1.o y 5° del artículo 12 serán aplicables en los casos en que se ejerciere violencia sobre las personas.
    En estos delitos no podrá estimarse que concurre la circunstancia atenuante del número 7° del artículo 11, por la mera restitución a la víctima de las especies robadas o hurtadas y, en todo caso, el Juez deberá considerar, especificada, la justificación del celo con que el delincuente ha obrado.



    § 5 bis. De la receptación

    ART. 456 bis A.

    El que conociendo su origen o no pudiendo menos que conocerlo, tenga en su poder, a cualquier título, especies hurtadas, robadas u objeto de abigeato o sustracción de madera, de receptación o de apropiación indebida del artículo 470, número 1°, las transporte, compre, venda, transforme o comercialice en cualquier forma, aun cuando ya hubiese dispuesto de ellas, sufrirá la pena de presidio menor en cualquiera de sus grados y multa de cinco a cien unidades tributarias mensuales.
    Para la determinación de la pena aplicable el tribunal tendrá especialmente en cuenta el valor de las especies, así como la gravedad del delito en que se obtuvieron, si éste era conocido por el autor.
    Cuando el objeto de la receptación sean vehículos motorizados o cosas que forman parte de redes de suministro de servicios públicos o domiciliarios, tales como electricidad, gas, agua, alcantarillado, colectores de aguas lluvia o telefonía, se impondrá la pena de presidio menor en su grado máximo y multa equivalente al valor de la tasación fiscal del vehículo o la pena de presidio menor en su grado máximo, y multa de cinco a veinte unidades tributarias mensuales, respectivamente. La sentencia condenatoria por delitos de este inciso dispondrá el comiso de los instrumentos, herramientas o medios empleados para cometerlos o para transformar o transportar los elementos sustraídos. Si dichos elementos son almacenados, ocultados o transformados en algún establecimiento de comercio con conocimiento del dueño o administrador, se podrá decretar, además, la clausura definitiva de dicho establecimiento, oficiándose a la autoridad competente.
    Sin perjuicio de lo dispuesto en el inciso anterior, se aplicará el máximum de la pena privativa de libertad allí señalada y multa equivalente al doble de la tasación fiscal, al autor de receptación de vehículos motorizados que conociere o no pudiere menos que conocer que en la apropiación de éste se ejerció sobre su legítimo tenedor alguna de las conductas descritas en el artículo 439. Lo dispuesto en este inciso no será aplicable a quien, por el mismo hecho, le correspondiere participación responsable por cualquiera de las hipótesis del delito de robo previstas en el artículo 433 y en el inciso primero del artículo 436.
    Se impondrá el grado máximo de la pena establecida en el inciso primero, cuando el autor haya incurrido en reiteración de esos hechos o sea reincidente en ellos. En los casos de reiteración o reincidencia en la receptación de los objetos señalados en el inciso tercero, se aplicará la pena privativa de libertad allí establecida, aumentada en un grado.
    Tratándose del delito de abigeato o sustracción de madera y la multa establecida en el inciso primero será de setenta y cinco a cien unidades tributarias mensuales y el juez podrá disponer la clausura definitiva del establecimiento.
    Si el valor de lo receptado excediere de cuatrocientas unidades tributarias mensuales, se impondrá el grado máximo de la pena o el máximun de la pena que corresponda en cada caso.





    § VI.

    De la usurpación.





    ART. 457.

    Al que con violencia en las personas ocupare una cosa inmueble o usurpare un derecho real que otro poseyere o tuviere legítimamente, y al que, hecha la ocupación en ausencia del legítimo poseedor o tenedor, vuelto éste le repeliere, además de las penas en que incurra por la violencia que causare, se le aplicará una multa de once a veinte unidades tributarias mensuales.
    Si tales actos se ejecutaren por el dueño o poseedor regular contra el que posee o tiene ilegítimamente la cosa, aunque con derecho aparente, la pena será multa de seis a diez unidades tributarias mensuales, sin perjuicio de las que correspondieren por la violencia causada.






    ART. 458.

    Cuando, en los casos del inciso primero del artículo anterior, el hecho se llevare a efecto sin violencia en las personas, la pena será multa de seis a diez unidades tributarias mensuales.









    ART. 459.

    Sufrirán las penas de presidio menor en sus grados medio a máximo y multa de veinte a cinco mil unidades tributarias mensuales, los que sin título legítimo e invadiendo derechos ajenos:
    1.° Sacaren aguas de represas, estanques u otros depósitos; de ríos, arroyos o fuentes, sean superficiales o subterráneas; de canales o acueductos, redes de agua potable e instalaciones domiciliarias de éstas, y se las apropiaren para hacer de ellas un uso cualquiera.
    2.° Rompieren o alteraren con igual fin diques, esclusas, compuertas, marcos u otras obras semejantes existentes en los ríos, arroyos, fuentes, depósitos, canales o acueductos.
    3.° Pusieren embarazo al ejercicio de los derechos que un tercero tuviere sobre dichas aguas.
    4.° Usurparen un derecho cualquiera referente al curso de ellas o turbaren a alguno en su legítima posesión.
    Las sanciones establecidas en este artículo no se aplicarán a quienes hagan uso del agua para consumo personal o familiar en los términos señalados en el artículo 56 del Código de Aguas.


    ART. 460.

    Cuando los simples delitos a que se refiere el artículo anterior se ejecutaren con violencia o intimidación en las personas, si el culpable no mereciere mayor pena por la violencia o intimidación que causare, sufrirá la de presidio menor en cualquiera de sus grados y multa de cincuenta a cinco mil unidades tributarias mensuales.









    Artículo 460 bis.- El que a sabiendas duplique la inscripción de su derecho en el Registro de Propiedad de Aguas del Conservador de Bienes Raíces sufrirá las penas de presidio menor en su grado mínimo, multa de once a veinte unidades tributarias mensuales, la revocación del título duplicado y la cancelación de la inscripción duplicada.



    ART. 461.

    Serán castigados con las penas del artículo 459, los que teniendo derecho para sacar aguas o usarlas se hubieren servido fraudulentamente, con tal fin, de orificios, conductos, marcos, compuertas o esclusas de una forma diversa a la establecida o de una capacidad superior a la medida a que tienen derecho.


    ART. 462.

    El que destruyere o alterare términos o límites de propiedades públicas o particulares con ánimo de lucrarse, será penado con presidio menor en su grado mínimo y multa de once a veinte unidades tributarias mensuales.








    § VII.

    De los delitos concursales y de las defraudaciones.



    ART. 463.

    Será castigado con la pena de presidio menor en cualquiera de sus grados el que, dentro de los dos años anteriores a la dictación de la resolución de liquidación a la que se refiere la ley N° 20.720, que Sustituye el régimen concursal vigente por una ley de reorganización y liquidación de empresas y personas, y perfecciona el rol de la superintendencia del ramo, o durante el tiempo que medie entre la notificación de la demanda de liquidación forzosa y la dictación de la respectiva resolución, conociendo el mal estado de sus negocios:

    1.° Redujere considerablemente su patrimonio destruyendo, dañando, inutilizando o dilapidando, activos o valores o renunciando sin razón a créditos.

    2.° Dispusiere de sumas relevantes en consideración a su patrimonio aplicándolas en juegos o apuestas o en negocios inusualmente riesgosos en relación con su actividad económica normal.

    3.° Diere créditos sin las garantías habituales en atención a su monto, o se desprendiere de garantías sin que se hubieren satisfecho los créditos caucionados.

    4.° Realizare otro acto manifiestamente contrario a las exigencias de una administración racional del patrimonio.

    Tratándose de una empresa deudora en el sentido de la ley N° 20.720, la pena señalada en el inciso anterior se impondrá también al que hubiere actuado con ignorancia inexcusable del mal estado de sus negocios.

    En el caso del número 4.° del inciso primero, las penas no serán impuestas si el hecho no hubiere contribuido relevantemente a ocasionar la insolvencia del deudor.




    ART. 463 bis.-

    Será castigado con la pena de presidio menor en su grado medio a presidio mayor en su grado mínimo, el deudor que realizare alguna de las siguientes conductas:

    1.° Favorecer a uno o más acreedores en desmedro de otro pagando deudas que no fueren actualmente exigibles u otorgando garantías para deudas contraídas previamente sin garantía, dentro de los dos años anteriores a la resolución de reorganización o liquidación o durante el tiempo que medie entre la notificación de la demanda de liquidación forzosa y la dictación de la respectiva resolución.

    2.° Percibir, apropiarse o distraer bienes que deban ser objeto de cualquier clase de procedimiento concursal de liquidación, después de dictada la resolución de liquidación.

    3.° Realizar actos de disposición de bienes de su patrimonio, reales o simulados, o constituir prenda, hipoteca u otro gravamen sobre ellos, después de la resolución de liquidación.

    4.° Ocultar total o parcialmente sus bienes o sus haberes, dentro de los dos años anteriores a la resolución de liquidación o reorganización, o con posterioridad a esa resolución.

    ART. 463 ter.-

    Será castigado con la pena de presidio menor en sus grados mínimo a medio el deudor que:

    1.° Durante cualquier clase de procedimiento concursal de reorganización o de liquidación, proporcionare al veedor o liquidador, en su caso, o a sus acreedores, información o antecedentes falsos o incompletos, en términos que no reflejen la verdadera situación de su activo o pasivo.

    2.° Dentro de los dos años anteriores a la dictación de la resolución de liquidación o durante el tiempo que medie entre la notificación de la demanda de liquidación forzosa y la dictación de la respectiva resolución, no hubiese llevado o conservado los libros de contabilidad y sus respaldos exigidos por la ley que deben ser puestos a disposición del liquidador una vez dictada la resolución de liquidación, o si hubiese ocultado, inutilizado, destruido o falseado la información en términos que ella no refleje la verdadera situación de su activo y pasivo.

    ART. 463 quáter.-
    Será castigado como autor de los delitos contemplados en los artículos 463, 463 bis y 463 ter quien, en la dirección o administración de los negocios del deudor, sometido a un procedimiento concursal de reorganización a un procedimiento concursal de reorganización simplificada, a un procedimiento concursal de liquidación o a un procedimiento concursal de liquidación simplificada, hubiese ejecutado alguno de los actos o incurrido en alguna de las omisiones allí señalados, o hubiese autorizado expresamente dichos actos u omisiones.



    ART. 464.

    Será castigado con la pena de presidio menor en su grado máximo a presidio mayor en su grado mínimo y con la sanción accesoria de inhabilidad especial perpetua para ejercer el cargo, el veedor o liquidador designado en cualquier clase de procedimiento concursal de reorganización o de liquidación que:

    1. Proporcionare ventajas indebidas al deudor, a un acreedor o a un tercero.

    2. Perpetrare cualquiera de los hechos previstos en los números 1 u 11 del artículo 470.

    ART. 464 bis.-

    El deudor, veedor, liquidador, o aquellos a los que se refiere el artículo 463 quáter, que se valiere de quien no tuviere esa calidad para perpetrar cualquiera de los delitos previstos en los artículos precedentes de este Párrafo será castigado como autor del respectivo delito.
    El que sin tener alguna de las calidades señaladas en el inciso precedente interviniere en la perpetración del delito será castigado como inductor o cómplice según las circunstancias.

    ART. 464 ter.-

    El que mediante engaño determinare a un deudor, veedor, liquidador, o aquellos a los que se refiere el artículo 463 quáter, a incurrir en cualquiera de los hechos previstos en los artículos precedentes de este Párrafo, será castigado con las mismas penas en ellos señalada.
    ART. 464 quáter.-

    Además de lo dispuesto en los artículos 27 a 31, el profesional que, con ocasión del ejercicio de su profesión, fuere penalmente responsable por haber intervenido en la perpetración de cualquiera de los delitos previstos en el presente Párrafo, será sancionado también con la pena accesoria de suspensión o inhabilitación para su ejercicio.
    La pena y su duración serán determinadas atendiendo a la pena principal impuesta conforme a las reglas previstas en los artículos 29 y 30 de este Código, para la inhabilitación o suspensión de cargo u oficio público.
    ART. 465.
    La persecución penal de los delitos contemplados en este Párrafo sólo podrá iniciarse previa instancia particular de la Superintendencia de Insolvencia y Reemprendimiento; del veedor o liquidador del proceso concursal respectivo; de cualquier acreedor que haya verificado su crédito si se tratare de un procedimiento concursal de liquidación o de liquidación simplificada, lo que se acreditará con copia autorizada del respectivo escrito y su proveído; o en el caso de un procedimiento concursal de reorganización o reorganización simplificada, de todo acreedor a quien le afecte el Acuerdo de Reorganización de conformidad a lo establecido en los artículos 66 y 286 F de la ley Nº 20.720.
    Si se tratare de delitos de este Párrafo cometidos por veedores o liquidadores, la Superintendencia de Insolvencia y Reemprendimiento deberá denunciarlos si alguno de los funcionarios de su dependencia toma conocimiento de aquéllos en el ejercicio de sus funciones. Además, podrá interponer querella criminal, entendiéndose para este efecto cumplidos los requisitos que establece el inciso tercero del artículo 111 del Código Procesal Penal.
    Cuando se celebren acuerdos reparatorios de conformidad al artículo 241 y siguientes del Código Procesal Penal, los términos de esos acuerdos deberán ser aprobados previamente por la junta de acreedores respectiva y las prestaciones que deriven de ellos beneficiarán a todos los acreedores, a prorrata de sus respectivos créditos, sin distinguir para ello la clase o categoría de los mismos.
    Conocerá de los delitos concursales regulados en este Párrafo el tribunal con competencia en lo criminal del domicilio del deudor.
   

    ART. 465 bis.-Derogado.

    ART. 466. Derogado.





    § VIII.

    Estafas y otros engaños.





    ART. 467.-

    El que para obtener provecho patrimonial para sí o para un tercero mediante engaño provocare un error en otro, haciéndolo incurrir en una disposición patrimonial consistente en ejecutar, omitir o tolerar alguna acción en perjuicio suyo o de un tercero será sancionado:

    1. Con presidio menor en su grado máximo y multa de veintiuna a trescientas unidades tributarias mensuales, si el perjuicio excede de cuatrocientas unidades tributarias mensuales y no pasa de cuarenta mil.

    2. Con presidio menor en sus grados medio a máximo y multa de once a quince unidades tributarias mensuales, si excede de cuarenta unidades tributarias mensuales y no pasa de cuatrocientas.

    3. Con presidio menor en su grado medio y multa de seis a diez unidades tributarias mensuales, si excede de cuatro unidades tributarias mensuales y no pasa de cuarenta.

    4. Con presidio menor en su grado mínimo y multa de cinco unidades tributarias mensuales, si excede de una unidad tributaria mensual y no pasa de cuatro.

    Si el perjuicio excede de cuarenta mil unidades tributarias mensuales, se aplicará la pena de presidio menor en su grado máximo a presidio mayor en su grado mínimo y multa de trescientas a quinientas unidades tributarias mensuales.




    ART. 468.

    Incurrirá en el delito previsto en el artículo anterior el que defraudare a otro usando de nombre fingido, atribuyéndose poder, influencia o crédito supuestos, aparentando bienes, crédito, comisión, empresa o negociación imaginarios, o valiéndose de cualquier otro engaño semejante.

    Las penas del artículo anterior serán aplicadas también al que para obtener un provecho para sí o para un tercero irrogue perjuicio patrimonial a otra persona:

    1. Manipulando los datos contenidos en un sistema informático o el resultado del procesamiento informático de datos a través de una intromisión indebida en la operación de éste.

    2. Utilizando sin la autorización del titular una o más claves confidenciales que habiliten el acceso u operación de un sistema informático, o

    3. Haciendo uso no autorizado de una tarjeta de pago ajena o de los datos codificados en una tarjeta de pago que la identifiquen y habiliten como medio de pago.

    Sin perjuicio de las penas que correspondan conforme al inciso anterior, sufrirá la pena de presidio menor en su grado medio y multa de seis a diez unidades tributarias mensuales el que obtenga indebidamente los datos codificados en una tarjeta de pago que la identifiquen y habiliten como medio de pago. La misma pena sufrirá el que los adquiera o ponga a disposición de otro a cualquier título.

    En la investigación de los delitos previstos en este artículo será aplicable lo dispuesto en el artículo 8 de la ley N° 20.009.

    Lo dispuesto en los incisos segundo y tercero de este artículo será aplicable si el hecho no tuviere mayor pena conforme a otra ley.


    ART. 469.

    Se impondrá respectivamente el máximum de las penas señaladas en el art. 467:
    1.° A los plateros y joyeros que cometieren defraudaciones alterando en su calidad, ley o peso los objetos relativos a su arte o comercio.
    2.° A los traficantes que defraudaren usando de pesos o medidas falsos en el despacho de los objetos de su tráfico.
    3.° A los comisionistas que cometieren defraudación alterando en sus cuentas los precios o las condiciones de los contratos, suponiendo gastos o exagerando los que hubieren hecho.
    4.° A los capitanes de buques que defrauden suponiendo gastos o exagerando los que hubieren hecho, o cometiendo cualquiera otro fraude en sus cuentas.
    5.° A los que cometieren defraudación con pretexto de supuestas remuneraciones a empleados públicos, sin perjuicio de la acción de calumnia que a éstos corresponda.
    6.° Al dueño de la cosa embargada, o a cualquier otro que, teniendo noticia del embargo, hubiere destruido fraudulentamente los objetos en que se ha hecho la traba.



    ART. 470.

    Las penas privativas de libertad del art. 467 se aplicarán también:
    1.° A los que en perjuicio de otro se apropiaren o distrajeren dinero, efectos o cualquiera otra cosa mueble que hubieren recibido en depósito, comisión o administración, o por otro título que produzca obligación de entregarla o devolverla.
    En cuanto a la prueba del depósito en el caso a que se refiere el art. 2.217 del Código Civil, se observará lo que en dicho artículo se dispone.
    2.° A los capitanes de buques que, fuera de los casos y sin las solemnidades prevenidas por la ley, vendieren dichos buques, tomaren dinero a la gruesa sobre su casco y quilla, giraren letras a cargo del naviero, enajenaren mercaderías o vituallas o tomaren provisiones pertenecientes a los pasajeros.
    3.° A los que cometieren alguna defraudación abusando de firma de otro en blanco y extendiendo con ella algún documento en perjuicio del mismo o de un tercero.
    4.° A los que defraudaren haciendo suscribir a otro con engaño algún documento.
    5.° A los que cometieren defraudaciones sustrayendo, ocultando, destruyendo o inutilizando en todo o en parte algún proceso, expediente, documento u otro papel de cualquiera clase.
    6.° A los que con datos falsos u ocultando antecedentes que les son conocidos, celebraren dolosamente contratos aleatorios basados en dichos datos o antecedentes.
    7.° A los que en el juego se valieren de fraude para asegurar la suerte.
    8.° A los que fraudulentamente obtuvieren del Fisco, de las municipalidades, de las Cajas de Previsión y de las instituciones centralizadas o descentralizadas del Estado, prestaciones improcedentes, tales como remuneraciones, bonificaciones, subsidios, pensiones, jubilaciones, asignaciones, devoluciones o imputaciones indebidas.
    9.° Al que, con ánimo de defraudar, con o sin representación de persona natural o jurídica dedicada al rubro inmobiliario o de la construcción, suscribiere o hiciere suscribir contrato de promesa de compraventa de inmueble dedicado a la vivienda, local comercial u oficina, sin cumplir con las exigencias establecidas por el artículo 138 bis de la Ley General de Urbanismo y Construcciones, siempre que se produzca un perjuicio patrimonial para el promitente comprador.
    10.° A los que maliciosamente obtuvieren para sí, o para un tercero, el pago total o parcialmente indebido de un seguro, sea simulando la existencia de un siniestro, provocándolo intencionalmente, presentándolo ante el asegurador como ocurrido por causas o en circunstancias distintas a las verdaderas, ocultando la cosa asegurada o aumentando fraudulentamente las pérdidas efectivamente sufridas.
    Si no se verifica el pago indebido por causas independientes de su voluntad, se aplicará el mínimo o, en su caso, el grado mínimo de la pena.
    La pena se determinará de acuerdo con el monto de lo indebidamente solicitado.
    11. Al que teniendo a su cargo la salvaguardia o la gestión del patrimonio de otra persona, o de alguna parte de éste, en virtud de la ley, de una orden de la autoridad o de un acto o contrato, le irrogare perjuicio, sea ejerciendo abusivamente facultades para disponer por cuenta de ella u obligarla, sea ejecutando u omitiendo cualquier otra acción de modo manifiestamente contrario al interés del titular del patrimonio afectado.
    Si el hecho recayere sobre el patrimonio de una persona en relación con la cual el sujeto fuere guardador, tutor o curador, o de una persona incapaz que el sujeto tuviere a su cargo en alguna otra calidad, se impondrá, según sea el caso, el máximum o el grado máximo de las penas señaladas en el artículo 467.
    En caso de que el patrimonio encomendado fuere el de una sociedad anónima abierta o especial u otro patrimonio administrado por esa sociedad, el administrador que realizare alguna de las conductas descritas en el párrafo primero de este numeral, irrogando perjuicio al patrimonio social, será sancionado con las penas señaladas en el artículo 467 aumentadas en un grado. Además, se impondrá la pena de inhabilitación especial temporal en su grado mínimo para desempeñarse como gerente, director, liquidador o administrador a cualquier título de una sociedad o entidad sometida a fiscalización de una Superintendencia o de la Comisión para el Mercado Financiero.
    En los casos previstos en este artículo se impondrá, además, pena de multa de la mitad al tanto de la defraudación.






    ART. 471.

    Será castigado con presidio o relegación menores en sus grados mínimos o multa de once a veinte unidades tributarias mensuales:
    1.º El dueño de una cosa mueble que la sustrajere de quien la tenga legítimamente en su poder, con perjuicio de éste o de un tercero.
    2.° El que otorgare en perjuicio de otro un contrato simulado.
    3.° Derogado.
    Los ejemplares, máquinas u objetos contrahechos, introducidos o expendidos fraudulentamente, se aplicarán al perjudicado y también las láminas o utensilios empleados en la ejecución del fraude, cuando solo pudieren usarse para cometerlo.

    ART. 472.

    El que suministre valores, de cualquiera manera que sea, a un interés que exceda del máximo que la ley permita estipular, será castigado con presidio o reclusión menores en cualquiera de sus grados.
    Se impondrá el grado máximo de la pena establecida en el inciso anterior cuando la conducta que allí se sanciona se realice simulando, de cualquier forma, que se suministran los valores a un interés permitido por la ley.
    Condenado por usura un extranjero, será expulsado del país; y condenado como reincidente en delito de usura un nacionalizado, se le cancelará su nacionalización y se le expulsará del país.
    En ambos casos la expulsión se hará después de cumplida la pena.
    En la sustanciación y fallo de los procesos instruidos para la investigación de estos delitos, los Tribunales apreciarán la prueba en conciencia.




    ART. 472 bis.-

    El que con abuso grave de una situación de necesidad, de la inexperiencia o de la incapacidad de discernimiento de otra persona, le pagare una remuneración manifiestamente desproporcionada e inferior al ingreso mínimo mensual previsto por la ley o le diere en arrendamiento un inmueble como morada recibiendo una contraprestación manifiestamente desproporcionada, será castigado con la pena de presidio o reclusión menor en cualquiera de sus grados.
    ART. 472 ter.-

    En los casos en que alguno de los hechos previstos en este Párrafo irrogare un perjuicio que exceda de ochenta mil unidades tributarias mensuales o afecte a un número considerable de personas, se podrá imponer la pena superior en un grado a la señalada por la ley.

    ART. 473.

    El que defraudare o perjudicare a otro usando de cualquier engaño que no se halle expresado en los artículos anteriores de este párrafo, será castigado con presidio o relegación menores en sus grados mínimos y multa de once a veinte unidades tributarias mensuales.








    § IX.

    Del incendio y otros estragos.





    Artículo 474. El que incendiare edificio, aeronave, buque, plataforma naval, automóviles de dos o más plazas, camiones, instalaciones de servicios sanitarios, de almacenamiento o transporte de combustibles, de distribución o generación de energía eléctrica, portuaria, aeronáutica o ferroviaria, incluyendo las de trenes subterráneos, u otro lugar, medio de transporte, instalación o bien semejante, siempre que hubiere personas en su interior, causando la muerte de una o más personas cuya presencia allí pudo prever, será castigado con presidio mayor en su grado máximo a presidio perpetuo.
    La misma pena se impondrá cuando del incendio no resultare muerte sino mutilación de miembro importante o lesión grave de las comprendidas en el número 1° del artículo 397.



    Artículo 475. El que incendiare edificio, aeronave, buque, plataforma naval, vehículos de transporte público de pasajeros, automóviles de dos o más plazas, camiones, instalaciones de servicios sanitarios, de almacenamiento o transporte de combustibles, de distribución o generación de energía eléctrica, portuaria, aeronáutica o ferroviaria, incluyendo las de trenes subterráneos, u otro lugar, medio de transporte, instalación o bien semejante, siempre que allí hubiere una o más personas y su presencia se pudiese prever, será castigado con presidio mayor en su grado medio a presidio perpetuo.



    ART. 476.

    Se castigará con presidio mayor en cualquiera de sus grados:
    1.° Al que incendiare un edificio o lugar destinado a servir de morada, que no estuviere actualmente habitado.
    2º Al que dentro de poblado ejecutare el incendio en edificio, aeronave, buque, plataforma naval, vehículos de transporte público de pasajeros, automóviles de dos o más plazas, camiones, instalaciones de servicios sanitarios, de almacenamiento o transporte de combustibles, de distribución o generación de energía eléctrica, portuaria, aeronáutica o ferroviaria, incluyendo las de trenes subterráneos, u otro lugar, medio de transporte, instalación o bien semejante, cuando no hubiere personas en su interior o su presencia no se pudiese prever.
    3.º Al que incendiare bosques, mieses, pastos, montes, cierros, plantíos o formaciones xerofíticas de aquellas definidas en la ley Nº 20.283.
    4.º Al que fuera de los casos señalados en los números anteriores provoque un incendio que afectare gravemente las condiciones de vida animal o vegetal de un Área Silvestre Protegida.



    Artículo 477.- El incendiario de objetos no comprendidos en los artículos anteriores será penado:
    1.º Con presidio menor en su grado máximo a presidio mayor en su grado mínimo y multa de once a quince unidades tributarias mensuales, si el daño causado a terceros excediere de cuarenta unidades tributarias mensuales.
    2.º Con presidio menor en sus grados medio a máximo y multa de seis a diez unidades tributarias mensuales, si el daño excediere de cuatro unidades tributarias mensuales y no pasare de cuarenta unidades tributarias mensuales.
    3.º Con presidio menor en sus grados mínimo a medio y multa de cinco unidades tributarias mensuales, si el daño excediere de una unidad tributaria mensual y no pasare de cuatro unidades tributarias mensuales.


    ART. 478.

    En caso de aplicarse el incendio a chozas, pajar o cobertizo deshabitado o a cualquier otro objeto cuyo valor no excediere de cuatro sueldos vitales, en tiempo y con circunstancias que manifiestamente excluyan todo peligro de propagación, el culpable no incurrirá en las penas señaladas en este párrafo; pero sí en las que mereciera por el daño que causare con arreglo a las disposiciones del párrafo siguiente.


    ART. 479.

    Cuando el fuego se comunicare del objeto que el culpable se propuso quemar, a otro u otros cuya destrucción, por su naturaleza o consecuencias, debe penarse con mayor severidad, se aplicará la pena más grave, siempre que los objetos incendiados estuvieren colocados de tal modo que el fuego haya debido comunicarse de unos a otros, atendidas las circunstancias del caso.


    ART. 480.

    Incurrirán respectivamente en las penas de este párrafo los que causen estragos por medio de sumersión o varamiento de nave, inundación, destrucción de puentes o máquinas de vapor, y en general por la aplicación de cualquier otro agente o medio de destrucción tan poderoso como los expresados.



    ART. 481.

    El que fuere aprehendido con artefactos, implementos o preparativos conocidamente dispuestos para incendiar o causar alguno de los estragos expresados en este párrafo, será castigado con presidio menor en sus grados mínimo a medio; salvo que pudiendo considerarse el hecho como tentativa de un delito determinado debiera castigarse con mayor pena.



    ART. 482.

    El culpable de incendio o estragos no se eximirá de las penas de los artículos anteriores, aunque para cometer el delito hubiere incendiado o destruido bienes de su pertenencia.
    Pero no incurrirá en tales penas el que rozare a fuego, incendiare rastrojos u otros objetos en tiempos y con circunstancias que manifiestamente excluyan todo propósito de propagación, y observando los reglamentos que se dicten sobre esta materia.



    ART. 483.

    Se presume responsable de un incendio al comerciante en cuya casa o establecimiento tiene origen aquél, si no justificare con sus libros, documentos u otra clase de prueba, que no reportaba provecho alguno del siniestro.
    Se presume también responsable de un incendio al comerciante cuyo seguro sea exageradamente superior al valor real del objeto asegurado en el momento de producirse el siniestro. En los casos de seguros con pólizas flotantes se presumirá responsable al comerciante que, en la declaración inmediatamente anterior al siniestro, declare valores manifiestamente superiores a sus existencias. Asimismo, se presume responsable si en todo o en parte a disminuido o retirado las cosas aseguradas del lugar señalado en la póliza respectiva sin motivo justificado o sin dar aviso previo al asegurador.
    Las presunciones de este artículo no obstan a la apreciación de la prueba en conciencia.

    ART. 483. a)

    El contador o cualquiera persona que falsee o adultere la contabilidad del comerciante que sufra un siniestro, será sancionado con la pena señalada en el inciso segundo del artículo 197; pero no le afectará responsabilidad al contador por las existencias y precios inventariados.

    ART. 483. b)

    A los comerciantes responsables del delito de incendio se les aplicará también una multa de veintiuna a cincuenta unidades tributarias mensuales, tomándose en cuenta para graduarla la naturaleza, entidad y gravedad del siniestro y las facultades económicas del condenado.

    Si no se paga la multa el condenado sufrirá por vía de sustitución y apremio, un día de reclusión por un un quinto de unidad tributaria mensual de multa, no pudiendo exceder la reclusión de seis meses.

    La multa impuesta se mantendrá en una cuenta especial a la orden de la Superintendencia de Compañia de Seguros Sociedades Anónimas y Bolsas de Comercio, la cual anualmente la distribuirá proporcionalmente entre los distintos Cuerpos de Bomberos en el país.

    § X.

    De los daños.





    ART. 484.

    Incurren en el delito de daños y están sujetos a las penas de este párrafo, los que en la propiedad ajena causaren alguno que no se halle comprendido en el párrafo anterior.
    ART 485.

    Serán castigados con la pena de reclusión menor en sus grados medio a máximo y multa de once a veinte unidades tributarias mensuales, los que causaren daño cuyo importe exceda de cuarenta unidades tributarias mensuales:

    1.° Con la mira de impedir el libre ejercicio de la autoridad o en venganza de sus determinaciones, bien se cometiere el delito contra empleados públicos, bien contra particulares que, como testigos o de cualquiera otra manera, hayan contribuido o puedan contribuir a la ejecución o aplicación de las leyes;
    2.° Produciendo, por cualquier medio, infección o contagio en animales o aves domésticas;
    3.° Empleando sustancias venenosas o corrosivas;
    4.° En cuadrilla y en despoblado;
    5.° En archivos, registros, bibliotecas o museos públicos;
    6.° En puentes, caminos, paseos u otros bienes de uso público;
    7.° En tumbas, signos conmemorativos, monumentos, estatuas, cuadros u otros objetos de arte colocados en edificios o lugares públicos;
    8.° Arruinando al perjudicado;
    9.° En medios de transporte público de pasajeros o en bienes o infraestructura asociada a dicha actividad.




    ART. 486.

    El que, con alguna de las circunstancias expresadas en el artículo anterior, causare daño cuyo importe exceda de cuatro unidades tributarias mensuales y no pase de cuarenta unidades tributarias mensuales, sufrirá la pena de reclusión menor en sus grados mínimo a medio y multa de seis a diez unidades tributarias mensuales.
    Cuando dicho importe no excediere de cuatro unidades tributarias mensuales ni bajare de una unidad tributaria mensual, la pena será reclusión menor en su grado mínimo y multa de cinco unidades tributarias mensuales.


    ART. 487.

    Los daños no comprendidos en los artículos anteriores, serán penados con reclusión menor en su grado mínimo o multa de once a veinte unidades tributarias mensuales.
    Esta disposición no es aplicable a los daños causados por el ganado y a los demás que deben calificarse de faltas, con arreglo a lo que se establece en el Libro tercero.






    ART. 488.

    Las disposiciones del presente párrafo sólo tendrán lugar cuando el hecho no pueda considerarse como otro delito que merezca mayor pena.



    § XI.

    Disposiciones generales.





    ART. 489.

    Están exentos de responsabilidad criminal y sujetos únicamente a la civil por los hurtos, defraudaciones o daños que recíprocamente se causaren:
    1.º Los parientes consanguíneos en toda la línea recta.
    2.º Los parientes consanguíneos hasta el segundo grado inclusive de la línea colateral.
    3.° Los parientes afines en toda la línea recta.
    4.° Derogado.
    5.° Los cónyuges.
    6.° Los convivientes civiles.

    La excepción de este artículo no es aplicable a los extraños que participaren del delito, ni tampoco entre cónyuges cuando se trate de los delitos de daños indicados en el párrafo anterior.

    Además, esta exención no será aplicable cuando la víctima sea una persona mayor de sesenta años.






    TÍTULO DÉCIMO.

    DE LOS CUASIDELITOS.




    ART. 490.

    El que por imprudencia temeraria ejecutare un hecho que, si mediara malicia, constituiría un crimen o un simple delito contra las personas, será penado:
    1.º Con reclusión o relegación menores en sus grados mínimos a medios, cuando el hecho importare crimen.
    2.° Con reclusión o relegación menores en sus grados mínimos o multa de once a veinte unidades tributarias mensuales, cuando importare simple delito.




    ART. 491.

    El médico, cirujano, farmacéutico, flebotomiano o matrona que causare mal a las personas por negligencia culpable en el desempeño de su profesión, incurrirá respectivamente en las penas del artículo anterior.
    Iguales penas se aplicarán al dueño de animales feroces que, por descuido culpable de su parte, causaren daño a las personas.



    ART. 492.

    Las penas del artículo 490 se impondrán también respectivamente al que, con infracción de los reglamentos y por mera imprudencia o negligencia, ejecutare un hecho o incurriere en una omisión que, a mediar malicia, constituiría un crimen o un simple delito contra las personas.
    A los responsables de cuasidelito de homicidio o lesiones, ejecutados por medio de vehículos a tracción mecánica o animal, se los sancionará, además de las penas indicadas en el artículo 490, con la suspensión del carné, permiso o autorización que los habilite para conducir vehículos, por un período de uno a dos años, si el hecho de mediar malicia constituyera un crimen, y de seis meses a un año, si constituyera simple delito. En caso de reincidencia, podrá condenarse al conductor a inhabilidad perpetua para conducir vehículos a tracción mecánica o animal, cancelándose el carné, permiso o autorización.


    ART. 493.

    Las disposiciones del presente párrafo no se aplicarán a los cuasidelitos especialmente penados en este Código.



    LIBRO TERCERO.


    TÍTULO PRIMERO

    DE LAS FALTAS.


    ART. 494.

    Sufrirán la pena de multa de una a cuatro unidades tributarias mensuales:

    1.° El que asistiendo a un espectáculo público provocare algún desorden o tomare parte en él.
    2.° El que excitare o dirigiere cencerradas u otras reuniones tumultuosas en ofensa de alguna persona o del sosiego de las poblaciones.
    3.° El que ensuciare, arrojare o abandonare basura, materiales o desechos de cualquier índole en playas, riberas de ríos o de lagos, parques nacionales, reservas nacionales, monumentos naturales o en otras áreas de conservación de la biodiversidad declaradas bajo protección oficial.
    La pena consistirá en la prestación de servicios en beneficio de la comunidad consistente en la limpieza de playas, lagos o ríos. Esta pena se regulará conforme a lo dispuesto en los artículos 49, 49 bis, 49 ter, 49 quáter, 49 quinquies y 49 sexies, debiendo existir consentimiento previo del condenado. En caso de no haber consentimiento, se aplicará la pena de multa.
    4.° El que amenazare a otro con armas blancas y el que riñendo con otro las sacare, como no sea con motivo justo.
    5.° El que causare lesiones leves, entendiéndose por tales las que, en concepto del tribunal, no se hallaren comprendidas en el art. 399, atendidas la calidad de las personas y circunstancias del hecho. En ningún caso el tribunal podrá calificar como leves las lesiones cometidas en contra de las personas mencionadas en el artículo 5° de la Ley sobre Violencia Intrafamiliar, ni aquéllas cometidas en contra de las personas a que se refiere el inciso primero del artículo 403 bis de este Código.
    6.° El que corriere carruajes o caballerías con peligro de las personas, haciéndolo en poblado, ya sea de noche o de día cuando haya aglomeración de gente.
    7.° El farmacéutico que despachare medicamentos en virtud de receta que no se halle debidamente autorizada.
    8.° El que habitualmente y después de apercibimiento ejerciere, sin título legal ni permiso de autoridad competente, las profesiones de médico, cirujano, farmacéutico o Dentista.
    9.° El facultativo que, notando en una persona o en un cadáver señales de envenenamiento o de otro delito grave, no diere parte a la autoridad oportunamente.
    10.° El médico, cirujano, farmacéutico, Dentista o matrona que incurriere en descuido culpable en el desempeño de su profesión, sin causar daño a las personas.
    11.° Los mismos individuos expresados en el numero anterior, que no prestaren los servicios de su profesión durante el turno que les señale la autoridad administrativa.
    12.° El médico, cirujano, farmacéutico, matrona o cualquiera otro que, llamado en clase de perito o testigo, se negare a practicar una operación propia de su profesión u oficio o a prestar una declaración requerida por la autoridad judicial, en los casos y en la forma que determine el Código de Procedimiento y sin perjuicio de los apremios legales.
    13.° El que encontrando perdido o abandonado a un menor de siete años no lo entregare a su familia o no lo recogiere o depositare en lugar seguro, dando cuenta a la autoridad en los dos últimos casos.
    14.° El que no socorriere o auxiliare a una persona que encontrare en despoblado herida, maltratada o en peligro de perecer, cuando pudiere hacerlo sin detrimento propio.
    15.° Los padres de familia o los que legalmente hagan sus veces que abandonen a sus hijos, no procurándoles la educación que permiten y requieren su clase y facultades.
    16.° El que sin estar legítimamente autorizado impidiere a otro con violencia hacer lo que la ley no prohíbe, o le compeliere a ejecutar lo que no quiera.
    17.° El que quebrantare los reglamentos o disposiciones de la autoridad sobre la custodia, conservación y trasporte de materias inflamables o corrosivas o productos químicos que puedan causar estragos.
    18.° El dueño de animales feroces que en lugar accesible al público los dejare sueltos o en disposición de causar mal.
    Para estos efectos, se comprenderán como feroces los animales potencialmente peligrosos.
    19.  El que ejecutare alguno de los hechos penados en los artículos 189, 233, 448, 467, 469, 470 y 477, siempre que el delito se refiera a valores que no excedan de una  unidad tributaria mensual.
    20.° El que con violencia se apoderare de una cosa perteneciente a su deudor para hacerse pago con ella.
    21.° El que con violencia en las cosas entrare a cazar o pescar en lugar cerrado, o en lugar abierto contra expresa prohibición intimada personalmente.

    Con todo, tratándose de las faltas mencionadas en el número 19, la multa no será inferior al valor malversado o defraudado, al de la cosa hurtada o del daño causado, en su caso, y podrá alcanzar el doble de ese valor, aun cuando supere una unidad tributaria mensual.







    ART. 494 bis.

    Los autores de hurto serán castigados con prisión en su grado mínimo a medio y multa de una a cuatro unidades tributarias mensuales, si el valor de la cosa hurtada no pasa de media unidad tributaria mensual.
    La falta de que trata este artículo se castigará con multa de una a cuatro unidades tributarias mensuales, si se encuentra en grado de frustrada. En estos casos, el tribunal podrá conmutar la multa por la realización de trabajos determinados en beneficio de la comunidad, señalando expresamente el tipo de trabajo, el lugar donde deba realizarse, su duración y la persona o institución encargada de controlar su cumplimiento. Los trabajos se realizarán, de preferencia, sin afectar la jornada laboral o de estudio que tenga el infractor, con un máximo de ocho horas semanales. La no realización cabal y oportuna de los trabajos determinados por el tribunal dejará sin efecto la conmutación por el solo ministerio de la ley, y deberá cumplirse íntegramente la sanción primitivamente aplicada.
    En los casos en que participen en el hurto individuos mayores de dieciocho años y menores de esa edad, se aplicará a los mayores la pena que les habría correspondido sin esa circunstancia, aumentada en un grado, si éstos se han prevalido de los menores en la perpetración de la falta.
    En caso de reincidencia en hurto falta frustrado, se duplicará la multa aplicada. Se entenderá que hay reincidencia cuando el responsable haya sido condenado previamente por delito de la misma especie, cualquiera haya sido la pena impuesta y su estado de cumplimiento. Si el responsable ha reincidido dos o más veces se triplicará la multa aplicada.
    La agravante regulada en el inciso precedente prescribirá de conformidad con lo dispuesto en el artículo 104. Tratándose de faltas, el término de la prescripción será de seis meses.


    ART. 494 ter.

    Comete acoso sexual el que realizare, en lugares públicos o de libre acceso público, y sin mediar el consentimiento de la víctima, un acto de significación sexual capaz de provocar una situación objetivamente intimidatoria, hostil o humillante, y que no constituya una falta o delito al que se imponga una pena más grave, que consistiere en:
    1. Actos de carácter verbal o ejecutados por medio de gestos. En este caso se impondrá una multa de una a tres unidades tributarias mensuales.
    2. Conductas consistentes en acercamientos o persecuciones, o actos de exhibicionismo obsceno o de contenido sexual explícito. En cualquiera de estos casos se impondrá la pena de prisión en su grado medio a máximo y multa de cinco a diez unidades tributarias mensuales.



    ART. 495.

    Serán castigados con multa de una unidad tributaria mensual:

    1.° El que contraviniere a las reglas que la autoridad dictare para conservar el orden público o evitar que se altere, salvo que el hecho constituya crimen o simple delito.
    2.° El que por quebrantar los reglamentos sobre espectáculos públicos ocasionare algún desorden.
    3.° El subordinado del orden civil que faltare al respeto y sumisión debidos a sus jefes o superiores.
    4.° El particular que cometiere igual falta respecto de cualquier funcionario revestido de autoridad pública, mientras ejerce sus funciones, y respecto de toda persona constituida en dignidad, aun cuando no sea en el ejercicio de sus funciones, siempre que fuere conocida o se anunciare como tal; sin perjuicio de imponer, tanto en este caso como en el anterior, la pena correspondiente al crimen o simple delito, si lo hubiere.
    5.° El que públicamente ofendiere el pudor con acciones o dichos deshonestos.
    6.° El cónyuge que escandalizare con sus disensiones domésticas después de haber sido amonestado por la autoridad.
    7.° El que infringiere los reglamentos de policía en lo concerniente a quienes ejercen el comercio sexual.
    8.° El que diere espectáculos públicos sin licencia de la autoridad, o traspasando la que se le hubiere concedido.
    9.° El que abriere establecimientos sin licencia de la autoridad, cuando sea necesaria.
    10.° El que en la exposición de niños quebrantare los reglamentos.
    11.° El que infringiere las reglas establecidas para la quema de bosques, rastrojos u otros productos de la tierra, o para evitar la propagación de fuego en máquinas de vapor, caleras, hornos u otros lugares semejantes.
    12.° El que infringiere los reglamentos sobre corta de bosques o arbolados.
    13.° El que infringiere las leyes o reglamentos sobre apertura, conservación y reparación de vías públicas.
    14.° El que en caminos públicos, calles, plazas, ferias u otros sitios semejantes de reunión estableciere rifas u otros juegos de envite o azar.
    15.° El que defraudare al público en la venta de mantenimientos, ya sea en calidad, ya en cantidad, por valor que no exceda de una unidad tributaria mensual, y el que vendiere bebidas o mantenimientos deteriorados o nocivos.
    16.° El traficante que tuviere medidas o pesos falsos, aunque con ellos no hubiere defraudado.
    17.° El que usare en su tráfico medidas o pesos no contrastados.
    18.° El dueño o encargado de fondas, cafés, confiterías u otros establecimientos destinados al despacho de comestibles o bebidas que faltare a los reglamentos de policía relativos a la conservación o uso de vasijas o útiles destinados para el servicio.
    19.° El que faltando a las órdenes de la autoridad, descuidare reparar o demoler edificios ruinosos.
    20.° El que infringiere las reglas de seguridad concernientes a la apertura de pozos o excavaciones y al depósito de materiales o escombros, o a la colocación de cualesquiera otros objetos en las calles, plazas, paseos públicos o en la parte exterior de los edificios que embaracen el tráfico o puedan causar daño a los transeúntes.
    21.° El que intencionalmente o con negligencia culpable causare daño que no exceda de una unidad tributaria mensual en bienes públicos o de propiedad particular.
    22.° El que aprovechando aguas de otro o distrayéndolas de su curso, causare daño que no exceda de una unidad tributaria mensual.

    Con todo, la multa para las faltas señaladas en los números 15, 21 y 22 será a lo menos equivalente al valor de lo defraudado o del daño causado y podrá llegar hasta el doble de ese valor, aunque exceda una unidad tributaria mensual.



    ART. 496.

    Sufrirán la pena de multa de una a cuatro unidades tributarias mensuales:

    1.° El que faltare a la obediencia debida a la autoridad dejando de cumplir las órdenes particulares que ésta le diere, en todos aquellos casos en que la desobediencia no tenga señalada mayor pena por este Código o por leyes especiales.
    2.° El que pudiendo, sin grave detrimento propio, prestar a la autoridad el auxilio que reclamare en casos de incendio, inundación, naufragio u otra calamidad, se negare a ello.
    3.° El que impidiere el ejercicio de las funciones fiscalizadoras de los inspectores municipales.
    4.° El que no diere los partes de defunción, contraviniendo a la ley o reglamentos.
    5.° El que ocultare su verdadero nombre y apellido a la autoridad o a persona que tenga derecho para exigir que los manifieste o se negare a manifestarlos o diere domicilio falso.
    6.° El que infringiere las reglas de policía dirigidas a asegurar el abastecimiento de los pueblos.
    7.° El que con rondas u otros esparcimientos nocturnos altere el sosiego público, desobedeciendo a la autoridad.
    8.° El que tomare parte en cencerradas u otras reuniones ofensivas a alguna persona, no estando comprendida en el núm. 2.º del art. 494.
    9.° El que se bañare quebrantando las reglas de decencia o seguridad establecidas por la autoridad.
    10.° El que riñere en público sin armas, salvo el caso de justa defensa propia o de un tercero.
    11.° El que injuriare a otro livianamente de obra o de palabra, no siendo por escrito y con publicidad.
    12.° Derogado.
    13.° El que corriere carruajes o caballerías dentro de una población, no siendo en los casos previstos por el núm. 6.° del art. 494.
    14.° El que infringiere los reglamentos relativos a carruajes públicos o de particulares.
    15.° El que infringiere las reglas de policía relativas a posadas, fondas, tabernas y otros establecimientos públicos.
    16.° El encargado de la guarda de un loco o demente que le dejare vagar por sitios públicos sin la debida seguridad.
    17.° El dueño de animales dañinos que los dejare sueltos o en disposición de causar mal en las poblaciones.
    18.° El que con su embriaguez molestare a tercero en público.
    19.° El que arrojare animales muertos en sitios vedados o quebrantando las reglas de policía.
    20.° El que infringiere las reglas de policía en la elaboración de objetos fétidos o insalubres, o los arrojare a las calles, plazas o paseos públicos.
    21.° El que arrojare escombros u objetos punzantes o cortantes en lugares públicos, contraviniendo a las reglas de policía.
    22.° El que no entregare a la policía de aseo las basuras o desperdicios que hubiere en el interior de su habitación.
    23.° El que echare en las acequias de las poblaciones objetos que, impidiendo el libre y fácil curso de las aguas, puedan ocasionar anegación.
    24.° El que tuviere en balcones, ventanas, azoteas u otros puntos exteriores de sus casas tiestos u otros objetos, con infracción de las reglas de policía.
    25.° El que arrojare a la calle por balcones, ventanas o por cualquiera otra parte agua u objetos que puedan causar daño.
    26.° El que tirare piedras u otros objetos arrojadizos en parajes públicos, con riesgo de los transeúntes, o lo hiciere a las casas o edificios, en perjuicio de los mismos o con peligro de las personas.
    27.° El que infringiere los reglamentos en materia de juegos o diversiones dentro de las poblaciones.
    28.° El que entrare con carruajes, caballerías o animales dañinos en heredades plantadas o sembradas.
    29.° El que en contravención a los reglamentos construyere chimeneas, estufas u hornos, o dejare de limpiarlos o cuidarlos.
    30.° El que, empleando el fuego, elevare globos sin permiso de la autoridad.
    31.° El que, habiendo recibido de buena fe moneda falsa o cercenada o títulos de crédito falsos, los circulare después de constarle su falsedad o cercenamiento, siempre que su valor no exceda de una unidad tributaria mensual.
    32.° Derogado.
    33.° El que entrare en heredad ajena para coger frutas y comerlas en el acto.
    34.° El que entrare sin violencia a cazar o pescar en sitio vedado o cerrado.
    35.° Derogado.
    36.° El que infringiere los reglamentos de caza o pesca en el modo y tiempo de ejecutar una u otra o de vender sus productos.
    37.° Los empresarios del alumbrado público que faltaren a las reglas establecidas para su servicio, y los particulares que infringieren dichas reglas.
    38.° El que indebidamente apagare el alumbrado público o del exterior de los edificios, o de los portales, teatros, u otros lugares de espectáculo o reunión, o el de las escaleras de los mismos.


    ART. 497.

    El dueño de ganados que entraren en heredad ajena cerrada y causaren daño, será castigado con multa, por cada cabeza de ganado:

    1.° De una unidad tributaria mensual, si fuere vacuno, caballar, mular o asnal.
    2.° De  un quinto de unidad tributaria mensual, si fuere lanar o cabrío y la heredad tuviere arbolado.
    4.° Del tanto del daño causado a un tercio mas, si fuere de otra especie no comprendida en los números anteriores.
    Esto mismo se observará si el ganado fuere lanar o cabrío y la heredad no tuviere arbolado.









NOTA
      El numeral 21° de la ley 11183, publicada el 10.06.1953, modifica el presente artículo en el sentido de: sustituir su numeral 1° tal como se transcribe; suprimiendo su numeral 2° original. El numeral 3°pasa a ser 2°, sustituyéndose parte de su contenido por la frase transcrita, pero sin alterar la numeración respecto de su actual numeral 4°. Por tal razón, en la construcción de su texto actualizado se mantiene su numeración sin tener un orden correlativo.

    TÍTULO SEGUNDO

    DISPOSICIONES COMUNES A LAS FALTAS.





    ART. 498.

    Los cómplices en las faltas serán castigados con una pena que no exceda de la mitad de la que corresponda a los autores.


    ART.499.

    Caerán en comiso:
    1.° Las armas que llevare el ofensor al hacer un daño o inferir injuria, si las hubiere mostrado.
    2.° Las bebidas y comestibles deteriorados y nocivos.
    3.° Los efectos falsificados, adulterados o averiados que se expendieren como legítimos o buenos.
    4.º Los comestibles en que se defraudare al público en cantidad o calidad.
    5.° Las medidas o pesos falsos.
    6.° Los enseres que sirvan para juegos o rifas.
    7.° Los efectos que se empleen para adivinaciones u otros engaños semejantes.


    ART. 500.

    El comiso de los instrumentos y efectos de las faltas, expresados en el artículo anterior, lo decretará el tribunal a su prudente arbitrio según los casos y circunstancias.



    ART. 501.

    En las ordenanzas municipales y en los reglamentos generales o particulares que dictare en lo sucesivo la autoridad administrativa no se establecerán mayores penas que las señaladas en este libro, aun cuando hayan de imponerse en virtud de atribuciones gubernativas, a no ser que se determine otra cosa por leyes especiales.


    TÍTULO FINAL.

    DE LA OBSERVANCIA DE ESTE CÓDIGO.





    ARTÍCULO FINAL.

    El presente Código comenzará a regir el primero de marzo de mil ochocientos setenta y cinco, y en esa fecha quedarán derogadas las leyes y demás disposiciones preexistentes sobre todas las materias que en él se tratan.


    Y por cuanto, oído el Consejo de Estado, he tenido a bien aprobarlo y sancionarlo; por tanto promúlguese y llévese a efecto en todas sus partes como ley de la República.

    FEDERICO ERRÁZURIZ.

    JOSÉ MARÍA BARCELÓ.




















    















    § VIII.

    Disposiciones comunes a los dos párrafos anteriores.








    ART. 421.

    Se comete el delito de calumnia o injuria no sólo manifiestamente, sino por medio de alegorías, caricaturas, emblemas o alusiones.


    ART. 422.

    La calumnia y la injuria se reputan hechas por escrito y con publicidad cuando se propagaren por medio de carteles o pasquines fijados en los sitios públicos; por papeles impresos, no sujetos a la ley de imprenta, litografías, grabados o manuscritos comunicados a más de cinco personas, o por alegorías, caricaturas, emblemas o alusiones reproducidos por medio de la litografía, el grabado, la fotografía u otro procedimiento cualquiera.



    ART. 423.

    El acusado de calumnia o injuria encubierta o equívoca que rehusare dar en juicio explicaciones satisfactorias acerca de ella, será castigado con las penas de los delitos de calumnia o injuria manifiesta.

    ART. 424. Derogado.

    ART. 425.

    Respecto de las calumnias o injurias publicadas por medio de periódicos extranjeros, podrán ser acusados los que, desde el territorio de la República, hubieren enviado los artículos o dado orden para su inserción, o contribuido a la introducción o expendición de esos periódicos en Chile con ánimo manifiesto de propagar la calumnia o injuria.
    ART. 426.

    La calumnia o injuria causada en juicio se juzgará disciplinariamente por el tribunal que conoce de la causa; sin perjuicio del derecho del ofendido para deducir, una vez que el proceso haya concluido, la acción penal correspondiente.


    ART. 427

    Las expresiones que puedan estimarse calumniosas o injuriosas, consignadas en un documento oficial, no destinado a la publicidad, sobre asuntos del servicio público, no dan derecho para acusar criminalmente al que las consignó.


    ART. 428.

    El condenado por calumnia o injuria puede ser relevado de la pena impuesta mediante perdón del acusador; pero la remisión no producirá efecto respecto de la multa una vez que ésta haya sido satisfecha.

    La calumnia o injuria se entenderá tácitamente remitida cuando hubieren mediado actos positivos que, en concepto del tribunal, importen reconciliación o abandono de la acción.

    ART. 429. Derogado.

    ART. 430.

    En el caso de calumnias o injurias recíprocas, se observarán las reglas siguientes:

    1.° Si las más graves de las calumnias o injurias recíprocamente inferidas merecieren igual pena, el tribunal las dará todas por compensadas.
    2.° Cuando la más grave de las calumnias o injurias imputadas por una de las partes, tuviere señalado mayor castigo que la más grave de las imputadas por la otra, al imponer la pena correspondiente a aquélla se rebajará la asignada para ésta.




    Art. 431.

    La acción de calumnia o injuria prescribe en un año, contado desde que el ofendido tuvo o pudo racionalmente tener conocimiento de la ofensa.

    La misma regla se observará respecto de las demás personas enumeradas en el artículo 108 del Código Procesal Penal; pero el tiempo trascurrido desde que el ofendido tuvo o pudo tener conocimiento de la ofensa hasta su muerte, se tomará en cuenta al computarse el año durante el cual pueden ejercitar esta acción las personas comprendidas en dicho artículo.

    No podrá entablarse acción de calumnia o injuria después de cinco años, contados desde que se cometió el delito. Pero si la calumnia o injuria hubiere sido causada en juicio, este plazo no obstará al cómputo del año durante el cual se podrá ejercer la acción.


    TÍTULO NOVENO.

    CRÍMENES Y SIMPLES DELITOS CONTRA LA PROPIEDAD.








    § I.

    De la apropiación de las cosas muebles ajenas contra la voluntad de su dueño.





    ART. 432.

    El que sin la voluntad de su dueño y con ánimo de lucrarse se apropia cosa mueble ajena usando de violencia o intimidación en las personas o de fuerza en las cosas, comete robo; si faltan la violencia, la intimidación y la fuerza, el delito se califica de hurto.
   


NOTA
      El N° 7 de la ley 11183, publicada el 10.06.1956, modifica el presente artículo, en el sentido de derogar su inciso segundo.

    § II.

    Del robo con violencia o intimidación en las personas.




    ART. 433.

    El culpable de robo con violencia o intimidación en las personas, sea que la violencia o la intimidación tenga lugar antes del robo para facilitar su ejecución, en el acto de cometerlo o después de cometido para favorecer su impunidad, será castigado:
    1°. Con presidio mayor en su grado máximo a presidio perpetuo calificado cuando, con motivo u ocasión del robo, se cometiere, además, homicidio o violación.
    2°. Con presidio mayor en su grado máximo a presidio perpetuo cuando, con motivo u ocasión del robo, se cometiere alguna de las lesiones comprendidas en los artículos 395, 396 y 397, número 1°.
    3°. Con presidio mayor en su grado medio a máximo cuando se cometieren lesiones de las que trata el número 2° del artículo 397 o cuando las víctimas fueren retenidas bajo rescate o por un lapso mayor a aquel que resulte necesario para la comisión del delito.

    ART. 434.

    Los que cometieren actos de piratería serán castigados con la pena de presidio mayor en su grado mínimo a presidio perpetuo.





    ART. 435.    Derogado.

    ART. 436.

    Fuera de los casos previstos en los artículos precedentes, los robos ejecutados con violencia o intimidación en las personas, serán penados con presidio mayor en sus grados mínimo a máximo, cualquiera que sea el valor de las especies sustraídas.
    Se considerará como robo y se castigará con la pena de presidio menor en sus grados medio a máximo, la apropiación de dinero u otras especies que los ofendidos lleven consigo, cuando se proceda por sorpresa o aparentando riñas en lugares de concurrencia o haciendo otras maniobras dirigidas a causar agolpamiento o confusión.
    También será considerado robo, y se sancionará con la pena de presidio menor en su grado máximo, la apropiación de vehículos motorizados, siempre que se valga de la sorpresa, de la distracción de la víctima o se genere por parte del autor cualquier maniobra distractora cuyo objeto sea que la víctima abandone el vehículo para facilitar su apropiación, en ambos casos, en el momento en que ésta se apreste a ingresar o hacer abandono de un lugar habitado, destinado a la habitación o sus dependencias, o su lugar de trabajo, salvo en aquellos casos en que medie violencia o intimidación, en los que se aplicará lo dispuesto en el inciso primero.


    ART. 437.    Derogado.


    ART. 438.

    El que para obtener un provecho patrimonial para sí o para un tercero constriña a otro con violencia o intimidación a suscribir, otorgar o entregar un instrumento público o privado que importe una obligación estimable en dinero, o a ejecutar, omitir o tolerar cualquier otra acción que importe una disposición patrimonial en perjuicio suyo o de un tercero, será castigado con las penas respectivamente señaladas en este párrafo para el culpable de robo.

    ART. 439.

    Para los efectos del presente párrafo se estimarán por violencia o intimidación en las personas los malos tratamientos de obra, las amenazas ya para hacer que se entreguen o manifiesten las cosas, ya para impedir la resistencia u oposición a que se quiten, o cualquier otro acto que pueda intimidar o forzar a la manifestación o entrega. Hará también violencia el que para obtener la entrega o manifestación alegare orden falsa de alguna autoridad, o la diere por sí fingiéndose ministro de justicia o funcionario público. Por su parte, hará también intimidación el que para apropiarse u obtener la entrega o manifestación de un vehículo motorizado o de las cosas ubicadas dentro del mismo, fracture sus vidrios, encontrándose personas en su interior; o amenace la integridad de niños que se encuentren al interior del vehículo, sin perjuicio de la prueba que se pudiere presentar en contrario.



    § III.

    Del robo con fuerza en las cosas.


    ART. 440.

    El culpable de robo con fuerza en las cosas efectuado en lugar habitado o destinado a la habitación o en sus dependencias, sufrirá la pena de presidio mayor en su grado mínimo si cometiere el delito:

    1.º Con escalamiento, entendiéndose que lo hay cuando se entra por vía no destinada al efecto, por forado o con rompimiento de pared o techos, o fractura de puertas o ventanas.
    2.º Haciendo uso de llaves falsas, o verdadera que hubiere sido sustraída, de ganzúas u otros instrumentos semejantes para entrar en el lugar del robo.
    3.º A Introduciéndose en el lugar del robo mediante la seducción de algún doméstico, o a favor de nombres supuestos o simulación de autoridad.
    4.° Eliminado.

    ART. 441.    Derogado.


    ART. 442.

    El robo en lugar no habitado, se castigará con presidio menor en sus grados medio a máximo, siempre que concurra alguna de las circunstancias siguientes:
    1.º A Escalamiento.
    2.º Fractura de puertas interiores, armarios, arcas u otra clase de muebles u objetos cerrados o sellados.
    3.º Haber hecho uso de llaves falsas, o verdadera que se hubiere sustraído, de ganzúas u otros instrumentos semejantes para entrar en el lugar del robo o abrir los muebles cerrados.


   
    ART. 443.

    Con la misma pena señalada en el artículo anterior se castigará el robo de cosas que se encuentren en bienes nacionales de uso público, en sitio no destinado a la habitación o en el interior de vehículos motorizados, si el autor hace uso de llaves falsas o verdaderas que se hayan substraído, de ganzúas u otros instrumentos semejantes o si se procede, mediante fractura de puertas, vidrios, cierros, candados u otros dispositivos de protección o si se utilizan medios de tracción.
    Si el delito a que se refiere el inciso precedente recayere sobre un vehículo motorizado, se impondrá la pena de presidio menor en su grado máximo.
    Se considerará robo y se castigará con la pena del inciso precedente la apropiación de un vehículo motorizado mediante la generación de cualquier maniobra distractora cuyo objeto sea que la víctima abandone el vehículo, fuera de los casos a los que se refiere el artículo 436.
    Si con ocasión de alguna de las conductas señaladas en el inciso primero, se produce la interrupción o interferencia del suministro de un servicio público o domiciliario, tales como electricidad, gas, agua, alcantarillado, colectores de aguas lluvia o telefonía, la pena se aplicará en su grado máximo.


    ART. 443 bis.

    El robo con fuerza de cajeros automáticos, dispensadores o contenedores de dinero, o del dinero y valores contenidos en ellos, será sancionado con la pena de presidio menor en su grado máximo. Para los efectos del presente artículo se entenderá que hay fuerza en las cosas si se ha procedido con alguno de los medios señalados en el artículo 440, Nos 1° y 2°; si se ha fracturado, destruido o dañado el cajero automático o dispensador o sus dispositivos de protección o sujeción mediante el uso de instrumentos contundentes o cortantes de cualquier tipo, incluyendo el empleo de medios químicos; o si se utilizan medios de tracción.

    ART. 444.

    Se presume autor de tentativa de robo al que se introdujere con forado, fractura, escalamiento, uso de llave falsa o de llave verdadera sustraída o de ganzúa en algún aposento, casa, edificio habitado o destinado a la habitación o en sus dependencias.


    ART. 445.

    El que fabricare, expendiere o tuviere en su poder llaves falsas, ganzúas u otros instrumentos destinados conocidamente para efectuar el delito de robo y no diere descargo suficiente sobre su fabricación, expendición, adquisición o conservación, será castigado con presidio menor en su grado mínimo.


    § IV.

    Del hurto.


    Artículo 446.- Los autores de hurto serán castigados:
    1.º Con presidio menor en sus grados medio a máximo y multa de once a quince unidades tributarias mensuales, si el valor de la cosa hurtada excediera de cuarenta unidades tributarias mensuales.
    2.º Con presidio menor en su grado medio y multa de seis a diez unidades tributarias mensuales, si el valor excediere de cuatro unidades tributarias mensuales y no pasare de cuarenta unidades tributarias mensuales.
    3.º Con presidio menor en su grado mínimo y multa de cinco unidades tributarias mensuales, si excediere de media unidad tributaria mensual y no pasare de cuatro unidades tributarias mensules.
    Si el valor de la cosa hurtada excediere de cuatrocientas unidades tributarias mensuales, se aplicará la pena de presidio menor en su grado máximo y multa de veintiuna a treinta unidades tributarias mensuales.


    ART. 447.

    En los casos del artículo anterior podrá aplicarse la pena inmediatamente superior en grado:
   
    1.° Si el hurto se cometiere por dependiente, criado o sirviente asalariado, bien sea en la casa en que sirve o bien en aquella a que lo hubiere llevado su amo o patrón.
    2.º Cuando se cometiere por obrero, oficial o aprendiz en la casa, taller o almacén de su maestro o de la persona para quien trabaja, o por individuo que trabaja habitualmente en la casa donde hubiere hurtado.
    3.° Si se cometiere por el posadero, fondista u otra persona que hospede gentes en cosas que hubieren llevado a la posada o fonda.
    4.° Cuando se cometiere por patrón o comandante de buque, lanchero, conductor o bodeguero de tren, guarda almacenes, carruajero, carretero o arriero en cosas que se hayan puesto en su buque, carro, bodega, etc.



    ART. 447 bis.-

    El hurto de cosas que forman parte de redes de suministro de servicios públicos o domiciliarios, tales como electricidad, gas, agua, alcantarillado, colectores de aguas lluvia o telefonía, será castigado con presidio menor en sus grados medio a máximo.
    Si con ocasión de alguna de las conductas señaladas en este artículo se produce la interrupción o interferencia del servicio, la pena se aplicará en su grado máximo.


    ART. 448.

    El que, hallándose una especie mueble, al parecer perdida, cuyo valor exceda de una unidad tributaria mensual, no la entregare a la autoridad o a su dueño, siempre que le conste quién sea éste, por hechos coexistentes posteriores al hallazgo, será castigado con presidio menor en su grado mínimo y multa de cinco unidades tributarias mensuales.

    También será castigado con presidio menor en su grado mínimo y multa de cinco unidades tributarias mensuales, el que hallare especies, al parecer perdidas o abandonadas, a consecuencia de naufragio, inundación, incendio, terremoto, accidente en ferrocarril u otra causa análoga, cuyo valor exceda la cantidad mencionada en el inciso anterior, y no las entregare a los dueños o a la autoridad en su defecto.
    § IV bis. Del Abigeato



    ART. 448 bis.
   
    El que robe o hurte uno o más caballos o bestias de silla o carga, o especies de ganado mayor, menor o colmenas, comete abigeato y será castigado con las penas señaladas en los Párrafos 2, 3 y 4.
    Asimismo, se considerará autor del delito de abigeato al que sin el consentimiento de quienes pueden disponer del ganado:

    1°. Altere o elimine marcas o señales en animales ajenos.

    2°. Marque, señale, contramarque o contraseñale animales ajenos.

    3°. Expida o porte certificados falsos para obtener guías o formularios o haga conducir animales ajenos sin estar debidamente autorizado.


    ART. 448 ter.
    Una vez determinada la pena que correspondería a los autores, cómplices y encubridores de abigeato sin el requisito de tratarse de la substracción de animales y considerando las circunstancias modificatorias de responsabilidad penal concurrentes, el juez deberá aumentarla en un grado y aplicará, en todo caso, la pena de comiso en los términos del artículo 31 de este Código.

    Cuando las especies substraídas tengan un valor que exceda las cinco unidades tributarias mensuales, se aplicará, además, la accesoria de multa de setenta y cinco a cien unidades tributarias mensuales.

    Si la pena consta de dos o más grados, el aumento establecido en el inciso primero se hará después de determinar la pena que habría correspondido al imputado, con prescindencia del requisito de tratarse de la substracción de animales.

    Será castigado como autor de abigeato el que beneficie o destruya una especie para apropiarse de toda ella o de alguna de sus partes.

    La regla del inciso primero de este artículo se observará también en los casos previstos en el artículo 448, si se trata de animales comprendidos en el artículo anterior.

    ART. 448 quáter.

    Se castigará como autor de abigeato a aquel en cuyo poder se encuentren animales o partes de los mismos referidos en este Párrafo, cuando no pueda justificar su adquisición o legítima tenencia y, del mismo modo, al que sea habido en predio ajeno, arreando, transportando, manteniendo cautivas, inmovilizadas o maniatadas dichas especies animales. El porte de armas, herramientas o utensilios comúnmente empleados para el faenamiento de animales por quien no diere descargo suficiente de su tenencia, se castigará de conformidad a lo establecido en el artículo 445.

    Las marcas registradas, señales conocidas, dispositivos de identificación individual oficial registrados ante el Servicio Agrícola y Ganadero u otras de carácter electrónico o tecnológico puestas sobre el animal, constituyen presunción de dominio a favor del dueño de la marca o señal.

    Para los efectos previstos en el inciso primero, en los casos de traslado de animales o de partes de los mismos, realizado en vehículos de transporte de carga, Carabineros de Chile deberá exigir, además del formulario de movimiento animal, la boleta, factura o guía de despacho correspondiente, a efectos de acreditar el dominio, posesión o legítima tenencia de las especies. Ante la imposibilidad de acreditar dicho dominio, posesión o legítima tenencia, según corresponda, por carecer de los mencionados documentos o por negarse a su exhibición, los funcionarios policiales se incautarán de las especies, sus partes y del medio de transporte, dando aviso a la fiscalía correspondiente para el inicio de la investigación que proceda, al Servicio de Impuestos Internos ante un eventual delito tributario, a la autoridad sanitaria competente para que instruya sumario sanitario y al Servicio Agrícola y Ganadero para determinar la eventual existencia de infracciones a la normativa agropecuaria.
   
    Ante la sospecha o la comisión de los delitos a que se refiere este párrafo, el Ministerio Público podrá, en lo pertinente, autorizar la correspondiente investigación bajo la técnica de entrega vigilada en los términos regulados en el Párrafo 3° bis del Título I del Libro II del Código Procesal Penal.

    ART. 448 quinquies.
    El que se apropie de las plumas, pelos, crines, cerdas, lanas o cualquier elemento del pelaje de animales ajenos, por cualquier medio que ello se realice, será castigado con presidio menor en sus grados mínimo a medio.

    ART. 448 sexies.

    Los vehículos motorizados o de otra clase, las herramientas y los instrumentos utilizados en la comisión del delito de abigeato, caerán en comiso.

    Durante el curso del procedimiento dichos bienes serán incautados de conformidad a las reglas generales, sin perjuicio del derecho establecido en el artículo 189 del Código Procesal Penal.

    § IV ter.
   
    De la sustracción de madera


     
    ART. 448 septies.

    El que robe o hurte troncos o trozas de madera comete el delito de sustracción de madera y será sancionado con las penas señaladas en los Párrafos II, III y IV del presente Título. Cuando la madera sustraída tenga un valor que exceda las 10 unidades tributarias mensuales se aplicará además la accesoria de multa de 75 a 100 unidades tributarias mensuales.
    Si la madera sustraída tiene un valor superior a las 50 unidades tributarias mensuales o si la sustracción obedece a un proceder sistemático u organizado, se podrán aplicar las técnicas especiales de investigación previstas en el artículo 226 bis del Código Procesal Penal.
    Los vehículos motorizados o de otra clase, las herramientas y los instrumentos utilizados en la comisión del delito, caerán en comiso.
     
    ART. 448 octies.

    Se castigará como autor de sustracción de madera, con las penas previstas en el artículo 446, a quien en cuyo poder se encuentren troncos o trozas de madera, cuando no pueda justificar su adquisición, su legítima tenencia o su labor en dichas faenas o actividades conexas destinadas a la tala de árboles y, del mismo modo, al que sea habido en predio ajeno, en idénticas faenas o actividades, sin consentimiento de su propietario ni autorización de tala.
    Asimismo, será sancionado con la pena de presidio menor en sus grados medio a máximo quien falsifique o maliciosamente haga uso de documentos falsos para obtener guías o formularios con miras a trasladar o comercializar madera de manera ilícita.
    § V.

    Disposiciones comunes a los cuatro Párrafos anteriores.





   
    ART. 449.

    Para determinar la pena de los delitos comprendidos en los Párrafos 1 a 4 ter, con excepción de aquellos contemplados en los artículos 448, inciso primero, y 448 quinquies, y del artículo 456 bis A, no se considerará lo establecido en los artículos 65 a 69 y se aplicarán las reglas que a continuación se señalan:
    1ª. Dentro del límite del grado o grados señalados por la ley como pena al delito, el tribunal determinará la cuantía de la pena en atención al número y entidad de las circunstancias atenuantes y agravantes concurrentes, así como a la mayor o menor extensión del mal causado, fundamentándolo en su sentencia.
    2ª. Tratándose de condenados reincidentes en los términos de las circunstancias agravantes de los numerales 15 y 16 del artículo 12, el tribunal deberá, para los efectos de lo señalado en la regla anterior, excluir el grado mínimo de la pena si ésta es compuesta, o el mínimum si consta de un solo grado.


    ART. 449 bis.-

    Será circunstancia agravante de los delitos contemplados en los Párrafos 1, 2, 3, 4, 4 bis y 4 ter de este Título, y del descrito en el artículo 456 bis A, el hecho de que el imputado haya actuado formando parte de una agrupación u organización de dos o más personas destinada a cometer dichos hechos punibles, siempre que ésta o aquélla no constituyere una asociación ilícita de que trata el Párrafo 10 del Título VI del Libro Segundo.





    ART. 449 ter.

    Cuando los delitos sancionados en los Párrafos 3 y 4 de este Título se perpetraren con ocasión de calamidad pública o alteración del orden público, sea que se actúe en grupo o individualmente pero amparado en este, se aumentará la pena privativa de libertad respectiva en un grado.
    Tratándose de la conducta sancionada en el inciso primero del artículo 436, y concurriendo las circunstancias descritas en el inciso anterior, se aplicará la pena privativa de libertad respectiva, con exclusión de su grado mínimo.




    ART. 449 quáter.

    Se aplicará en todo caso la regla 2ª del artículo 449, aun cuando el responsable no sea reincidente, si los delitos señalados en dicho artículo se cometen en circunstancias tales que contribuyan a la sustracción o destrucción de todo o la mayor parte de aquello que había o se guardaba en algún establecimiento de comercio o industrial o del propio establecimiento. En estos casos el hecho se denominará saqueo.
    Si el responsable fuere reincidente en los términos de las circunstancias agravantes de los numerales 15 y 16 del artículo 12, el juez podrá considerar suficiente fundamento esta circunstancia para la imposición del máximo de la pena resultante. 



    ART. 450.

    Los delitos a que se refiere al Párrafo 2 y el artículo 440 del Párrafo 3 de este Título se castigarán como consumados desde que se encuentren en grado de tentativa.
    La misma regla se aplicará a los delitos sancionados en los Párrafos 3, 4, 4 bis y 4 ter de este Título cuando se cometieren con las circunstancias señaladas en el inciso primero de los artículos 449 ter o 449 quáter.
   




    ART. 450 bis.
    En el robo con violencia o intimidación en las personas no procederá la atenuante de responsabilidad penal contenida en el artículo 11, N° 7.

    Artículo 451.- En los casos de reiteración de hurtos, aunque se trate de faltas, a una misma persona, o a distintas personas en una misma casa, establecimiento de comercio, centro comercial, feria, recinto o lugar el tribunal calificará el ilícito y hará la regulación de la pena tomando por base el importe total de los objetos sustraídos, y la impondrá al delincuente en su grado superior.
    Esta regla es sin perjuicio de lo dispuesto en el art. 447.


    ART. 452.

    El que después de haber sido condenado por robo o hurto cometiere cualquiera de estos delitos, además de las penas que le correspondan por el hecho o hechos en que hubiere reincidido, el tribunal podrá imponerle la de sujeción a la vigilancia de la autoridad dentro de los límites fijados en el art. 25.



    ART. 453.

    Cuando se reunieren en un hecho varias de las circunstancias a que se señala pena diversa según los párrafos precedentes, se aplicará la de las circunstancias que en aquel caso particular la merezcan más grave, pudiendo el tribunal aumentarla en un grado.


    ART. 454.
    Se presumirá autor del robo o hurto de una cosa aquel en cuyo poder se encuentre, salvo que justifique su legítima adquisición o que la prueba de su irreprochable conducta anterior establezca una presunción en contrario.


    ART. 455.

    Cuando del proceso no resulte probado el valor de la cosa sustraída ni pudiere estimarse por peritos u otro arbitrio legal, el tribunal hará su regulación prudencialmente.



    ART. 455 bis.

    Si en el momento de producirse el robo o hurto de un vehículo motorizado, se encontrare en su interior un infante o una persona que no pudiere abandonar el vehículo por sus propios medios, y el autor del robo o hurto inicia la conducción del mismo, se aplicará la pena de presidio mayor en sus grados medio a máximo.



    ART. 456.

    Si antes de perseguir al responsable o antes de decretar su prisión devolviere voluntariamente la cosa robada o hurtada, no hallándose comprendido en los casos de los arts. 433 y 434, se le aplicará la pena inmediatamente inferior en grado a la señalada para el delito.
    ART. 456 BIS.

    En los delitos de robo y hurto serán circunstancias agravantes las siguientes:
    1°) Ejecutar el delito en sitios faltos de vigilancia policial, obscuros, solitarios, sin tránsito habitual o que por cualquiera otra condición favorezcan la impunidad.
    2°) Ser la víctima niño, anciano, inválido o persona en manifiesto estado de inferioridad física;
    3°) Ejecutar el delito usando un vehículo motorizado sin placa patente delantera, trasera o ambas; o con cualquiera oculta o con vidrios oscuros o polarizados, en contravención a la ley N° 18.290, de Tránsito; o en el que se haya utilizado cualquier otra práctica, técnica, intervención, herramienta, dispositivo o condición que favorezca su impunidad;
    4°) Ejercer la violencia en las personas que intervengan en defensa de la víctima, salvo que este hecho importe otro delito; y
    5°) Actuar con personas exentas de responsabilidad criminal, según el número 1.o del artículo 10.
    Las circunstancias agravantes de los números 1.o y 5° del artículo 12 serán aplicables en los casos en que se ejerciere violencia sobre las personas.
    En estos delitos no podrá estimarse que concurre la circunstancia atenuante del número 7° del artículo 11, por la mera restitución a la víctima de las especies robadas o hurtadas y, en todo caso, el Juez deberá considerar, especificada, la justificación del celo con que el delincuente ha obrado.



    § 5 bis. De la receptación

    ART. 456 bis A.

    El que conociendo su origen o no pudiendo menos que conocerlo, tenga en su poder, a cualquier título, especies hurtadas, robadas u objeto de abigeato o sustracción de madera, de receptación o de apropiación indebida del artículo 470, número 1°, las transporte, compre, venda, transforme o comercialice en cualquier forma, aun cuando ya hubiese dispuesto de ellas, sufrirá la pena de presidio menor en cualquiera de sus grados y multa de cinco a cien unidades tributarias mensuales.
    Para la determinación de la pena aplicable el tribunal tendrá especialmente en cuenta el valor de las especies, así como la gravedad del delito en que se obtuvieron, si éste era conocido por el autor.
    Cuando el objeto de la receptación sean vehículos motorizados o cosas que forman parte de redes de suministro de servicios públicos o domiciliarios, tales como electricidad, gas, agua, alcantarillado, colectores de aguas lluvia o telefonía, se impondrá la pena de presidio menor en su grado máximo y multa equivalente al valor de la tasación fiscal del vehículo o la pena de presidio menor en su grado máximo, y multa de cinco a veinte unidades tributarias mensuales, respectivamente. La sentencia condenatoria por delitos de este inciso dispondrá el comiso de los instrumentos, herramientas o medios empleados para cometerlos o para transformar o transportar los elementos sustraídos. Si dichos elementos son almacenados, ocultados o transformados en algún establecimiento de comercio con conocimiento del dueño o administrador, se podrá decretar, además, la clausura definitiva de dicho establecimiento, oficiándose a la autoridad competente.
    Sin perjuicio de lo dispuesto en el inciso anterior, se aplicará el máximum de la pena privativa de libertad allí señalada y multa equivalente al doble de la tasación fiscal, al autor de receptación de vehículos motorizados que conociere o no pudiere menos que conocer que en la apropiación de éste se ejerció sobre su legítimo tenedor alguna de las conductas descritas en el artículo 439. Lo dispuesto en este inciso no será aplicable a quien, por el mismo hecho, le correspondiere participación responsable por cualquiera de las hipótesis del delito de robo previstas en el artículo 433 y en el inciso primero del artículo 436.
    Se impondrá el grado máximo de la pena establecida en el inciso primero, cuando el autor haya incurrido en reiteración de esos hechos o sea reincidente en ellos. En los casos de reiteración o reincidencia en la receptación de los objetos señalados en el inciso tercero, se aplicará la pena privativa de libertad allí establecida, aumentada en un grado.
    Tratándose del delito de abigeato o sustracción de madera y la multa establecida en el inciso primero será de setenta y cinco a cien unidades tributarias mensuales y el juez podrá disponer la clausura definitiva del establecimiento.
    Si el valor de lo receptado excediere de cuatrocientas unidades tributarias mensuales, se impondrá el grado máximo de la pena o el máximun de la pena que corresponda en cada caso.





    § VI.

    De la usurpación.





    ART. 457.

    Al que con violencia en las personas ocupare una cosa inmueble o usurpare un derecho real que otro poseyere o tuviere legítimamente, y al que, hecha la ocupación en ausencia del legítimo poseedor o tenedor, vuelto éste le repeliere, además de las penas en que incurra por la violencia que causare, se le aplicará una multa de once a veinte unidades tributarias mensuales.
    Si tales actos se ejecutaren por el dueño o poseedor regular contra el que posee o tiene ilegítimamente la cosa, aunque con derecho aparente, la pena será multa de seis a diez unidades tributarias mensuales, sin perjuicio de las que correspondieren por la violencia causada.






    ART. 458.

    Cuando, en los casos del inciso primero del artículo anterior, el hecho se llevare a efecto sin violencia en las personas, la pena será multa de seis a diez unidades tributarias mensuales.









    ART. 459.

    Sufrirán las penas de presidio menor en sus grados medio a máximo y multa de veinte a cinco mil unidades tributarias mensuales, los que sin título legítimo e invadiendo derechos ajenos:
    1.° Sacaren aguas de represas, estanques u otros depósitos; de ríos, arroyos o fuentes, sean superficiales o subterráneas; de canales o acueductos, redes de agua potable e instalaciones domiciliarias de éstas, y se las apropiaren para hacer de ellas un uso cualquiera.
    2.° Rompieren o alteraren con igual fin diques, esclusas, compuertas, marcos u otras obras semejantes existentes en los ríos, arroyos, fuentes, depósitos, canales o acueductos.
    3.° Pusieren embarazo al ejercicio de los derechos que un tercero tuviere sobre dichas aguas.
    4.° Usurparen un derecho cualquiera referente al curso de ellas o turbaren a alguno en su legítima posesión.
    Las sanciones establecidas en este artículo no se aplicarán a quienes hagan uso del agua para consumo personal o familiar en los términos señalados en el artículo 56 del Código de Aguas.


    ART. 460.

    Cuando los simples delitos a que se refiere el artículo anterior se ejecutaren con violencia o intimidación en las personas, si el culpable no mereciere mayor pena por la violencia o intimidación que causare, sufrirá la de presidio menor en cualquiera de sus grados y multa de cincuenta a cinco mil unidades tributarias mensuales.









    Artículo 460 bis.- El que a sabiendas duplique la inscripción de su derecho en el Registro de Propiedad de Aguas del Conservador de Bienes Raíces sufrirá las penas de presidio menor en su grado mínimo, multa de once a veinte unidades tributarias mensuales, la revocación del título duplicado y la cancelación de la inscripción duplicada.



    ART. 461.

    Serán castigados con las penas del artículo 459, los que teniendo derecho para sacar aguas o usarlas se hubieren servido fraudulentamente, con tal fin, de orificios, conductos, marcos, compuertas o esclusas de una forma diversa a la establecida o de una capacidad superior a la medida a que tienen derecho.


    ART. 462.

    El que destruyere o alterare términos o límites de propiedades públicas o particulares con ánimo de lucrarse, será penado con presidio menor en su grado mínimo y multa de once a veinte unidades tributarias mensuales.








    § VII.

    De los delitos concursales y de las defraudaciones.



    ART. 463.

    Será castigado con la pena de presidio menor en cualquiera de sus grados el que, dentro de los dos años anteriores a la dictación de la resolución de liquidación a la que se refiere la ley N° 20.720, que Sustituye el régimen concursal vigente por una ley de reorganización y liquidación de empresas y personas, y perfecciona el rol de la superintendencia del ramo, o durante el tiempo que medie entre la notificación de la demanda de liquidación forzosa y la dictación de la respectiva resolución, conociendo el mal estado de sus negocios:

    1.° Redujere considerablemente su patrimonio destruyendo, dañando, inutilizando o dilapidando, activos o valores o renunciando sin razón a créditos.

    2.° Dispusiere de sumas relevantes en consideración a su patrimonio aplicándolas en juegos o apuestas o en negocios inusualmente riesgosos en relación con su actividad económica normal.

    3.° Diere créditos sin las garantías habituales en atención a su monto, o se desprendiere de garantías sin que se hubieren satisfecho los créditos caucionados.

    4.° Realizare otro acto manifiestamente contrario a las exigencias de una administración racional del patrimonio.

    Tratándose de una empresa deudora en el sentido de la ley N° 20.720, la pena señalada en el inciso anterior se impondrá también al que hubiere actuado con ignorancia inexcusable del mal estado de sus negocios.

    En el caso del número 4.° del inciso primero, las penas no serán impuestas si el hecho no hubiere contribuido relevantemente a ocasionar la insolvencia del deudor.




    ART. 463 bis.-

    Será castigado con la pena de presidio menor en su grado medio a presidio mayor en su grado mínimo, el deudor que realizare alguna de las siguientes conductas:

    1.° Favorecer a uno o más acreedores en desmedro de otro pagando deudas que no fueren actualmente exigibles u otorgando garantías para deudas contraídas previamente sin garantía, dentro de los dos años anteriores a la resolución de reorganización o liquidación o durante el tiempo que medie entre la notificación de la demanda de liquidación forzosa y la dictación de la respectiva resolución.

    2.° Percibir, apropiarse o distraer bienes que deban ser objeto de cualquier clase de procedimiento concursal de liquidación, después de dictada la resolución de liquidación.

    3.° Realizar actos de disposición de bienes de su patrimonio, reales o simulados, o constituir prenda, hipoteca u otro gravamen sobre ellos, después de la resolución de liquidación.

    4.° Ocultar total o parcialmente sus bienes o sus haberes, dentro de los dos años anteriores a la resolución de liquidación o reorganización, o con posterioridad a esa resolución.

    ART. 463 ter.-

    Será castigado con la pena de presidio menor en sus grados mínimo a medio el deudor que:

    1.° Durante cualquier clase de procedimiento concursal de reorganización o de liquidación, proporcionare al veedor o liquidador, en su caso, o a sus acreedores, información o antecedentes falsos o incompletos, en términos que no reflejen la verdadera situación de su activo o pasivo.

    2.° Dentro de los dos años anteriores a la dictación de la resolución de liquidación o durante el tiempo que medie entre la notificación de la demanda de liquidación forzosa y la dictación de la respectiva resolución, no hubiese llevado o conservado los libros de contabilidad y sus respaldos exigidos por la ley que deben ser puestos a disposición del liquidador una vez dictada la resolución de liquidación, o si hubiese ocultado, inutilizado, destruido o falseado la información en términos que ella no refleje la verdadera situación de su activo y pasivo.

    ART. 463 quáter.-
    Será castigado como autor de los delitos contemplados en los artículos 463, 463 bis y 463 ter quien, en la dirección o administración de los negocios del deudor, sometido a un procedimiento concursal de reorganización a un procedimiento concursal de reorganización simplificada, a un procedimiento concursal de liquidación o a un procedimiento concursal de liquidación simplificada, hubiese ejecutado alguno de los actos o incurrido en alguna de las omisiones allí señalados, o hubiese autorizado expresamente dichos actos u omisiones.



    ART. 464.

    Será castigado con la pena de presidio menor en su grado máximo a presidio mayor en su grado mínimo y con la sanción accesoria de inhabilidad especial perpetua para ejercer el cargo, el veedor o liquidador designado en cualquier clase de procedimiento concursal de reorganización o de liquidación que:

    1. Proporcionare ventajas indebidas al deudor, a un acreedor o a un tercero.

    2. Perpetrare cualquiera de los hechos previstos en los números 1 u 11 del artículo 470.

    ART. 464 bis.-

    El deudor, veedor, liquidador, o aquellos a los que se refiere el artículo 463 quáter, que se valiere de quien no tuviere esa calidad para perpetrar cualquiera de los delitos previstos en los artículos precedentes de este Párrafo será castigado como autor del respectivo delito.
    El que sin tener alguna de las calidades señaladas en el inciso precedente interviniere en la perpetración del delito será castigado como inductor o cómplice según las circunstancias.

    ART. 464 ter.-

    El que mediante engaño determinare a un deudor, veedor, liquidador, o aquellos a los que se refiere el artículo 463 quáter, a incurrir en cualquiera de los hechos previstos en los artículos precedentes de este Párrafo, será castigado con las mismas penas en ellos señalada.
    ART. 464 quáter.-

    Además de lo dispuesto en los artículos 27 a 31, el profesional que, con ocasión del ejercicio de su profesión, fuere penalmente responsable por haber intervenido en la perpetración de cualquiera de los delitos previstos en el presente Párrafo, será sancionado también con la pena accesoria de suspensión o inhabilitación para su ejercicio.
    La pena y su duración serán determinadas atendiendo a la pena principal impuesta conforme a las reglas previstas en los artículos 29 y 30 de este Código, para la inhabilitación o suspensión de cargo u oficio público.
    ART. 465.
    La persecución penal de los delitos contemplados en este Párrafo sólo podrá iniciarse previa instancia particular de la Superintendencia de Insolvencia y Reemprendimiento; del veedor o liquidador del proceso concursal respectivo; de cualquier acreedor que haya verificado su crédito si se tratare de un procedimiento concursal de liquidación o de liquidación simplificada, lo que se acreditará con copia autorizada del respectivo escrito y su proveído; o en el caso de un procedimiento concursal de reorganización o reorganización simplificada, de todo acreedor a quien le afecte el Acuerdo de Reorganización de conformidad a lo establecido en los artículos 66 y 286 F de la ley Nº 20.720.
    Si se tratare de delitos de este Párrafo cometidos por veedores o liquidadores, la Superintendencia de Insolvencia y Reemprendimiento deberá denunciarlos si alguno de los funcionarios de su dependencia toma conocimiento de aquéllos en el ejercicio de sus funciones. Además, podrá interponer querella criminal, entendiéndose para este efecto cumplidos los requisitos que establece el inciso tercero del artículo 111 del Código Procesal Penal.
    Cuando se celebren acuerdos reparatorios de conformidad al artículo 241 y siguientes del Código Procesal Penal, los términos de esos acuerdos deberán ser aprobados previamente por la junta de acreedores respectiva y las prestaciones que deriven de ellos beneficiarán a todos los acreedores, a prorrata de sus respectivos créditos, sin distinguir para ello la clase o categoría de los mismos.
    Conocerá de los delitos concursales regulados en este Párrafo el tribunal con competencia en lo criminal del domicilio del deudor.
   

    ART. 465 bis.-Derogado.

    ART. 466. Derogado.





    § VIII.

    Estafas y otros engaños.





    ART. 467.-

    El que para obtener provecho patrimonial para sí o para un tercero mediante engaño provocare un error en otro, haciéndolo incurrir en una disposición patrimonial consistente en ejecutar, omitir o tolerar alguna acción en perjuicio suyo o de un tercero será sancionado:

    1. Con presidio menor en su grado máximo y multa de veintiuna a trescientas unidades tributarias mensuales, si el perjuicio excede de cuatrocientas unidades tributarias mensuales y no pasa de cuarenta mil.

    2. Con presidio menor en sus grados medio a máximo y multa de once a quince unidades tributarias mensuales, si excede de cuarenta unidades tributarias mensuales y no pasa de cuatrocientas.

    3. Con presidio menor en su grado medio y multa de seis a diez unidades tributarias mensuales, si excede de cuatro unidades tributarias mensuales y no pasa de cuarenta.

    4. Con presidio menor en su grado mínimo y multa de cinco unidades tributarias mensuales, si excede de una unidad tributaria mensual y no pasa de cuatro.

    Si el perjuicio excede de cuarenta mil unidades tributarias mensuales, se aplicará la pena de presidio menor en su grado máximo a presidio mayor en su grado mínimo y multa de trescientas a quinientas unidades tributarias mensuales.




    ART. 468.

    Incurrirá en el delito previsto en el artículo anterior el que defraudare a otro usando de nombre fingido, atribuyéndose poder, influencia o crédito supuestos, aparentando bienes, crédito, comisión, empresa o negociación imaginarios, o valiéndose de cualquier otro engaño semejante.

    Las penas del artículo anterior serán aplicadas también al que para obtener un provecho para sí o para un tercero irrogue perjuicio patrimonial a otra persona:

    1. Manipulando los datos contenidos en un sistema informático o el resultado del procesamiento informático de datos a través de una intromisión indebida en la operación de éste.

    2. Utilizando sin la autorización del titular una o más claves confidenciales que habiliten el acceso u operación de un sistema informático, o

    3. Haciendo uso no autorizado de una tarjeta de pago ajena o de los datos codificados en una tarjeta de pago que la identifiquen y habiliten como medio de pago.

    Sin perjuicio de las penas que correspondan conforme al inciso anterior, sufrirá la pena de presidio menor en su grado medio y multa de seis a diez unidades tributarias mensuales el que obtenga indebidamente los datos codificados en una tarjeta de pago que la identifiquen y habiliten como medio de pago. La misma pena sufrirá el que los adquiera o ponga a disposición de otro a cualquier título.

    En la investigación de los delitos previstos en este artículo será aplicable lo dispuesto en el artículo 8 de la ley N° 20.009.

    Lo dispuesto en los incisos segundo y tercero de este artículo será aplicable si el hecho no tuviere mayor pena conforme a otra ley.


    ART. 469.

    Se impondrá respectivamente el máximum de las penas señaladas en el art. 467:
    1.° A los plateros y joyeros que cometieren defraudaciones alterando en su calidad, ley o peso los objetos relativos a su arte o comercio.
    2.° A los traficantes que defraudaren usando de pesos o medidas falsos en el despacho de los objetos de su tráfico.
    3.° A los comisionistas que cometieren defraudación alterando en sus cuentas los precios o las condiciones de los contratos, suponiendo gastos o exagerando los que hubieren hecho.
    4.° A los capitanes de buques que defrauden suponiendo gastos o exagerando los que hubieren hecho, o cometiendo cualquiera otro fraude en sus cuentas.
    5.° A los que cometieren defraudación con pretexto de supuestas remuneraciones a empleados públicos, sin perjuicio de la acción de calumnia que a éstos corresponda.
    6.° Al dueño de la cosa embargada, o a cualquier otro que, teniendo noticia del embargo, hubiere destruido fraudulentamente los objetos en que se ha hecho la traba.



    ART. 470.

    Las penas privativas de libertad del art. 467 se aplicarán también:
    1.° A los que en perjuicio de otro se apropiaren o distrajeren dinero, efectos o cualquiera otra cosa mueble que hubieren recibido en depósito, comisión o administración, o por otro título que produzca obligación de entregarla o devolverla.
    En cuanto a la prueba del depósito en el caso a que se refiere el art. 2.217 del Código Civil, se observará lo que en dicho artículo se dispone.
    2.° A los capitanes de buques que, fuera de los casos y sin las solemnidades prevenidas por la ley, vendieren dichos buques, tomaren dinero a la gruesa sobre su casco y quilla, giraren letras a cargo del naviero, enajenaren mercaderías o vituallas o tomaren provisiones pertenecientes a los pasajeros.
    3.° A los que cometieren alguna defraudación abusando de firma de otro en blanco y extendiendo con ella algún documento en perjuicio del mismo o de un tercero.
    4.° A los que defraudaren haciendo suscribir a otro con engaño algún documento.
    5.° A los que cometieren defraudaciones sustrayendo, ocultando, destruyendo o inutilizando en todo o en parte algún proceso, expediente, documento u otro papel de cualquiera clase.
    6.° A los que con datos falsos u ocultando antecedentes que les son conocidos, celebraren dolosamente contratos aleatorios basados en dichos datos o antecedentes.
    7.° A los que en el juego se valieren de fraude para asegurar la suerte.
    8.° A los que fraudulentamente obtuvieren del Fisco, de las municipalidades, de las Cajas de Previsión y de las instituciones centralizadas o descentralizadas del Estado, prestaciones improcedentes, tales como remuneraciones, bonificaciones, subsidios, pensiones, jubilaciones, asignaciones, devoluciones o imputaciones indebidas.
    9.° Al que, con ánimo de defraudar, con o sin representación de persona natural o jurídica dedicada al rubro inmobiliario o de la construcción, suscribiere o hiciere suscribir contrato de promesa de compraventa de inmueble dedicado a la vivienda, local comercial u oficina, sin cumplir con las exigencias establecidas por el artículo 138 bis de la Ley General de Urbanismo y Construcciones, siempre que se produzca un perjuicio patrimonial para el promitente comprador.
    10.° A los que maliciosamente obtuvieren para sí, o para un tercero, el pago total o parcialmente indebido de un seguro, sea simulando la existencia de un siniestro, provocándolo intencionalmente, presentándolo ante el asegurador como ocurrido por causas o en circunstancias distintas a las verdaderas, ocultando la cosa asegurada o aumentando fraudulentamente las pérdidas efectivamente sufridas.
    Si no se verifica el pago indebido por causas independientes de su voluntad, se aplicará el mínimo o, en su caso, el grado mínimo de la pena.
    La pena se determinará de acuerdo con el monto de lo indebidamente solicitado.
    11. Al que teniendo a su cargo la salvaguardia o la gestión del patrimonio de otra persona, o de alguna parte de éste, en virtud de la ley, de una orden de la autoridad o de un acto o contrato, le irrogare perjuicio, sea ejerciendo abusivamente facultades para disponer por cuenta de ella u obligarla, sea ejecutando u omitiendo cualquier otra acción de modo manifiestamente contrario al interés del titular del patrimonio afectado.
    Si el hecho recayere sobre el patrimonio de una persona en relación con la cual el sujeto fuere guardador, tutor o curador, o de una persona incapaz que el sujeto tuviere a su cargo en alguna otra calidad, se impondrá, según sea el caso, el máximum o el grado máximo de las penas señaladas en el artículo 467.
    En caso de que el patrimonio encomendado fuere el de una sociedad anónima abierta o especial u otro patrimonio administrado por esa sociedad, el administrador que realizare alguna de las conductas descritas en el párrafo primero de este numeral, irrogando perjuicio al patrimonio social, será sancionado con las penas señaladas en el artículo 467 aumentadas en un grado. Además, se impondrá la pena de inhabilitación especial temporal en su grado mínimo para desempeñarse como gerente, director, liquidador o administrador a cualquier título de una sociedad o entidad sometida a fiscalización de una Superintendencia o de la Comisión para el Mercado Financiero.
    En los casos previstos en este artículo se impondrá, además, pena de multa de la mitad al tanto de la defraudación.






    ART. 471.

    Será castigado con presidio o relegación menores en sus grados mínimos o multa de once a veinte unidades tributarias mensuales:
    1.º El dueño de una cosa mueble que la sustrajere de quien la tenga legítimamente en su poder, con perjuicio de éste o de un tercero.
    2.° El que otorgare en perjuicio de otro un contrato simulado.
    3.° Derogado.
    Los ejemplares, máquinas u objetos contrahechos, introducidos o expendidos fraudulentamente, se aplicarán al perjudicado y también las láminas o utensilios empleados en la ejecución del fraude, cuando solo pudieren usarse para cometerlo.

    ART. 472.

    El que suministre valores, de cualquiera manera que sea, a un interés que exceda del máximo que la ley permita estipular, será castigado con presidio o reclusión menores en cualquiera de sus grados.
    Se impondrá el grado máximo de la pena establecida en el inciso anterior cuando la conducta que allí se sanciona se realice simulando, de cualquier forma, que se suministran los valores a un interés permitido por la ley.
    Condenado por usura un extranjero, será expulsado del país; y condenado como reincidente en delito de usura un nacionalizado, se le cancelará su nacionalización y se le expulsará del país.
    En ambos casos la expulsión se hará después de cumplida la pena.
    En la sustanciación y fallo de los procesos instruidos para la investigación de estos delitos, los Tribunales apreciarán la prueba en conciencia.




    ART. 472 bis.-

    El que con abuso grave de una situación de necesidad, de la inexperiencia o de la incapacidad de discernimiento de otra persona, le pagare una remuneración manifiestamente desproporcionada e inferior al ingreso mínimo mensual previsto por la ley o le diere en arrendamiento un inmueble como morada recibiendo una contraprestación manifiestamente desproporcionada, será castigado con la pena de presidio o reclusión menor en cualquiera de sus grados.
    ART. 472 ter.-

    En los casos en que alguno de los hechos previstos en este Párrafo irrogare un perjuicio que exceda de ochenta mil unidades tributarias mensuales o afecte a un número considerable de personas, se podrá imponer la pena superior en un grado a la señalada por la ley.

    ART. 473.

    El que defraudare o perjudicare a otro usando de cualquier engaño que no se halle expresado en los artículos anteriores de este párrafo, será castigado con presidio o relegación menores en sus grados mínimos y multa de once a veinte unidades tributarias mensuales.








    § IX.

    Del incendio y otros estragos.





    Artículo 474. El que incendiare edificio, aeronave, buque, plataforma naval, automóviles de dos o más plazas, camiones, instalaciones de servicios sanitarios, de almacenamiento o transporte de combustibles, de distribución o generación de energía eléctrica, portuaria, aeronáutica o ferroviaria, incluyendo las de trenes subterráneos, u otro lugar, medio de transporte, instalación o bien semejante, siempre que hubiere personas en su interior, causando la muerte de una o más personas cuya presencia allí pudo prever, será castigado con presidio mayor en su grado máximo a presidio perpetuo.
    La misma pena se impondrá cuando del incendio no resultare muerte sino mutilación de miembro importante o lesión grave de las comprendidas en el número 1° del artículo 397.



    Artículo 475. El que incendiare edificio, aeronave, buque, plataforma naval, vehículos de transporte público de pasajeros, automóviles de dos o más plazas, camiones, instalaciones de servicios sanitarios, de almacenamiento o transporte de combustibles, de distribución o generación de energía eléctrica, portuaria, aeronáutica o ferroviaria, incluyendo las de trenes subterráneos, u otro lugar, medio de transporte, instalación o bien semejante, siempre que allí hubiere una o más personas y su presencia se pudiese prever, será castigado con presidio mayor en su grado medio a presidio perpetuo.



    ART. 476.

    Se castigará con presidio mayor en cualquiera de sus grados:
    1.° Al que incendiare un edificio o lugar destinado a servir de morada, que no estuviere actualmente habitado.
    2º Al que dentro de poblado ejecutare el incendio en edificio, aeronave, buque, plataforma naval, vehículos de transporte público de pasajeros, automóviles de dos o más plazas, camiones, instalaciones de servicios sanitarios, de almacenamiento o transporte de combustibles, de distribución o generación de energía eléctrica, portuaria, aeronáutica o ferroviaria, incluyendo las de trenes subterráneos, u otro lugar, medio de transporte, instalación o bien semejante, cuando no hubiere personas en su interior o su presencia no se pudiese prever.
    3.º Al que incendiare bosques, mieses, pastos, montes, cierros, plantíos o formaciones xerofíticas de aquellas definidas en la ley Nº 20.283.
    4.º Al que fuera de los casos señalados en los números anteriores provoque un incendio que afectare gravemente las condiciones de vida animal o vegetal de un Área Silvestre Protegida.



    Artículo 477.- El incendiario de objetos no comprendidos en los artículos anteriores será penado:
    1.º Con presidio menor en su grado máximo a presidio mayor en su grado mínimo y multa de once a quince unidades tributarias mensuales, si el daño causado a terceros excediere de cuarenta unidades tributarias mensuales.
    2.º Con presidio menor en sus grados medio a máximo y multa de seis a diez unidades tributarias mensuales, si el daño excediere de cuatro unidades tributarias mensuales y no pasare de cuarenta unidades tributarias mensuales.
    3.º Con presidio menor en sus grados mínimo a medio y multa de cinco unidades tributarias mensuales, si el daño excediere de una unidad tributaria mensual y no pasare de cuatro unidades tributarias mensuales.


    ART. 478.

    En caso de aplicarse el incendio a chozas, pajar o cobertizo deshabitado o a cualquier otro objeto cuyo valor no excediere de cuatro sueldos vitales, en tiempo y con circunstancias que manifiestamente excluyan todo peligro de propagación, el culpable no incurrirá en las penas señaladas en este párrafo; pero sí en las que mereciera por el daño que causare con arreglo a las disposiciones del párrafo siguiente.


    ART. 479.

    Cuando el fuego se comunicare del objeto que el culpable se propuso quemar, a otro u otros cuya destrucción, por su naturaleza o consecuencias, debe penarse con mayor severidad, se aplicará la pena más grave, siempre que los objetos incendiados estuvieren colocados de tal modo que el fuego haya debido comunicarse de unos a otros, atendidas las circunstancias del caso.


    ART. 480.

    Incurrirán respectivamente en las penas de este párrafo los que causen estragos por medio de sumersión o varamiento de nave, inundación, destrucción de puentes o máquinas de vapor, y en general por la aplicación de cualquier otro agente o medio de destrucción tan poderoso como los expresados.



    ART. 481.

    El que fuere aprehendido con artefactos, implementos o preparativos conocidamente dispuestos para incendiar o causar alguno de los estragos expresados en este párrafo, será castigado con presidio menor en sus grados mínimo a medio; salvo que pudiendo considerarse el hecho como tentativa de un delito determinado debiera castigarse con mayor pena.



    ART. 482.

    El culpable de incendio o estragos no se eximirá de las penas de los artículos anteriores, aunque para cometer el delito hubiere incendiado o destruido bienes de su pertenencia.
    Pero no incurrirá en tales penas el que rozare a fuego, incendiare rastrojos u otros objetos en tiempos y con circunstancias que manifiestamente excluyan todo propósito de propagación, y observando los reglamentos que se dicten sobre esta materia.



    ART. 483.

    Se presume responsable de un incendio al comerciante en cuya casa o establecimiento tiene origen aquél, si no justificare con sus libros, documentos u otra clase de prueba, que no reportaba provecho alguno del siniestro.
    Se presume también responsable de un incendio al comerciante cuyo seguro sea exageradamente superior al valor real del objeto asegurado en el momento de producirse el siniestro. En los casos de seguros con pólizas flotantes se presumirá responsable al comerciante que, en la declaración inmediatamente anterior al siniestro, declare valores manifiestamente superiores a sus existencias. Asimismo, se presume responsable si en todo o en parte a disminuido o retirado las cosas aseguradas del lugar señalado en la póliza respectiva sin motivo justificado o sin dar aviso previo al asegurador.
    Las presunciones de este artículo no obstan a la apreciación de la prueba en conciencia.

    ART. 483. a)

    El contador o cualquiera persona que falsee o adultere la contabilidad del comerciante que sufra un siniestro, será sancionado con la pena señalada en el inciso segundo del artículo 197; pero no le afectará responsabilidad al contador por las existencias y precios inventariados.

    ART. 483. b)

    A los comerciantes responsables del delito de incendio se les aplicará también una multa de veintiuna a cincuenta unidades tributarias mensuales, tomándose en cuenta para graduarla la naturaleza, entidad y gravedad del siniestro y las facultades económicas del condenado.

    Si no se paga la multa el condenado sufrirá por vía de sustitución y apremio, un día de reclusión por un un quinto de unidad tributaria mensual de multa, no pudiendo exceder la reclusión de seis meses.

    La multa impuesta se mantendrá en una cuenta especial a la orden de la Superintendencia de Compañia de Seguros Sociedades Anónimas y Bolsas de Comercio, la cual anualmente la distribuirá proporcionalmente entre los distintos Cuerpos de Bomberos en el país.

    § X.

    De los daños.





    ART. 484.

    Incurren en el delito de daños y están sujetos a las penas de este párrafo, los que en la propiedad ajena causaren alguno que no se halle comprendido en el párrafo anterior.
    ART 485.

    Serán castigados con la pena de reclusión menor en sus grados medio a máximo y multa de once a veinte unidades tributarias mensuales, los que causaren daño cuyo importe exceda de cuarenta unidades tributarias mensuales:

    1.° Con la mira de impedir el libre ejercicio de la autoridad o en venganza de sus determinaciones, bien se cometiere el delito contra empleados públicos, bien contra particulares que, como testigos o de cualquiera otra manera, hayan contribuido o puedan contribuir a la ejecución o aplicación de las leyes;
    2.° Produciendo, por cualquier medio, infección o contagio en animales o aves domésticas;
    3.° Empleando sustancias venenosas o corrosivas;
    4.° En cuadrilla y en despoblado;
    5.° En archivos, registros, bibliotecas o museos públicos;
    6.° En puentes, caminos, paseos u otros bienes de uso público;
    7.° En tumbas, signos conmemorativos, monumentos, estatuas, cuadros u otros objetos de arte colocados en edificios o lugares públicos;
    8.° Arruinando al perjudicado;
    9.° En medios de transporte público de pasajeros o en bienes o infraestructura asociada a dicha actividad.




    ART. 486.

    El que, con alguna de las circunstancias expresadas en el artículo anterior, causare daño cuyo importe exceda de cuatro unidades tributarias mensuales y no pase de cuarenta unidades tributarias mensuales, sufrirá la pena de reclusión menor en sus grados mínimo a medio y multa de seis a diez unidades tributarias mensuales.
    Cuando dicho importe no excediere de cuatro unidades tributarias mensuales ni bajare de una unidad tributaria mensual, la pena será reclusión menor en su grado mínimo y multa de cinco unidades tributarias mensuales.


    ART. 487.

    Los daños no comprendidos en los artículos anteriores, serán penados con reclusión menor en su grado mínimo o multa de once a veinte unidades tributarias mensuales.
    Esta disposición no es aplicable a los daños causados por el ganado y a los demás que deben calificarse de faltas, con arreglo a lo que se establece en el Libro tercero.






    ART. 488.

    Las disposiciones del presente párrafo sólo tendrán lugar cuando el hecho no pueda considerarse como otro delito que merezca mayor pena.



    § XI.

    Disposiciones generales.





    ART. 489.

    Están exentos de responsabilidad criminal y sujetos únicamente a la civil por los hurtos, defraudaciones o daños que recíprocamente se causaren:
    1.º Los parientes consanguíneos en toda la línea recta.
    2.º Los parientes consanguíneos hasta el segundo grado inclusive de la línea colateral.
    3.° Los parientes afines en toda la línea recta.
    4.° Derogado.
    5.° Los cónyuges.
    6.° Los convivientes civiles.

    La excepción de este artículo no es aplicable a los extraños que participaren del delito, ni tampoco entre cónyuges cuando se trate de los delitos de daños indicados en el párrafo anterior.

    Además, esta exención no será aplicable cuando la víctima sea una persona mayor de sesenta años.






    TÍTULO DÉCIMO.

    DE LOS CUASIDELITOS.




    ART. 490.

    El que por imprudencia temeraria ejecutare un hecho que, si mediara malicia, constituiría un crimen o un simple delito contra las personas, será penado:
    1.º Con reclusión o relegación menores en sus grados mínimos a medios, cuando el hecho importare crimen.
    2.° Con reclusión o relegación menores en sus grados mínimos o multa de once a veinte unidades tributarias mensuales, cuando importare simple delito.




    ART. 491.

    El médico, cirujano, farmacéutico, flebotomiano o matrona que causare mal a las personas por negligencia culpable en el desempeño de su profesión, incurrirá respectivamente en las penas del artículo anterior.
    Iguales penas se aplicarán al dueño de animales feroces que, por descuido culpable de su parte, causaren daño a las personas.



    ART. 492.

    Las penas del artículo 490 se impondrán también respectivamente al que, con infracción de los reglamentos y por mera imprudencia o negligencia, ejecutare un hecho o incurriere en una omisión que, a mediar malicia, constituiría un crimen o un simple delito contra las personas.
    A los responsables de cuasidelito de homicidio o lesiones, ejecutados por medio de vehículos a tracción mecánica o animal, se los sancionará, además de las penas indicadas en el artículo 490, con la suspensión del carné, permiso o autorización que los habilite para conducir vehículos, por un período de uno a dos años, si el hecho de mediar malicia constituyera un crimen, y de seis meses a un año, si constituyera simple delito. En caso de reincidencia, podrá condenarse al conductor a inhabilidad perpetua para conducir vehículos a tracción mecánica o animal, cancelándose el carné, permiso o autorización.


    ART. 493.

    Las disposiciones del presente párrafo no se aplicarán a los cuasidelitos especialmente penados en este Código.



    LIBRO TERCERO.


    TÍTULO PRIMERO

    DE LAS FALTAS.


    ART. 494.

    Sufrirán la pena de multa de una a cuatro unidades tributarias mensuales:

    1.° El que asistiendo a un espectáculo público provocare algún desorden o tomare parte en él.
    2.° El que excitare o dirigiere cencerradas u otras reuniones tumultuosas en ofensa de alguna persona o del sosiego de las poblaciones.
    3.° El que ensuciare, arrojare o abandonare basura, materiales o desechos de cualquier índole en playas, riberas de ríos o de lagos, parques nacionales, reservas nacionales, monumentos naturales o en otras áreas de conservación de la biodiversidad declaradas bajo protección oficial.
    La pena consistirá en la prestación de servicios en beneficio de la comunidad consistente en la limpieza de playas, lagos o ríos. Esta pena se regulará conforme a lo dispuesto en los artículos 49, 49 bis, 49 ter, 49 quáter, 49 quinquies y 49 sexies, debiendo existir consentimiento previo del condenado. En caso de no haber consentimiento, se aplicará la pena de multa.
    4.° El que amenazare a otro con armas blancas y el que riñendo con otro las sacare, como no sea con motivo justo.
    5.° El que causare lesiones leves, entendiéndose por tales las que, en concepto del tribunal, no se hallaren comprendidas en el art. 399, atendidas la calidad de las personas y circunstancias del hecho. En ningún caso el tribunal podrá calificar como leves las lesiones cometidas en contra de las personas mencionadas en el artículo 5° de la Ley sobre Violencia Intrafamiliar, ni aquéllas cometidas en contra de las personas a que se refiere el inciso primero del artículo 403 bis de este Código.
    6.° El que corriere carruajes o caballerías con peligro de las personas, haciéndolo en poblado, ya sea de noche o de día cuando haya aglomeración de gente.
    7.° El farmacéutico que despachare medicamentos en virtud de receta que no se halle debidamente autorizada.
    8.° El que habitualmente y después de apercibimiento ejerciere, sin título legal ni permiso de autoridad competente, las profesiones de médico, cirujano, farmacéutico o Dentista.
    9.° El facultativo que, notando en una persona o en un cadáver señales de envenenamiento o de otro delito grave, no diere parte a la autoridad oportunamente.
    10.° El médico, cirujano, farmacéutico, Dentista o matrona que incurriere en descuido culpable en el desempeño de su profesión, sin causar daño a las personas.
    11.° Los mismos individuos expresados en el numero anterior, que no prestaren los servicios de su profesión durante el turno que les señale la autoridad administrativa.
    12.° El médico, cirujano, farmacéutico, matrona o cualquiera otro que, llamado en clase de perito o testigo, se negare a practicar una operación propia de su profesión u oficio o a prestar una declaración requerida por la autoridad judicial, en los casos y en la forma que determine el Código de Procedimiento y sin perjuicio de los apremios legales.
    13.° El que encontrando perdido o abandonado a un menor de siete años no lo entregare a su familia o no lo recogiere o depositare en lugar seguro, dando cuenta a la autoridad en los dos últimos casos.
    14.° El que no socorriere o auxiliare a una persona que encontrare en despoblado herida, maltratada o en peligro de perecer, cuando pudiere hacerlo sin detrimento propio.
    15.° Los padres de familia o los que legalmente hagan sus veces que abandonen a sus hijos, no procurándoles la educación que permiten y requieren su clase y facultades.
    16.° El que sin estar legítimamente autorizado impidiere a otro con violencia hacer lo que la ley no prohíbe, o le compeliere a ejecutar lo que no quiera.
    17.° El que quebrantare los reglamentos o disposiciones de la autoridad sobre la custodia, conservación y trasporte de materias inflamables o corrosivas o productos químicos que puedan causar estragos.
    18.° El dueño de animales feroces que en lugar accesible al público los dejare sueltos o en disposición de causar mal.
    Para estos efectos, se comprenderán como feroces los animales potencialmente peligrosos.
    19.  El que ejecutare alguno de los hechos penados en los artículos 189, 233, 448, 467, 469, 470 y 477, siempre que el delito se refiera a valores que no excedan de una  unidad tributaria mensual.
    20.° El que con violencia se apoderare de una cosa perteneciente a su deudor para hacerse pago con ella.
    21.° El que con violencia en las cosas entrare a cazar o pescar en lugar cerrado, o en lugar abierto contra expresa prohibición intimada personalmente.

    Con todo, tratándose de las faltas mencionadas en el número 19, la multa no será inferior al valor malversado o defraudado, al de la cosa hurtada o del daño causado, en su caso, y podrá alcanzar el doble de ese valor, aun cuando supere una unidad tributaria mensual.







    ART. 494 bis.

    Los autores de hurto serán castigados con prisión en su grado mínimo a medio y multa de una a cuatro unidades tributarias mensuales, si el valor de la cosa hurtada no pasa de media unidad tributaria mensual.
    La falta de que trata este artículo se castigará con multa de una a cuatro unidades tributarias mensuales, si se encuentra en grado de frustrada. En estos casos, el tribunal podrá conmutar la multa por la realización de trabajos determinados en beneficio de la comunidad, señalando expresamente el tipo de trabajo, el lugar donde deba realizarse, su duración y la persona o institución encargada de controlar su cumplimiento. Los trabajos se realizarán, de preferencia, sin afectar la jornada laboral o de estudio que tenga el infractor, con un máximo de ocho horas semanales. La no realización cabal y oportuna de los trabajos determinados por el tribunal dejará sin efecto la conmutación por el solo ministerio de la ley, y deberá cumplirse íntegramente la sanción primitivamente aplicada.
    En los casos en que participen en el hurto individuos mayores de dieciocho años y menores de esa edad, se aplicará a los mayores la pena que les habría correspondido sin esa circunstancia, aumentada en un grado, si éstos se han prevalido de los menores en la perpetración de la falta.
    En caso de reincidencia en hurto falta frustrado, se duplicará la multa aplicada. Se entenderá que hay reincidencia cuando el responsable haya sido condenado previamente por delito de la misma especie, cualquiera haya sido la pena impuesta y su estado de cumplimiento. Si el responsable ha reincidido dos o más veces se triplicará la multa aplicada.
    La agravante regulada en el inciso precedente prescribirá de conformidad con lo dispuesto en el artículo 104. Tratándose de faltas, el término de la prescripción será de seis meses.


    ART. 494 ter.

    Comete acoso sexual el que realizare, en lugares públicos o de libre acceso público, y sin mediar el consentimiento de la víctima, un acto de significación sexual capaz de provocar una situación objetivamente intimidatoria, hostil o humillante, y que no constituya una falta o delito al que se imponga una pena más grave, que consistiere en:
    1. Actos de carácter verbal o ejecutados por medio de gestos. En este caso se impondrá una multa de una a tres unidades tributarias mensuales.
    2. Conductas consistentes en acercamientos o persecuciones, o actos de exhibicionismo obsceno o de contenido sexual explícito. En cualquiera de estos casos se impondrá la pena de prisión en su grado medio a máximo y multa de cinco a diez unidades tributarias mensuales.



    ART. 495.

    Serán castigados con multa de una unidad tributaria mensual:

    1.° El que contraviniere a las reglas que la autoridad dictare para conservar el orden público o evitar que se altere, salvo que el hecho constituya crimen o simple delito.
    2.° El que por quebrantar los reglamentos sobre espectáculos públicos ocasionare algún desorden.
    3.° El subordinado del orden civil que faltare al respeto y sumisión debidos a sus jefes o superiores.
    4.° El particular que cometiere igual falta respecto de cualquier funcionario revestido de autoridad pública, mientras ejerce sus funciones, y respecto de toda persona constituida en dignidad, aun cuando no sea en el ejercicio de sus funciones, siempre que fuere conocida o se anunciare como tal; sin perjuicio de imponer, tanto en este caso como en el anterior, la pena correspondiente al crimen o simple delito, si lo hubiere.
    5.° El que públicamente ofendiere el pudor con acciones o dichos deshonestos.
    6.° El cónyuge que escandalizare con sus disensiones domésticas después de haber sido amonestado por la autoridad.
    7.° El que infringiere los reglamentos de policía en lo concerniente a quienes ejercen el comercio sexual.
    8.° El que diere espectáculos públicos sin licencia de la autoridad, o traspasando la que se le hubiere concedido.
    9.° El que abriere establecimientos sin licencia de la autoridad, cuando sea necesaria.
    10.° El que en la exposición de niños quebrantare los reglamentos.
    11.° El que infringiere las reglas establecidas para la quema de bosques, rastrojos u otros productos de la tierra, o para evitar la propagación de fuego en máquinas de vapor, caleras, hornos u otros lugares semejantes.
    12.° El que infringiere los reglamentos sobre corta de bosques o arbolados.
    13.° El que infringiere las leyes o reglamentos sobre apertura, conservación y reparación de vías públicas.
    14.° El que en caminos públicos, calles, plazas, ferias u otros sitios semejantes de reunión estableciere rifas u otros juegos de envite o azar.
    15.° El que defraudare al público en la venta de mantenimientos, ya sea en calidad, ya en cantidad, por valor que no exceda de una unidad tributaria mensual, y el que vendiere bebidas o mantenimientos deteriorados o nocivos.
    16.° El traficante que tuviere medidas o pesos falsos, aunque con ellos no hubiere defraudado.
    17.° El que usare en su tráfico medidas o pesos no contrastados.
    18.° El dueño o encargado de fondas, cafés, confiterías u otros establecimientos destinados al despacho de comestibles o bebidas que faltare a los reglamentos de policía relativos a la conservación o uso de vasijas o útiles destinados para el servicio.
    19.° El que faltando a las órdenes de la autoridad, descuidare reparar o demoler edificios ruinosos.
    20.° El que infringiere las reglas de seguridad concernientes a la apertura de pozos o excavaciones y al depósito de materiales o escombros, o a la colocación de cualesquiera otros objetos en las calles, plazas, paseos públicos o en la parte exterior de los edificios que embaracen el tráfico o puedan causar daño a los transeúntes.
    21.° El que intencionalmente o con negligencia culpable causare daño que no exceda de una unidad tributaria mensual en bienes públicos o de propiedad particular.
    22.° El que aprovechando aguas de otro o distrayéndolas de su curso, causare daño que no exceda de una unidad tributaria mensual.

    Con todo, la multa para las faltas señaladas en los números 15, 21 y 22 será a lo menos equivalente al valor de lo defraudado o del daño causado y podrá llegar hasta el doble de ese valor, aunque exceda una unidad tributaria mensual.



    ART. 496.

    Sufrirán la pena de multa de una a cuatro unidades tributarias mensuales:

    1.° El que faltare a la obediencia debida a la autoridad dejando de cumplir las órdenes particulares que ésta le diere, en todos aquellos casos en que la desobediencia no tenga señalada mayor pena por este Código o por leyes especiales.
    2.° El que pudiendo, sin grave detrimento propio, prestar a la autoridad el auxilio que reclamare en casos de incendio, inundación, naufragio u otra calamidad, se negare a ello.
    3.° El que impidiere el ejercicio de las funciones fiscalizadoras de los inspectores municipales.
    4.° El que no diere los partes de defunción, contraviniendo a la ley o reglamentos.
    5.° El que ocultare su verdadero nombre y apellido a la autoridad o a persona que tenga derecho para exigir que los manifieste o se negare a manifestarlos o diere domicilio falso.
    6.° El que infringiere las reglas de policía dirigidas a asegurar el abastecimiento de los pueblos.
    7.° El que con rondas u otros esparcimientos nocturnos altere el sosiego público, desobedeciendo a la autoridad.
    8.° El que tomare parte en cencerradas u otras reuniones ofensivas a alguna persona, no estando comprendida en el núm. 2.º del art. 494.
    9.° El que se bañare quebrantando las reglas de decencia o seguridad establecidas por la autoridad.
    10.° El que riñere en público sin armas, salvo el caso de justa defensa propia o de un tercero.
    11.° El que injuriare a otro livianamente de obra o de palabra, no siendo por escrito y con publicidad.
    12.° Derogado.
    13.° El que corriere carruajes o caballerías dentro de una población, no siendo en los casos previstos por el núm. 6.° del art. 494.
    14.° El que infringiere los reglamentos relativos a carruajes públicos o de particulares.
    15.° El que infringiere las reglas de policía relativas a posadas, fondas, tabernas y otros establecimientos públicos.
    16.° El encargado de la guarda de un loco o demente que le dejare vagar por sitios públicos sin la debida seguridad.
    17.° El dueño de animales dañinos que los dejare sueltos o en disposición de causar mal en las poblaciones.
    18.° El que con su embriaguez molestare a tercero en público.
    19.° El que arrojare animales muertos en sitios vedados o quebrantando las reglas de policía.
    20.° El que infringiere las reglas de policía en la elaboración de objetos fétidos o insalubres, o los arrojare a las calles, plazas o paseos públicos.
    21.° El que arrojare escombros u objetos punzantes o cortantes en lugares públicos, contraviniendo a las reglas de policía.
    22.° El que no entregare a la policía de aseo las basuras o desperdicios que hubiere en el interior de su habitación.
    23.° El que echare en las acequias de las poblaciones objetos que, impidiendo el libre y fácil curso de las aguas, puedan ocasionar anegación.
    24.° El que tuviere en balcones, ventanas, azoteas u otros puntos exteriores de sus casas tiestos u otros objetos, con infracción de las reglas de policía.
    25.° El que arrojare a la calle por balcones, ventanas o por cualquiera otra parte agua u objetos que puedan causar daño.
    26.° El que tirare piedras u otros objetos arrojadizos en parajes públicos, con riesgo de los transeúntes, o lo hiciere a las casas o edificios, en perjuicio de los mismos o con peligro de las personas.
    27.° El que infringiere los reglamentos en materia de juegos o diversiones dentro de las poblaciones.
    28.° El que entrare con carruajes, caballerías o animales dañinos en heredades plantadas o sembradas.
    29.° El que en contravención a los reglamentos construyere chimeneas, estufas u hornos, o dejare de limpiarlos o cuidarlos.
    30.° El que, empleando el fuego, elevare globos sin permiso de la autoridad.
    31.° El que, habiendo recibido de buena fe moneda falsa o cercenada o títulos de crédito falsos, los circulare después de constarle su falsedad o cercenamiento, siempre que su valor no exceda de una unidad tributaria mensual.
    32.° Derogado.
    33.° El que entrare en heredad ajena para coger frutas y comerlas en el acto.
    34.° El que entrare sin violencia a cazar o pescar en sitio vedado o cerrado.
    35.° Derogado.
    36.° El que infringiere los reglamentos de caza o pesca en el modo y tiempo de ejecutar una u otra o de vender sus productos.
    37.° Los empresarios del alumbrado público que faltaren a las reglas establecidas para su servicio, y los particulares que infringieren dichas reglas.
    38.° El que indebidamente apagare el alumbrado público o del exterior de los edificios, o de los portales, teatros, u otros lugares de espectáculo o reunión, o el de las escaleras de los mismos.


    ART. 497.

    El dueño de ganados que entraren en heredad ajena cerrada y causaren daño, será castigado con multa, por cada cabeza de ganado:

    1.° De una unidad tributaria mensual, si fuere vacuno, caballar, mular o asnal.
    2.° De  un quinto de unidad tributaria mensual, si fuere lanar o cabrío y la heredad tuviere arbolado.
    4.° Del tanto del daño causado a un tercio mas, si fuere de otra especie no comprendida en los números anteriores.
    Esto mismo se observará si el ganado fuere lanar o cabrío y la heredad no tuviere arbolado.









NOTA
      El numeral 21° de la ley 11183, publicada el 10.06.1953, modifica el presente artículo en el sentido de: sustituir su numeral 1° tal como se transcribe; suprimiendo su numeral 2° original. El numeral 3°pasa a ser 2°, sustituyéndose parte de su contenido por la frase transcrita, pero sin alterar la numeración respecto de su actual numeral 4°. Por tal razón, en la construcción de su texto actualizado se mantiene su numeración sin tener un orden correlativo.

    TÍTULO SEGUNDO

    DISPOSICIONES COMUNES A LAS FALTAS.





    ART. 498.

    Los cómplices en las faltas serán castigados con una pena que no exceda de la mitad de la que corresponda a los autores.


    ART.499.

    Caerán en comiso:
    1.° Las armas que llevare el ofensor al hacer un daño o inferir injuria, si las hubiere mostrado.
    2.° Las bebidas y comestibles deteriorados y nocivos.
    3.° Los efectos falsificados, adulterados o averiados que se expendieren como legítimos o buenos.
    4.º Los comestibles en que se defraudare al público en cantidad o calidad.
    5.° Las medidas o pesos falsos.
    6.° Los enseres que sirvan para juegos o rifas.
    7.° Los efectos que se empleen para adivinaciones u otros engaños semejantes.


    ART. 500.

    El comiso de los instrumentos y efectos de las faltas, expresados en el artículo anterior, lo decretará el tribunal a su prudente arbitrio según los casos y circunstancias.



    ART. 501.

    En las ordenanzas municipales y en los reglamentos generales o particulares que dictare en lo sucesivo la autoridad administrativa no se establecerán mayores penas que las señaladas en este libro, aun cuando hayan de imponerse en virtud de atribuciones gubernativas, a no ser que se determine otra cosa por leyes especiales.


    TÍTULO FINAL.

    DE LA OBSERVANCIA DE ESTE CÓDIGO.





    ARTÍCULO FINAL.

    El presente Código comenzará a regir el primero de marzo de mil ochocientos setenta y cinco, y en esa fecha quedarán derogadas las leyes y demás disposiciones preexistentes sobre todas las materias que en él se tratan.


    Y por cuanto, oído el Consejo de Estado, he tenido a bien aprobarlo y sancionarlo; por tanto promúlguese y llévese a efecto en todas sus partes como ley de la República.

    FEDERICO ERRÁZURIZ.

    JOSÉ MARÍA BARCELÓ.